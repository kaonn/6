\section{Fold}

Now we can characterize the complexity of some familiar functions. 
First, add lists to the language: 

\begin{align*}
    \mathsf{Val} \quad v &::= \listty{A}
    \mid \nil
    \mid \cons{u}{v}\\
    \mathsf{Comp} \quad M &::= \lrec{v}{M_0}{x,xs,f}{M_1}
\end{align*}

\begin{mathpar}

\inferrule{
}{
  \lrec{\nil}{M_0}{x,xs,f}{M_1} \mapsto M_0
}

\inferrule{
}{
  \lrec{\cons{u}{v}}{M_0}{x,xs,f}{M_1} \mapsto 
  \seq{\thunk{\lrec{v}{M_0}{x,xs,f}{M_1}}}{f}{[u/x,v/xs]M_1}
}
\end{mathpar}

For convenience, we define a resource monoid on pairs of natural numbers (from 
Section 3.3.1 in~\cite{Hoffmann11}: 
\[
(p,q) \cdot (p',q') \triangleq \begin{cases}
(p + p' - q, q') \quad\text{ if } $p' > q$\\
(p, q - p' + q') \quad\text{ else }
\end{cases}
\]

Then we have some properties:

\begin{lemma}(Resource Monoid)
\begin{enumerate}
\item If $p \ge c + q$ and $p' \ge c' + q'$ then $r \ge c+c' + s$ where 
$(r,s) = (p,q) \cdot (p',q')$
\item If $p > q$, then $(p,q)^n = (np - (n-1)q, q)$. Otherwise, 
$(p,q)^n = (p, nq - (n-1)p)$. 
\end{enumerate}
\end{lemma}

We can lift the monoid to the computation layer:

\begin{verbatim}
ressum = \(p,q),(p',q') ->
  if p' > q then ret((p+p'-q,q'))
  else ret((p, q+q'-p'))
\end{verbatim}

Suppose we define fold as follows:

\[
fold \triangleq \lam{f,b,l}{}{\lrec{l}{\ret{b}}{x,xs,z}{f(x,z)}}
\]

Then we have the following:

\begin{lemma}(Fold)\label{lemma:fold}
Given
\begin{gather*}
\isVal{f}{\arrabtc{\dprodc{A}{B}}{B}{x.P,x.y.Q}}\\
\isVal{b}{B}\\
\isVal{l}{\listtyc{A}}
\end{gather*}
let 
\begin{verbatim}
T = fold (\(a,(v,p,q)) ->
  v' <- f(a,v);
  p' <- [(a,v)/x]P;
  q' <- [(a,v)/x,v'/y]Q;
  ressum (p,q) (p',q')
) (0,0) l
\end{verbatim}

then \isCComp{fold\,f\,b\,l}{B}{\fst{T} + 3n+2}{\snd{T}}
\end{lemma}

\begin{proof}
Let $l = [a_0,...,a_{n-1}]$. Unfolding the computation of fold:
\begin{verbatim}
fold f b l 
==>^3 lrec{l}(...) 
==> v1 <- lrec{[a1,...,a{n-})]}(...);
    f(a0,v1)
==> v1 <- 
      v2 <- lrec{[a2,...a{n-1}]}(...);
      f(a1,v2);
    f(a0,v1)
... // n-2 steps
==> v1 <- 
      v2 <- 
        ...
        vn <- lrec{[]}(...);
        f(a{n-1},vn);
        ...
      f(a1,v2);
    f(a0,v1)
==> v1 <- 
      v2 <- 
        ...
        vn <- ret(b);
        f(a{n-1},vn);
        ...
      f(a1,v2);
    f(a0,v1)
==> v1 <- 
      v2 <- 
        ...
        v{n-1} <- f(a{n-1},b);
        ...
      f(a1,v2);
    f(a0,v1)
\end{verbatim}
So essentially we have to compute 
\[
f(a_0,f(a_1,...,f(a_{n-1},b)...))
\]
Let each \evalCost{f(a_i,...)}{c_i}{v_i}, and $v_n \triangleq b$. We know by assumption that 
\begin{gather*}
\eval{[(a_i,v_{i+1})/x]P}{\bar{p_i}},\\
\eval{[(a_i,v_{i+1})/x,v_i/y]Q}{\bar{q_i}},\\
p_i \ge c_i + q_i
\end{gather*}

Thus the total cost for the function calls is $c = sum_i(c_i)$, which is bounded by
the sum in the resource monoid: 
\[
(p_{i-1}, q_{i-1}) \cdot (p_{i-2}, q_{i-2}) \cdot \dots \cdot (p_0,q_0) 
\]

This can be computed as another fold:

\begin{verbatim}
t = fold (\(a,(v,p,q)) ->
  v' <- f(a,v);
  p' <- [(a,v)/x]P;
  q' <- [(a,v)/x,v'/y]Q;
  ressum (p,q) (p',q')
) (0,0) l

p,q = t.1, t.2
\end{verbatim}
Where $p \ge c + q$.

Furthermore, we need 1 unit for each function application, and 1 unit for each sequence, 
which accounts for another $2n-1$ units.

Hence in total we need $3 + n + c + 2n-1 = c + 3n+2$, and is bounded by the given cost specification.
\end{proof}

We can define map in terms of fold: 

\[
map \triangleq \lam{f,l}{}{fold\,(\lam{(a,b)}{}{\bind{f(a)}{v}{\ret{\cons{v}{b}}}})\,\nil\,l}
\]

In particular, assuming \isVal{f}{\arrabtc{A}{B}{\_.\ret{p},\_.\_.\ret{q}}}, we can deduce that 
\[
\isVal{f'}{\arrabtc{\dprodc{A}{B}}{\listtyc{B}}{\_.\ret{p+2},\_.\_.\ret{q}}}
\]

where $f'$ is the function argument to fold.
Then the total function cost can be bounded by $(p+2,q)^n = (n(p+2) - (n-1)q,q) = (n(p-q+2) + q,q)$,
i.e. it suffices to start map with $n(p-q+2)$ units, where $n = |l|$.

