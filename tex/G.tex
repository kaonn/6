% --------------------------------------------------------------
% This is all preamble stuff that you don't have to worry about.
% Head down to where it says "Start here"
% --------------------------------------------------------------
 
\documentclass[12pt]{article}
 
\usepackage[margin=1in]{geometry} 
\usepackage{amsmath,amsthm,amssymb,tikz}
\usepackage{mathpartir}
\usepackage{stix}
\usepackage{tabularx}
\usepackage{booktabs,colortbl}
\usepackage{makecell}
\usepackage{tgpagella}
\usepackage[normalem]{ulem}
\usepackage{xargs}
\usepackage{enumitem}
\usepackage{wasysym}
 
\newcommand{\N}{\mathbb{N}}
\newcommand{\Z}{\mathbb{Z}}
\newcommand{\R}{\mathbb{R}}
 
\newenvironment{theorem}[2][Theorem]{\begin{trivlist}
\item[\hskip \labelsep {\bfseries #1}\hskip \labelsep {\bfseries #2.}]}{\end{trivlist}}
\newtheorem{lemma}{Lemma}[section]
\newtheorem{definition}{Definition}[section]
\newtheorem{corollary}{Corollary}[lemma]
\newenvironment{exercise}[2][Exercise]{\begin{trivlist}
\item[\hskip \labelsep {\bfseries #1}\hskip \labelsep {\bfseries #2.}]}{\end{trivlist}}
\newenvironment{reflection}[2][Reflection]{\begin{trivlist}
\item[\hskip \labelsep {\bfseries #1}\hskip \labelsep {\bfseries #2.}]}{\end{trivlist}}
\newenvironment{proposition}[2][Proposition]{\begin{trivlist}
\item[\hskip \labelsep {\bfseries #1}\hskip \labelsep {\bfseries #2.}]}{\end{trivlist}}
\iffalse
\newenvironment{corollary}[2][Corollary]{\begin{trivlist}
\item[\hskip \labelsep {\bfseries #1}\hskip \labelsep {\bfseries #2.}]}{\end{trivlist}}
\fi
\newenvironment{problem}[2][Problem]{\begin{trivlist}
\item[\hskip \labelsep {\bfseries #1}\hskip \labelsep {\bfseries #2.}]}{\end{trivlist}}

\newcommand*{\rulefiller}{%
  \arrayrulecolor[gray]{0.8}% change to cell colour
  \specialrule{\heavyrulewidth}{0pt}{-\heavyrulewidth}% "invisible" rule
  \arrayrulecolor{black}% revert to regular line colour
}

\newcommand{\map}[3]{{\color{red}{\varphi}} #1 : #2.\; #3}
\newcommand{\zero}{{\color{red}{\mathtt{zero}}}}
\newcommand{\suc}[1]{{\color{red}{\mathtt{suc}}}(#1)}
\newcommand{\pair}[2]{{\textcolor{red}{\langle}} #1,#2 {\textcolor{red}{\rangle}}}
\newcommand{\ind}[5]{{\color{red}{\mathtt{ind}}}(#1)(#2;#3.#4.#5)}

\newcommand{\lam}[3]{\ensuremath{{\color{red}{\lambda}} #1 .\; #3}}
\newcommand{\seq}[3]{\mathtt{seq}(#1;#2.\;#3)}
\newcommand{\bind}[3]{#2 \leftarrow #1;\;#3}
\newcommand{\rec}[5]{\mathtt{rec}(#1)(#2;#3.#4.#5)}
\newcommand{\case}[5]{\mathtt{case}(#1)(#2.#3; #4.#5)}
\newcommand{\thunk}[1]{{\color{red}{\mathtt{thunk}}}(#1)}
\newcommand{\comp}[1]{{\color{red}{\lift{\mathtt{comp}}}}(#1)}
\newcommand{\compc}[1]{{\color{red}{\mathtt{comp}}}(#1)}
\newcommand{\ccomp}[3]{{\color{red}{\mathtt{comp}}}(#1, #2 \diamond #3)}
\newcommand{\force}[1]{\mathtt{force}(#1)}
\newcommand{\ret}[1]{\mathtt{ret}(#1)}
\newcommand{\fst}[1]{#1 \cdot \mathtt{1}}
\newcommand{\snd}[1]{#1 \cdot \mathtt{2}}

\newcommand{\nat}{\mathtt{nat}}
\newcommand{\pure}[2]{#1 \rhd #2}
\newcommand{\arr}[4]{\ensuremath{(#1\diamond#2) \to (#3\diamond#4)}}
\newcommand{\dprod}[2]{\ensuremath{#1 \times #2}}
\newcommand{\unit}{\mathtt{unit}}
\newcommand{\dprodc}[2]{\ensuremath{#1 \lift{\times} #2}}
\newcommand{\eqty}[3]{\ensuremath{\mathtt{eq}_{#1}(#2, \allowbreak #3)}}
\newcommand{\eqtyc}[3]{\ensuremath{\lift{\mathtt{eq}}_{#1}(#2, \allowbreak #3)}}

\newcommand{\z}{\ensuremath{\mathtt{int}}}
\newcommand{\step}[2]{\ensuremath{#1 \mapsto #2}}
\newcommand{\stepIn}[3]{\ensuremath{#1 \mapsto^{#2} #3}}
\newcommand{\eqto}[2]{#1 \Rightarrow #2}
\newcommand{\eval}[2]{\ensuremath{#1 \Downarrow #2}}
\newcommand{\evalCost}[3]{\ensuremath{#1 \Downarrow^{#2} #3}}

\newcommand{\val}[1]{\ensuremath{#1 \;\mathsf{val}}}
\newcommand{\final}[1]{\ensuremath{#1 \;\mathsf{final}}}

\newcommand{\isOf}[2]{#1 {:} #2}
\newcommand{\gammaToVal}[3]{\ensuremath{#1 \gg #2 \in_0 #3}}
\newcommand{\gammaToTypeComp}[3]{\ensuremath{#1 \gg #3 \; \mathsf{type}}}
\newcommand{\gammaToTypeCompFixed}[3]{\ensuremath{#1 \rhd_{#2} #3 \; \mathsf{type}}}
\newcommand{\gammaToComp}[5]{\ensuremath{#1 \diamond #2 \gg #3 \in #4 \diamond #5}}
\newcommand{\gammaToCompFixed}[6]{\ensuremath{#1 \diamond #2 \rhd_{#3} #4 \in (#5 \diamond #6)}}
\newcommand{\gammaToPot}[2]{\ensuremath{#1 \gg #2 \; \mathsf{pot}}}
\newcommand{\gammaToPotFixed}[3]{\ensuremath{#1 \rhd_{#2} #3 \; \mathsf{pot}}}

\newcommand{\gammaToPotTyped}[3]{\ensuremath{#1 \gg #2 \; \mathsf{pot}\; #3}}

\newcommand{\isVal}[2]{\ensuremath{#1 \in_0 #2}}
\newcommand{\isType}[1]{\ensuremath{#1 \; \mathsf{type}_0}}
\newcommand{\isComp}[2]{\ensuremath{#1 \in #2}}
\newcommand{\isTypeComp}[1]{\ensuremath{#1 \; \mathsf{type}}}
\newcommand{\isPot}[1]{\ensuremath{#1 \; \mathsf{pot}}}
\newcommand{\isPotTyped}[2]{\ensuremath{#1 \; \mathsf{pot} \; #2}}
\newcommand{\isCtx}[1]{\ensuremath{#1\; \mathsf{ctx} }}
\newcommand{\isSub}[2]{\ensuremath{#1 \in #2}}

\newcommand{\subst}[2]{[#1]#2}
\newcommand{\lsub}[3]{\left< #1/#2 \right>#3}

\newcommand{\zp}{\mathbb{0}}
\newcommand{\constp}[1]{\ensuremath{\mathbb{#1}}}
\newcommand{\toNat}[1]{\underline{#1}}
\newcommand{\toNum}[1]{\overline{#1}}
\newcommand{\pmin}{\mathtt{min}}
\newcommand{\sub}{\mathtt{sub}}
\newcommand{\plus}{\mathtt{plus}}
\newcommand{\mult}{\mathtt{mult}}


\newcommand{\fact}[2]{#1$^{\textbf {\color{blue}{#2}}}$}
\newcommand{\applyAss}[1]{assumption \textbf{\color{blue}{#1}}}

\newcommand{\eqType}[2]{\ensuremath{#1 \doteq #2\; \mathsf{type}}}
\newcommand{\eqComp}[3]{\ensuremath{#1 \doteq #2 \in #3}}
\newcommand{\eqCComp}[5]{\ensuremath{#1 \doteq #2 \in #3 {\color{purple}{\;[#4; #5]}}}}
\newcommand{\eqVal}[3]{\ensuremath{#1 \doteq #2 \in_0 #3}}

\newcommand{\openEqComp}[4]{\ensuremath{#1 \gg #2 \doteq #3 \in #4}}
\newcommand{\openEqVal}[4]{\ensuremath{#1 \gg #2 \doteq #3 \in_0 #4}}
\newcommand{\openComp}[3]{\ensuremath{#1 \gg #2 \in #3}}
\newcommand{\openTypeComp}[2]{\ensuremath{#1 \gg #2 \; \mathsf{type}}}
\newcommand{\openEqTypeComp}[3]{\ensuremath{#1 \gg #2 \doteq  #3 \; \mathsf{type}}}

\newcommand{\eqCtx}[2]{\ensuremath{#1 \doteq #2}}
\newcommand{\eqInst}[3]{\ensuremath{#1 \sim #2 \in #3}}

\newbox\qqBoxA
\newdimen\qqCornerHgt
\setbox\qqBoxA=\hbox{$\ulcorner$}
\global\qqCornerHgt=\ht\qqBoxA
\newdimen\qqArgHgt
\def\Quinequote #1{%
    \setbox\qqBoxA=\hbox{$#1$}%
    \qqArgHgt=\ht\qqBoxA%
    \ifnum     \qqArgHgt<\qqCornerHgt \qqArgHgt=0pt%
    \else \advance \qqArgHgt by -\qqCornerHgt%
    \fi \raise\qqArgHgt\hbox{$\ulcorner$} \box\qqBoxA %
    \raise\qqArgHgt\hbox{$\urcorner$}}
\newcommand{\lift}[1]{\ensuremath{\Quinequote{#1}}}

\newcommand{\triv}{\star}
\newcommand{\inl}[1]{\mathtt{inl}(#1)}
\newcommand{\inr}[1]{\mathtt{inr}(#1)}
\newcommand{\sigmaty}[2]{\Sigma #1.#2}
\newcommand{\sigmatyc}[2]{\lift{\Sigma} #1.#2}
\newcommand{\pityc}[2]{\ensuremath{\lift{\Pi} #1.#2}}
\newcommand{\intty}[2]{\cap #1. #2}
\newcommand{\inttyc}[2]{\ensuremath{\lift{\cap} #1. #2}}
\newcommand{\subsetty}[2]{\{ #1 \mid #2\}}

\newcommand{\sameType}[4]{\ensuremath{#1 \sim #2 \downarrow #3 \in #4}}
\newcommand{\sameTypeOne}[6]{\ensuremath{\isOf{#1}{#2} \rhd #3 \sim #4 \downarrow #5 \in #6}}
\newcommand{\sameComp}[4]{\ensuremath{#1 \rhd #2 \sim #3 \in #4}}
\newcommand{\equivComp}[3]{\ensuremath{#1 \rhd #2 \asymp #3}}
\newcommand{\sameCComp}[6]{\ensuremath{#1 \rhd #2 \sim #3 \in #4\; {\color{purple}{[#5; #6]}}}}

\newcommand{\type}[1]{\mathtt{Type}_{#1}}
\newcommand{\term}{\mathsf{Term}}
\newcommand{\evalTo}[1]{{#1}^{\Downarrow}}
\newcommand{\class}[1]{\ensuremath{[#1]}}
\newcommand{\cost}[1]{\ensuremath{\mathcal{C}}(#1)}
\newcommand{\fequiv}[2]{\ensuremath{#1 \asymp #2}}
\newcommand{\arrabt}[3]{\ensuremath{\Pi(#1; #2)}}
\newcommand{\relpot}[4]{\ensuremath{\mathtt{relpot}(#1.#2; #3.#4)}}
\newcommand{\relpotty}[3]{\ensuremath{\mathtt{relpotty}(#1;#2.#3)}}
\begin{document}
 
% --------------------------------------------------------------
%                         Start here
% --------------------------------------------------------------
 
%\renewcommand{\qedsymbol}{\filledbox}
 
\title{}%replace X with the appropriate number
\author{ %replace with your name
} %if necessary, replace with your course title
 
%\maketitle

\section{Language}

\begin{align*}
    \mathsf{Exp} \quad E &::= \lam{x}{A}{M} 
    \mid \suc{E}
    \mid \zero 
    \mid \susp{E}
    \mid \ccomp{E_1}{E}{a.E_2}
    \mid x
    \mid \nat
    \mid \pair{E_1}{E_2}\\
    &\mid \arrabt{A}{a.B}{a.P,a.b.Q}
    \mid \dprod{\isOf{x}{M}}{M'}\\
    \mathsf{Comp} \quad M &::= 
     v_1 \; v_2
    \mid \rec{v}{M_0}{x}{f}{M_1}
    \mid \seq{v}{x}{M}
    \mid \fst{v}
    \mid \snd{v}
    \mid \ret{v}\\
\end{align*}

We start with a simple language consisting of functions, natural numbers, and pairs. Later on we will add
various constructs to illustrate cost analysis.
\section{Semantics}
We start with the usual small-step operational semantics, adjusted so that consecutive computations are 
strung together by a sequencing operation.
\[
\fbox{$\step{M}{M'}$} \quad\quad \fbox{$\final{M}$}
\]

\begin{mathpar}

\inferrule{
}{
    \final{\ret{v}}
}

\inferrule{
}{
    \step{(\lam{x}{A}{M}) v}{[v/x]M}
}

\inferrule{
}{
    \step{\rec{\zero}{M_0}{x}{f}{M_1}}{M_0}
}

\inferrule{
}{
    \step{\rec{\suc{v}}{M_0}{x}{f}{M_1}}{\seq{\thunk{\rec{v}{M_0}{x}{f}{M_1}}}{f}{[v/x]M_1}}
}

\inferrule{
}{
    \step{\fst{\pair{v}{w}}}{\ret{v}}
}

\inferrule{
}{
    \step{\snd{\pair{v}{w}}}{\ret{w}}
}

\inferrule{
    \step{M_1}{M_1'}
}{
    \step{\seq{\thunk{M_1}}{x}{M_2}}{\seq{\thunk{M_1'}}{x}{M_2}}
}

\inferrule{
}{
    \step{\seq{\thunk{\ret{v}}}{x}{M_2}}{[v/x]M_2}
}
\end{mathpar}

\evalCost{M}{c}{v} means $\exists v. \stepIn{M}{c}{\ret{v}}$ and \eval{M}{v} means $\exists v.
\stepIn{M}{*}{\ret{v}}$

Define a translation from values to natural numbers: 
$\toNat{n} = \begin{cases} 0 \text{ if } n = \zero\\
1+\toNat{v} \text{ if } n = \suc{v} \end{cases}$



%\section{Ground Judgments}

\begin{center}
\begin{tabularx}{1.0\textwidth}{lXll}
    \toprule
 Judgment & English & Presupposition & Meaning\\ \midrule
 \isType{A_0} & $A_0$ is a canonical type & --- & what are canonical values of $A_0$? \\ \midrule
 \isTypeComp{\star}{A} & $A$ computes to a type  & --- & \eval{A}{A_0} and \isType{A_0}\\ \midrule
 \isVal{v}{A} & $v$ is a canonical value of type $A$ & \isTypeComp{\star}{A} & 
    \makecell[tl]{Given that \eval{A}{A_0},\\ $v$ is a canonical value of $A_0$} \\ \midrule
 \isComp{\varphi}{M}{A}{\varrho} & $M$ computes to a value of type $A$ & 
    \makecell[tl]{\isTypeComp{\star}{A},\\\isPot{\varphi},\\ \isPot{\varrho}}&
    \makecell[tl]{\evalCost{M}{c}{v}, \isVal{v}{B}, \eval{\varphi}{p},\\ \eval{[v/y]\varrho}{p'},\\ and $\toNat{p} \ge c + \toNat{p'}$}\\
 \isType{\nat} & $\nat$ is a canonical type & --- & 
    \makecell[tl]{\isVal{\zero}{\nat}\\ \isVal{\suc{v}}{\ret{\nat}} given that\\ \isVal{v}{\ret{\nat}}}\\ \midrule
\isPot{\varphi}& $\varphi$ computes a potential & --- & 
    \makecell[tl]{\isComp{p}{\varphi}{\ret{\nat}}{p'}\\ for some $p,p'$}\\
 \iffalse
 \isPotTyped{\varrho}{A} & $\varrho$ is a potential function for $A$ & \isTypeComp{\star}{A} &
    \makecell[cl]{$\varrho$ is $i.M$ and for all \isVal{a}{A},\\ \eval{[a/i]M}{n} and \isVal{n}{\nat}}\\
    \fi
    \bottomrule
\end{tabularx}
\end{center}

Let it be the case that
\begin{gather*}
\isTypeComp{\star}{A_1}\\
\isTypeComp{\star}{[a_1/x_1]A_2} \text{ given } \isVal{a_1}{A_1},\\
\dots\dots\\
\isTypeComp{\star}{[a_1/x_1,\dots,a_{n-1}/x_{n-1}]A_n}\\
\text{ given } \\ 
    \isVal{a_1}{A_1},\\
    \dots,\\
    \isVal{a_{n-1}}{[a_1/x_1,\dots,a_{n-2}/x_{n-2}]A_{n-1}}
\end{gather*}

And 

\[
\isVal{a_1}{A_1},\dots,\isVal{a_n}{[a_1/x_1,\dots,a_{n_1}/x_{n-1}]A_n}
\]
Then for $\mathcal{J} \in \{\isTypeComp{\star}{A},\; \isVal{v}{A},\; \isPot{\varphi}\}$, the shorthand
\[\isOf{x_1}{A_1},\dots,\isOf{x_n}{A_n} \rhd_n \mathcal{J}
\]
means
\[[a_1/x_1,\dots,a_n/x_n]\mathcal{J}\]

If moreover, it is the case that 
\begin{gather*}
    \isPot{[a_1/x_1,\dots,a_n/x_n]\varphi}\\
    \isTypeComp{\star}{[a_1/x_1,\dots,a_n/x_n]B}\\
    \isPot{[a_1/x_1,\dots,a_n/x_n,b/y]\varrho} \text{ for all } \isVal{b}{[a_1/x_1,\dots,a_n/x_n]B}\\
\end{gather*}

Then the notation
\[
\isOf{x_1}{A_1},\dots,\isOf{x_n}{A_n} / \varphi \rhd_n M \in \isOf{y}{B}/\varrho
\]
means that 
\begin{gather*}
\eval{[a_1/x_1,\dots,a_n/x_n]\varphi}{p}\\ 
\evalCost{[a_1/x_1,\dots,a_n/x_n]M}{c}{v} \text{ for some } \isVal{v}{[a_1/x_1,\dots,a_n/x_n]B}\\
\eval{[a_1/x_1,\dots,a_n/x_n,v/y]\varrho}{p'}
\end{gather*}
satisfying $\toNat{p} \ge c + \toNat{p'}$.



%\section{Types}

\begin{itemize}
    \item \isType{\dprod{\isOf{x}{A}}{B}}
    \begin{enumerate}
        \item \isTypeComp{\star}{A}
        \item \gammaToTypeCompFixed{\isOf{x}{A}}{1}{B}
    \end{enumerate}
    \item \isType{\comp{A}}
    \begin{enumerate}
        \item \isType{A}
    \end{enumerate}
    \item \isVal{\thunk{M}}{\comp{A}}, given:
    \begin{enumerate}
        \item \isComp{\varphi}{M}{\isOf{x}{A}}{\varrho} for some $\varphi, \varrho$
    \end{enumerate}
        \item \isType{\ccomp{\varphi}{\isOf{x}{A}}{\varrho}}
         \begin{enumerate}
        \item \isType{A}
        \item \isPot{\varphi}
        \item \gammaToPotFixed{\isOf{x}{A}}{1}{\varrho}
        \end{enumerate}
        \item \isVal{\thunk{M}}{\ccomp{\varphi}{\isOf{x}{A}}{\varrho}}
        \begin{enumerate}
            \item \isComp{\varphi}{M}{\isOf{x}{A}}{\varrho}
        \end{enumerate}

    \item \isType{\arr{\isOf{x}{A}}{\varphi}{\isOf{y}{B}}{\varrho}}
\begin{enumerate}
    \item \isTypeComp{\star}{A}
    \item \gammaToTypeCompFixed{\isOf{x}{A}}{1}{B}
    \item \gammaToPotFixed{\isOf{x}{A}}{1}{\varphi}
    \item \gammaToPotFixed{\isOf{x}{A},\isOf{y}{B}}{2}{\varrho}
\end{enumerate}

    \item $\lam{x}{A}{M}$ is a canonical element of \arr{\isOf{x}{A}}{\varphi}{\isOf{y}{B}}{\varrho}, given:
\[
    \gammaToCompFixed{\isOf{x}{A}}{\varphi}{1}{M}{\isOf{y}{B}}{\varrho}
    \]
By the previous remarks, this is the case iff:
given \isVal{a}{A}, it is the case that:
\begin{enumerate}
    \item \eval{[a/x]\varphi}{p}
    \item \evalCost{[a/x]M}{c}{v} for some \isVal{v}{[a/x]B}
    \item \eval{[a/x,v/y]\varrho}{p'}
    \item $\toNat{p} \ge c + \toNat{p'}$
\end{enumerate}

\end{itemize}



%
\section{Hypothetical Judgments}

\begin{center}
\begin{tabularx}{1.0\textwidth}{lXl}
    \toprule
 Judgment & Presupposition & Meaning\\ \midrule
  \rowcolor[gray]{0.7}
 \isCtx{\cdot} & --- & ---\\
  \rowcolor[gray]{0.7}
 \isCtx{\isOf{x}{A},\Gamma} & --- & \isTypeComp{\star}{A}, and for all \isVal{a}{A}, \isCtx{[a/x]\Gamma}\\
  \isSub{\cdot}{\cdot} & --- & --- \\
 \isSub{x\mapsto M,\gamma}{\isOf{x}{A},\Gamma} & --- & \isVal{M}{A} and \isSub{\gamma}{[M/x]\Gamma}\\
 \rowcolor[gray]{0.7}
 \gammaToTypeComp{\cdot}{\star}{B} & --- & \isTypeComp{\star}{B}\\
  \rowcolor[gray]{0.7}
 \gammaToTypeComp{\isOf{x}{A},\Gamma}{\star}{B} & \isCtx{\isOf{x}{A},\Gamma} & for all \isVal{a}{A}, $[a/x]\left(\gammaToTypeComp{\Gamma}{\star}{B}\right)$ \\ 
 \gammaToVal{\cdot}{v}{B} & --- & \isVal{v}{B} \\
 \gammaToVal{\isOf{x}{A},\Gamma}{v}{B} & \isCtx{\isOf{x}{A},\Gamma} & for all \isVal{a}{A}, 
    $[a/x]\left(\gammaToVal{\Gamma}{v}{B}\right)$\\
\rowcolor[gray]{0.7}
 \gammaToPot{\cdot}{\varphi} & --- & \isPot{\varphi} \\ 
 \rowcolor[gray]{0.7}
 \gammaToPot{\isOf{x}{A},\Gamma}{\varphi} & \isCtx{\isOf{x}{A},\Gamma} & for all \isVal{a}{A},                       $[a/x]\left(\gammaToPot{\Gamma}{\varphi}\right)$\\
 \gammaToComp{\cdot}{\varphi}{M}{\isOf{y}{B}}{\varrho} & \isTypeComp{\star}{B}, \isPot{\varphi}, and \gammaToPot{\isOf{y}{B}}{\varrho}
    & \makecell[tl]{\evalCost{M}{c}{v},\\ \isVal{v}{B},\\ \eval{\varphi}{p},\\ \eval{[v/y]\varrho}{p'},\\ and $\toNat{p} \ge c + \toNat{p'}$}\\
  \gammaToComp{\isOf{x}{A},\Gamma}{\varphi}{M}{\isOf{y}{B}}{\varrho} & 
  \makecell[tl]{\isCtx{\isOf{x}{A},\Gamma},\\
  \gammaToTypeComp{\isOf{x}{A},\Gamma}{\star}{B},\\
  \gammaToPot{\isOf{x}{A},\Gamma}{\varphi},\\ 
  \gammaToPot{\isOf{x}{A},\Gamma,\isOf{y}{B}}{\varrho}}
  &\makecell[tl]{for all \isVal{a}{A},\\ $[a/x]\left(\gammaToComp{\Gamma}{\varphi}{M}{\isOf{y}{B}}{\varrho}\right)$}\\
    \bottomrule
\end{tabularx}
\end{center}

%\input{nats.tex}
%
\section{Arithmetic}

\begin{lemma}(Plus)
\isVal{\plus}
{\ret{\arr{\isOf{x}{\ret{\nat}}}{\zp}{\ret{\arr{\ret{\nat}}{\varphi}{\ret{\nat}}{\varrho}}}{\zp}}},
where 
\begin{gather*}
    \varphi_0 = \zp\\
    \varphi_1 = \ret{\suc{\zero}}\\
    \varphi= \rec{x}{\varphi_0^+}{z}{g}{\bind{\varphi_1}{\phi_1}{\plus\;\phi_1\;g}^+}\\
    \varrho_0 = \zp\\
    \varrho_1 = \zp\\
    \varrho= \rec{x}{\varrho_0}{z}{g}{\bind{\varrho_1}{\rho_1}{\plus\;\rho_1\;g}}\\
\end{gather*}
\end{lemma}

\begin{proof}
By definition, it suffices to show\\
\gammaToCompFixed{\isOf{x}{\ret{\nat}}}{\zp}{1}{\ret{\lam{y}{b}{\rec{x}{\ret{y}}{\_}{f}{\ret{\suc{f}}}}}}{\ret{\arr{\ret{\nat}}{\varphi}{\ret{\nat}}{\varrho}}}{\zp}\\
Let \isVal{n}{\ret{\nat}}. 
Then by Lemma~\ref{lemma:ret}, it suffices to show
\[
\isVal{\lam{y}{b}{\rec{n}{\ret{y}}{\_}{f}{\ret{\suc{f}}}}}{\ret{\arr{\ret{\nat}}{[n/x]\varphi}{\ret{\nat}}{\varrho}}}\]

By definition, we need to show 
\[
\gammaToCompFixed{\isOf{y}{\ret{\nat}}}{[n/x]\varphi}{1}{\rec{n}{\ret{y}}{\_}{f}{\ret{\suc{f}}}}{\ret{\nat}}{\varrho}
\]

Let \isVal{m}{\ret{\nat}} and unfold with $[m/y]$.  
By Lemma~\ref{lemma:recursor}, it suffices to show
\begin{enumerate}
    \item \gammaToComp{\cdot}{\zp}{\ret{m}}{\isOf{y}{\ret{\nat}}}{\zp}: follows from Lemma~\ref{lemma:ret}.
    \item \gammaToComp{\isOf{z}{\ret{\nat}},\isOf{f}{\ccomp{\zp}{\isOf{w}{\ret{\nat}}}{\zp}} }{\varphi_1}{\ret{\suc{f}}}{\ret{\nat}}{\varrho_1}.\\
    Let \isVal{z}{\ret{\nat}}. STS 
    \gammaToComp{\isOf{f}{\ccomp{\zp}{\isOf{w}{\ret{\nat}}}{\zp}} }{\varphi_1}{\ret{\suc{f}}}{\ret{\nat}}{\varrho_1}.
    Let \isVal{f}{\ccomp{\zp}{\isOf{w}{\ret{\nat}}}{\zp}}. 
    Then $f$ is $\thunk{F}$ for some $F$ s.t. \isComp{\zp}{F}{\isOf{w}{\ret{\nat}}}{\zp}. 
    This implies that $\evalCost{F}{0}{u}$, and hence $F = \ret{u}$. 
         Thus,
        \begin{align*}
            &\lsub{F}{f}{[\zero/z]\ret{\suc{f}}} = \lsub{\ret{u}}{f}{\ret{\suc{f}}} \mapsto \ret{\suc{u}}
        \end{align*} 
        Further, 
        \begin{align*}
           &\eval{\varphi_1 = \ret{\suc{\zero}}}{\suc{\zero}}
        \end{align*}
        Similarly, \eval{\varrho_1}{\zero}.
        The result holds since $\toNat{\suc{\zero}} \ge 1 + \toNat{\zero}$.
\end{enumerate}

Simplified recurrence for $\plus$: $\Phi(\plus)(x) = 2x + 1$.
\end{proof}

Recall $\mult \triangleq \lam{x}{a}{\ret{\lam{y}{b}{\rec{x}{\ret{\zero}}{\_}{f}{\plus\; y\; f}}}}
$. 
\begin{lemma}(Mult)\label{lemma:mult}
\isVal{\mult}{\ret{\arr{\isOf{x}{\ret{\nat}}}{\zp}{\ret{\arr{\isOf{y}{\ret{\nat}}}{\varphi}{\ret{\nat}}{\varrho}}}{\zp}}},
where 
\begin{gather*}
    \varphi_0 = \zp\\
    \varphi_1 = \bind{\mult\;\toNum{2}\;y}{r}{\plus\;r\;\toNum{5}}\\
    \varphi= \rec{x}{\varphi_0^+}{z}{g}{\bind{\varphi_1}{\phi_1}{\plus\;\phi_1\;g}^+}\\
    \varrho_0 = \zp\\
    \varrho_1 = \zp\\
    \varrho= \rec{x}{\varrho_0}{z}{g}{\bind{\varrho_1}{\rho_1}{\plus\;\rho_1\;g}}\\
\end{gather*}
\end{lemma}

\begin{proof}
It suffices to show 

\gammaToCompFixed{\isOf{x}{\ret{\nat}}}{\zp}{1}{\ret{\lam{y}{b}{\rec{x}{\ret{\zero}}{\_}{f}{\plus\; y\; f}}}}{\ret{\arr{\isOf{y}{\ret{\nat}}}{\varphi}{\ret{\nat}}{\varrho}}}{\zp}

Let \isVal{n}{\ret{\nat}}. Instantiate with $[n/x]$, by Lemma~\ref{lemma:ret}, STS
\isVal{\lam{y}{b}{\rec{n}{\ret{\zero}}{\_}{f}{\plus\; y\; f}}}{\ret{\arr{\isOf{y}{\ret{\nat}}}{\varphi}{\ret{\nat}}{\varrho}}}. Again,
NTS
\gammaToCompFixed{\isOf{y}{\ret{\nat}}}{\varphi}{1}{\rec{n}{\ret{\zero}}{\_}{f}{\plus\; y\; f}}{\ret{\nat}}{\varrho}. Let \isVal{m}{\ret{\nat}}. Instantiating with $[m/y]$, STS

\[
\gammaToComp{\cdot}{\varphi}{\rec{n}{\ret{\zero}}{\_}{f}{\plus\; m\; f}}{\ret{\nat}}{\varrho}
\]
By Lemma~\ref{lemma:recursor'}, STS
\begin{enumerate}
    \item \gammaToComp{\cdot}{\zp}{\ret{\zero}}{\isOf{y}{\ret{\nat}}}{\zp}: follows from
    Lemma~\ref{lemma:ret}
    \item \gammaToComp{\isOf{z}{\ret{\nat}},\isOf{f}{\ccomp{\zp}{\isOf{w}{\ret{\nat}}}{\zp}} }{\varphi_1}{\plus\; y\; f}{\ret{\nat}}{\varrho_1}.\\
    Let \isVal{z}{\ret{\nat}} and \isVal{f}{\ccomp{\zp}{\isOf{w}{\ret{\nat}}}{\zp}}. 
    Then $f = \thunk{F}$ s.t. \isComp{\zp}{F}{\isOf{w}{\ret{\nat}}}{\zp}. Hence,
    $F = \ret{u}$ for some \isVal{u}{\ret{\nat}}. Thus,
    \begin{align*}
    &\lsub{\ret{u}}{f}{\plus\; y\; f}\\
    \mapsto& \plus\; y\; u = \bind{\plus\; y}{r}{r\; u}\\
    \mapsto& \bind{\lam{y'}{b}{\rec{y}{\ret{y'}}{\_}{f}{\ret{\suc{f}}}}}{r}{r\; u}\\
    \mapsto& \lam{y'}{b}{\rec{y}{\ret{y'}}{\_}{f}{\ret{\suc{f}}}}\; u\\
    \mapsto& \rec{y}{\ret{u}}{\_}{f}{\ret{\suc{f}}}\\
    \mapsto^{2\toNat{y}+1}& \ret{v}
    \end{align*}
    Hence \evalCost{\lsub{\ret{u}}{f}{\plus\; y\; f}}{2\toNat{y}+5}{v}.
    Furthermore, \eval{\varphi_1 = \bind{\mult\;\toNum{2}\;y}{r}{\plus\;r\;\toNum{5}}}{\toNum{2\toNat{y}+5}},
    and \eval{\varrho_1}{\zero}. The result follows since 
    $\toNat{\toNum{2\toNat{y}+5}} \ge 2\toNat{y}+5 + \toNat{\zero}$.
\end{enumerate}

Simplified recurrence for $\mult$: $\Phi(\mult)(x)(y) = x^2y + (6-y)x + 1$.



\end{proof}



%\newcommand{\leaf}{\ensuremath{\mathtt{leaf}}}
\newcommand{\single}[1]{\ensuremath{\mathtt{single}(#1)}}
\newcommand{\ttwo}[2]{\ensuremath{\mathtt{t2}(#1,#2)}}
\newcommand{\tthree}[3]{\ensuremath{\mathtt{t3}(#1,#2,#3)}}
\newcommand{\ttrec}[8]{\mathtt{23TREC}(#1)(#2 \mid #3.#4 \mid #5.#6 \mid #7.#8)}
\newcommand{\tttree}[2]{\ensuremath{\mathtt{23tree}(#1,#2)}}
\newcommand{\word}{\mathtt{word}}
\newcommand{\mw}[1]{\mathtt{mw}(#1)}
\newcommand{\wrec}[4]{\mathtt{WREC}(#1)(#2; #3.#4)}
\newcommand{\join}{\mathsf{join}}
\newcommand{\cmp}{\mathsf{cmp}}
\newcommand{\ifnat}[3]{\mathsf{if}(#1;#2;#3)}
\newcommand{\const}[1]{\lam{\_}{\_}{#1}}
\newcommand{\eq}{\mathsf{eq}}
\newcommand{\macro}{@\!\!=}
\section{Trees}

Augment the language with the following values and commands:

\begin{align*}
    \mathsf{Node} \quad n &::= 
         \leaf
    \mid \single{v}
    \mid \ttwo{v_l}{v_r}
    \mid \tthree{v_l}{v_m}{v_r}\\
    \mathsf{Val} \quad v &::= \dots
    \mid \tttree{v}{w}
    \mid (\isOf{n}{\tttree{v}{w}})
    \mid \word
    \mid \mw{0} \dots \mw{2^{64}-1}\\
    \mathsf{Comp} \quad M &::= 
    \dots
    \mid \wrec{v}{M_0}{f}{M_1}\\
    &\mid \ttrec{v}{M_0}{a}{M_1}{l,r,f_l,f_r}{M_2}{l,m,r,f_l,f_m,f_r}{M_3}\\
\end{align*}

The semantics is extended with the computation rules:

\begin{mathpar}
\inferrule{
}{
  \step{\wrec{\mw{0}}{M_0}{f}{M_1}}{M_0}
}

\inferrule{
  F \triangleq \lam{d}{\_}{\bind{\mw{w}-d}{x}{\wrec{x}{M_0}{f}{M_1}}}
}{
  \step{\wrec{\mw{w}}{M_0}{f}{M_1}}{[F/f]M_1}
}

\inferrule{
}{
 \step{\ttrec{\leaf}{M_0}{a}{M_1}{l,r,f_l,f_r}{M_2}{l,m,r,f_l,f_m,f_r}{M_3}}{M_0}
}

\inferrule{
}{
 \step{\ttrec{\single{v}}{M_0}{a}{M_1}{l,r,f_l,f_r}{M_2}{l,m,r,f_l,f_m,f_r}{M_3}}{[v/a]M_1}
}

\inferrule{
  t = \isOf{\ttwo{l}{r}}{\tttree{s}{d}}\\
  \eval{d > \mw{0}}{\suc{\zero}}\\
  \eval{s > \mw{0}}{\suc{\zero}}\\
  v_l = \isOf{l}{\tttree{S_l}{D_l}}\\
  v_r = \isOf{r}{\tttree{S_r}{D_r}}\\
  \eval{d_l+1}{d}\\
  \eval{d_r+1}{d}\\
  \eval{s_l+s_r}{s}\\\\
  F_l \triangleq \thunk{\ttrec{v_l}{M_0}{a}{M_1}{l,r,f_l,f_r}{M_2}{l,m,r,f_l,f_m,f_r}{M_3}}\\
  F_r \triangleq \thunk{\ttrec{v_r}{M_0}{a}{M_1}{l,r,f_l,f_r}{M_2}{l,m,r,f_l,f_m,f_r}{M_3}}\\
}{
  \ttrec{t}{M_0}{a}{M_1}{l,r,f_l,f_r}{M_2}{l,m,r,f_l,f_m,f_r}{M_3} \mapsto
  \ret{\thunk{[v_l/l,v_r/r,F_l/f_l,F_r/f_r]M_2}}
}

\inferrule{
  t = \isOf{\tthree{l}{m}{r}}{\tttree{s}{d}}\\
  \eval{d > \mw{0}}{\suc{\zero}}\\
  \eval{s > \mw{0}}{\suc{\zero}}\\
  v_l = \isOf{l}{\tttree{S_l}{D_l}}\\
  v_m = \isOf{m}{\tttree{S_m}{D_m}}\\
  v_r = \isOf{r}{\tttree{S_r}{D_r}}\\
  \eval{d_l+1}{d}\\
  \eval{d_m+1}{d}\\
  \eval{d_r+1}{d}\\
  \eval{s_l+s_m+s_r}{s}\\\\
  F_l \triangleq \thunk{\ttrec{n_l}{M_0}{a}{M_1}{l,r,f_l,f_r}{M_2}{l,m,r,f_l,f_m,f_r}{M_3}}\\
  F_m \triangleq \thunk{\ttrec{n_m}{M_0}{a}{M_1}{l,r,f_l,f_r}{M_2}{l,m,r,f_l,f_m,f_r}{M_3}}\\
  F_r \triangleq \thunk{\ttrec{n_r}{M_0}{a}{M_1}{l,r,f_l,f_r}{M_2}{l,m,r,f_l,f_m,f_r}{M_3}}\\
}{
 \ttrec{t}{M_0}{a}{M_1}{l,r,f_l,f_r}{M_2}{l,m,r,f_l,f_m,f_r}{M_3} \mapsto\\
  \ret{\thunk{[v_l/l,v_m/m,v_r/r,F_l/f_l,F_m/f_m,F_r/f_r]M_3}}
}

\end{mathpar}

Augment the types:
\begin{itemize}
\item \isType{\tttree{s}{d}} 
  \begin{enumerate}
  \item \isType{A}
  \item \isVal{s}{\word}
  \item \isVal{d}{\word}
  \end{enumerate}

\item \isVal{\isOf{\leaf}{\tttree{\mw{0}}{\mw{0}}}}{\tttree{\mw{0}}{\mw{0}}}
\item \isVal{\isOf{\single{a}}{\tttree{\mw{1}}{\mw{1}}}}{\tttree{\mw{1}}{\mw{1}}}
  \begin{enumerate}
  \item \isVal{a}{A}
  \end{enumerate}
\item \isVal{\isOf{\ttwo{l}{r}}{\tttree{s}{d}}}{\tttree{s}{d}}
  \begin{enumerate}
  \item \isVal{l}{\tttree{s_l}{d}}
  \item \isVal{r}{\tttree{s_r}{d}}
  \item \eval{s_l+s_r}{s}, \eval{d_l+\mw{1}}{d}, \eval{d_r+\mw{1}}{d}
  \end{enumerate}

\item \isVal{\isOf{\tthree{l}{m}{r}}{\tttree{s}{d}}}{\tttree{s}{d}}
  \begin{enumerate}
  \item \isVal{l}{\tttree{s_l}{d}}
  \item \isVal{m}{\tttree{s_m}{d}}
  \item \isVal{r}{\tttree{s_r}{d}}
  \item \eval{s_l+s_m+s_r}{s}, \eval{d_l+\mw{1}}{d}, \eval{d_m+\mw{1}}{d}, \eval{d_r+\mw{1}}{d}
  \end{enumerate} 
\end{itemize}

\iffalse
\begin{lemma}(Balance)\label{lemma:depth}
  \isVal{\texttt{depth}}{\arr{\isOf{s}{\word}}{\zp}{
    \arr{\isOf{d}{\word}}{\zp}{\arr{\tttree{s}{d}}{\constp{2}}{\word}{\zp}}{\zp}}{\zp}}
\end{lemma}

\begin{proof}
  By 3 applications of Lemma~\ref{lemma:abs} and Lemma~\ref{lemma:ret}, 
  it suffices to show that 
  \gammaToComp{\cdot}{\mathsf{to\_nat}\;d}
  {\ttrec{t}{\ret{0}}{\_}{\ret{0}}{dl}{dl+1}{dl}{dl+1}}{\word}{\zp}
  given \isVal{s}{\word}, \isVal{t}{\word}, and \isVal{t}{\tttree{s}{d}}.
  Case on $t$:
  \begin{itemize}
    \item $t = \isOf{\leaf}{\tttree{\mw{0}}{\mw{0}}}$:\\
      Then \evalCost{\ttrec{t}{\ret{0}}{\_}{\ret{0}}{dl}{dl+1}{dl}{dl+1}}{1}{\mw{0}}, 
      and \constp{2} suffices.
    \item 
  \end{itemize}
\end{proof}
\fi

\begin{verbatim}

to_nat = \d. WREC(d){zero; p.f. r <- f 1; ret(suc(r))}

if(v;M_t;M_f) @= rec(v){M_f; _,_.M_t}

size := (\t. tmatch(t){
    s.d.ret(s)
  | s.d.a.ret(s)
  | s.d.l.r.ret(s)
  | s.d.l.m.r.ret(s)
});

depth := (\t. tmatch(t){
    s.d.ret(d)
  | s.d.a.ret(d)
  | s.d.l.r.ret(d)
  | s.d.l.m.r.ret(d)
});

T := (t2(single(0):(1,1),single(1):(1,1)):(2,2));
T1 := (t3(T,T,T):(6,3));

balance2 := (\T1. ret(\T2. tmatch(T1){
    s.d.ret(0);
  | s.d.a.ret(0);
  | s.d.l.r.
    s2 <- size T2; 
    s' <- s + s2;
    ret(t3(l,r,T2):(s',d));
  | s.d.l.m.r.
    sl <- size l; // 3
    sm <- size m; // 3
    sl' <- sl + sm; //2
    sr <- size r; //3
    s2 <- size T2; //3
    sr' <- sr + s2; //2
    d' <- d + 1; //2
    s' <- sl' + sr'; //a 2
    ret(t2(t2(l,m):(sl',d), t2(r,T2):(sr',d)):(s',d'));
}));

balance3 := (\T1. ret(\T2. ret(\T3.
  tmatch(T1){
    s.d.ret(0);
  | s.d.a.ret(0);
  | s.d.l.r.
    s2 <- size T2;
    s3 <- size T3;
    s23 <- s2 + s3;
    s' <- s + s23;
    d' <- d + 1;
    ret(t2(t2(l,r):(s,d),t2(T2,T3):(s23,d)):(s',d'));
  | s.d.l.m.r.
    sl <- size l; // 3
    sm <- size m; // 3
    sl' <- sl + sm; // 2
    sr <- size r; // 3
    s2 <- size T2; // 3
    s3 <- size T3; // 3
    s23 <- s2 + s3; // 2
    d' <- d + 1; // 2
    s' <- s + s23; // 2
    ret(t2(t3(l,m,r):(s,d), t2(T2,T3):(s23,d)):(s',d'));
})));

insertleft := (\x.ret(\t. 
  trec(t){
     ret(thunk(ret(single(x):(1,1))));
   | a.ret(thunk(ret(t2(single(x):(1,1), single(a):(1,1)):(2,2))));
   | l.r.fl.fr.
      ret(thunk(
        seq(fl; Fl.
        d <- depth Fl;
        dl <- depth l;
        b <- d > dl;
        if(b)
        then{
          balance2 Fl r
          }
        else{
          sl <- size Fl;
          sr <- size r;
          s' <- sl + sr;
          d' <- dl + 1;
          ret(t2(Fl,r):(s',d'));
          };)
      ));

   | l.m.r.fl.fm.fr. 
      ret(thunk(
        seq(fl; Fl. 
        d <- depth Fl;
        dl <- depth l;
        b <- d > dl;
        if(b) 
        then{
          balance3 Fl m r
          }
        else{
          sl <- size Fl;
          sm <- size m;
          sr <- size r;
          s' <- sl + sm;
          s'' <- s' + sr;
          d' <- dl + 1;
          ret(t3(Fl,m,r):(s'',d'));
          };)
      ));
  };
));

// 6 defs
T' <- insertleft 5 T1;
seq(T'; t'.ret(t'));
\end{verbatim}

\iffalse
full_left = \t.n. 
  23TREC(t){
    leaf => ret(zero);
    single a => ret(zero);
    t2(l,r,sl,sr,dl,dr,fl,fr) => 
      d <- dl + 1;
      b <- n == d;
      if(b; ret(zero); ret(fl));
    t3(l,m,r,sl,sm,sr,dl,dm,dr,fl,fm,fr) => 
      d <- dl + 1;
      b <- n == d;
      if(b; ret(suc(zero)); ret(fl));
  }


mk2 = \a.b. 
  case a of
    | inl l =>
      case b of
      | inl r =>
        d1 <- depth l;
        d2 <- depth r;
        b <- d1 < d2;
        if(b;
          23TREC(r){
            leaf => ret(!);
            single a => ret(inl(single(a)));
            t2(rl,rr,_,_) => ret(inl(t3(l,rl,rr)));
            t3(rl,rm,rr) => ret(inl(t2(t2(l,rl),t2(rm,rr))));
          };
          b <- d1 > d2;
          if(b;
            23TREC(l){
              leaf => ret(!);
              single a => ret(inl(single(a)));
              t2(ll,lr,_,_) => ret(inl(t3(ll,lr,r)));
              t3(ll,lm,lr) => ret(inl(t2(t2(ll,lm),t2(lr,r))));
            };
            ret(inl(t2(l,r)))
          )
        )
      | inr (r1,r2) => 
        d1 <- depth l;
        d2 <- depth r1;
        b <- d1 < d2;
        if(b;
          23TREC(r1){
            leaf => ret(!);
            single a => ret(inl(t2(r1,r2)));
            t2(r1l,r1r,_,_) => ret(inl(t2(t3(l,r1l,r1r),r2)));
            t3(r1l,r1m,r1r) => ret(inl(t3(t2(l,r1l),t2(r1m,r1r),r2)));
          };
          b <- d1 > d2;
          if(b;
            23TREC(l){
              leaf => ret(!);
              single a => ret(inl(single(a)));
              t2(ll,lr,_,_) => ret(inl(t2(l,t2(r1,r2))));
              t3(ll,lm,lr) => ret(inl(t2(t3(ll,lm,lr),t2(r1,r2))));
            };
            ret(inl(t3(l,r1,r2)))
          )
        )
    | inr (l1,l2) => 
      case b of
      | inl r =>
        d1 <- depth l1;
        d2 <- depth r;
        b <- d1 < d2;
        if(b;
          ret(inl(t2(t2(l1,l2),r)));
          b <- d1 > d2;
          if(b;
            23TREC(l2){
              leaf => ret(!);
              single a => ret(inl(t2(l1,single(a))));
              t2(rl,rr,_,_) => ret(inl(t2(l1,t3(rl,rr,r))));
              t3(rl,rm,rr) => ret(inl(t3(l,t2(rl,rm), t2(rr,r))));
            };
            ret(inl(t3(l1,l2,r)))
          )
        )
      | inr(r1,r2) => 
        d1 <- depth l1;
        d2 <- depth r1;
        b <- d1 < d2;
        if(b;
          ret(inl(t3(t2(l1,l2),r1,r2)));
          b <- d1 > d2;
          if(b;
            ret(inl(t3(l1,l2,t2(r1,r2))));
            ret(inl(t2(t2(l1,l2),t2(r1,r2))));
          )
        )

join' = \tr1.
  23TREC(tr1){
    leaf => \tr2. ret(inl(tr2));
    single a => \tr2.
      23TREC(tr2){
        leaf => ret(inl(tr1));
        single b => ret(inr(single a, single b));
        t2(l,r,fl,fr) => mk2(fl,r);
        t3(l,m,r,fl,fm,fr) => mk3(fl,m,r);
      }
    t2(l,r,fl,fr) => \tr2.
      d1 <- depth tr1;
      d2 <- depth tr2;
      b <- d1 < d2;
      if(b;
        23TREC(tr2){
         leaf => ret(!);
         single a => ret(!);
         t2(l,r,gl,gr) => mk2(gl,r);
         t3(l,m,r,gl,gm,gr) => mk3(gl,m,r);
        };
        b <- d1 > d2;
        if(b;
          r' <- fr tr2;
          mk2(l,r');;
          mk2(tr1,tr2)
        )
      )
    t3(l,m,r,fl,fm,fr) => \tr2.
      d1 <- depth tr1;
      d2 <- depth tr2;
      b <- d1 < d2;
      if(b;
        23TREC(tr2){
         leaf => ret(!);
         single a => ret(!);
         t2(l,r,gl,gr) => mk2(gl,r);
         t3(l,m,r,gl,gm,gr) => mk3(gl,m,r);
        };
        b <- d1 > d2;
        if(b;
          r' <- fr tr2;
          mk3(l,m,r');;
          mk2(tr1,tr2)
        )
      )
  }
\fi

\iffalse
  \left
  \mathsf{full\_left} \triangleq 
    \lam{t}{\_}{
      \lam{n}{\_}{
      \ttrec{t}{\const{\zero}}{\_}{\const{\zero}}
        {\_,\_,s_l,d_l,\_,\_,f_l,\_}\\{
          \bind{d_l+1}{d}{\bind{\eq\;n\;d}{b}{\ifnat{b}{\ret{b}}{\ret{f_l}}}}}\\
        {\_,\_,\_,s_l,d_l,\_,\_,\_,\_,f_l,\_,\_}{
          \bind{d_l+1}{d}{\bind{\eq\;n\;d}{b}{\ifnat{b}{\ret{\zero}}{\ret{f_l}}}}}
        }
        }\\
  \mathsf{full\_right} \triangleq 
    \lam{t}{}{
      \ttrec{t}{\const{\zero}}{\_}{\const{\zero}}
        {\_}{\lam{n}{\_}{\eq\;n\;d}}
          {\_,\_,f_l,f_r}{\bind{n-1}{n'}{f_r\; n'}}
        }\\
\join \triangleq 
  \lam{t_1}{\tttree{s_1}{d_1}}{
    \lam{t_2}{\tttree{s_2}{d_2}}{
    }}
\fi

%
\section{Construction}

Notation:

\begin{enumerate}
  \item $A \sim A' \downarrow \alpha \in \tau$ when \eval{A}{v}, \eval{A'}{v'}, and 
  $\tau(v,v',\alpha)$.
  \item $a : \alpha \rhd B \sim B' \downarrow \beta \in \tau$ when 
  for all $v,v'$ s.t. $\alpha(v,v')$, \sameType{[v/a]B}{[v'/a]B'}{\beta_{v,v'}}{\tau}.
\item \sameCComp{}{M}{M'}{\alpha}{P}{a.Q} when \evalCost{M}{c}{v}, \evalCost{M'}{c'}{v'}, 
  $\alpha(v,v')$. 
    Furthermore, \eval{P}{\bar{p}}, \eval{[v/a]Q}{\bar{q}}, and 
    $p \ge c + q$ and $p \ge c' + q$.
  \item \sameComp{}{M}{M'}{\alpha} when \sameCComp{}{M}{M'}{\alpha}{P}{a.Q} for some 
    $P$ and $a.Q$. 
  \item $\sameCComp{\isOf{a}{\alpha}}{M}{M'}{\beta}{P}{b.Q}$ when 
  for all $v,v'$ s.t. $\alpha(v,v')$, 
    \evalCost{[v/a]M}{c}{u}, \evalCost{[v'/a]M'}{c'}{u'}, and 
    $\beta_{v,v'}(u,u')$.
    Furthermore, \eval{[v/a]P}{\bar{p}}, \eval{[v/a,u/b]Q}{\bar{q}}, and 
    $p \ge c + q$ and $p \ge c' + q$.
\item \sameComp{\isOf{a}{\alpha}, \isOf{b}{\beta}}{M}{M'}{\gamma} when 
  for all $v,v'$ s.t. $\alpha(v,v')$, 
  for all $u,u'$ s.t. $\beta_{v,v'}(u,u')$, 
  \eval{[v/a,u/b]M}{w}, \eval{[v'/a,u'/b]M'}{w'}, and $\gamma_{{v,v}_{u,u'}}(w,w')$.
  \item $\omega = \mu \alpha.\, \{(\zero,\zero)\} \cup \{(\suc{v},\suc{v'}) \mid \alpha(v,v')\}$
\end{enumerate}

Construct the type systems $\tau_n$ as the least fixed-point(s) of the 
monotone function defined by the various clauses: 

\begin{align*}
  \text{Nat} &= \{(\nat,\nat,\phi) \mid \phi = \omega\}\\
  \iffalse
  \text{Pot}(\tau,\pi) &= 
    \{(\relpotty{A}{a}{B}, \relpotty{A'}{a}{B'}, \phi) \mid &\\
    &\exists \alpha.\, \sameType{A}{A'}{\alpha}{\tau} \\
    &\land\exists \beta.\, \sameTypeOne{a}{\alpha}{B}{B'}{\beta}{\pi} \\
    &\land \phi = \kappa(\alpha,a.\beta)
    \}\\
    \fi
  \text{Fun}(\tau, \pi) &=
  \{(\arrabt{A}{a.B}{a.P,a.b.Q}, \arrabt{A'}{a.B'}{a.P', a.b.Q'}, \phi) 
  \mid & \\
  &\exists \alpha.\, \sameType{A}{A'}{\alpha}{\tau} \\
  &\land\exists \beta.\, \sameTypeOne{a}{\alpha}{B}{B'}{\beta}{\pi} \\
  &\land\sameComp{\isOf{a}{\alpha}}{P}{P'}{\omega} \\
  &\land\sameComp{\isOf{a}{\alpha},\isOf{b}{\beta}}{Q}{Q'}{\omega} \\
  \land\phi &= \{(\lam{a}{\_}{M}, \lam{a}{}{M'}) \mid
  \sameCComp{\isOf{a}{\alpha}}{M}{M'}{\beta}{P}{b.Q}
   \}
  \}\\
  \text{Comp}(\tau) &= \{
  (\ccomp{P}{\isOf{a}{A}}{Q}, \ccomp{P'}{\isOf{a}{A'}}{Q'}, \phi) \mid \\
  &\exists \alpha.\, \sameType{A}{A'}{\alpha}{\tau}\\
  &\land \sameComp{}{P}{P'}{\omega}\\ 
  &\land \sameComp{\isOf{a}{\alpha}}{Q}{Q'}{\omega}\\
  \land\phi &= \{(\thunk{M}, \thunk{M'}) \mid \sameCComp{}{M}{M'}{\alpha}{P}{a.Q} \}
  \}\\
  \text{Eq}(\tau) &= \{
    (\eqty{A}{M}{N}, \eqty{A'}{M'}{N'}, \phi) \mid \\
  &\exists \alpha.\, \sameType{A}{A'}{\alpha}{\tau}\\
  &\land\sameComp{}{M}{M'}{\alpha}\\
  &\land\sameComp{}{N}{N'}{\alpha}\\
  \land\phi &= \{(\triv, \triv) \mid \sameComp{}{M}{N}{\alpha}
  \}
  \}\\
  \text{Intersection}(\tau) &= \{
    (\intty{\isOf{x}{A}}{B}, \intty{\isOf{x}{A'}}{B'},\phi) \mid \\
    &\exists \alpha.\, \sameType{A}{A'}{\alpha}{\tau} \\
    &\land\exists \beta.\, \sameTypeOne{a}{\alpha}{B}{B'}{\beta}{\tau} \\
    \land\phi &= \{(u,u') \mid \forall \alpha(v,v').\, \beta_{v,v'}(u,u')\}
  \}\\
  \text{Subset}(\tau) &= \{
    (\subsetty{\isOf{x}{A}}{B}, \subsetty{\isOf{x}{A'}}{B'},\phi) \mid \\
    &\exists \alpha.\, \sameType{A}{A'}{\alpha}{\tau} \\
    &\land\exists \beta.\, \sameTypeOne{a}{\alpha}{B}{B'}{\beta}{\tau} \\
    \land\phi &= \{(v,v') \mid \alpha(v,v') \land \exists u,u'.\beta_{v,v'}(u,u')\}
  \}\\
  \text{Type}(\tau) &= \{(\type{i}, \type{i}, \phi) \mid \exists \phi.\, \tau(\type{i}, \type{i}, \phi)\}\\
  v_n &= \{(\type{i}, \type{i}, \phi) \mid i < n \land \phi = \{(A,B) 
  \mid \exists \alpha.\, \sameType{A}{B}{\alpha}{\tau_i}\}\}\\
  \text{Types}(v,\tau) &= \text{Fun}(\tau,\tau) \cup \text{Pot}(\tau,\tau) \cup \text{Nat} \cup \text{Comp}(\tau) \cup \text{Type}(v)\\
  \tau_n &= \mu \tau.\, \text{Types}(v_n,\tau)
\end{align*}

Let the set of possible type systems be $\mathcal{D} = 
\mathcal{P}(\term \times \term \times \mathcal{P}(\term \times \term))$.
Since the various clauses have disjoint images, it suffices to show that each is monotone.

% TODO: monotone 
\iffalse
\begin{lemma}\label{lemma:monotone}
For all $v \in \mathcal{D}$, $f : \mathcal{D} \to \mathcal{D}, \tau \mapsto \text{Types}(v,\tau)$ is a monotone function.
\end{lemma}

\begin{proof}

\begin{itemize}
\item $\text{Fun}$:\\
Suppose $\tau \subseteq \tau'$. NTS $\text{Fun}(\tau,\tau) \subseteq \text{Fun}(\tau',\tau')$. 
Suppose $(\arr{A}{\varphi}{B}{\varrho}, \arr{A'}{\varphi'}{B'}{\varrho'}, \phi)  \in \text{Fun}(\tau,\tau)$. 
Unrolling the definition, we know:
\begin{enumerate}
  \item there is a $\alpha$ s.t. \sameType{A}{A'}{\alpha}{\tau}. Then by assumption, \sameType{A}{A'}{\alpha}{\tau'}. 
  \item there is a $\beta$ s.t. \fact{\sameTypeOne{a}{\alpha}{B}{B'}{\beta}{\tau}}{1}. We show
\sameTypeOne{a}{\alpha}{B}{B'}{\beta}{\tau'}. Let $v,v'$ s.t. $\alpha(v,v')$, and show 
\sameType{[v/a]B}{[v'/a]B'}{\beta_{v,v'}}{\tau'}. But this holds by \applyAss{1}.
\end{enumerate}
Since the next three constraints are determined by $\alpha$ and $\beta$ which did not change, we conclude 
$(\arr{A}{\varphi}{B}{\varrho}, \arr{A'}{\varphi'}{B'}{\varrho'}, \phi)  \in \text{Fun}(\tau',\tau')$.
\end{itemize}
\end{proof}
\fi

A type system is a possible type system $\tau$ which has the following properties:
\begin{enumerate}
\item Unicity: $\tau(A,B,\phi)$ and $\tau(A,B,\phi')$ implies $\phi = \phi'$.
\item PER valuation: $\tau(A,B,\phi)$ implies $\phi$ is a PER.
\item Symmetry: $\tau(A,B,\phi)$ implies $\tau(B,A,\phi)$. 
\item Transitivity: $\tau(A,B,\phi)$ and $\tau(B,C,\phi)$ implies $\tau(A,C,\phi)$.
\end{enumerate}

\begin{lemma}(TYPES)\label{lemma:types}
If $v$ is a type system, then $\tau^* = \mu \tau.\, \text{Types}(v,\tau)$ is a type system.
\end{lemma}


\begin{lemma}
$\tau_n$ is a type system for all $n$.
\end{lemma}

\begin{proof}
  Induction on $n$ with Lemma~\ref{lemma:types}
\end{proof}

\newcommand{\nseq}[1]{\ensuremath{\mathcal{B}_{#1}}}
\newcommand{\baire}{\mathcal{B}}
\newcommand{\squash}[1]{\ensuremath{\downarrow \! #1}}

\newcommand\independent{\protect\mathpalette{\protect\independenT}{\perp}}
\def\independenT#1#2{\mathrel{\rlap{$#1#2$}\mkern2mu{#1#2}}}
\newcommand{\empseq}{\independent}
\newcommand{\dbr}[6]{\mathtt{dbr}(#1,#2,#3,#4,#5,#6)}
\newcommand{\tol}{\ulcorner\to}
\newcommand{\tor}{\to\urcorner}
\newcommand{\tolr}{\lift{\to}}
\newcommand{\embed}{\Uparrow}
\newcommand{\ifthen}[3]{\mathtt{if}(#1;#2;#3)}


\section{Bar Induction}

\begin{definition}(Admissibility of Bar Induction)\label{lemma:birule}
  If 
  \begin{align*}
    &\text{(wfb)}\quad \openTypeComp{\Gamma, \isOf{n}{\lift{\nat}}, \isOf{s}{\lift{\nseq{n}}}}{B}\\
    &\text{(wfs)}\quad \openTypeComp{\Gamma, \isOf{n}{\lift{\nat}}, \isOf{s}{\lift{\nseq{n}}}}{R}\\
    &\text{(init)}\quad \openComp{\Gamma}{M_0}{R(\zero,\empseq)}\\
    &\text{(bar)}\quad \openComp{\Gamma,\isOf{s}{\lift{\baire}}, 
      \isOf{z}{\lift{\intty{\isOf{m}{\lift{\nat}}}{\squash{R(m,s)}}}}}{M_1}
      {\squash{\lift{\sigmaty{\isOf{n}{\lift{\nat}}}{B(n,s)}}}}\\ 
    &\text{(base)}\quad \openComp{\Gamma,\isOf{n}{\lift{\nat}}, \isOf{s}{\lift{\nseq{n}}},
    \isOf{z}{\squash{R(n,s)}}, \isOf{b}{B(n,s)}}{M_2}{P(n,s)}\\
    &\text{(ind)} \quad \openComp{\Gamma,\isOf{n}{\lift{\nat}}, \isOf{s}{\lift{\nseq{n}}},
    \isOf{z}{\squash{R(n,s)}},\\
    &\qquad\qquad\isOf{i}{\lift{\pityc{\isOf{m}{\lift{\nat}}}{
      R(\suc{n}, s \oplus_n m) \tolr P(\suc{n}, s \oplus_n m)}}}}{M_2}{P(n,s)}
  \end{align*}
  Then 
  \[
  \openComp{\Gamma}{\_}{\squash{P(0,\empseq)}}
  \]
\end{definition}

\begin{proof}
  By LEM, we get $\lift{\neg} \squash{P(0,\empseq)}$, which gives $\lift{\neg} P(0,\empseq)$.
  Taking the contrapositive of (ind), we have a function $f$ inhabiting
  \begin{align*}
    &\pityc{\isOf{n}{\lift{\nat}}}{
    \pityc{\isOf{s}{\lift{\nseq{n}}}}{
      \pityc{\isOf{z}{\squash{R(n,s)}}}{\\
        &\lift{\neg}P(n,s) 
      \tolr \sigmatyc{\isOf{m}{\lift{\nat}}}{R(\suc{n}, s \oplus_n m) 
      \lift{\times} \lift{\neg} P(\suc{n}, s \oplus_n m)}}}}
  \end{align*}
  Since $\lift{\neg}P(0,\empseq)$ and $R(0,\empseq)$, we can instantiate $f$ to obtain 
  \isVal{m}{\lift{\nat}} s.t. $R(1,\empseq \oplus_0 m)$ and $\lift{\neg}P(1,\empseq \oplus_0 m)$.
  Iterating this gives us a sequence \isVal{\alpha}{\lift{\baire}} s.t. 
  \inttyc{\isOf{m}{\lift{\nat}}}{\squash{R(m,\alpha)}} is inhabited. 
  It also shows that for all \isVal{n}{\lift{\nat}}, $\lift{\neg}P(n,\alpha)$.
  By (bar), we know it's true that there is \isVal{n}{\lift{\nat}} s.t. $B(n,\alpha)$.
  But by (base), we know that $P(n,\alpha)$, which is a contradiction.
\end{proof}

\begin{definition}(DBR)
  \begin{mathpar}
  \inferrule{}{
    \dbr{dec}{base}{ind}{n}{s}{r} \mapsto \\
  \seq{\thunk{dec\;n\;s}}{x}{
    \case{x}{p}{base\;n\;s\;r\;p}{\_}{ind\;n\;s\;\;r\;(\lam{t,r'}{}{\dbr{dec}{base}{ind}{\suc{n}}{s\oplus_n t}{r'}})}
}}
  \end{mathpar}
\end{definition}

\begin{lemma}
  $\mathtt{dbr}$ implements the following induction principle: 
  \begin{align*}
    \mathsf{BID} = 
    &\ulcorner\pityc{\isOf{R,B,P}{\pityc{\isOf{n}{\lift{\nat}}}{(\embed\nseq{n}) \tolr (\embed\type{})}}}{\\
          &\pityc{\isOf{s}{\lift{\baire}}}{\intty{\isOf{m}{\lift{\nat}}}{\squash{R(m,s)}} 
      \tolr \squash{\lift{\sigmaty{\isOf{n}{\lift{\nat}}}{B(n,s)}}}}\\
      \tolr &(\pityc{\isOf{n}{\lift{\nat}}}{\pityc{\isOf{s}{\lift{\nseq{n}}}}
      {\\
      & \squash{R(n,s)} \tolr (\pityc{\isOf{m}{\lift{\nat}}}{R(\suc{n}, s \oplus_n m) 
        \tolr P(\suc{n}, s \oplus_n m)}) \tolr P(n,s)}})\\
      \tolr &(\pityc{\isOf{n}{\lift{\nat}}}{\pityc{\isOf{s}{\lift{\nseq{n}}}}{\lift{B(n,s) + \neg B(n,s)}}}) \\
      \tolr &(\pityc{\isOf{n}{\lift{\nat}}}{\pityc{\isOf{s}{\lift{\nseq{n}}}}{\squash{R(n,s)} \tolr 
      B(n,s) \tolr P(n,s)}})\\
      \tolr & R(0,\empseq)\\
      \tolr & P(0,\empseq)\urcorner
    }
  \end{align*}
\end{lemma}

\begin{proof}
  Need to show that 
  \isVal{\lam{R,B,P,bar,ind,dec,base,r}{}{\dbr{dec}{base}{ind}{\zero}{\empseq}{r}}}{\mathsf{BID}}.
  We have assumptions:
  \begin{align*}
    &\isVal{R}{\pityc{\isOf{n}{\lift{\nat}}}{\embed\nseq{n} \tolr \embed\type{}}}\\
    &\isVal{B}{\pityc{\isOf{n}{\lift{\nat}}}{\embed\nseq{n} \tolr \embed\type{}}}\\
    &\isVal{P}{\pityc{\isOf{n}{\lift{\nat}}}{\embed\nseq{n} \tolr \embed\type{}}}\\
    &\isVal{bar}{\pityc{\isOf{s}{\lift{\baire}}}{\intty{\isOf{m}{\lift{\nat}}}{\squash{R(m,s)}} 
      \tolr \squash{\lift{\sigmaty{\isOf{n}{\lift{\nat}}}{B(n,s)}}}}}\\
    &\isVal{ind}{\pityc{\isOf{n}{\lift{\nat}}}{\pityc{\isOf{s}{\lift{\nseq{n}}}}
      {\\
      &\qquad\squash{R(n,s)} \tolr (\pityc{\isOf{m}{\lift{\nat}}}{R(\suc{n}, s \oplus_n m) 
        \tolr P(\suc{n}, s \oplus_n m)}) \tolr P(n,s)}}}\\
    &\isVal{dec}{\pityc{\isOf{n}{\lift{\nat}}}{\pityc{\isOf{s}{\lift{\nseq{n}}}}{\lift{B(n,s) + \neg B(n,s)}}}}\\
    &\isVal{base}{\pityc{\isOf{n}{\lift{\nat}}}{\pityc{\isOf{s}{\lift{\nseq{n}}}}{\squash{R(n,s)} \tolr 
      B(n,s) \tolr P(n,s)}}}\\
    &\isVal{r}{R(0,\empseq)}
  \end{align*}
  We need to find a term $M$ s.t. \isComp{M}{P(\zero,\empseq)}.
  Define \[
    Q \triangleq \lam{n,s}{}{\lift{\eqty{P(n,s)}{\dbr{dec}{base}{ind}{n}{s}{r}}{\dbr{dec}{base}{ind}{n}{s}{r}}}}\]. 
  By Rule~\ref{lemma:birule}, we know that $\squash{Q(\zero,\empseq)}$ is inhabited.
  This means that \isVal{\triv}{Q(\zero,\empseq)}, which implies that\\
  \eqComp{\dbr{dec}{base}{ind}{\zero}{\empseq}{r}}{\dbr{dec}{base}{ind}{\zero}{\empseq}{r}}{P(\zero,\empseq)}.
  Hence it suffices to show that the hypotheses of Rule~\ref{lemma:birule} holds. 
  Premises (wfb), (wfs), (init), and (bar) follow directly from assumptions $B,R,r,$ and $bar$.
  For (base), let \isVal{n}{\lift{\nat}}, \isVal{s}{\lift{\nseq{n}}}, \isVal{z}{\squash{R(n,s)}},
  and \isVal{b}{B(n,s)}. We need to show 
  \begin{align*}
    Q(n,s) &\iff \eqty{P(n,s)}{\dbr{dec}{base}{ind}{n}{s}{r}}{\dbr{dec}{base}{ind}{n}{s}{r}} \\
           &\iff \isComp{\dbr{dec}{base}{ind}{n}{s}{r}}{P(n,s)}
  \end{align*}
  We case on the result of $dec\;n\;s$.
  \begin{itemize}
    \item $\eqComp{dec\;n\;s}{\ret{\inl{p}}}{\lift{B(n,s) + \neg B(n,s)}}$ and \isVal{p}{B(n,s)}:\\
      Then $\dbr{dec}{base}{ind}{n}{s}{r} \mapsto^* base\;n\;s\;r\;p$. By assumption,
      \isComp{base\;n\;s\;r\;p}{P(n,s)}, and by Lemma~\ref{lemma:headexp},  
      \isComp{\dbr{dec}{base}{ind}{n}{s}{r}}{P(n,s)}.
    \item $\eqComp{dec\;n\;s}{\ret{\inr{p}}}{\lift{B(n,s) + \neg B(n,s)}}$ and \isVal{p}{\lift{\neg} B(n,s)}:\\
      Then \isComp{p\;b}{\bot}, which is a contradiction.
  \end{itemize}
  Hence the base case holds.
  For (ind), let \isVal{n}{\lift{\nat}}, \isVal{s}{\lift{\nseq{n}}},
    \isVal{z}{\squash{R(n,s)}},\\
    \isVal{i}{\lift{\pityc{\isOf{m}{\lift{\nat}}}{R(\suc{n}, s \oplus_n m) \tolr Q(\suc{n}, s \oplus_n m)}}}.
    We need to show $Q(n,s) \iff \isComp{\dbr{dec}{base}{ind}{n}{s}{r}}{P(n,s)}$.
    Again we case on $dec\;n\;s$. In case $B(n,s)$, we proceed as before. Otherwise, we know 
    \eqComp{dec\;n\;s}{\ret{\inr{p}}}{\lift{B(n,s) + \neg B(n,s)}} and \isVal{p}{\lift{\neg} B(n,s)}.
    Then  $\dbr{dec}{base}{ind}{n}{s}{r} \mapsto^* ind\;n\;s\;\;r\;(\lam{t,r'}{}{\dbr{dec}{base}{ind}{\suc{n}}{s\oplus_n t}{r'}})$.
    By definition, 
    \isComp{ind\;n\;s\;\;r\;(\lam{t,r'}{}{\dbr{dec}{base}{ind}{\suc{n}}{s\oplus_n t}{r'}})}{P(n,s)}, 
    given that we can show 
    \[ \isVal{\lam{t,r'}{}{\dbr{dec}{base}{ind}{\suc{n}}{s\oplus_n t}{r'}}}{
      \pityc{\isOf{m}{\lift{\nat}}}{R(\suc{n}, s \oplus_n m) \tolr P(\suc{n}, s \oplus_n m)}}\]
    Hence let \isVal{t}{\lift{\nat}} and \isVal{r'}{R(\suc{n},s \oplus_n m)}. Suffices to show 
    \isComp{\dbr{dec}{base}{ind}{\suc{n}}{s\oplus_n t}{r'}}{P(\suc{n}, s \oplus_n m)}.
    By assumption, we know \isComp{i\;t\;r'}{Q(\suc{n},s \oplus_n m)},
    which implies \isComp{\dbr{dec}{base}{ind}{\suc{n}}{s \oplus_n m}{r'}}{P(\suc{n},s \oplus_n m)}.
    Hence by Lemma~\ref{lemma:headexp}, we have shown that 
    \isComp{\dbr{dec}{base}{ind}{n}{s}{r}}{P(n,s)}.
  \end{proof}

\section{GCD}

Notation: let $M =_{\nat} N$ when \eqtyc{\lift{\nat}}{M}{N}.
We can use $\mathtt{dbr}$ to define gcd:
\[
  gcdProp(x,y) \triangleq \sigmatyc{\isOf{d,l,k}{\lift{\nat}}}{\ret{d} =_{\nat} kx + ly}
\]

Define the spread $R$, bar $B$, and $P$: 

\begin{align*}
  R(n,s) \triangleq &
  \ifthen{eq_b\;n\;0}{\ret{\top}\\}{
    &\ifthen{eq_b\;n\;1}{s(0) =_{\nat} \ret{x}\\}{
      &\ifthen{eq_b\;n\;2}{
        s(0) =_{\nat} \ret{x} \lift{\times} s(1) =_{\nat} \ret{y}\\
      }{
        &s(0) =_{\nat} \ret{x} \lift{\times} s(1) =_{\nat} \ret{y}
        \lift{\times} mod\, s(n-3)\, s(n-2) =_{\nat} s(n-1)
      }
    }
  }\\
  B(n,s) \triangleq & n > 1 \lift{\times} (s(n-1) =_{\nat} \ret{\zero}) \\
  P(n,s) \triangleq &\bind{n \le_b 2\\}{x}{
    &\ifthen{x}{gcdProp(x,y)}
  {\bind{s(n-1)}{s_1}{\bind{s(n-2)}{s_2}{gcdProp(s_2,s_1)}}}}
\end{align*}

The predicates $R,B,P$ are well-formed by the various formation rules. 
Additionally, we need to show the following: 
\begin{enumerate}
  \item \isVal{bar}{\pityc{\isOf{s}{\lift{\baire}}}{\intty{\isOf{m}{\lift{\nat}}}{\squash{R(m,s)}} 
      \tolr \squash{\lift{\sigmaty{\isOf{n}{\lift{\nat}}}{B(n,s)}}}}}:\\
    Strengthen the property by considering the starting sum of the sequence: 
    \[
    \pityc{\isOf{i,j}{\lift{\nat}}}{
      \pityc{\isOf{s}{\lift{\baire}}}{\intty{\isOf{m}{\lift{\nat}}}{\squash{R(m,s)}} 
      \tolr \bind{s(j) + s(\suc{j})}{v}{v \le i}
      \tolr \squash{\lift{\sigmaty{\isOf{n}{\lift{\nat}}}{B(n,s)}}}}}
  \]
    Let \isVal{j}{\lift{\nat}},
    \isVal{s}{\lift{\baire}}, \isVal{a}{\inttyc{\isOf{m}{\lift{\nat}}}{\squash{R(m,s)}}}, and 
    \isVal{p}{\bind{s(j) + s(\suc{j})}{v}{v \le i}}.
    Proceed by induction:
    \begin{itemize}
      \item $i = \zero$: 
    Need to show a term in \squash{\lift{\sigmaty{\isOf{n}{\lift{\nat}}}{B(n,s)}}}.
    Let \eval{s(j) + s(\suc{j})}{v}. \isVal{p}{\bind{s(j) + s(\suc{j})}{v}{v \le \zero}} implies that 
    $\ret{v} =_{\nat} \ret{\zero}$, which implies $s(j) =_{\nat} \ret{\zero}$ and 
    $s(\suc{j}) =_{\nat} \ret{\zero}$. 
    Hence we know \isComp{\ret{(\triv, \triv)}}{B(\suc{\suc{j}},s)}.
      \item $i = \suc{i'}$ for some \isVal{i'}{\lift{\nat}}:
    By induction we have a term $f$ inhabiting
        \begin{align*}
  \compc{
    &\pityc{\isOf{j}{\lift{\nat}}}{\\
          &\pityc{\isOf{s}{\lift{\baire}}}{\intty{\isOf{m}{\lift{\nat}}}{\squash{R(m,s)}} 
        \tolr \bind{s(j) + s(\suc{j})}{v}{v \le i'}
        \tolr \squash{\lift{\sigmaty{\isOf{n}{\lift{\nat}}}{B(n,s)}}}}}}
        \end{align*}
    By the spread law, we know:
        \begin{gather*}
          s(j) - s(\suc{j}) =_{\nat} s(\suc{\suc{j}})\\
          s(\suc{j}) - s(\suc{\suc{j}}) =_{\nat} s(\suc{\suc{\suc{j}}})
        \end{gather*}
        Let \eval{s(j) + s(\suc{j})}{v} and \eval{s(\suc{\suc{j}}) + s(\suc{\suc{\suc{j}}})}{u}.
        We case on $s(\suc{j})$ and $s(\suc{\suc{j}})$. If \eval{s(\suc{j})}{\zero}, 
        again \isComp{\ret{(\triv, \triv)}}{B(\suc{\suc{j}},s)}.
        Otherwise, we know $u < v \le \suc{i'}$, and hence $u \le i'$. 
        Then \isComp{f\,\suc{\suc{j}}\, s\,a\,\triv}{\squash{\lift{\sigmaty{\isOf{n}{\lift{\nat}}}{B(n,s)}}}}
        .
    \end{itemize}
  \item \isVal{ind}{\pityc{\isOf{n}{\lift{\nat}}}{\pityc{\isOf{s}{\lift{\nseq{n}}}} {\\
      \qquad\squash{R(n,s)} \tolr (\pityc{\isOf{m}{\lift{\nat}}}{R(\suc{n}, s \oplus_n m) 
        \tolr P(\suc{n}, s \oplus_n m)}) \tolr P(n,s)}}}\\
  Given \isVal{n}{\lift{\nat}}, \isVal{s}{\lift{\nseq{n}}}, 
    \isVal{r}{\squash{R(n,s)}}, 
    \isVal{i}{\pityc{\isOf{m}{\lift{\nat}}}{R(\suc{n}, s \oplus_n m)
        \tolr P(\suc{n}, s \oplus_n m)}},
    need to show $P(n,s)$. Proceed with induction on $n$.
    \begin{itemize}
      \item $n = \zero$:
        Suffices to show $P(0,s) \mapsto^* gcdProp(x,y)$.
        Note that \isComp{i\, x\, \triv}{P(1,[x])}, since 
        $R(1,[x]) \mapsto^* s(0) =_{\nat} \ret{x} \mapsto^* 
        \ret{x} =_{\nat} \ret{x}$. Thus 
        \isComp{i\, x\, \triv}{gcdProp(x,y)} since $P(0,s) \mapsto^* gcdProp(x,y)$.
      \item $n = \suc{n'}$ for \isVal{n'}{\lift{\nat}}:
        By induction, we have a term 
        \begin{gather*}
        \isVal{f}{\pityc{\isOf{s'}{\lift{\nseq{n'}}}} {
          \squash{R(n',s')} \tolr (\pityc{\isOf{m}{\lift{\nat}}}{R(\suc{n'}, s' \oplus_{n'} m) \tolr P(\suc{n'}, s' \oplus_{n'} m)})\\ \tolr P(n',s')}}
        \end{gather*}
        Suffices to show a term inhabiting $P(\suc{n'}, s)$.
        Now case on $n'$: 
        \begin{itemize}
          \item $n' = \zero$: then again it suffices to show 
            $P(\suc{n'}, s) \mapsto^* gcdProp(x,y)$.
            Since \isVal{r}{\squash{R(\suc{\zero},s)}}, we know
            $s(0) =_{\nat} \ret{x}$.
            Note that \isComp{i\, y\, (\triv,\triv)}{P(\suc{\suc{\zero}},[x,y])}, since 
            \[R(\suc{\suc{\zero}}, [x,y]) \mapsto^* s(0) =_{\nat} \ret{x} \lift{\times}
            s(1) =_{\nat} \ret{y}\]. Done since 
            $P(\suc{\suc{\zero}},[x,y]) \mapsto^* gcdProp(x,y)$.
          \item $n' = \suc{\zero}$: suffices to show
            $P(\suc{\suc{n'}}, s) \mapsto^* gcdProp(x,y)$.
            Since \isVal{r}{\squash{R(\suc{\suc{\zero}},s)}}, we know
            $s(0) =_{\nat} \ret{x}$ and $s(1) =_{\nat} \ret{y}$.
            Let \eval{x-y}{v}.
            Note that \isComp{\bind{x-y}{d}{i\, d\, (\triv,\triv,\triv)}}{
              \bind{x-y}{d}{P(3,[x,y,d])}}, since 
            \[R(3,s\oplus_3 v) \mapsto^* \ret{x} =_{\nat} \ret{x} \lift{\times} 
            \ret{y} =_{\nat} \ret{y} \lift{\times} \ret{v} =_{\nat} v\].
            Furthermore, note that 
            \begin{align*} 
              &\bind{x-y}{d}{P(3,[x,y,d])} \\
              \mapsto &gcdProp(y,v)
            \end{align*}
            Hence \isComp{\bind{x-y}{d}{\bind{i\, d\, (\triv,\triv,\triv)}{r}{
              \fst{\snd{r}}}}}{gcdProp(y,v)}.
            So we know the type $gcdProp(y,v) = 
            \sigmatyc{\isOf{d,k,l}{\lift{\z}}}{d =_{\z} k x + l v}$ is inhabited.
            To obtain $gcdProp(x,y)$, we case on $x \ge_b y$. If this is the case,
            then: 
            \begin{align*}
              d &=_{\z} k y + l v\\
                &= k y + l (x - y)\\
                & = k y + l x - l y\\
                & = k y - l y + l x\\
                & = (k - l)y + l x\\
                & = l x + (k-l)y
            \end{align*}
            Similarly for $x <_b y$, we know $d = (-l) x + (k+l)y$.
            So given \isVal{g}{gcdProp(y,v)}, we know 
            \begin{align*}
              \isComp{\bind{\fst{g}\\}{&d}{
                \bind{\fst{\snd{g}}\\}{&k}{
                  \bind{\fst{\snd{\snd{g}}}\\}{&l}{
                    &\ifthen{x \ge_b y}{(d,l,k-l,\triv)}{(d,-l,k+l,\triv)}}
            }}}{gcdProp(x,y)}
            \end{align*}
            Hence 
            \begin{align*}
              \isComp{
                \bind{\bind{x-y}{d}{i\, d\, (\triv,\triv,\triv)\\}}{&g}{
              \bind{\fst{g}\\}{&d}{
                \bind{\fst{\snd{g}}\\}{&k}{
                  \bind{\fst{\snd{\snd{g}}}\\}{&l}{
                    &\ifthen{x \ge_b y}{(d,l,k-l,\triv)}{(d,-l,k+l,\triv)}}}
            }}}{gcdProp(x,y)}
            \end{align*}
          \item $n' > \suc{\zero}$:
            Let \eval{s(n-1)}{s_1} and \eval{s(n-2)}{s_2}.
            Now we need show 
            \begin{align*}
              P(\suc{n'},s) \mapsto^* gcdProp(s_2,s_1)
            \end{align*}
            is inhabited. Then we have 
            \isComp{\bind{|s_2 - s_1|}{d}{i\, d\, (\triv,\triv,\triv)}}{
              P(\suc{\suc{n'}}, s \oplus_{\suc{n'}} d)}. 
            Note that $P(\suc{\suc{n'}}, s \oplus_{\suc{n'}} d) \mapsto^* 
            gcdProp(s_1,v)$, where \eval{|s_2 - s_1|}{v}.
            Similar to above, we case on $s_2 \ge s_1$, and derive the following:
            \begin{align*}
              \isComp{
                \bind{\bind{|s_2-s_1|}{d}{i\, d\, (\triv,\triv,\triv)\\}}{&g}{
              \bind{\fst{g}\\}{&d}{
                \bind{\fst{\snd{g}}\\}{&k}{
                  \bind{\fst{\snd{\snd{g}}}\\}{&l}{
                    &\ifthen{s_2 \ge_b s_1}{(d,l,k-l,\triv)}{(d,-l,k+l,\triv)}}}
            }}}{gcdProp(s_2,s_1)}
            \end{align*}
        \end{itemize}
        Thus realizer can be defined as 
        \begin{align*}
          ind \triangleq &\lam{n,s,r,i}{}{\\
            \ifthen{&eq\,n\,0}{i\,x\,\triv\\}{
              \ifthen{&eq\,n\,1}{i\,y\,(\triv,\triv)\\}{
                \ifthen{&eq\,n\,2\\}{
          \bind{\bind{x-y}{d}{i\, d\, (\triv,\triv,\triv)\\}}{&g}{
              \bind{\fst{g}\\}{&d}{
                \bind{\fst{\snd{g}}\\}{&k}{
                  \bind{\fst{\snd{\snd{g}}}\\}{&l}{
                    &\ifthen{x \ge_b y}{(d,l,k-l,\triv)}{(d,-l,k+l,\triv)}}}
            }\\}
          }{
            \bind{\bind{|s_2-s_1|}{d}{i\, d\, (\triv,\triv,\triv)\\}}{&g}{
              \bind{\fst{g}\\}{&d}{
                \bind{\fst{\snd{g}}\\}{&k}{
                  \bind{\fst{\snd{\snd{g}}}\\}{&l}{
                    &\ifthen{s_2 \ge_b s_1}{(d,l,k-l,\triv)}{(d,-l,k+l,\triv)}}}
            }}
          }
              }
            }
          }
        \end{align*}
    \end{itemize}
  \item \isVal{dec}{\pityc{\isOf{n}{\lift{\nat}}}{\pityc{\isOf{s}{\lift{\nseq{n}}}}{\lift{B(n,s) + \lift{\neg} B(n,s)}}}}:
  Let \isVal{n}{\lift{\nat}} and \isVal{s}{\lift{\nseq{n}}}.
  Case on $n$:
  \begin{itemize}
    \item $n = \zero, \suc{\zero}$:
      Prove the right disjunct:
      $B(n,s) \mapsto^* \bot \lift{\times} \_$, so 
      \isVal{\lam{b}{}{\bind{\fst{b}}{v}{absurd\,v}}}{\lift{\neg} B(n,s)}.
    \item $n \ge_b 2$:
      Let \eval{s(n-1)}{v}. If $v = \zero$, we can prove the left disjunct:
      \isVal{(\triv, \triv)}{B(n,s)}. Otherwise, we can prove 
      \isVal{\lam{b}{}{\triv}}{\lift{\neg}B(n,s)}.
  \end{itemize}
  The realizer for dec is 
  \begin{align*}
    dec &\triangleq \lam{n,s}{}{
      \ifthen{n \le_b 1}{\lam{b}{}{\bind{\fst{b}}{v}{absurd\,v}}}
      {
        \lam{b}{}{\triv}
      }
    }
  \end{align*}

  \item \isVal{base}{\pityc{\isOf{n}{\lift{\nat}}}{\pityc{\isOf{s}{\lift{\nseq{n}}}}{\squash{R(n,s)} \tolr 
      B(n,s) \tolr P(n,s)}}}:
    Let \isVal{n}{\lift{\nat}}, \isVal{s}{\lift{\nseq{n}}}, 
    \isVal{r}{\squash{R(n,s)}}, and \isVal{b}{B(n,s)}. We need to show 
    $P(n,s)$. Case on $n$: 
    \begin{itemize}
      \item $n = \zero,\suc{\zero}$:
        Then $B(n,s) \mapsto^* \bot \lift{\times}  (s(n-1) =_{\nat} \ret{\zero})$, so 
        \isComp{\bind{\fst{b}}{v}{absurd\,v}}{P(n,s)}.
      \item $n = \suc{\suc{\zero}}$: 
        Then $B(n,s) \mapsto^* \top \lift{\times}  (s(n-1) =_{\nat} \ret{\zero})$.
        We need to show 
        $P(n,s) \mapsto^* gcdProp(x,y)$. By $\squash{R(n,s)}$, we know 
        $s(0) =_{\nat} \ret{x}$ and $s(1) =_{\nat} \ret{y}$. Hence we know 
        $\ret{y} =_{\nat} \ret{\zero}$. Since $x = 1 \cdot x + 0 \cdot 0$,  
        \isComp{\ret{(x,\suc{\zero},\zero)}}{gcdProp(x,\zero)}.
      \item $n \ge_b 3$: 
        Then $B(n,s) \mapsto^* \top \lift{\times}  (s(n-1) =_{\nat} \ret{\zero})$.
        Let \eval{s(n-2)}{s_2}. We need to show 
        $P(n,s) \mapsto^* gcdProp(s_2,\zero)$.
        Since $s_2 = 1 \cdot s_2 + 0 \cdot 0$,  
        \isComp{\ret{(s_2,\suc{\zero},\zero)}}{gcdProp(s_2,\zero)}.
    \end{itemize}
    Hence the realizer for base is:
    \begin{align*}
      base &\triangleq \lam{n,s,r,b}{}{\\
        \ifthen{&n \le_b 1}{\bind{\fst{b}}{v}{absurd\,v}\\}{
          \ifthen{&eq\,n\,2}{\ret{(x,\suc{\zero},\zero)}\\}{
            &\bind{s(n-2)}{s_2}{(s_2,\suc{\zero},\zero)}
          }
        }}
    \end{align*}
  \item \isVal{r}{R(0,\empseq)}:
    Since $R(0,\empseq) \mapsto^* \top$, $r \triangleq \triv$ suffices.
\end{enumerate}

Finally, we can define gcd as
\[
  gcd(x,y) \triangleq \dbr{dec}{base}{ind}{0}{\empseq}{r}
\]




\begin{definition}(Types and Equality)
  Given a type system $\tau$, 
  \begin{enumerate}
    \item \eqType{A}{B} when \sameType{A}{B}{\alpha}{\tau} for some $\alpha$.
    \item \eqCComp{M}{M'}{A}{P}{a.Q} when presupposing \eqType{A}{A},
      \sameType{A}{A}{\alpha}{\tau} and \sameCComp{}{M}{M'}{\alpha}{P}{a.Q}.
    \item \eqVal{v}{v'}{A} when presupposing \eqType{A}{A},
      \sameType{A}{A}{\alpha}{\tau} and $\alpha(v,v')$.
  \end{enumerate}
\end{definition}

\begin{definition}(Context)
  \eqCtx{\Gamma}{\Gamma'} when
  \begin{enumerate}
    \item \eqCtx{\cdot}{\cdot}
    \item \eqCtx{\isOf{a}{A},\Gamma}{\isOf{a}{A'},\Gamma'} when \eqType{A}{A'} and for all 
      \eqVal{v}{v'}{A}, \eqCtx{[v/a]\Gamma}{[v'/a]\Gamma'}
  \end{enumerate}
\end{definition}

\begin{definition}(Substitution)
  \eqInst{\gamma}{\gamma'}{\Gamma} when
  \begin{enumerate}
    \item \eqInst{\cdot}{\cdot}{\cdot}
    \item \eqInst{v,\gamma}{v',\gamma'}{\isOf{a}{A},\Gamma} when
      \eqVal{v}{v'}{A}, \eqInst{\gamma}{\gamma'}{[v/a]\Gamma}
  \end{enumerate}
\end{definition}

\begin{definition}(Open Judgments)
  \begin{enumerate}
    \item \openEqTypeComp{\Gamma}{A}{A'} when for all \eqInst{\gamma}{\gamma'}{\Gamma}, 
      \eqType{\gamma A}{\gamma A'}
    \item \openEqCComp{\Gamma}{M}{M'}{A}{P}{a.Q} when for all \eqInst{\gamma}{\gamma'}{\Gamma}, 
      \eqCComp{\gamma M}{\gamma M'}{\gamma A}{\gamma P}{a.\gamma Q}
    \item \openEqVal{\Gamma}{v}{v'}{A} when for all \eqInst{\gamma}{\gamma'}{\Gamma}, 
      \eqVal{\gamma v}{\gamma v'}{\gamma A}
  \end{enumerate}
\end{definition}

\begin{lemma}(Sequence)
  If \isCComp{M_1}{A_1}{P_1}{a_1.Q_1} and 
  \openCComp{\isOf{x}{A_1}}{M_2}{A_2}{P_2}{a_2.Q_2},\\ then 
  \isCComp{\seq{\thunk{M_1}}{x}{M_2}}{\seq{\thunk{M_1}}{x}{A_2}}
  {\fst{R}^+}{\_.\snd{R}}
  where 
  \[ 
  R \triangleq 
\seq{\thunk{M_1}}{x}{\seq{\thunk{M_2}}{y}{
    (P_1,[x/a_1]Q_1) \cdot (P_2,[y/a_2]Q_2)
  }}
  \]
\end{lemma}

\begin{proof}
  First, we need to show \eqType{\seq{\thunk{M_1}}{x}{A_2}}{\seq{\thunk{M_1}}{x}{A_2}}.
  By definition, STS \sameType{\seq{\thunk{M_1}}{x}{A_2}}{\seq{\thunk{M_1}}{x}{A_2}}{\alpha}{\tau} for some $\alpha$. 
  By assumption, we know \sameType{A_1}{A_1}{\alpha_1}{\tau} for some $\alpha_1$ and 
  that \eval{M_1}{v_1} and $\alpha_1(v_1,v_1)$. Hence $\eqVal{v_1}{v_1}{A_1}$.
  By definition, \eqInst{(v_1,\cdot)}{(v_1,\cdot)}{(\isOf{x}{A_1},\cdot)}.
  The second assumption gives \openEqTypeComp{\isOf{x}{A_1}}{A_2}{A_2}. 
  Applying the definition with the above instance gives 
  \eqType{[v_1/x]A_2}{[v_1/x]A_2}. This means \sameType{[v_1/x]A_2}{[v_1/x]A_2}{\alpha}{\tau}
  for some $\alpha$. Since
  \stepIn{\seq{\thunk{M_1}}{x}{A_2}}{*}{[v_1/x]A_2}, we are done.

  Next, we need to show 
  \isCComp{\seq{\thunk{M_1}}{x}{M_2}}{\seq{\thunk{M_1}}{x}{A_2}}
  {\fst{R}}{\_.\snd{R}}.
  STS 
  \sameCComp{}{\seq{\thunk{M_1}}{x}{M_2}}{\seq{\thunk{M_1}}{x}{M_2}}{\alpha}
  {\fst{R}}{\snd{R}} for the $\alpha$
  above. By assumption, we have 
  \eqCComp{[v_1/x]M_2}{[v_1/x]M_2}{[v_1/x]A_2}{[v_1/x]P_2}{a_2.[v_1/x]Q_2}.
  Hence we know 
  \begin{mathpar}
    \eval{P_1}{\bar{p_1}}

    \eval{[v_1/a_1]A_1}{\bar{q_1}}

    \eval{[v_1/x]P_2}{\bar{p_2}}

    \eval{[v_1/x]M_2}{v_2} 
    
    \eval{[v_1/x, v_2/a_2]}{Q_2}

    \eval{R}{(\bar{r},\bar{s})}
  \end{mathpar}
  Note that 
  \begin{align*}
    &\seq{\thunk{M_1}}{x}{M_2}\\
    \mapsto^{c_1}& \seq{\thunk{\ret{v_1}}}{x}{M_2}\\
    \mapsto &[v_1/x]M_2\\
    \mapsto^{c_2}& v_2
  \end{align*}
  where 
  $\alpha(v_2,v_2)$, 
  $p_1 \ge c_1 + q_1$, and $p_2 \ge c_2 + q_2$. 
  If $p_2 > q_1$, then $r = p_1 - q_1 + p_2$ and $s = q_2$, and it suffices 
  to show 
  \[
    p_1 - q_1 + p_2 + 1\ge c_1 + 1 + c_2 + q_2 
  \]
  which holds given the two equations from the assumption.
  Otherwise, $r = p_1$ and $s = q_1 - p_2 + q_2$, and we need to show
  \[
  p_1 + 1\ge c_1 + 1 + c_2 + q_1 - p_2 + q_2
  \]
  which also holds. Thus 
  \sameCComp{}{\seq{\thunk{M_1}}{x}{M_2}}{\seq{\thunk{M_1}}{x}{M_2}}{\alpha}
  {\fst{R}^+}{\snd{R}}, and we are done.
\end{proof}

\begin{lemma}(Nat Formation)
  \eqType{\lift{\nat}}{\lift{\nat}}.
\end{lemma}

\begin{lemma}(Nat Introduction)
  \begin{enumerate}
    \item \eqVal{\zero}{\zero}{\nat}. 
    \item \eqVal{\suc{n}}{\suc{n'}}{\nat} given \eqVal{n}{n'}{\nat}.
  \end{enumerate}
\end{lemma}

\begin{lemma}(Comp Formation)
  Given \eqType{A}{A'}, \eqComp{P}{P'}{\nat}, \openEqComp{\isOf{a}{A}}{Q}{Q'}{\nat}, then
  \eqType{\ccomp{P}{\isOf{a}{A}}{Q}}{\ccomp{P'}{\isOf{a}{A'}}{Q'}}.
\end{lemma}

\begin{lemma}(Comp Introduction)
  If \eqCComp{M}{M'}{A}{P}{a.Q}, then
  \eqVal{\thunk{M}}{\thunk{M'}}{\ccomp{P}{\isOf{a}{A}}{Q}}.
\end{lemma}

\begin{proof}
  Suppose \eqCComp{M}{M'}{A}{P}{a.Q}. This means
  \evalCost{M}{c}{v}, \evalCost{M'}{c'}{v'}, \eval{P}{\bar{p}}, 
  \eval{[v/a]Q}{\bar{q}}, \eval{[v'/a]Q}{\bar{q'}}, and 
  $p \ge c + q$ and $p \ge c' + q'$.
  By definition, $\tau(\ccomp{P}{\isOf{a}{A}}{Q}, \ccomp{P}{\isOf{a}{A}}{Q}, \phi)$, and 
  the above conditions suffice for $\phi(\thunk{M},\thunk{M'})$. Hence 
  \eqVal{\thunk{M}}{\thunk{M'}}{\ccomp{P}{\isOf{a}{A}}{Q}}.
\end{proof}

\begin{lemma}(Promise)
  If \isVal{u}{\ccomp{P}{\isOf{a}{A}}{Q}}, 
  then $u$ is $\thunk{M}$ for some $M$ s.t. \eqCComp{M}{M}{A}{P}{a.Q}.
\end{lemma}

\begin{proof}
  By assumption, $\tau(\ccomp{P}{\isOf{a}{A}}{Q}, \ccomp{P}{\isOf{a}{A}}{Q}, \phi)$ and 
  $\phi(u,u)$ for some $\phi$. The result follows from the definition of $\phi$.
\end{proof}

\begin{lemma}(Head Expansion)\label{lemma:headexp}
  \begin{enumerate}
    \item \isTypeComp{A} and \stepIn{A}{*}{A'} implies \eqType{A}{A'}.
    \item \isCComp{M'}{A}{P}{a.Q} and \step{M}{M'} implies 
      \eqCComp{M}{M'}{A}{P^+}{a.Q}.
  \end{enumerate}
\end{lemma}

\begin{proof} \ 
  \begin{enumerate}
    \item Suppose \isTypeComp{A} and \stepIn{A}{*}{A'}. This means 
      \sameType{A}{A}{\alpha}{\tau} for some $\alpha$, which means 
      \eval{A}{A_0} and $\tau(A_0,A_0,\tau)$. Since $\Downarrow$ deterministic, 
      \eval{A'}{A_0}, and hence also \sameType{A}{A'}{\alpha}{\tau}. By definition,
      \eqType{A}{A'}.
    \item Suppose \isCComp{M'}{A}{P}{a.Q} and \step{M}{M'}. This means 
      \sameType{A}{A}{\alpha}{\tau} and 
      \sameCComp{}{M'}{M'}{\alpha}{P}{a.Q} for some $\alpha$.
      This means \evalCost{M'}{c}{v}, $\alpha(v,v)$, 
      \eval{P}{\bar{p}}, \eval{[v'/a]Q}{\bar{q}}, and 
      $p \ge c + q$. 
      Again, since $\Downarrow$ is 
      deterministic, \step{M}{M'} and 
      \evalCost{M'}{c}{v}. Hence 
      \sameCComp{}{M}{M'}{\alpha}{P^+}{a.Q}. Thus 
      \eqCComp{M}{M'}{A}{P^+}{a.Q}.
  \end{enumerate}
\end{proof}

\iffalse
\begin{lemma}(Recursor)
  Given 
  \begin{gather*}
    \isVal{n}{\ret{\nat}}\\
    \openTypeComp{\isOf{x}{\ret{\nat}}}{B}\\
    \isComp{M_0}{[\zero/x]B}\\
    \openComp{\isOf{z}{\ret{\nat}}, \isOf{f}{\ret{\ccomp{\zp}{[z/x]B}{\zp}}}}{M_1}{[\suc{z}/x]B}\\
    \isComp{P_0}{\ret{\nat}}\\
    \openComp{\isOf{z}{\ret{\nat}}}{P_1}{\ret{\nat}}\\
    \isComp{Q_0}{\ret{\nat}}\\
    \openComp{\isOf{z}{\ret{\nat}}}{Q_1}{\ret{\nat}}\\
    \evalCost{M_0}{c_0}{v_0}, \eval{P_0}{\bar{p_0}}, \eval{Q_0}{\bar{q_0}}, 
    p_0 \ge c_0 + q_0\\
    \forall \isVal{z}{\ret{\nat}}, \isVal{f}{\ret{\ccomp{\zp}{[z/x]B}{\zp}}}.\,
    \evalCost{M_1}{c_1}{v_1}, \eval{P_1}{\bar{p_1}}, \eval{Q_1}{\bar{q_1}},
    p_1 \ge c_1 + q_1
  \end{gather*}
  Then 
  \begin{gather*}
  \evalCost{\rec{n}{M_0}{z}{f}{M_1}}{c}{v},\\
    \eval{\rec{n}{P_0^+}{z}{g}{\bind{P_1}{p_1}{\seq{g}{r}{\plus\;p_1\;r}}}^+}{\bar{p}},\\
    \eval{\rec{n}{Q_0^+}{z}{g}{\bind{Q_1}{q_1}{\seq{g}{r}{\plus\;q_1\;r}}}^+}{\bar{q}}, 
  \end{gather*}
  and $p \ge c + q$.
\end{lemma}

\begin{proof}
  Induction on $n$.
  \begin{itemize}
    \item $n = \zero$:
      \begin{align*}
        &\rec{\zero}{M_0}{z}{f}{M_1}\\
        \mapsto& M_0\\
        \mapsto^{c_0}& \ret{v_0}\\
      \end{align*}
      Further,
  \eval{\rec{\zero}{P_0^+}{z}{g}{\bind{P_1}{p_1}{\seq{g}{r}{\plus\;p_1\;r}}}^+}{\overline{p_0+1}},
      \eval{\rec{\zero}{Q_0^+}{z}{g}{\bind{Q_1}{q_1}{\seq{g}{r}{\plus\;q_1\;r}}}^+}{\overline{q_0+1}}, 
      
  and the result holds since $p_0 \ge c_0 + q_0$.
\item $n = \suc{n'}$ for \isVal{n'}{\ret{\nat}}:\\
  Then \step{\rec{\suc{n'}}{M_0}{z}{f}{M_1}}{[n'/z, \thunk{\rec{n'}{M_0}{z}{f}{M_1}}/f]M_1}.
  By induction: 
  \begin{gather*}
  \evalCost{\rec{n'}{M_0}{z}{f}{M_1}}{c'}{v'},\\
    \eval{\rec{n'}{P_0^+}{z}{g}{\bind{P_1}{p_1}{\seq{g}{r}{\plus\;p_1\;r}}}^+}{\bar{p'}},\\
    \eval{\rec{n'}{Q_0^+}{z}{g}{\bind{Q_1}{q_1}{\seq{g}{r}{\plus\;q_1\;r}}}^+}{\bar{q'}}, 
  \end{gather*}
      and $p' \ge c' + q'$.
  \end{itemize}
\end{proof}
\fi


\end{document}
