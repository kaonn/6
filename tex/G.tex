% --------------------------------------------------------------
% This is all preamble stuff that you don't have to worry about.
% Head down to where it says "Start here"
% --------------------------------------------------------------
 
\documentclass[12pt]{article}
 
\usepackage[margin=1in]{geometry} 
\usepackage{amsmath,amsthm,amssymb,tikz}
\usepackage{mathpartir}
\usepackage{stix}
\usepackage{tabularx}
\usepackage{booktabs,colortbl}
\usepackage{makecell}
\usepackage{tgpagella}
\usepackage[normalem]{ulem}
 
\newcommand{\N}{\mathbb{N}}
\newcommand{\Z}{\mathbb{Z}}
\newcommand{\R}{\mathbb{R}}
 
\newenvironment{theorem}[2][Theorem]{\begin{trivlist}
\item[\hskip \labelsep {\bfseries #1}\hskip \labelsep {\bfseries #2.}]}{\end{trivlist}}
\newtheorem{lemma}{Lemma}[section]
\newtheorem{corollary}{Corollary}[lemma]
\newenvironment{exercise}[2][Exercise]{\begin{trivlist}
\item[\hskip \labelsep {\bfseries #1}\hskip \labelsep {\bfseries #2.}]}{\end{trivlist}}
\newenvironment{reflection}[2][Reflection]{\begin{trivlist}
\item[\hskip \labelsep {\bfseries #1}\hskip \labelsep {\bfseries #2.}]}{\end{trivlist}}
\newenvironment{proposition}[2][Proposition]{\begin{trivlist}
\item[\hskip \labelsep {\bfseries #1}\hskip \labelsep {\bfseries #2.}]}{\end{trivlist}}
\iffalse
\newenvironment{corollary}[2][Corollary]{\begin{trivlist}
\item[\hskip \labelsep {\bfseries #1}\hskip \labelsep {\bfseries #2.}]}{\end{trivlist}}
\fi
\newenvironment{problem}[2][Problem]{\begin{trivlist}
\item[\hskip \labelsep {\bfseries #1}\hskip \labelsep {\bfseries #2.}]}{\end{trivlist}}

\newcommand*{\rulefiller}{%
  \arrayrulecolor[gray]{0.8}% change to cell colour
  \specialrule{\heavyrulewidth}{0pt}{-\heavyrulewidth}% "invisible" rule
  \arrayrulecolor{black}% revert to regular line colour
}

\newcommand{\map}[3]{{\color{red}{\varphi}} #1 : #2.\; #3}
\newcommand{\zero}{{\color{red}{\mathtt{zero}}}}
\newcommand{\suc}[1]{{\color{red}{\mathtt{suc}}}(#1)}
\newcommand{\pair}[2]{{\textcolor{red}{\langle}} #1,#2 {\textcolor{red}{\rangle}}}
\newcommand{\ind}[5]{{\color{red}{\mathtt{ind}}}(#1)(#2;#3.#4.#5)}

\newcommand{\lam}[3]{\ensuremath{{\color{red}{\lambda}} #1 .\; #3}}
\newcommand{\seq}[3]{\mathtt{seq}(#1;#2.\;#3)}
\newcommand{\bind}[3]{#2 \leftarrow #1;\;#3}
\newcommand{\rec}[5]{\mathtt{rec}(#1)(#2;#3.#4.#5)}
\newcommand{\thunk}[1]{{\color{red}{\mathtt{thunk}}}(#1)}
\newcommand{\comp}[1]{{\color{red}{\mathtt{comp}}}(#1)}
\newcommand{\ccomp}[3]{{\color{red}{\mathtt{comp}}}(#1, #2 \diamond #3)}
\newcommand{\force}[1]{\mathtt{force}(#1)}
\newcommand{\ret}[1]{\mathtt{ret}(#1)}
\newcommand{\fst}[1]{#1 \cdot \mathtt{1}}
\newcommand{\snd}[1]{#1 \cdot \mathtt{2}}

\newcommand{\nat}{\mathtt{nat}}
\newcommand{\pure}[2]{#1 \rhd #2}
\newcommand{\arr}[4]{\ensuremath{(#1\diamond#2) \to (#3\diamond#4)}}
\newcommand{\dprod}[2]{\ensuremath{#1 \times #2}}

\newcommand{\step}[2]{\ensuremath{#1 \mapsto #2}}
\newcommand{\stepIn}[3]{\ensuremath{#1 \mapsto^{#2} #3}}
\newcommand{\eqto}[2]{#1 \Rightarrow #2}
\newcommand{\eval}[2]{\ensuremath{#1 \Downarrow #2}}
\newcommand{\evalCost}[3]{\ensuremath{#1 \Downarrow^{#2} #3}}

\newcommand{\val}[1]{\ensuremath{#1 \;\mathsf{val}}}
\newcommand{\final}[1]{\ensuremath{#1 \;\mathsf{final}}}

\newcommand{\isOf}[2]{#1 {:} #2}
\newcommand{\gammaToVal}[3]{\ensuremath{#1 \gg #2 \in_0 #3}}
\newcommand{\gammaToTypeComp}[3]{\ensuremath{#1 \gg #3 \; \mathsf{type}}}
\newcommand{\gammaToTypeCompFixed}[3]{\ensuremath{#1 \rhd_{#2} #3 \; \mathsf{type}}}
\newcommand{\gammaToComp}[5]{\ensuremath{#1 \diamond #2 \gg #3 \in #4 \diamond #5}}
\newcommand{\gammaToCompFixed}[6]{\ensuremath{#1 \diamond #2 \rhd_{#3} #4 \in (#5 \diamond #6)}}
\newcommand{\gammaToPot}[2]{\ensuremath{#1 \gg #2 \; \mathsf{pot}}}
\newcommand{\gammaToPotFixed}[3]{\ensuremath{#1 \rhd_{#2} #3 \; \mathsf{pot}}}

\newcommand{\gammaToPotTyped}[3]{\ensuremath{#1 \gg #2 \; \mathsf{pot}\; #3}}

\newcommand{\isVal}[2]{\ensuremath{#1 \in_0 #2}}
\newcommand{\isType}[1]{\ensuremath{#1 \; \mathsf{type}_0}}
\newcommand{\isComp}[4]{\ensuremath{#1\; \mathsf{ wit }\; #2 \in #3 \;\mathsf{ with }\; #4}}
\newcommand{\isTypeComp}[2]{\ensuremath{#2 \; \mathsf{type}}}
\newcommand{\isPot}[1]{\ensuremath{#1 \; \mathsf{pot}}}
\newcommand{\isPotTyped}[2]{\ensuremath{#1 \; \mathsf{pot} \; #2}}
\newcommand{\isCtx}[1]{\ensuremath{#1\; \mathsf{ctx} }}
\newcommand{\isSub}[2]{\ensuremath{#1 \in #2}}

\newcommand{\subst}[2]{[#1]#2}
\newcommand{\lsub}[3]{\left< #1/#2 \right>#3}

\newcommand{\zp}{\mathbb{0}}
\newcommand{\constp}[1]{\ensuremath{\mathbb{#1}}}
\newcommand{\toNat}[1]{\underline{#1}}
\newcommand{\toNum}[1]{\overline{#1}}
\newcommand{\pmin}{\mathtt{min}}
\newcommand{\sub}{\mathtt{sub}}
\newcommand{\plus}{\mathtt{plus}}
\newcommand{\mult}{\mathtt{mult}}


\newcommand{\fact}[2]{#1$^{\textbf {\color{blue}{#2}}}$}
\newcommand{\applyAss}[1]{assumption \textbf{\color{blue}{#1}}}
\begin{document}
 
% --------------------------------------------------------------
%                         Start here
% --------------------------------------------------------------
 
%\renewcommand{\qedsymbol}{\filledbox}
 
\title{}%replace X with the appropriate number
\author{ %replace with your name
} %if necessary, replace with your course title
 
%\maketitle

%\section{Language}

\begin{align*}
    \mathsf{Exp} \quad E &::= \lam{x}{A}{M} 
    \mid \suc{E}
    \mid \zero 
    \mid \susp{E}
    \mid \ccomp{E_1}{E}{a.E_2}
    \mid x
    \mid \nat
    \mid \pair{E_1}{E_2}\\
    &\mid \arrabt{A}{a.B}{a.P,a.b.Q}
    \mid \dprod{\isOf{x}{M}}{M'}\\
    \mathsf{Comp} \quad M &::= 
     v_1 \; v_2
    \mid \rec{v}{M_0}{x}{f}{M_1}
    \mid \seq{v}{x}{M}
    \mid \fst{v}
    \mid \snd{v}
    \mid \ret{v}\\
\end{align*}

We start with a simple language consisting of functions, natural numbers, and pairs. Later on we will add
various constructs to illustrate cost analysis.
\section{Semantics}
We start with the usual small-step operational semantics, adjusted so that consecutive computations are 
strung together by a sequencing operation.
\[
\fbox{$\step{M}{M'}$} \quad\quad \fbox{$\final{M}$}
\]

\begin{mathpar}

\inferrule{
}{
    \final{\ret{v}}
}

\inferrule{
}{
    \step{(\lam{x}{A}{M}) v}{[v/x]M}
}

\inferrule{
}{
    \step{\rec{\zero}{M_0}{x}{f}{M_1}}{M_0}
}

\inferrule{
}{
    \step{\rec{\suc{v}}{M_0}{x}{f}{M_1}}{\seq{\thunk{\rec{v}{M_0}{x}{f}{M_1}}}{f}{[v/x]M_1}}
}

\inferrule{
}{
    \step{\fst{\pair{v}{w}}}{\ret{v}}
}

\inferrule{
}{
    \step{\snd{\pair{v}{w}}}{\ret{w}}
}

\inferrule{
    \step{M_1}{M_1'}
}{
    \step{\seq{\thunk{M_1}}{x}{M_2}}{\seq{\thunk{M_1'}}{x}{M_2}}
}

\inferrule{
}{
    \step{\seq{\thunk{\ret{v}}}{x}{M_2}}{[v/x]M_2}
}
\end{mathpar}

\evalCost{M}{c}{v} means $\exists v. \stepIn{M}{c}{\ret{v}}$ and \eval{M}{v} means $\exists v.
\stepIn{M}{*}{\ret{v}}$

Define a translation from values to natural numbers: 
$\toNat{n} = \begin{cases} 0 \text{ if } n = \zero\\
1+\toNat{v} \text{ if } n = \suc{v} \end{cases}$



%\section{Ground Judgments}

\begin{center}
\begin{tabularx}{1.0\textwidth}{lXll}
    \toprule
 Judgment & English & Presupposition & Meaning\\ \midrule
 \isType{A_0} & $A_0$ is a canonical type & --- & what are canonical values of $A_0$? \\ \midrule
 \isTypeComp{\star}{A} & $A$ computes to a type  & --- & \eval{A}{A_0} and \isType{A_0}\\ \midrule
 \isVal{v}{A} & $v$ is a canonical value of type $A$ & \isTypeComp{\star}{A} & 
    \makecell[tl]{Given that \eval{A}{A_0},\\ $v$ is a canonical value of $A_0$} \\ \midrule
 \isComp{\varphi}{M}{A}{\varrho} & $M$ computes to a value of type $A$ & 
    \makecell[tl]{\isTypeComp{\star}{A},\\\isPot{\varphi},\\ \isPot{\varrho}}&
    \makecell[tl]{\evalCost{M}{c}{v}, \isVal{v}{B}, \eval{\varphi}{p},\\ \eval{[v/y]\varrho}{p'},\\ and $\toNat{p} \ge c + \toNat{p'}$}\\
 \isType{\nat} & $\nat$ is a canonical type & --- & 
    \makecell[tl]{\isVal{\zero}{\nat}\\ \isVal{\suc{v}}{\ret{\nat}} given that\\ \isVal{v}{\ret{\nat}}}\\ \midrule
\isPot{\varphi}& $\varphi$ computes a potential & --- & 
    \makecell[tl]{\isComp{p}{\varphi}{\ret{\nat}}{p'}\\ for some $p,p'$}\\
 \iffalse
 \isPotTyped{\varrho}{A} & $\varrho$ is a potential function for $A$ & \isTypeComp{\star}{A} &
    \makecell[cl]{$\varrho$ is $i.M$ and for all \isVal{a}{A},\\ \eval{[a/i]M}{n} and \isVal{n}{\nat}}\\
    \fi
    \bottomrule
\end{tabularx}
\end{center}

Let it be the case that
\begin{gather*}
\isTypeComp{\star}{A_1}\\
\isTypeComp{\star}{[a_1/x_1]A_2} \text{ given } \isVal{a_1}{A_1},\\
\dots\dots\\
\isTypeComp{\star}{[a_1/x_1,\dots,a_{n-1}/x_{n-1}]A_n}\\
\text{ given } \\ 
    \isVal{a_1}{A_1},\\
    \dots,\\
    \isVal{a_{n-1}}{[a_1/x_1,\dots,a_{n-2}/x_{n-2}]A_{n-1}}
\end{gather*}

And 

\[
\isVal{a_1}{A_1},\dots,\isVal{a_n}{[a_1/x_1,\dots,a_{n_1}/x_{n-1}]A_n}
\]
Then for $\mathcal{J} \in \{\isTypeComp{\star}{A},\; \isVal{v}{A},\; \isPot{\varphi}\}$, the shorthand
\[\isOf{x_1}{A_1},\dots,\isOf{x_n}{A_n} \rhd_n \mathcal{J}
\]
means
\[[a_1/x_1,\dots,a_n/x_n]\mathcal{J}\]

If moreover, it is the case that 
\begin{gather*}
    \isPot{[a_1/x_1,\dots,a_n/x_n]\varphi}\\
    \isTypeComp{\star}{[a_1/x_1,\dots,a_n/x_n]B}\\
    \isPot{[a_1/x_1,\dots,a_n/x_n,b/y]\varrho} \text{ for all } \isVal{b}{[a_1/x_1,\dots,a_n/x_n]B}\\
\end{gather*}

Then the notation
\[
\isOf{x_1}{A_1},\dots,\isOf{x_n}{A_n} / \varphi \rhd_n M \in \isOf{y}{B}/\varrho
\]
means that 
\begin{gather*}
\eval{[a_1/x_1,\dots,a_n/x_n]\varphi}{p}\\ 
\evalCost{[a_1/x_1,\dots,a_n/x_n]M}{c}{v} \text{ for some } \isVal{v}{[a_1/x_1,\dots,a_n/x_n]B}\\
\eval{[a_1/x_1,\dots,a_n/x_n,v/y]\varrho}{p'}
\end{gather*}
satisfying $\toNat{p} \ge c + \toNat{p'}$.



%\section{Types}

\begin{itemize}
    \item \isType{\dprod{\isOf{x}{A}}{B}}
    \begin{enumerate}
        \item \isTypeComp{\star}{A}
        \item \gammaToTypeCompFixed{\isOf{x}{A}}{1}{B}
    \end{enumerate}
    \item \isType{\comp{A}}
    \begin{enumerate}
        \item \isType{A}
    \end{enumerate}
    \item \isVal{\thunk{M}}{\comp{A}}, given:
    \begin{enumerate}
        \item \isComp{\varphi}{M}{\isOf{x}{A}}{\varrho} for some $\varphi, \varrho$
    \end{enumerate}
        \item \isType{\ccomp{\varphi}{\isOf{x}{A}}{\varrho}}
         \begin{enumerate}
        \item \isType{A}
        \item \isPot{\varphi}
        \item \gammaToPotFixed{\isOf{x}{A}}{1}{\varrho}
        \end{enumerate}
        \item \isVal{\thunk{M}}{\ccomp{\varphi}{\isOf{x}{A}}{\varrho}}
        \begin{enumerate}
            \item \isComp{\varphi}{M}{\isOf{x}{A}}{\varrho}
        \end{enumerate}

    \item \isType{\arr{\isOf{x}{A}}{\varphi}{\isOf{y}{B}}{\varrho}}
\begin{enumerate}
    \item \isTypeComp{\star}{A}
    \item \gammaToTypeCompFixed{\isOf{x}{A}}{1}{B}
    \item \gammaToPotFixed{\isOf{x}{A}}{1}{\varphi}
    \item \gammaToPotFixed{\isOf{x}{A},\isOf{y}{B}}{2}{\varrho}
\end{enumerate}

    \item $\lam{x}{A}{M}$ is a canonical element of \arr{\isOf{x}{A}}{\varphi}{\isOf{y}{B}}{\varrho}, given:
\[
    \gammaToCompFixed{\isOf{x}{A}}{\varphi}{1}{M}{\isOf{y}{B}}{\varrho}
    \]
By the previous remarks, this is the case iff:
given \isVal{a}{A}, it is the case that:
\begin{enumerate}
    \item \eval{[a/x]\varphi}{p}
    \item \evalCost{[a/x]M}{c}{v} for some \isVal{v}{[a/x]B}
    \item \eval{[a/x,v/y]\varrho}{p'}
    \item $\toNat{p} \ge c + \toNat{p'}$
\end{enumerate}

\end{itemize}



%
\section{Hypothetical Judgments}

\begin{center}
\begin{tabularx}{1.0\textwidth}{lXl}
    \toprule
 Judgment & Presupposition & Meaning\\ \midrule
  \rowcolor[gray]{0.7}
 \isCtx{\cdot} & --- & ---\\
  \rowcolor[gray]{0.7}
 \isCtx{\isOf{x}{A},\Gamma} & --- & \isTypeComp{\star}{A}, and for all \isVal{a}{A}, \isCtx{[a/x]\Gamma}\\
  \isSub{\cdot}{\cdot} & --- & --- \\
 \isSub{x\mapsto M,\gamma}{\isOf{x}{A},\Gamma} & --- & \isVal{M}{A} and \isSub{\gamma}{[M/x]\Gamma}\\
 \rowcolor[gray]{0.7}
 \gammaToTypeComp{\cdot}{\star}{B} & --- & \isTypeComp{\star}{B}\\
  \rowcolor[gray]{0.7}
 \gammaToTypeComp{\isOf{x}{A},\Gamma}{\star}{B} & \isCtx{\isOf{x}{A},\Gamma} & for all \isVal{a}{A}, $[a/x]\left(\gammaToTypeComp{\Gamma}{\star}{B}\right)$ \\ 
 \gammaToVal{\cdot}{v}{B} & --- & \isVal{v}{B} \\
 \gammaToVal{\isOf{x}{A},\Gamma}{v}{B} & \isCtx{\isOf{x}{A},\Gamma} & for all \isVal{a}{A}, 
    $[a/x]\left(\gammaToVal{\Gamma}{v}{B}\right)$\\
\rowcolor[gray]{0.7}
 \gammaToPot{\cdot}{\varphi} & --- & \isPot{\varphi} \\ 
 \rowcolor[gray]{0.7}
 \gammaToPot{\isOf{x}{A},\Gamma}{\varphi} & \isCtx{\isOf{x}{A},\Gamma} & for all \isVal{a}{A},                       $[a/x]\left(\gammaToPot{\Gamma}{\varphi}\right)$\\
 \gammaToComp{\cdot}{\varphi}{M}{\isOf{y}{B}}{\varrho} & \isTypeComp{\star}{B}, \isPot{\varphi}, and \gammaToPot{\isOf{y}{B}}{\varrho}
    & \makecell[tl]{\evalCost{M}{c}{v},\\ \isVal{v}{B},\\ \eval{\varphi}{p},\\ \eval{[v/y]\varrho}{p'},\\ and $\toNat{p} \ge c + \toNat{p'}$}\\
  \gammaToComp{\isOf{x}{A},\Gamma}{\varphi}{M}{\isOf{y}{B}}{\varrho} & 
  \makecell[tl]{\isCtx{\isOf{x}{A},\Gamma},\\
  \gammaToTypeComp{\isOf{x}{A},\Gamma}{\star}{B},\\
  \gammaToPot{\isOf{x}{A},\Gamma}{\varphi},\\ 
  \gammaToPot{\isOf{x}{A},\Gamma,\isOf{y}{B}}{\varrho}}
  &\makecell[tl]{for all \isVal{a}{A},\\ $[a/x]\left(\gammaToComp{\Gamma}{\varphi}{M}{\isOf{y}{B}}{\varrho}\right)$}\\
    \bottomrule
\end{tabularx}
\end{center}

%
\section{Lemmas}
Define:
\begin{gather*}
    \zp \triangleq \ret{\zero}\\
    \plus \triangleq \lam{x}{a}{\ret{\lam{y}{b}{\rec{x}{\ret{y}}{\_}{f}{\ret{\suc{f}}}}}}\\
    \mult \triangleq \lam{x}{a}{\ret{\lam{y}{b}{\rec{x}{\ret{y}}{\_}{f}{\plus\; y\; f}}}}\\
    \sub \triangleq \lam{x}{a}{\rec{x}{\ret{\lam{y}{b}{\ret{\zero}}}}{z}{f}{\ret{\lam{y}{b}{\rec{y}{\ret{\suc{p}}}{y'}{\_}{f y'}}}}}\\
    \pmin \triangleq \lam{x}{a}{\ret{\lam{y}{b}{\seq{\sub\; x\; y}{d}{\rec{d}{\ret{x}}{\_}{\_}{\ret{y}}}}}}
\end{gather*}

\begin{lemma}\textnormal{(basis)}\label{lemma:basic}
\isPot{\zp}, and for all \isVal{x}{\ret{\nat}} and \isVal{y}{\ret{\nat}}, \isPot{\plus\; x\; y}, 
 \isPot{\mult\; x\; y}, \isPot{\sub\; x\; y} and \isPot{\pmin\; x\; y}.
\end{lemma}

Notation:
\begin{gather*}
\bind{M_1}{x}{M_2} \triangleq \seq{\thunk{M_1}}{x}{M_2}\\
f\;v_1\;\dots\;v_n \triangleq \bind{f\;v_1}{r_1}{\bind{r_1\;v_2}{r_2}{\dots,\bind{r_{n-2}\;v_{n-1}}{r_{n-1}}{r_{n-1}\;v_n}}}\\
\varphi^+ \triangleq \bind{\varphi}{v}{\ret{\suc{v}}}\\
\end{gather*}

\begin{lemma}\textnormal{(Recursor)}\label{lemma:recursor}
Given 
\begin{gather}
\gammaToPot{\isOf{x}{\ret{\nat}}}{\varphi_0},\\
\gammaToPot{\isOf{x}{\ret{\nat}},\isOf{z}{\ret{\nat}}, \isOf{g}{\comp{\nat}}}{\varphi_1},\\
\gammaToTypeComp{\isOf{x}{\ret{\nat}}}{\star}{B},\\
\gammaToPot{\isOf{x}{\ret{\nat}},\isOf{y}{B}}{\varrho_0},\\
\gammaToPot{\isOf{x}{\ret{\nat}}, \isOf{z}{\ret{\nat}}, \isOf{g}{\comp{\nat}}, \isOf{y}{B}}{\varrho_1},\\
\gammaToComp{\cdot}{[\zero/x]\varphi_0}{M_0}{\isOf{y}{[\zero/x]B}}{[\zero/x]\varrho_0},\\
\gammaToComp{\isOf{z}{\ret{\nat}}}{[\suc{z}/x]\lsub{\rec{z}{\varphi_0}{z}{g}{\varphi_1}}{g}{\varphi_1}}
{\lsub{\rec{z}{M_0}{z}{f}{M_1}}{f}{M_1}\\}{\isOf{y}{[\suc{z}/x]B}}{[\suc{z}/x]\lsub{\rec{z}{\varrho_0}{z}{g}{\varrho_1}}{g}{\varrho_1}}
\end{gather}
Then 
\gammaToComp{\isOf{x}{\ret{\nat}}}{\rec{x}{\varphi_0}{z}{g}{\varphi_1}^+}{\rec{x}{M_0}{z}{f}{M_1}}{\isOf{y}{B}}{
\rec{x}{\varrho_0}{z}{g}{\varrho_1}}
\end{lemma}

\begin{proof}
Presuppositions: assumed by given. Let \isVal{n}{\ret{\nat}}. STS \gammaToComp{\cdot}{[n/x]\rec{x}{\varphi_0}{z}{g}{\varphi_1}^+}{\rec{n}{M_0}{z}{f}{M_1}}{\isOf{y}{[n/x]B}}{
[n/x]\rec{x}{\varrho_0}{z}{g}{\varrho_1}}
. Presuppositions: 
\begin{enumerate}
    \item \isTypeComp{\star}{[n/x]B}. By instantiating (2) with $n$ for $x$. 
    \item \isPot{[n/x]\rec{x}{\varphi_0}{z}{g}{\varphi_1}^+}. By instantiating (1) with $n$ for $x$. 
    \item \gammaToPot{\isOf{y}{[n/x]B}}{[n/x]\rec{x}{\varrho_0}{z}{g}{\varrho_1}}
\end{enumerate}
Next, induction on $n$: 
\begin{itemize}
    \item $n$ is $\zero$:\\
    Then \step{\rec{\zero}{M_0}{z}{f}{M_1}}{M_0}. By (6), we know \evalCost{M_0}{c}{u} for some \isVal{u}{[\zero/x]B}.
    By head expansion, \evalCost{\rec{\zero}{M_0}{z}{f}{M_1}}{c+1}{u}. 
    Furthermore, we know \eval{[\zero/x]\varphi_0}{p}, \eval{[\zero/x,u/y]\varrho_0}{p'}, and $\toNat{p} \ge c + \toNat{p'}$. Furthermore, we know that 
    \begin{align*}
        &[\zero/x]\rec{x}{\varphi_0}{z}{g}{\varphi_1}^+\\ 
        \mapsto & [\zero/x]\varphi_0^+\\
        \mapsto^* &\seq{\thunk{\ret{p}}}{v}{\ret{\suc{v}}}\\
        \mapsto& \ret{\suc{p}}\\
    \end{align*}. Since $\toNat{\suc{p}} = \toNat{p} + 1$, STS that\\
    \eval{[\zero/x,u/y]\rec{x}{\varrho_0}{z}{g}{\varrho_1}}{p''}
    and that $p' \ge p''$. Evidently this is the case since
    \step{[\zero/x,u/y]\rec{x}{\varrho_0}{z}{g}{\varrho_1}}{\eval{[\zero/x,u/y]\varrho_0}{p'}}.
    \item $n$ is $\suc{n'}$ for some $n'$:\\
    Then \step{\rec{\suc{n'}}{M_0}{z}{f}{M_1}}{\lsub{\rec{n'}{M_0}{z}{f}{M_1}}{f}{[n'/z]M_1}}. 
    Instantiating (7) with $[n'/z]$, we know that \evalCost{\lsub{\rec{n'}{M_0}{z}{f}{M_1}}{f}{[n'/z]M_1}}{c}{u}, 
    and by head expansion, \evalCost{\rec{\suc{n'}}{M_0}{z}{f}{M_1}}{c+1}{u}. Furthermore, we know
    \begin{align*}
        &\rec{\suc{n'}}{[\suc{n'}/x]\varphi_0}{z}{g}{[\suc{n'}/x]\varphi_1}^+\\
        \mapsto& [\suc{z}/x]\lsub{\rec{z}{\varphi_0}{z}{g}{\varphi_1}}{g}{\varphi_1}^+\\
        \mapsto^*& \ret{p}^+\\
        \mapsto& \ret{\suc{p}}
    \end{align*} and that 
    \begin{align*}
        &\rec{\suc{n'}}{[\suc{n'}/x]\varrho_0}{z}{g}{[\suc{n'}/x]\varrho_1}\\
        \mapsto &[\suc{z}/x]\lsub{\rec{z}{\varrho_0}{z}{g}{\varrho_1}}{g}{\varrho_1}\\
        \mapsto^* & \ret{p'}
    \end{align*}
    such that $\toNat{p} \ge c + \toNat{p'}$, which suffices for 
    $\toNat{\suc{p}} \ge c+1 + \toNat{p'}$.
\end{itemize}
\end{proof}

\newpage
\begin{lemma}\textnormal{(Recursor')}\label{lemma:recursor'}
Given 
\begin{gather}
\gammaToPot{\cdot}{\varphi_0},\\
\gammaToPot{\isOf{z}{\ret{\nat}}}{\varphi_1},\\
\gammaToTypeComp{\isOf{x}{\ret{\nat}}}{\star}{B},\\
\gammaToPot{\cdot}{\varrho_0},\\
\gammaToPot{\isOf{z}{\ret{\nat}}}{\varrho_1},\\
\gammaToComp{\cdot}{\varphi_0}{M_0}{\isOf{y}{[\zero/x]B}}{\varrho_0},\\
\gammaToComp{\isOf{z}{\ret{\nat}},\isOf{f}{\ret{\ccomp{\zp}{\isOf{w}{[z/x]B}}{\zp}}}}{\varphi_1}
{M_1}{\isOf{y}{[\suc{z}/x]B}}{\varrho_1},\\
\gammaToVal{\cdot}{n}{\ret{\nat}}
\end{gather}
Then 
\gammaToComp{\cdot}{\rec{n}{\varphi_0^+}{z}{g}{\bind{\varphi_1}{\phi_1}{\plus\; g\; \phi_1}^+}}{\rec{n}{M_0}{z}{f}{M_1}}{\isOf{y}{[n/x]B}}{
\rec{n}{\varrho_0}{z}{g}{\bind{\varrho_1}{\rho_1}{\plus\;\rho_1\; g}}}
\end{lemma}

\begin{proof}
Presuppositions: 
\begin{enumerate}
    \item \isTypeComp{\star}{[n/x]B}. By instantiating (11) with $n$ for $x$. 
    \item \isPot{\rec{n}{\varphi_0^+}{z}{g}{\bind{\varphi_1}{\phi_1}{\plus\;\phi_1\;g}^+}}. TODO
    \item \isPot{\rec{n}{\varrho_0}{z}{g}{\bind{\varrho_1}{\rho_1}{\plus\;\rho_1\; g}}}. TODO
\end{enumerate}
Next, induction on $n$: 
\begin{itemize}
    \item $n$ is $\zero$:\\
    Then \step{\rec{\zero}{M_0}{z}{f}{M_1}}{M_0}. By (14), we know \evalCost{M_0}{c}{u} 
    for some \isVal{u}{[\zero/x]B}.
    By head expansion, \evalCost{\rec{\zero}{M_0}{z}{f}{M_1}}{c+1}{u}. 
    Furthermore, we know \eval{\varphi_0}{p}, \eval{[u/y]\varrho_0}{p'}, and $\toNat{p} \ge c + \toNat{p'}$. Furthermore, we know that 
    \begin{align*}
        &\rec{\zero}{\varphi_0^+}{z}{g}{\bind{\varphi_1}{\phi_1}{\plus\;\phi_1\;g}^+}\\ 
        \mapsto & \varphi_0^+\\
        \mapsto^* &\seq{\thunk{\ret{p}}}{v}{\ret{\suc{v}}}\\
        \mapsto& \ret{\suc{p}}
    \end{align*}
    Since $\toNat{\suc{p}} = \toNat{p} + 1$, STS that
    \eval{[u/y]\rec{\zero}{\varrho_0}{z}{g}{\bind{\varrho_1}{\rho_1}{\plus\;\rho_1\;g}}}{p''}
    and that $p' \ge p''$. Evidently this is the case since
    \step{[u/y]\rec{\zero}{\varrho_0}{z}{g}{\bind{\varrho_1}{\rho_1}{\plus\; \rho_1\; g}}}{\eval{[u/y]\varrho_0}{p'}}.
    \item $n$ is $\suc{n'}$ for some $n'$:\\
    Then \step{\rec{\suc{n'}}{M_0}{z}{f}{M_1}}{\lsub{\rec{n'}{M_0}{z}{f}{M_1}}{f}{[n'/p]M_1}}. 
    By induction, we know that 
    \gammaToComp{\cdot}{\rec{n'}{\varphi_0^+}{z}{g}{\bind{\varphi_1}{\phi_1}{\plus\;\phi_1\;g}^+}}
    {\rec{n'}{M_0}{z}{f}{M_1}}{\isOf{y}{[n'/x]B}}
    {\rec{n'}{\varrho_0}{z}{g}{\plus\;g\;\varrho_1}}. This implies that 
    \begin{gather}
    \evalCost{\rec{n'}{M_0}{z}{f}{M_1}}{c}{u}, \\
    \eval{\rec{n'}{\varphi_0^+}{z}{g}{\bind{\varphi_1}{\phi_1}{\plus\;\phi_1\;g}^+}}{p},\\
    \eval{\rec{n'}{\varrho_0}{z}{g}{\bind{\varrho_1}{\rho_1}{\plus\;\rho_1\;g}}}{p'}, \text{ and that}\\
    \toNat{p} \ge c + \toNat{p'}
    \end{gather}
    Instantiating (15) with $[n'/p, \thunk{\ret{u}}/f]$, we know that
    \evalCost{\lsub{\ret{u}}{f}{[n'/p]M_1}}{c'}{u'},
    and by head expansion, \evalCost{\rec{\suc{n'}}{M_0}{z}{f}{M_1}}{c+1+c'}{u'}. Furthermore, we know
    \begin{align}
        &\eval{[n'/z]\varphi_1}{q},\\
        &\eval{[n'/z]\varrho_1}{q'}, \text{ and that }\\
        &q \ge c' + q'
    \end{align}
    Expanding our current potential, we have:
    \begin{align*}
        &\rec{\suc{n'}}{\varphi_0^+}{z}{g}{\bind{\varphi_1}{\phi_1}{\plus\;\phi_1\;g}^+}\\
        \mapsto& \lsub{\rec{n'}{\varphi_0^+}{z}{g}{\bind{\varphi_1}{\phi_1}{\plus\;\phi_1\;g}^+}}{g}{[n'/p]\bind{\varphi_1}{\phi_1}{\plus\;\phi_1\;g}^+}\\
        \mapsto^*& \lsub{\ret{p}}{g}{\bind{[n'/z]\varphi_1}{\phi_1}{\plus\;\phi_1\;g}^+}\\
        \mapsto& \bind{[n'/z]\varphi_1}{\phi_1}{\plus\;\phi_1\;p}^+\\
        \mapsto^*& \bind{\ret{q}}{\phi_1}{\plus\;\phi_1\;p}^+\\
        \mapsto& \plus\; q \; p ^+
    \end{align*} and that 
    \begin{align*}
        &\rec{\suc{n'}}{\varrho_0}{z}{g}{\bind{\varrho_1}{\rho_1}{\plus\;\rho_1\; g}}\\
        \mapsto& \lsub{\rec{n'}{\varrho_0}{z}{g}{\bind{\varrho_1}{\rho_1}{\plus\;\rho_1\; g}}}{g}{[n'/z]\bind{\varrho_1}{\rho_1}{\plus\;\rho_1\; g}}\\
        \mapsto^*& \lsub{\ret{p'}}{g}{\bind{[n'/z]\varrho_1}{\rho_1}{\plus\;\rho_1\; g}}\\
        \mapsto& \bind{[n'/z]\varrho_1}{\rho_1}{\plus\;\rho_1\; p'}\\
        \mapsto^*& \bind{\ret{q'}}{\rho_1}{\plus\;\rho_1\; p'}\\
        \mapsto& \plus\;q'\; p'
    \end{align*}
    Now it suffices to show $\toNat{\plus\; q \; p ^+} \ge c + 1 + c' + \toNat{\plus\;q'\; p'}$,
    which holds by (20) and (23).
\end{itemize}
\end{proof}

\begin{lemma}\textnormal{(Return)}\label{lemma:ret}
If \gammaToVal{\Gamma}{v}{B}, then \gammaToComp{\Gamma}{\zp}{\ret{v}}{\isOf{y}{B}}{\zp}
\end{lemma}

\begin{proof}
Induction on the length of $\Gamma$.
\begin{itemize}
    \item $\Gamma = \cdot$\\
    Suppose \gammaToVal{\cdot}{v}{B}. By definition, this means \isVal{v}{B}. STS 
    \gammaToComp{\cdot}{\zp}{\ret{v}}{\isOf{y}{B}}{\zp}. We know that 
    $\evalCost{\ret{v}}{0}{v}$, $\eval{\zp}{\zero}$, $\eval{[v/y]\zp}{\zero}$, and 
    $\toNat{\zero} \ge 0 + \toNat{\zero}$.
    \item $\Gamma = \isOf{x}{A},\Gamma'$\\
    Suppose \gammaToVal{\isOf{x}{A},\Gamma'}{v}{B}. This means 
    \fact{$[a/x]\left(\gammaToVal{\Gamma}{v}{B}\right)$ for all \isVal{a}{A}}{1}. 
    NTS \gammaToComp{\isOf{x}{A},\Gamma'}{\zp}{\ret{v}}{\isOf{y}{B}}{\zp}. Let
    \isVal{a}{A}. STS $[a/x]\left(\gammaToComp{\Gamma}{\varphi}{\ret{v}}{\isOf{y}{B}}{\varrho}\right)$, which 
    holds by IH when instantiated with \applyAss{1}.
\end{itemize}
\end{proof}

\begin{lemma}\textnormal{(Context)}\label{lemma:ctx}
If \isCtx{\Gamma[\isOf{x}{A}]\Gamma'}, then for all \isSub{\gamma}{\Gamma}, \isTypeComp{\star}{\subst{\gamma}{A}}.
\end{lemma}

\begin{proof}
Induction on length of $\Gamma$.
\begin{itemize}
    \item $\Gamma = \cdot$\\
    Suppose \isCtx{[\isOf{x}{A}]\Gamma'} and \isSub{\gamma}{\cdot}. The former implies \isTypeComp{\star}{A},
    together with the latter which implies $\gamma = \cdot$, we have \isTypeComp{\star}{A\gamma}.
    \item $\Gamma = \isOf{x'}{A'},\Gamma''$\\
    \isCtx{\isOf{x'}{A'},\Gamma''[\isOf{x}{A}]\Gamma'} and \isSub{\gamma}{\isOf{x'}{A'},\Gamma''}. 
    Then $\gamma$ must 
    be $x' \mapsto M, \gamma'$ for some \isVal{M}{A'} and \isSub{\gamma'}{[M/x']\Gamma''}. 
    Further, \isCtx{[M/x'](\Gamma''[\isOf{x}{A}]\Gamma')}. By IH,
    \isTypeComp{\star}{[\gamma'][M/x']A}. This suffices since $[\gamma]A = [\gamma'][M/x']A$.
\end{itemize}
\end{proof}

\begin{lemma}\textnormal{(Abstraction)}\label{lemma:abs}
If \gammaToComp{\Gamma[\isOf{x}{A}]}{\varphi}{M}{\isOf{y}{B}}{\varrho}, then 
\gammaToVal{\Gamma}{\lam{x}{A}{M}}{\ret{\arr{\isOf{x}{A}}{\varphi}{\isOf{y}{B}}{\varrho}}}
\end{lemma}

\begin{proof}
Induction on length of $\Gamma$.
\begin{itemize}
    \item $\Gamma = \cdot$\\
    Suppose \gammaToComp{\isOf{x}{A}}{\varphi}{M}{\isOf{y}{B}}{\varrho}. This means
     \fact{\gammaToComp{\cdot}{[a/x]\varphi}{[a/x]M}{\isOf{y}{[a/x]B}}{[a/x]\varrho} for all \isVal{a}{A}}{1}.
     NTS \gammaToVal{\cdot}{\lam{x}{A}{M}}{\ret{\arr{A}{\varphi}{\isOf{y}{B}}{\varrho}}}. STS
     \isVal{\lam{x}{A}{M}}{\ret{\arr{A}{\varphi}{\isOf{y}{B}}{\varrho}}}. First, NTS 
     \isTypeComp{\star}{\arr{\isOf{x}{A}}{\varphi}{\isOf{y}{B}}{\varrho}}:
     \begin{enumerate}
         \item \isTypeComp{\star}{A} by presupposition with Lemma~\ref{lemma:ctx}
         \item NTS \gammaToTypeCompFixed{\isOf{x}{A}}{1}{B}. Let \isVal{a}{A}. STS 
         \isTypeComp{\star}{[a/x]B}. By presupposition, \gammaToTypeComp{\isOf{x}{A},\cdot}{\star}{B}. This implies
         \gammaToTypeComp{\cdot}{\star}{[a/x]B}, which means \isTypeComp{\star}{[a/x]B}.
         \item NTS \gammaToPotFixed{\isOf{x}{A}}{1}{\varphi}. Let \isVal{a}{A}. STS \isPot{[a/x]\varphi}. 
         By presupposition, \gammaToPot{\isOf{x}{A},\cdot}{\varphi}.
         This implies \gammaToPot{\cdot}{[a/x]\varphi}, which means \isPot{[a/x]\varphi}.
         \item NTS \gammaToPotFixed{\isOf{x}{A},\isOf{y}{B}}{1}{\varrho}. 
         Let \isVal{a}{A} and \isVal{b}{[a/x]B}. STS \isPot{[a/x,b/y]\varrho}. By presupposition, 
         \gammaToPot{\isOf{x}{A},\cdot[\isOf{y}{B}]}{\varrho}. This implies 
         \gammaToPot{\cdot[\isOf{y}{[a/x]B}]}{[a/x]\varrho}, which means \gammaToPot{\isOf{y}{[a/x]B}}{[a/x]\varrho}.
         Next, this implies \gammaToPot{\cdot}{[b/y][a/x]\varrho}, which means \isPot{[b/y][a/x]\varrho}. 
     \end{enumerate}
     Next, NTS \gammaToCompFixed{\isOf{x}{A}}{\varphi}{1}{M}{\isOf{y}{B}}{\varrho}.
     Let \isVal{a}{A}. NTS 
     \begin{enumerate}
         \item \eval{[a/x]\varphi}{p}. By presupposition on \applyAss{1}, \gammaToPot{\cdot}{[a/x]\varphi}. This 
         means \isPot{[a/x]\varphi}, which ensures a return value. 
         \item \evalCost{[a/x]M}{c}{v} for some \isVal{v}{[a/x]B}. By definition of \applyAss{1}, 
         \evalCost{[a/x]M}{c}{v} and \isVal{v}{[a/x]B}
         \item \eval{[a/x,v/y]\varrho}{p'}. By presupposition on \applyAss{1}, 
         \gammaToPot{\isOf{y}{[a/x]B}}{[a/x]\varrho}. This 
         means \isPot{[a/x,b/y]\varrho} for all \isVal{b}{B}. Thus take $b$ to be $v$, and we are guaranteed a $p'$.
         \item $\toNat{p} \ge c + \toNat{p'}$. Follows from definition of \applyAss{1}.
     \end{enumerate}
    \item $\Gamma = \isOf{x'}{A'},\Gamma'$\\
    Suppose \gammaToComp{\isOf{x'}{A'},\Gamma'[\isOf{x}{A}]}{\varphi}{M}{\isOf{y}{B}}{\varrho}. This means
    \fact{$[a'/x']\left(\gammaToComp{\Gamma'[\isOf{x}{A}]}{\varphi}{M}{\isOf{y}{B}}{\varrho}\right)$ for all \isVal{a'}{A'}}{2}. 
    NTS \gammaToVal{\isOf{x'}{A'},\Gamma'}{\lam{x}{A}{M}}{\ret{\arr{\isOf{x}{A}}{\varphi}{\isOf{y}{B}}{\varrho}}}. 
    This means 
    $[a'/x']\left(\gammaToVal{\Gamma'}{\lam{x}{A}{M}}{\ret{\arr{\isOf{x}{A}}{\varphi}{\isOf{y}{B}}{\varrho}}}\right)$
    for all \isVal{a'}{A'}. Suppose \isVal{a'}{A'}. Then the result holds by \applyAss{2} on IH.
\end{itemize}
\end{proof}

\begin{lemma}\textnormal{(Application)}
If \isVal{f}{\ret{\arr{\isOf{x}{A}}{\varphi}{\isOf{y}{B}}{\varrho}}} and \isVal{a}{A}, then 
\gammaToComp{\cdot}{[a/x]\varphi^+}{f\;v}{\isOf{y}{[a/x]B}}{[a/x]\varrho}.
\end{lemma}

\begin{proof}
Given \isVal{f}{\ret{\arr{\isOf{x}{A}}{\varphi}{\isOf{y}{B}}{\varrho}}} and \isVal{a}{A}, 
we know that $f = \lam{x}{a}{M}$ s.t. 
\[
    \gammaToCompFixed{\isOf{x}{A}}{\varphi}{1}{M}{\isOf{y}{B}}{\varrho}
\]
Instantiating this with $[a/x]$, we know
\begin{enumerate}
    \item \eval{[a/x]\varphi}{p}
    \item \evalCost{[a/x]M}{c}{v} for some \isVal{v}{[a/x]B}
    \item \eval{[a/x,v/y]\varrho}{p'}
    \item $\toNat{p} \ge c + \toNat{p'}$
\end{enumerate}

Since \step{\lam{x}{a}{M}\; a}{[a/x]M}, it suffices to show 
$\step{[a/x]\varphi^+}{q}$ s.t $\toNat{q} \ge \toNat{p} + 1$, which 
holds since $\eval{[a/x]\varphi^+}{\suc{p}}$. 

\end{proof}

\begin{lemma}\textnormal{(Sequence)}\label{lemma:seq}
If \gammaToComp{\cdot}{\varphi}{M_1}{\isOf{x}{A}}{\varrho} and 
\gammaToComp{\isOf{x}{A}}{\varrho}{M_2}{\isOf{y}{B}}{\varsigma}, then 
\gammaToComp{\isOf{x}{A}}{\varphi^+}{\seq{\thunk{M_1}}{x}{M_2}}{\isOf{y}{B}}{\varsigma}.
\end{lemma}

\begin{proof}

\end{proof}

\begin{corollary}\textnormal{(Potential Sequence)}\label{cor:potseq}
If \isPot{\varphi} and
\gammaToPot{\isOf{x}{\ret{\nat}}}{\varrho}, then 
\isPot{\seq{\thunk{\varphi}}{x}{\varrho}}.
\end{corollary}

\begin{lemma}\textnormal{(Pair)}\label{lemma:pair}
If \gammaToVal{\cdot}{v}{A}, \gammaToTypeComp{\isOf{x}{A}}{\star}{B}, and \gammaToVal{\cdot}{w}{[v/x]B},
then \gammaToVal{\cdot}{\pair{v}{w}}{\ret{\dprod{\isOf{x}{A}}{B}}}. 
\end{lemma}



%
\section{Arithmetic}

\begin{lemma}(Plus)
\isVal{\plus}
{\ret{\arr{\isOf{x}{\ret{\nat}}}{\zp}{\ret{\arr{\ret{\nat}}{\varphi}{\ret{\nat}}{\varrho}}}{\zp}}},
where 
\begin{gather*}
    \varphi_0 = \zp\\
    \varphi_1 = \ret{\suc{\zero}}\\
    \varphi= \rec{x}{\varphi_0^+}{z}{g}{\bind{\varphi_1}{\phi_1}{\plus\;\phi_1\;g}^+}\\
    \varrho_0 = \zp\\
    \varrho_1 = \zp\\
    \varrho= \rec{x}{\varrho_0}{z}{g}{\bind{\varrho_1}{\rho_1}{\plus\;\rho_1\;g}}\\
\end{gather*}
\end{lemma}

\begin{proof}
By definition, it suffices to show\\
\gammaToCompFixed{\isOf{x}{\ret{\nat}}}{\zp}{1}{\ret{\lam{y}{b}{\rec{x}{\ret{y}}{\_}{f}{\ret{\suc{f}}}}}}{\ret{\arr{\ret{\nat}}{\varphi}{\ret{\nat}}{\varrho}}}{\zp}\\
Let \isVal{n}{\ret{\nat}}. 
Then by Lemma~\ref{lemma:ret}, it suffices to show
\[
\isVal{\lam{y}{b}{\rec{n}{\ret{y}}{\_}{f}{\ret{\suc{f}}}}}{\ret{\arr{\ret{\nat}}{[n/x]\varphi}{\ret{\nat}}{\varrho}}}\]

By definition, we need to show 
\[
\gammaToCompFixed{\isOf{y}{\ret{\nat}}}{[n/x]\varphi}{1}{\rec{n}{\ret{y}}{\_}{f}{\ret{\suc{f}}}}{\ret{\nat}}{\varrho}
\]

Let \isVal{m}{\ret{\nat}} and unfold with $[m/y]$.  
By Lemma~\ref{lemma:recursor}, it suffices to show
\begin{enumerate}
    \item \gammaToComp{\cdot}{\zp}{\ret{m}}{\isOf{y}{\ret{\nat}}}{\zp}: follows from Lemma~\ref{lemma:ret}.
    \item \gammaToComp{\isOf{z}{\ret{\nat}},\isOf{f}{\ccomp{\zp}{\isOf{w}{\ret{\nat}}}{\zp}} }{\varphi_1}{\ret{\suc{f}}}{\ret{\nat}}{\varrho_1}.\\
    Let \isVal{z}{\ret{\nat}}. STS 
    \gammaToComp{\isOf{f}{\ccomp{\zp}{\isOf{w}{\ret{\nat}}}{\zp}} }{\varphi_1}{\ret{\suc{f}}}{\ret{\nat}}{\varrho_1}.
    Let \isVal{f}{\ccomp{\zp}{\isOf{w}{\ret{\nat}}}{\zp}}. 
    Then $f$ is $\thunk{F}$ for some $F$ s.t. \isComp{\zp}{F}{\isOf{w}{\ret{\nat}}}{\zp}. 
    This implies that $\evalCost{F}{0}{u}$, and hence $F = \ret{u}$. 
         Thus,
        \begin{align*}
            &\lsub{F}{f}{[\zero/z]\ret{\suc{f}}} = \lsub{\ret{u}}{f}{\ret{\suc{f}}} \mapsto \ret{\suc{u}}
        \end{align*} 
        Further, 
        \begin{align*}
           &\eval{\varphi_1 = \ret{\suc{\zero}}}{\suc{\zero}}
        \end{align*}
        Similarly, \eval{\varrho_1}{\zero}.
        The result holds since $\toNat{\suc{\zero}} \ge 1 + \toNat{\zero}$.
\end{enumerate}

Simplified recurrence for $\plus$: $\Phi(\plus)(x) = 2x + 1$.
\end{proof}

Recall $\mult \triangleq \lam{x}{a}{\ret{\lam{y}{b}{\rec{x}{\ret{\zero}}{\_}{f}{\plus\; y\; f}}}}
$. 
\begin{lemma}(Mult)\label{lemma:mult}
\isVal{\mult}{\ret{\arr{\isOf{x}{\ret{\nat}}}{\zp}{\ret{\arr{\isOf{y}{\ret{\nat}}}{\varphi}{\ret{\nat}}{\varrho}}}{\zp}}},
where 
\begin{gather*}
    \varphi_0 = \zp\\
    \varphi_1 = \bind{\mult\;\toNum{2}\;y}{r}{\plus\;r\;\toNum{5}}\\
    \varphi= \rec{x}{\varphi_0^+}{z}{g}{\bind{\varphi_1}{\phi_1}{\plus\;\phi_1\;g}^+}\\
    \varrho_0 = \zp\\
    \varrho_1 = \zp\\
    \varrho= \rec{x}{\varrho_0}{z}{g}{\bind{\varrho_1}{\rho_1}{\plus\;\rho_1\;g}}\\
\end{gather*}
\end{lemma}

\begin{proof}
It suffices to show 

\gammaToCompFixed{\isOf{x}{\ret{\nat}}}{\zp}{1}{\ret{\lam{y}{b}{\rec{x}{\ret{\zero}}{\_}{f}{\plus\; y\; f}}}}{\ret{\arr{\isOf{y}{\ret{\nat}}}{\varphi}{\ret{\nat}}{\varrho}}}{\zp}

Let \isVal{n}{\ret{\nat}}. Instantiate with $[n/x]$, by Lemma~\ref{lemma:ret}, STS
\isVal{\lam{y}{b}{\rec{n}{\ret{\zero}}{\_}{f}{\plus\; y\; f}}}{\ret{\arr{\isOf{y}{\ret{\nat}}}{\varphi}{\ret{\nat}}{\varrho}}}. Again,
NTS
\gammaToCompFixed{\isOf{y}{\ret{\nat}}}{\varphi}{1}{\rec{n}{\ret{\zero}}{\_}{f}{\plus\; y\; f}}{\ret{\nat}}{\varrho}. Let \isVal{m}{\ret{\nat}}. Instantiating with $[m/y]$, STS

\[
\gammaToComp{\cdot}{\varphi}{\rec{n}{\ret{\zero}}{\_}{f}{\plus\; m\; f}}{\ret{\nat}}{\varrho}
\]
By Lemma~\ref{lemma:recursor'}, STS
\begin{enumerate}
    \item \gammaToComp{\cdot}{\zp}{\ret{\zero}}{\isOf{y}{\ret{\nat}}}{\zp}: follows from
    Lemma~\ref{lemma:ret}
    \item \gammaToComp{\isOf{z}{\ret{\nat}},\isOf{f}{\ccomp{\zp}{\isOf{w}{\ret{\nat}}}{\zp}} }{\varphi_1}{\plus\; y\; f}{\ret{\nat}}{\varrho_1}.\\
    Let \isVal{z}{\ret{\nat}} and \isVal{f}{\ccomp{\zp}{\isOf{w}{\ret{\nat}}}{\zp}}. 
    Then $f = \thunk{F}$ s.t. \isComp{\zp}{F}{\isOf{w}{\ret{\nat}}}{\zp}. Hence,
    $F = \ret{u}$ for some \isVal{u}{\ret{\nat}}. Thus,
    \begin{align*}
    &\lsub{\ret{u}}{f}{\plus\; y\; f}\\
    \mapsto& \plus\; y\; u = \bind{\plus\; y}{r}{r\; u}\\
    \mapsto& \bind{\lam{y'}{b}{\rec{y}{\ret{y'}}{\_}{f}{\ret{\suc{f}}}}}{r}{r\; u}\\
    \mapsto& \lam{y'}{b}{\rec{y}{\ret{y'}}{\_}{f}{\ret{\suc{f}}}}\; u\\
    \mapsto& \rec{y}{\ret{u}}{\_}{f}{\ret{\suc{f}}}\\
    \mapsto^{2\toNat{y}+1}& \ret{v}
    \end{align*}
    Hence \evalCost{\lsub{\ret{u}}{f}{\plus\; y\; f}}{2\toNat{y}+5}{v}.
    Furthermore, \eval{\varphi_1 = \bind{\mult\;\toNum{2}\;y}{r}{\plus\;r\;\toNum{5}}}{\toNum{2\toNat{y}+5}},
    and \eval{\varrho_1}{\zero}. The result follows since 
    $\toNat{\toNum{2\toNat{y}+5}} \ge 2\toNat{y}+5 + \toNat{\zero}$.
\end{enumerate}

Simplified recurrence for $\mult$: $\Phi(\mult)(x)(y) = x^2y + (6-y)x + 1$.



\end{proof}



\newcommand{\leaf}{\ensuremath{\mathtt{leaf}}}
\newcommand{\single}[1]{\ensuremath{\mathtt{single}(#1)}}
\newcommand{\ttwo}[2]{\ensuremath{\mathtt{t2}(#1,#2)}}
\newcommand{\tthree}[3]{\ensuremath{\mathtt{t3}(#1,#2,#3)}}
\newcommand{\ttrec}[8]{\mathtt{23TREC}(#1)(#2 \mid #3.#4 \mid #5.#6 \mid #7.#8)}
\newcommand{\tttree}[3]{\ensuremath{\mathtt{23tree}(#1,#2,#3)}}
\newcommand{\word}{\mathtt{word}}
\newcommand{\mw}[1]{\mathtt{mw}(#1)}
\newcommand{\wrec}[4]{\mathtt{WREC}(#1)(#2; #3.#4)}
\newcommand{\join}{\mathsf{join}}
\newcommand{\cmp}{\mathsf{cmp}}
\newcommand{\ifnat}[3]{\mathsf{if}(#1;#2;#3)}
\newcommand{\const}[1]{\lam{\_}{\_}{#1}}
\newcommand{\eq}{\mathsf{eq}}
\newcommand{\macro}{@\!\!=}
\newcommand{\wtonat}[1]{\mathsf{nat}(#1)}
\section{Trees}

Augment the language with the following values and commands:

\begin{align*}
    \mathsf{Node} \quad n &::= 
         \leaf
    \mid \single{v}
    \mid \ttwo{v_l}{v_r}
    \mid \tthree{v_l}{v_m}{v_r}\\
    \mathsf{Val} \quad v &::= \dots
    \mid \tttree{A}{v}{w}
    \mid (\isOf{n}{\tttree{A}{v}{w}})
    \mid \word
    \mid \mw{0} \dots \mw{2^{64}-1}\\
    \mathsf{Comp} \quad M &::= 
    \dots
    \mid \wrec{v}{M_0}{f}{M_1}\\
    &\mid \ttrec{v}{M_0}{a}{M_1}{l,r,f_l,f_r}{M_2}{l,m,r,f_l,f_m,f_r}{M_3}\\
\end{align*}

The semantics is extended with the computation rules:

\begin{mathpar}
\inferrule{
}{
  \step{\wrec{\mw{0}}{M_0}{f}{M_1}}{M_0}
}

\inferrule{
}{
  \step{\wtonat{\mw{w}}}{\toNum{w}}
}

\inferrule{
}{
 \step{\ttrec{\leaf}{M_0}{a}{M_1}{l,r,f_l,f_r}{M_2}{l,m,r,f_l,f_m,f_r}{M_3}}{M_0}
}

\inferrule{
}{
 \step{\ttrec{\single{v}}{M_0}{a}{M_1}{l,r,f_l,f_r}{M_2}{l,m,r,f_l,f_m,f_r}{M_3}}{[v/a]M_1}
}

\inferrule{
  t = \isOf{\ttwo{l}{r}}{\tttree{A}{s}{d}}\\
  \eval{d > \mw{0}}{\suc{\zero}}\\
  \eval{s > \mw{0}}{\suc{\zero}}\\
  v_l = \isOf{l}{\tttree{A}{S_l}{D_l}}\\
  v_r = \isOf{r}{\tttree{A}{S_r}{D_r}}\\
  \eval{d_l+1}{d}\\
  \eval{d_r+1}{d}\\
  \eval{s_l+s_r}{s}\\\\
  F_l \triangleq \thunk{\ttrec{v_l}{M_0}{a}{M_1}{l,r,f_l,f_r}{M_2}{l,m,r,f_l,f_m,f_r}{M_3}}\\
  F_r \triangleq \thunk{\ttrec{v_r}{M_0}{a}{M_1}{l,r,f_l,f_r}{M_2}{l,m,r,f_l,f_m,f_r}{M_3}}\\
}{
  \ttrec{t}{M_0}{a}{M_1}{l,r,f_l,f_r}{M_2}{l,m,r,f_l,f_m,f_r}{M_3} \mapsto
  [v_l/l,v_r/r,F_l/f_l,F_r/f_r]M_2
}

\inferrule{
  t = \isOf{\tthree{l}{m}{r}}{\tttree{A}{s}{d}}\\
  \eval{d > \mw{0}}{\suc{\zero}}\\
  \eval{s > \mw{0}}{\suc{\zero}}\\
  v_l = \isOf{l}{\tttree{A}{S_l}{D_l}}\\
  v_m = \isOf{m}{\tttree{A}{S_m}{D_m}}\\
  v_r = \isOf{r}{\tttree{A}{S_r}{D_r}}\\
  \eval{d_l+1}{d}\\
  \eval{d_m+1}{d}\\
  \eval{d_r+1}{d}\\
  \eval{s_l+s_m+s_r}{s}\\\\
  F_l \triangleq \thunk{\ttrec{n_l}{M_0}{a}{M_1}{l,r,f_l,f_r}{M_2}{l,m,r,f_l,f_m,f_r}{M_3}}\\
  F_m \triangleq \thunk{\ttrec{n_m}{M_0}{a}{M_1}{l,r,f_l,f_r}{M_2}{l,m,r,f_l,f_m,f_r}{M_3}}\\
  F_r \triangleq \thunk{\ttrec{n_r}{M_0}{a}{M_1}{l,r,f_l,f_r}{M_2}{l,m,r,f_l,f_m,f_r}{M_3}}\\
}{
 \ttrec{t}{M_0}{a}{M_1}{l,r,f_l,f_r}{M_2}{l,m,r,f_l,f_m,f_r}{M_3} \mapsto\\
  \ret{\thunk{[v_l/l,v_m/m,v_r/r,F_l/f_l,F_m/f_m,F_r/f_r]M_3}}
}

\end{mathpar}

Augment the types:
\begin{itemize}
\item \isType{\tttree{A}{s}{d}} 
  \begin{enumerate}
  \item \isType{A}
  \item \isVal{s}{\word}
  \item \isVal{d}{\word}
  \end{enumerate}

\item \isVal{\isOf{\leaf}{\tttree{A}{\mw{0}}{\mw{0}}}}{\tttree{A}{\mw{0}}{\mw{0}}}
\item \isVal{\isOf{\single{a}}{\tttree{A}{\mw{1}}{\mw{1}}}}{\tttree{A}{\mw{1}}{\mw{1}}}
  \begin{enumerate}
  \item \isVal{a}{A}
  \end{enumerate}
\item \isVal{\isOf{\ttwo{l}{r}}{\tttree{A}{s}{d}}}{\tttree{A}{s}{d}}
  \begin{enumerate}
  \item \isVal{l}{\tttree{A}{s_l}{d}}
  \item \isVal{r}{\tttree{A}{s_r}{d}}
  \item \eval{s_l+s_r}{s}, \eval{d_l+\mw{1}}{d}, \eval{d_r+\mw{1}}{d}
  \end{enumerate}

\item \isVal{\isOf{\tthree{l}{m}{r}}{\tttree{A}{s}{d}}}{\tttree{A}{s}{d}}
  \begin{enumerate}
  \item \isVal{l}{\tttree{A}{s_l}{d}}
  \item \isVal{m}{\tttree{A}{s_m}{d}}
  \item \isVal{r}{\tttree{A}{s_r}{d}}
  \item \eval{s_l+s_m+s_r}{s}, \eval{d_l+\mw{1}}{d}, \eval{d_m+\mw{1}}{d}, \eval{d_r+\mw{1}}{d}
  \end{enumerate} 
\end{itemize}

\begin{verbatim}
map := (\f.ret(\t.
  trec(t){
    ret(leaf:(0,0));
  | a.
    r <- f a; // phi_f(a) + 2
    ret(single(r):(1,1));
  | l.r.fl.fr.
    seq(fl; Fl. // 1
      seq(fr; Fr. // 1
        sl <- size Fl; // 3
        sr <- size Fr; // 3
        d <- depth Fl; // 3
        s' <- sl + sr; // 2
        d' <- d + 1; // 2
        ret(t2(Fl,Fr):(s',d'));
      )
    )
  | l.m.r.fl.fm.fr.
  seq(fl; Fl. // 1
    seq(fm; Fm. // 1
      seq(fr; Fr. // 1
        sl <- size Fl; // 3
        sm <- size Fm; // 3
        sr <- size Fr; // 3
        d <- depth Fl; //3 
        s' <- sl + sm; // 2
        s'' <- s' + sr; // 2
        d' <- d + 1; // 2
        ret(t3(Fl,Fm,Fr):(s'',d'));
      )
    )
  )
  }
));

num2 := (\t. trec(t){
    ret(0)
  | a.ret(0)
  | l.r.fl.fr.
    seq(fl; Fl. // 1
      seq(fr; Fr. // 1
        n <- Fl + Fr;
        n' <- n + 1;
        ret(n');
      )
    )
  | l.m.r.fl.fm.fr.
  seq(fl; Fl. // 1
    seq(fm; Fm. // 1
      seq(fr; Fr. // 1
        n <- Fl + Fm;
        n' <- n + Fr;
        ret(n');
      )
    )
  )
});

num3 := (\t. trec(t){
    ret(0)
  | a.ret(0)
  | l.r.fl.fr.
    seq(fl; Fl. // 1
      seq(fr; Fr. // 1
        n <- Fl + Fr;
        ret(n);
      )
    )
  | l.m.r.fl.fm.fr.
  seq(fl; Fl. // 1
    seq(fm; Fm. // 1
      seq(fr; Fr. // 1
        n <- Fl + Fm;
        n' <- n + Fr;
        n'' <- n' + 1;
        ret(n'');
      )
    )
  )
});
\end{verbatim}

\begin{lemma}(Map)\label{lemma:map}
  $\isVal{\mathtt{map}}{\arr{\arr{\isOf{a}{A}}{\varphi_1}{\isOf{b}{B}}{\varrho_1}}{\zp}{\arr{\isOf{t}{\tttree{A}{s}{d}}}{\varphi}{\tttree{B}{s}{d}}{\varrho}}{\zp}}$
  where $\varphi \triangleq \ttrec{t}{\zp^+}{a}{\bind{\varphi_1}{p}{\plus\;p\;\toNum{3}}}{l,r,f_l,f_r}{\bind{f_l}{F_l}{\bind{f_r}{F_r}{\bind{\plus\;F_l\;F_r}{p}{\plus\;p\;\toNum{16}}}}}{l,m,r,f_l,f_m,f_r}
    {\bind{f_l}{F_l}{\bind{f_m}{F_m}{\bind{f_r}{F_r}{\bind{\plus\;F_l\;F_m}{p}{\bind{\plus\;p\;F_r}{p'}{\plus\;p'\;\toNum{22}}}}}}}$.
  and $\varrho \triangleq \ttrec{t}{\zp}{a}{\varrho_1}{l,r,f_l,f_r}{\bind{f_l}{F_l}{\bind{f_r}{F_r}{\plus\;F_l\;F_r}}}{l,m,r,f_l,f_m,f_r}
    {\bind{f_l}{F_l}{\bind{f_m}{F_m}{\bind{f_r}{F_r}{\bind{\plus\;F_l\;F_m}{p}{\plus\;p\;F_r}}}}}$.\\
    Informally:
    $\Phi(map)(t) = \begin{cases}
      1 \text{ if } t = \leaf\\
      \varphi_1 + 3 \text{ if } t = \single{a}\\ 
      \Phi(map)(l) + \Phi(map)(r) + 16 \text{ if } t = \ttwo{l}{r}\\
      \Phi(map)(l) + \Phi(map)(m) + \Phi(map)(r) + 22 \text{ if } t = \tthree{l}{m}{r}\\
    \end{cases}$  
\end{lemma}
  Instantiate with \isVal{f}{\arr{\isOf{a}{A}}{\varphi_1}{\isOf{b}{B}}{\varrho_1}}. By Lemma~\ref{lemma:ret}, suffices to show 
  \isVal{\lam{t}{\_}{\dots}}{\arr{\tttree{A}{s}{d}}{\varphi}{\tttree{B}{s}{d}}{\varrho}}. Instantiate with 
  \isVal{t}{\tttree{A}{s}{d}}, and suffices to show 
  \gammaToComp{\cdot}{\varphi}{\mathtt{trec}(\dots)}{\tttree{B}{s}{d}}{\zp}.
\begin{proof}
Induction on $t$.
  \begin{itemize}
    \item $t = \leaf$: then \evalCost{\texttt{trec(t){leaf => ret(leaf)\dots}}}{1}{\texttt{ret(leaf)}}. 
      The potential $\zp^+$ suffices. 
    \item $t = \single{a}$: then 
      \begin{align*}
        &\texttt{trec(t){r <- f a; ret(single(r));\dots}}\\
        \mapsto& \texttt{r <- f a; ret(single(r));}\\
        \mapsto& \texttt{r <- (...); ret(single(r));}\\
        \mapsto^c& \texttt{r <- ret(v); ret(single(r));}\\
        \mapsto& \texttt{ret(single(v));}
      \end{align*}
      and \eval{\varphi}{p}, \eval{\varrho}{p'}, such that $\toNat{p} \ge c + \toNat{p'}$.
      Thus \eval{\bind{\varphi_1}{p}{\plus\;p\;\toNum{3}}}{\toNat{p}+3} suffices to cover the cost and
      leaves the same the remaining potential $\varrho$.
    \item $t = \ttwo{l}{r}$, for $\isVal{l}{\tttree{A}{s_1}{d'}}$ and $\isVal{r}{\tttree{A}{s_2}{d'}}$. Then 
      \begin{align*}
        &\texttt{trec(t){\dots}} \\
        \mapsto& \texttt{seq(fl; Fl.\dots)}\\
        \mapsto^{c_l}& \texttt{seq(thunk(ret(v1))); Fl.\dots}\\
        \mapsto& \texttt{seq(fr; Fr.\dots)}\\
        \mapsto^{c_r}& \texttt{seq(thunk(ret(v2)); Fr.\dots)}\\
        \mapsto& \texttt{sl <- size Fl;\dots}\\
        \mapsto^3& \texttt{sr <- size Fr;\dots}\\
        \mapsto^3& \texttt{d <- depth Fl;\dots}\\
        \mapsto^3& \texttt{s' <- sl + sr;\dots}\\
        \mapsto^2& \texttt{d' <- d + 1;\dots}\\
        \mapsto^2& \texttt{ret(t2(Fl,Fr):(s',d'));\dots}\\
      \end{align*}
      Hence the cost is $c = c_l+c_r + 16$.
      Let $F_l,F_r$ be the recursive results of $\varphi$, and $G_l,G_r$ of $\varrho$.
      . By induction, 
      $\toNat{F_l} \ge c_l + \toNat{G_l}$ and $\toNat{F_r} \ge c_r + \toNat{G_r}$.
      Thus the potential \eval{\varphi}{\toNat{F_l} + \toNat{F_r} + 16} is sufficient 
      to cover the cost and remaining potential.
    \item $t = \tthree{l}{m}{r}$: similar to above. 
  \end{itemize}
\end{proof}

\begin{lemma}
  Let \isVal{t}{\tttree{A}{s}{d}}, \eval{\texttt{size(t)}}{n}, \eval{\texttt{num2(t)}}{n_2}, and \eval{\texttt{num3(t)}}{n_3}.
  Given \isVal{f}{\arr{\isOf{\_}{A}}{\varphi_1}{B}{\varrho_1}},
  if \eval{\Phi(map_f)(t)}{v}, then \eval{\phi(map_f)(n,n_2,n_3)}{v}, where
  $\phi(map_f)(n,n_2,n_3) \triangleq 1 + (3 + \varphi_1)\cdot n + 16n_2 + 22n_3$.
\end{lemma}

\begin{proof}
  Induction on $t$.
  \begin{itemize}
    \item $t = \leaf$: then \eval{\Phi(map_f)(t)}{1}, and \eval{\texttt{size(t)}}{0}, \eval{\texttt{num2(t)}}{0}, and \eval{\texttt{num3(t)}}{0}.
      Thus \eval{\phi(map_f)(0,0,0)}{1}. 
    \item $t = \single{a}$:  then \step{\Phi(map_f)(t)}{3 + \varphi_1},  and \eval{\texttt{size(t)}}{1}, \eval{\texttt{num2(t)}}{0}, and \eval{\texttt{num3(t)}}{0}.
      Thus \step{\phi(map_f)(1,0,0)}{3 + \varphi_1}. 
    \item $t = \ttwo{l}{r}$: then \step{\Phi(map_f)(t)}{\Phi(map_f)(l) + \Phi(map_f)(r) + 16}, and \step{\texttt{size(t)}}{size(l) + size(r)}, 
      \step{\texttt{num2(t)}}{1 + num2(l) + num2(r)}, and \step{\texttt{num3(t)}}{num3(l) + num3(r)}.
      Thus 
      \begin{align*}
        & \phi(map_f)(size(l) + size(r),1 + num2(l) + num2(r),num3(l) + num3(r))\\
        &\mapsto 1 + (3 + \varphi)(size(l) + size(r)) + 16(1 + num2(l) + num2(r)) + 22(num3(l) + num3(r))\\
        &\mapsto \left(1 + (3 + \varphi)size(l) + 16(num2(l)) + 22(num3(l))\right) + \\
                 &\left(1 + (3 + \varphi)size(r) + 16(num2(r)) + 22(num3(r))\right) + 16\\
        &= \Phi(map_f)(l) + \Phi(map_f)(r) + 16 \tag{I.H.}
      \end{align*}
      \item $t = \tthree{l}{m}{r}$: symmetric.
  \end{itemize}
\end{proof}

\begin{verbatim}
foldl := (\f. ret(\t. trec(t){
    ret(\init. ret(init))
  | a.ret(\init. r <- f(init,a); ret(r);)
  | l.r.fl.fr.
    ret(\init.
    seq(fl; Fl. // 1
      bl <- Fl init; // ?
      seq(fr; Fr. // 1
        br <- Fr bl;
        ret(br);
      );
      )
    )
  | l.m.r.fl.fm.fr.
  ret(\init.
  seq(fl; Fl. // 1
    bl <- Fl init; // ?
    seq(fm; Fm. // 1
      bm <- Fm bl;
      seq(fr; Fr. // 1
        br <- Fr bm;
        ret(br);
      );
    );
  ))
}));
\end{verbatim}

\begin{lemma}(Fold?)
  \isVal{\texttt{foldl}}{\arr{\arr{\isOf{\_}{\dprod{A}{B}}}{\varphi_1}{B}{\varrho_1}}{\zp}{\arr{\isOf{t}{\tttree{A}{s}{d}}}{\zp}{\arr{\isOf{init}{B}}{\varphi}{B}{\zp}}{\zp}}{\zp}}
  where \\
  $\varphi(\texttt{foldl}\;f\;init)(t) = \begin{cases}
    1 \text{ if } t = \leaf\\
    \varphi_1 + 2 \text{ if } t = \single{a} \\
    \varphi(\texttt{foldl}\;f\;init)(l) + \varphi(\texttt{foldl}\;f\;b)(r) + 8 \text{ if } t = \ttwo{l}{r}\\
    \varphi(\texttt{foldl}\;f\;init)(l) + \varphi(\texttt{foldl}\;f\;bl)(m) + \varphi(\texttt{foldl}\;f\;br)(r) + 11\text{ if } t = \tthree{l}{m}{r}
  \end{cases}$
\end{lemma}

\iffalse

\begin{lemma}(Tree Recursor)\label{lemma:trec}
  Given 
  \begin{gather*}
    \gammaToVal{\cdot}{t}{\tttree{A}{s}{d}}\\
    \gammaToComp{\cdot}{\varphi_0}{M_0}{[\leaf/x]B}{\varrho_0}\\
    \gammaToComp{\isOf{a}{A}}{\varphi_1}{M_1}{[\single{a}/x]B}{\varrho_0}\\
    \gammaToComp{\isOf{l}{\ret{\tttree{A}{s_1}{d}}},\isOf{r}{\ret{\tttree{A}{s_2}{d}}},\isOf{f_l}{\ccomp{\zp}{[l/x]B}{\zp}},\isOf{f_r}{\ccomp{\zp}{[r/x]B}{\zp}}}
      {\varphi_2\\}{M_2}{[\ttwo{l}{r}/x]B}{\varrho_2} \tag{*}\\
    \gammaToComp{\isOf{l}{\ret{\tttree{A}{s_1}{d}}},\isOf{m}{\ret{\tttree{A}{s_2}{d}}},\isOf{r}{\ret{\tttree{A}{s_3}{d}}},\isOf{f_l}{\ccomp{\zp}{[l/x]B}{\zp}},\\
      \isOf{f_m}{\ccomp{\zp}{[m/x]B}{\zp}},\isOf{f_r}{\ccomp{\zp}{[r/x]B}{\zp}}}
      {\varphi_3\\}{M_3}{[\tthree{l}{m}{r}/x]B}{\varrho_3}
  \end{gather*}
  Then \gammaToComp{\cdot}{\varphi}{\ttrec{t}{M_0}{a}{M_1}{l,r,f_l,f_r}{M_2}{l,m,r,f_l,f_m,f_r}{M_3}}{[t/x]B}{\varrho} where
  \begin{gather*}
  \varphi \triangleq \ttrec{t}{\varphi_0^+}{a}{\varphi_1^+}{l,r,f_l,f_r}{?\varphi_2^+}{l,m,r}{?\varphi_3^+}\\
  \varrho \triangleq \ttrec{t}{\varrho_0}{a}{\varrho_1}{l,r,f_l,f_r}{\varrho_2}{l,m,r,f_l,f_m,f_r}{\varrho_3}
  \end{gather*}
\end{lemma}

\begin{proof}
Induction on structure of $t$:
  \begin{itemize}
    \item $t = \leaf$, $t = \single{a}$: similar to Lemma~\ref{lemma:recursor'}
    \item $t = \ttwo{l}{r}$, for some \isOf{l}{\tttree{A}{s_1}{d}}, \isOf{r}{\tttree{A}{s_2}{d}}:
      Then \step{\ttrec{t}{M_0}{a}{M_1}{l,r,f_l,f_r}{M_2}{l,m,r,f_l,f_m,f_r}{M_3}}{[v_l/l,v_r/r,F_l/f_l,F_r/f_r]M_2}. 
      where $F_l \triangleq \thunk{\ret{\thunk{\ttrec{v_l}{M_0}{a}{M_1}{l,r,f_l,f_r}{M_2}{l,m,r,f_l,f_m,f_r}{M_3}}}}$ 
      and $F_r \triangleq \thunk{\ret{\thunk{\ttrec{v_r}{M_0}{a}{M_1}{l,r,f_l,f_r}{M_2}{l,m,r,f_l,f_m,f_r}{M_3}}}}$. 
      By Lemma~\ref{lemma:ret}, $\isOf{F_l}{\ccomp{\zp}{[l/x]B}{\zp}}$,
      Instantiating * gives \evalCost{[v_l/l,v_r/r,F_l/f_l,F_r/f_r]M_2}{c}{}

  \end{itemize}
\end{proof}

\fi

\begin{verbatim}

to_nat = \d. WREC(d){zero; p.f. r <- f 1; ret(suc(r))}

if(v;M_t;M_f) @= rec(v){M_f; _,_.M_t}

size := (\t. tmatch(t){
    s.d.ret(s)
  | s.d.a.ret(s)
  | s.d.l.r.ret(s)
  | s.d.l.m.r.ret(s)
});

depth := (\t. tmatch(t){
    s.d.ret(d)
  | s.d.a.ret(d)
  | s.d.l.r.ret(d)
  | s.d.l.m.r.ret(d)
});

T := (t2(single(0):(1,1),single(1):(1,1)):(2,2));
T1 := (t3(T,T,T):(6,3));

balance2 := (\T1. ret(\T2. tmatch(T1){
    s.d.ret(0);
  | s.d.a.ret(0);
  | s.d.l.r.
    s2 <- size T2; 
    s' <- s + s2;
    ret(t3(l,r,T2):(s',d));
  | s.d.l.m.r.
    sl <- size l; // 3
    sm <- size m; // 3
    sl' <- sl + sm; //2
    sr <- size r; //3
    s2 <- size T2; //3
    sr' <- sr + s2; //2
    d' <- d + 1; //2
    s' <- sl' + sr'; //a 2
    ret(t2(t2(l,m):(sl',d), t2(r,T2):(sr',d)):(s',d'));
}));

balance3 := (\T1. ret(\T2. ret(\T3.
  tmatch(T1){
    s.d.ret(0);
  | s.d.a.ret(0);
  | s.d.l.r.
    s2 <- size T2;
    s3 <- size T3;
    s23 <- s2 + s3;
    s' <- s + s23;
    d' <- d + 1;
    ret(t2(t2(l,r):(s,d),t2(T2,T3):(s23,d)):(s',d'));
  | s.d.l.m.r.
    sl <- size l; // 3
    sm <- size m; // 3
    sl' <- sl + sm; // 2
    sr <- size r; // 3
    s2 <- size T2; // 3
    s3 <- size T3; // 3
    s23 <- s2 + s3; // 2
    d' <- d + 1; // 2
    s' <- s + s23; // 2
    ret(t2(t3(l,m,r):(s,d), t2(T2,T3):(s23,d)):(s',d'));
})));

insertleft := (\x.ret(\t. 
  trec(t){
     ret(thunk(ret(single(x):(1,1))));
   | a.ret(thunk(ret(t2(single(x):(1,1), single(a):(1,1)):(2,2))));
   | l.r.fl.fr.
      ret(thunk(
        seq(fl; Fl.
        d <- depth Fl;
        dl <- depth l;
        b <- d > dl;
        if(b)
        then{
          balance2 Fl r
          }
        else{
          sl <- size Fl;
          sr <- size r;
          s' <- sl + sr;
          d' <- dl + 1;
          ret(t2(Fl,r):(s',d'));
          };)
      ));

   | l.m.r.fl.fm.fr. 
      ret(thunk(
        seq(fl; Fl. 
        d <- depth Fl;
        dl <- depth l;
        b <- d > dl;
        if(b) 
        then{
          balance3 Fl m r
          }
        else{
          sl <- size Fl;
          sm <- size m;
          sr <- size r;
          s' <- sl + sm;
          s'' <- s' + sr;
          d' <- dl + 1;
          ret(t3(Fl,m,r):(s'',d'));
          };)
      ));
  };
));

// 6 defs
T' <- insertleft 5 T1;
seq(T'; t'.ret(t'));
\end{verbatim}

\iffalse
full_left = \t.n. 
  23TREC(t){
    leaf => ret(zero);
    single a => ret(zero);
    t2(l,r,sl,sr,dl,dr,fl,fr) => 
      d <- dl + 1;
      b <- n == d;
      if(b; ret(zero); ret(fl));
    t3(l,m,r,sl,sm,sr,dl,dm,dr,fl,fm,fr) => 
      d <- dl + 1;
      b <- n == d;
      if(b; ret(suc(zero)); ret(fl));
  }


mk2 = \a.b. 
  case a of
    | inl l =>
      case b of
      | inl r =>
        d1 <- depth l;
        d2 <- depth r;
        b <- d1 < d2;
        if(b;
          23TREC(r){
            leaf => ret(!);
            single a => ret(inl(single(a)));
            t2(rl,rr,_,_) => ret(inl(t3(l,rl,rr)));
            t3(rl,rm,rr) => ret(inl(t2(t2(l,rl),t2(rm,rr))));
          };
          b <- d1 > d2;
          if(b;
            23TREC(l){
              leaf => ret(!);
              single a => ret(inl(single(a)));
              t2(ll,lr,_,_) => ret(inl(t3(ll,lr,r)));
              t3(ll,lm,lr) => ret(inl(t2(t2(ll,lm),t2(lr,r))));
            };
            ret(inl(t2(l,r)))
          )
        )
      | inr (r1,r2) => 
        d1 <- depth l;
        d2 <- depth r1;
        b <- d1 < d2;
        if(b;
          23TREC(r1){
            leaf => ret(!);
            single a => ret(inl(t2(r1,r2)));
            t2(r1l,r1r,_,_) => ret(inl(t2(t3(l,r1l,r1r),r2)));
            t3(r1l,r1m,r1r) => ret(inl(t3(t2(l,r1l),t2(r1m,r1r),r2)));
          };
          b <- d1 > d2;
          if(b;
            23TREC(l){
              leaf => ret(!);
              single a => ret(inl(single(a)));
              t2(ll,lr,_,_) => ret(inl(t2(l,t2(r1,r2))));
              t3(ll,lm,lr) => ret(inl(t2(t3(ll,lm,lr),t2(r1,r2))));
            };
            ret(inl(t3(l,r1,r2)))
          )
        )
    | inr (l1,l2) => 
      case b of
      | inl r =>
        d1 <- depth l1;
        d2 <- depth r;
        b <- d1 < d2;
        if(b;
          ret(inl(t2(t2(l1,l2),r)));
          b <- d1 > d2;
          if(b;
            23TREC(l2){
              leaf => ret(!);
              single a => ret(inl(t2(l1,single(a))));
              t2(rl,rr,_,_) => ret(inl(t2(l1,t3(rl,rr,r))));
              t3(rl,rm,rr) => ret(inl(t3(l,t2(rl,rm), t2(rr,r))));
            };
            ret(inl(t3(l1,l2,r)))
          )
        )
      | inr(r1,r2) => 
        d1 <- depth l1;
        d2 <- depth r1;
        b <- d1 < d2;
        if(b;
          ret(inl(t3(t2(l1,l2),r1,r2)));
          b <- d1 > d2;
          if(b;
            ret(inl(t3(l1,l2,t2(r1,r2))));
            ret(inl(t2(t2(l1,l2),t2(r1,r2))));
          )
        )

join' = \tr1.
  23TREC(tr1){
    leaf => \tr2. ret(inl(tr2));
    single a => \tr2.
      23TREC(tr2){
        leaf => ret(inl(tr1));
        single b => ret(inr(single a, single b));
        t2(l,r,fl,fr) => mk2(fl,r);
        t3(l,m,r,fl,fm,fr) => mk3(fl,m,r);
      }
    t2(l,r,fl,fr) => \tr2.
      d1 <- depth tr1;
      d2 <- depth tr2;
      b <- d1 < d2;
      if(b;
        23TREC(tr2){
         leaf => ret(!);
         single a => ret(!);
         t2(l,r,gl,gr) => mk2(gl,r);
         t3(l,m,r,gl,gm,gr) => mk3(gl,m,r);
        };
        b <- d1 > d2;
        if(b;
          r' <- fr tr2;
          mk2(l,r');;
          mk2(tr1,tr2)
        )
      )
    t3(l,m,r,fl,fm,fr) => \tr2.
      d1 <- depth tr1;
      d2 <- depth tr2;
      b <- d1 < d2;
      if(b;
        23TREC(tr2){
         leaf => ret(!);
         single a => ret(!);
         t2(l,r,gl,gr) => mk2(gl,r);
         t3(l,m,r,gl,gm,gr) => mk3(gl,m,r);
        };
        b <- d1 > d2;
        if(b;
          r' <- fr tr2;
          mk3(l,m,r');;
          mk2(tr1,tr2)
        )
      )
  }
\fi

\iffalse
  \left
  \mathsf{full\_left} \triangleq 
    \lam{t}{\_}{
      \lam{n}{\_}{
      \ttrec{t}{\const{\zero}}{\_}{\const{\zero}}
        {\_,\_,s_l,d_l,\_,\_,f_l,\_}\\{
          \bind{d_l+1}{d}{\bind{\eq\;n\;d}{b}{\ifnat{b}{\ret{b}}{\ret{f_l}}}}}\\
        {\_,\_,\_,s_l,d_l,\_,\_,\_,\_,f_l,\_,\_}{
          \bind{d_l+1}{d}{\bind{\eq\;n\;d}{b}{\ifnat{b}{\ret{\zero}}{\ret{f_l}}}}}
        }
        }\\
  \mathsf{full\_right} \triangleq 
    \lam{t}{}{
      \ttrec{t}{\const{\zero}}{\_}{\const{\zero}}
        {\_}{\lam{n}{\_}{\eq\;n\;d}}
          {\_,\_,f_l,f_r}{\bind{n-1}{n'}{f_r\; n'}}
        }\\
\join \triangleq 
  \lam{t_1}{\tttree{A}{s_1}{d_1}}{
    \lam{t_2}{\tttree{A}{s_2}{d_2}}{
    }}
\fi


\end{document}
