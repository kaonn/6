% --------------------------------------------------------------
% This is all preamble stuff that you don't have to worry about.
% Head down to where it says "Start here"
% --------------------------------------------------------------
 
\documentclass[12pt]{article}
 
\usepackage[margin=1in]{geometry} 
\usepackage{amsmath,amsthm,amssymb,tikz}
\usepackage{mathpartir}
\usepackage{stix}
\usepackage{tabularx}
\usepackage{booktabs,colortbl}
\usepackage{makecell}
\usepackage{tgpagella}
\usepackage[normalem]{ulem}
\usepackage{xargs}
\usepackage{enumitem}
\usepackage{wasysym}
\usepackage{todonotes}
 
\newcommand{\N}{\mathbb{N}}
\newcommand{\Z}{\mathbb{Z}}
\newcommand{\R}{\mathbb{R}}
 
\newenvironment{theorem}[2][Theorem]{\begin{trivlist}
\item[\hskip \labelsep {\bfseries #1}\hskip \labelsep {\bfseries #2.}]}{\end{trivlist}}
\newtheorem{lemma}{Lemma}[section]
\newtheorem{definition}{Definition}[section]
\newtheorem{corollary}{Corollary}[lemma]
\newenvironment{exercise}[2][Exercise]{\begin{trivlist}
\item[\hskip \labelsep {\bfseries #1}\hskip \labelsep {\bfseries #2.}]}{\end{trivlist}}
\newenvironment{reflection}[2][Reflection]{\begin{trivlist}
\item[\hskip \labelsep {\bfseries #1}\hskip \labelsep {\bfseries #2.}]}{\end{trivlist}}
\newenvironment{proposition}[2][Proposition]{\begin{trivlist}
\item[\hskip \labelsep {\bfseries #1}\hskip \labelsep {\bfseries #2.}]}{\end{trivlist}}
\iffalse
\newenvironment{corollary}[2][Corollary]{\begin{trivlist}
\item[\hskip \labelsep {\bfseries #1}\hskip \labelsep {\bfseries #2.}]}{\end{trivlist}}
\fi
\newenvironment{problem}[2][Problem]{\begin{trivlist}
\item[\hskip \labelsep {\bfseries #1}\hskip \labelsep {\bfseries #2.}]}{\end{trivlist}}

\newcommand*{\rulefiller}{%
  \arrayrulecolor[gray]{0.8}% change to cell colour
  \specialrule{\heavyrulewidth}{0pt}{-\heavyrulewidth}% "invisible" rule
  \arrayrulecolor{black}% revert to regular line colour
}

\newcommand{\map}[3]{{\color{red}{\varphi}} #1 : #2.\; #3}
\newcommand{\zero}{{\color{red}{\mathtt{zero}}}}
\newcommand{\suc}[1]{{\color{red}{\mathtt{suc}}}(#1)}
\newcommand{\pair}[2]{{\textcolor{red}{\langle}} #1,#2 {\textcolor{red}{\rangle}}}
\newcommand{\ind}[5]{{\color{red}{\mathtt{ind}}}(#1)(#2;#3.#4.#5)}

\newcommand{\lam}[3]{\ensuremath{{\color{red}{\lambda}} #1 .\; #3}}
\newcommand{\seq}[3]{\mathtt{seq}(#1;#2.\;#3)}
%\newcommand{\bind}[3]{#2 \leftarrow #1;\;#3}
\newcommand{\bind}[3]{\mathtt{do}\; #1 \; \mathtt{as}\; #2\; \mathtt{in}\; #3}
\newcommand{\rec}[5]{\mathtt{rec}(#1)(#2;#3.#4.#5)}
\newcommand{\case}[5]{\mathtt{case}(#1)(#2.#3; #4.#5)}
\newcommand{\thunk}[1]{{\color{red}{\mathtt{thunk}}}(#1)}
\newcommand{\comp}[1]{{\color{red}{\mathtt{comp}}}(#1)}
\newcommand{\compc}[1]{{\color{red}{\mathtt{comp}}}(#1)}
\newcommand{\ccomp}[3]{{\color{red}{\mathtt{comp}}}(#1, #2 \diamond #3)}
\newcommand{\force}[1]{\mathtt{force}(#1)}
\newcommand{\ret}[1]{\mathtt{ret}(#1)}
\newcommand{\fst}[1]{#1 \cdot \mathtt{1}}
\newcommand{\snd}[1]{#1 \cdot \mathtt{2}}

\newcommand{\nat}{\mathtt{nat}}
\newcommand{\pure}[2]{#1 \rhd #2}
\newcommand{\arr}[4]{\ensuremath{(#1\diamond#2) \to (#3\diamond#4)}}
\newcommand{\dprod}[2]{\ensuremath{#1 \times #2}}
\newcommand{\unit}{\mathtt{unit}}
\newcommand{\dprodc}[2]{\ensuremath{#1 \lift{\times} #2}}
\newcommand{\eqty}[3]{\ensuremath{\mathtt{eq}_{#1}(#2, \allowbreak #3)}}
\newcommand{\eqtyc}[3]{\ensuremath{\lift{\mathtt{eq}}_{#1}(#2, \allowbreak #3)}}

\newcommand{\z}{\ensuremath{\mathtt{int}}}
\newcommand{\step}[2]{\ensuremath{#1 \mapsto #2}}
\newcommand{\stepIn}[3]{\ensuremath{#1 \mapsto^{#2} #3}}
\newcommand{\eqto}[2]{#1 \Rightarrow #2}
\newcommand{\eval}[2]{\ensuremath{#1 \Downarrow #2}}
\newcommand{\evalCost}[3]{\ensuremath{#1 \Downarrow^{#2} #3}}

\newcommand{\val}[1]{\ensuremath{#1 \;\mathsf{val}}}
\newcommand{\final}[1]{\ensuremath{#1 \;\mathsf{final}}}

\newcommand{\isOf}[2]{#1 {:} #2}
\newcommand{\gammaToVal}[3]{\ensuremath{#1 \gg #2 \in_0 #3}}
\newcommand{\gammaToTypeComp}[3]{\ensuremath{#1 \gg #3 \; \mathsf{type}}}
\newcommand{\gammaToTypeCompFixed}[3]{\ensuremath{#1 \rhd_{#2} #3 \; \mathsf{type}}}
\newcommand{\gammaToComp}[5]{\ensuremath{#1 \diamond #2 \gg #3 \in #4 \diamond #5}}
\newcommand{\gammaToCompFixed}[6]{\ensuremath{#1 \diamond #2 \rhd_{#3} #4 \in (#5 \diamond #6)}}
\newcommand{\gammaToPot}[2]{\ensuremath{#1 \gg #2 \; \mathsf{pot}}}
\newcommand{\gammaToPotFixed}[3]{\ensuremath{#1 \rhd_{#2} #3 \; \mathsf{pot}}}

\newcommand{\gammaToPotTyped}[3]{\ensuremath{#1 \gg #2 \; \mathsf{pot}\; #3}}

\newcommand{\isVal}[2]{\ensuremath{#1 \in_0 #2}}
\newcommand{\isType}[1]{\ensuremath{#1 \; \mathsf{type}_0}}
\newcommand{\isComp}[2]{\ensuremath{#1 \in #2}}
\newcommand{\isCComp}[4]{\ensuremath{#1 \in #2 {\color{purple}{\;[#3; #4]}}}}
\newcommand{\isTypeComp}[1]{\ensuremath{#1 \; \mathsf{type}}}
\newcommand{\isPot}[1]{\ensuremath{#1 \; \mathsf{pot}}}
\newcommand{\isPotTyped}[2]{\ensuremath{#1 \; \mathsf{pot} \; #2}}
\newcommand{\isCtx}[1]{\ensuremath{#1\; \mathsf{ctx} }}
\newcommand{\isSub}[2]{\ensuremath{#1 \in #2}}

\newcommand{\subst}[2]{[#1]#2}
\newcommand{\lsub}[3]{\left< #1/#2 \right>#3}

\newcommand{\zp}{\mathbb{0}}
\newcommand{\constp}[1]{\ensuremath{\mathbb{#1}}}
\newcommand{\toNat}[1]{\underline{#1}}
\newcommand{\toNum}[1]{\overline{#1}}
\newcommand{\pmin}{\mathtt{min}}
\newcommand{\sub}{\mathtt{sub}}
\newcommand{\plus}{\mathtt{plus}}
\newcommand{\mult}{\mathtt{mult}}


\newcommand{\fact}[2]{#1$^{\textbf {\color{blue}{#2}}}$}
\newcommand{\applyAss}[1]{assumption \textbf{\color{blue}{#1}}}

\newcommand{\eqType}[2]{\ensuremath{#1 \doteq #2\; \mathsf{type}}}
\newcommand{\eqComp}[3]{\ensuremath{#1 \doteq #2 \in #3}}
\newcommand{\eqCComp}[5]{\ensuremath{#1 \doteq #2 \in #3 {\color{purple}{\;[#4; #5]}}}}
\newcommand{\eqVal}[3]{\ensuremath{#1 \doteq #2 \in_0 #3}}

\newcommand{\openEqComp}[4]{\ensuremath{#1 \gg #2 \doteq #3 \in #4}}
\newcommand{\openEqCComp}[6]{\ensuremath{#1 \gg #2 \doteq #3 \in #4 {\color{purple}{\;[#5; #6]}}}}
\newcommand{\openEqVal}[4]{\ensuremath{#1 \gg #2 \doteq #3 \in_0 #4}}
\newcommand{\openComp}[3]{\ensuremath{#1 \gg #2 \in #3}}
\newcommand{\openCComp}[5]{\ensuremath{#1 \gg #2 \in #3 {\color{purple}{\;[#4; #5]}}}}
\newcommand{\openTypeComp}[2]{\ensuremath{#1 \gg #2 \; \mathsf{type}}}
\newcommand{\openEqTypeComp}[3]{\ensuremath{#1 \gg #2 \doteq  #3 \; \mathsf{type}}}

\newcommand{\eqCtx}[2]{\ensuremath{#1 \doteq #2}}
\newcommand{\eqInst}[3]{\ensuremath{#1 \sim #2 \in #3}}

\newbox\qqBoxA
\newdimen\qqCornerHgt
\setbox\qqBoxA=\hbox{$\ulcorner$}
\global\qqCornerHgt=\ht\qqBoxA
\newdimen\qqArgHgt
\def\Quinequote #1{%
    \setbox\qqBoxA=\hbox{$#1$}%
    \qqArgHgt=\ht\qqBoxA%
    \ifnum     \qqArgHgt<\qqCornerHgt \qqArgHgt=0pt%
    \else \advance \qqArgHgt by -\qqCornerHgt%
    \fi \raise\qqArgHgt\hbox{$\ulcorner$} \box\qqBoxA %
    \raise\qqArgHgt\hbox{$\urcorner$}}
\newcommand{\lift}[1]{\ensuremath{\Quinequote{#1}}}

\newcommand{\triv}{\star}
\newcommand{\inl}[1]{\mathtt{inl}(#1)}
\newcommand{\inr}[1]{\mathtt{inr}(#1)}
\newcommand{\sigmaty}[2]{\Sigma #1.#2}
\newcommand{\sigmatyc}[2]{\lift{\Sigma} #1.#2}
\newcommand{\pityc}[2]{\ensuremath{\lift{\Pi} #1.#2}}
\newcommand{\intty}[2]{\cap #1. #2}
\newcommand{\inttyc}[2]{\ensuremath{\lift{\cap} #1. #2}}
\newcommand{\subsetty}[2]{\{ #1 \mid #2\}}

\newcommand{\sameType}[4]{\ensuremath{#1 \sim #2 \downarrow #3 \in #4}}
\newcommand{\sameTypeOne}[6]{\ensuremath{\isOf{#1}{#2} \rhd #3 \sim #4 \downarrow #5 \in #6}}
\newcommand{\sameComp}[4]{\ensuremath{#1 \rhd #2 \sim #3 \in #4}}
\newcommand{\equivComp}[3]{\ensuremath{#1 \rhd #2 \asymp #3}}
\newcommand{\sameCComp}[6]{\ensuremath{#1 \rhd #2 \sim #3 \in #4\; {\color{purple}{[#5; #6]}}}}

\newcommand{\type}[1]{\mathtt{Type}_{#1}}
\newcommand{\term}{\mathsf{Term}}
\newcommand{\evalTo}[1]{{#1}^{\Downarrow}}
\newcommand{\class}[1]{\ensuremath{[#1]}}
\newcommand{\cost}[1]{\ensuremath{\mathcal{C}}(#1)}
\newcommand{\fequiv}[2]{\ensuremath{#1 \asymp #2}}
\newcommand{\arrabt}[3]{\ensuremath{\Pi(#1, #2, #3)}}
\newcommand{\arrabtc}[3]{\ensuremath{\lift{\Pi(#1, #2, #3)}}}
\newcommand{\relpot}[4]{\ensuremath{\mathtt{relpot}(#1.#2; #3.#4)}}
\newcommand{\relpotty}[3]{\ensuremath{\mathtt{relpotty}(#1;#2.#3)}}

%bar.tex
\newcommand{\nseq}[1]{\ensuremath{\mathcal{B}_{#1}}}
\newcommand{\nseqc}[3]{\ensuremath{\mathcal{B}_{#1}^{[#2,#3]}}}
\newcommand{\baire}{\mathcal{B}}
\newcommand{\squash}[1]{\ensuremath{\downarrow \! #1}}

\newcommand\independent{\protect\mathpalette{\protect\independenT}{\perp}}
\def\independenT#1#2{\mathrel{\rlap{$#1#2$}\mkern2mu{#1#2}}}
\newcommand{\empseq}{\independent}
\newcommand{\dbr}[6]{\mathtt{dbr}(#1,#2,#3,#4,#5,#6)}
\newcommand{\dbrp}[6]{\underline{\mathtt{dbr}}(#1,#2,#3,#4,#5,#6)}
\newcommand{\tol}{\ulcorner\to}
\newcommand{\tor}{\to\urcorner}
\newcommand{\tolr}{\lift{\to}}
\newcommand{\embed}{\Uparrow}
\newcommand{\ifthen}[3]{\mathtt{if}(#1;#2;#3)}

% stack.tex
\newcommand{\doState}[2]{\ensuremath{#1\,\rhd\, #2}}
\newcommand{\retState}[2]{\ensuremath{#1\,\lhd\, #2}}
\newcommand{\thunkcst}[1]{\ensuremath{\{#1\}}}
\newcommand{\extSeq}[2]{\ensuremath{#1 \nearrow #2}}

%examples
\newcommand{\listty}[1]{\ensuremath{\mathtt{list}(#1)}}
\newcommand{\listtyc}[1]{\ensuremath{\lift{\mathtt{list}}(#1)}}
\newcommand{\nil}{\ensuremath{\mathtt{nil}}}
\newcommand{\cons}[2]{\ensuremath{\mathtt{cons}(#1;#2)}}
\newcommand{\lrec}[4]{\mathtt{lrec}(#1)(#2;#3.#4)}

%gcd
\newcommand{\ttcst}{\mathtt{tt}}
\newcommand{\ffcst}{\mathtt{ff}}
\newcommand{\pos}[1]{\mathtt{pos}(#1)}
\begin{document}
 
% --------------------------------------------------------------
%                         Start here
% --------------------------------------------------------------
 
%\renewcommand{\qedsymbol}{\filledbox}
 
\title{}%replace X with the appropriate number
\author{ %replace with your name
} %if necessary, replace with your course title
 
%\maketitle

\section{Language}

\begin{align*}
    \mathsf{Exp} \quad E &::= \lam{x}{A}{M} 
    \mid \suc{E}
    \mid \zero 
    \mid \susp{E}
    \mid \ccomp{E_1}{E}{a.E_2}
    \mid x
    \mid \nat
    \mid \pair{E_1}{E_2}\\
    &\mid \arrabt{A}{a.B}{a.P,a.b.Q}
    \mid \dprod{\isOf{x}{M}}{M'}\\
    \mathsf{Comp} \quad M &::= 
     v_1 \; v_2
    \mid \rec{v}{M_0}{x}{f}{M_1}
    \mid \seq{v}{x}{M}
    \mid \fst{v}
    \mid \snd{v}
    \mid \ret{v}\\
\end{align*}

We start with a simple language consisting of functions, natural numbers, and pairs. Later on we will add
various constructs to illustrate cost analysis.
\section{Semantics}
We start with the usual small-step operational semantics, adjusted so that consecutive computations are 
strung together by a sequencing operation.
\[
\fbox{$\step{M}{M'}$} \quad\quad \fbox{$\final{M}$}
\]

\begin{mathpar}

\inferrule{
}{
    \final{\ret{v}}
}

\inferrule{
}{
    \step{(\lam{x}{A}{M}) v}{[v/x]M}
}

\inferrule{
}{
    \step{\rec{\zero}{M_0}{x}{f}{M_1}}{M_0}
}

\inferrule{
}{
    \step{\rec{\suc{v}}{M_0}{x}{f}{M_1}}{\seq{\thunk{\rec{v}{M_0}{x}{f}{M_1}}}{f}{[v/x]M_1}}
}

\inferrule{
}{
    \step{\fst{\pair{v}{w}}}{\ret{v}}
}

\inferrule{
}{
    \step{\snd{\pair{v}{w}}}{\ret{w}}
}

\inferrule{
    \step{M_1}{M_1'}
}{
    \step{\seq{\thunk{M_1}}{x}{M_2}}{\seq{\thunk{M_1'}}{x}{M_2}}
}

\inferrule{
}{
    \step{\seq{\thunk{\ret{v}}}{x}{M_2}}{[v/x]M_2}
}
\end{mathpar}

\evalCost{M}{c}{v} means $\exists v. \stepIn{M}{c}{\ret{v}}$ and \eval{M}{v} means $\exists v.
\stepIn{M}{*}{\ret{v}}$

Define a translation from values to natural numbers: 
$\toNat{n} = \begin{cases} 0 \text{ if } n = \zero\\
1+\toNat{v} \text{ if } n = \suc{v} \end{cases}$



%\section{Ground Judgments}

\begin{center}
\begin{tabularx}{1.0\textwidth}{lXll}
    \toprule
 Judgment & English & Presupposition & Meaning\\ \midrule
 \isType{A_0} & $A_0$ is a canonical type & --- & what are canonical values of $A_0$? \\ \midrule
 \isTypeComp{\star}{A} & $A$ computes to a type  & --- & \eval{A}{A_0} and \isType{A_0}\\ \midrule
 \isVal{v}{A} & $v$ is a canonical value of type $A$ & \isTypeComp{\star}{A} & 
    \makecell[tl]{Given that \eval{A}{A_0},\\ $v$ is a canonical value of $A_0$} \\ \midrule
 \isComp{\varphi}{M}{A}{\varrho} & $M$ computes to a value of type $A$ & 
    \makecell[tl]{\isTypeComp{\star}{A},\\\isPot{\varphi},\\ \isPot{\varrho}}&
    \makecell[tl]{\evalCost{M}{c}{v}, \isVal{v}{B}, \eval{\varphi}{p},\\ \eval{[v/y]\varrho}{p'},\\ and $\toNat{p} \ge c + \toNat{p'}$}\\
 \isType{\nat} & $\nat$ is a canonical type & --- & 
    \makecell[tl]{\isVal{\zero}{\nat}\\ \isVal{\suc{v}}{\ret{\nat}} given that\\ \isVal{v}{\ret{\nat}}}\\ \midrule
\isPot{\varphi}& $\varphi$ computes a potential & --- & 
    \makecell[tl]{\isComp{p}{\varphi}{\ret{\nat}}{p'}\\ for some $p,p'$}\\
 \iffalse
 \isPotTyped{\varrho}{A} & $\varrho$ is a potential function for $A$ & \isTypeComp{\star}{A} &
    \makecell[cl]{$\varrho$ is $i.M$ and for all \isVal{a}{A},\\ \eval{[a/i]M}{n} and \isVal{n}{\nat}}\\
    \fi
    \bottomrule
\end{tabularx}
\end{center}

Let it be the case that
\begin{gather*}
\isTypeComp{\star}{A_1}\\
\isTypeComp{\star}{[a_1/x_1]A_2} \text{ given } \isVal{a_1}{A_1},\\
\dots\dots\\
\isTypeComp{\star}{[a_1/x_1,\dots,a_{n-1}/x_{n-1}]A_n}\\
\text{ given } \\ 
    \isVal{a_1}{A_1},\\
    \dots,\\
    \isVal{a_{n-1}}{[a_1/x_1,\dots,a_{n-2}/x_{n-2}]A_{n-1}}
\end{gather*}

And 

\[
\isVal{a_1}{A_1},\dots,\isVal{a_n}{[a_1/x_1,\dots,a_{n_1}/x_{n-1}]A_n}
\]
Then for $\mathcal{J} \in \{\isTypeComp{\star}{A},\; \isVal{v}{A},\; \isPot{\varphi}\}$, the shorthand
\[\isOf{x_1}{A_1},\dots,\isOf{x_n}{A_n} \rhd_n \mathcal{J}
\]
means
\[[a_1/x_1,\dots,a_n/x_n]\mathcal{J}\]

If moreover, it is the case that 
\begin{gather*}
    \isPot{[a_1/x_1,\dots,a_n/x_n]\varphi}\\
    \isTypeComp{\star}{[a_1/x_1,\dots,a_n/x_n]B}\\
    \isPot{[a_1/x_1,\dots,a_n/x_n,b/y]\varrho} \text{ for all } \isVal{b}{[a_1/x_1,\dots,a_n/x_n]B}\\
\end{gather*}

Then the notation
\[
\isOf{x_1}{A_1},\dots,\isOf{x_n}{A_n} / \varphi \rhd_n M \in \isOf{y}{B}/\varrho
\]
means that 
\begin{gather*}
\eval{[a_1/x_1,\dots,a_n/x_n]\varphi}{p}\\ 
\evalCost{[a_1/x_1,\dots,a_n/x_n]M}{c}{v} \text{ for some } \isVal{v}{[a_1/x_1,\dots,a_n/x_n]B}\\
\eval{[a_1/x_1,\dots,a_n/x_n,v/y]\varrho}{p'}
\end{gather*}
satisfying $\toNat{p} \ge c + \toNat{p'}$.



%\section{Types}

\begin{itemize}
    \item \isType{\dprod{\isOf{x}{A}}{B}}
    \begin{enumerate}
        \item \isTypeComp{\star}{A}
        \item \gammaToTypeCompFixed{\isOf{x}{A}}{1}{B}
    \end{enumerate}
    \item \isType{\comp{A}}
    \begin{enumerate}
        \item \isType{A}
    \end{enumerate}
    \item \isVal{\thunk{M}}{\comp{A}}, given:
    \begin{enumerate}
        \item \isComp{\varphi}{M}{\isOf{x}{A}}{\varrho} for some $\varphi, \varrho$
    \end{enumerate}
        \item \isType{\ccomp{\varphi}{\isOf{x}{A}}{\varrho}}
         \begin{enumerate}
        \item \isType{A}
        \item \isPot{\varphi}
        \item \gammaToPotFixed{\isOf{x}{A}}{1}{\varrho}
        \end{enumerate}
        \item \isVal{\thunk{M}}{\ccomp{\varphi}{\isOf{x}{A}}{\varrho}}
        \begin{enumerate}
            \item \isComp{\varphi}{M}{\isOf{x}{A}}{\varrho}
        \end{enumerate}

    \item \isType{\arr{\isOf{x}{A}}{\varphi}{\isOf{y}{B}}{\varrho}}
\begin{enumerate}
    \item \isTypeComp{\star}{A}
    \item \gammaToTypeCompFixed{\isOf{x}{A}}{1}{B}
    \item \gammaToPotFixed{\isOf{x}{A}}{1}{\varphi}
    \item \gammaToPotFixed{\isOf{x}{A},\isOf{y}{B}}{2}{\varrho}
\end{enumerate}

    \item $\lam{x}{A}{M}$ is a canonical element of \arr{\isOf{x}{A}}{\varphi}{\isOf{y}{B}}{\varrho}, given:
\[
    \gammaToCompFixed{\isOf{x}{A}}{\varphi}{1}{M}{\isOf{y}{B}}{\varrho}
    \]
By the previous remarks, this is the case iff:
given \isVal{a}{A}, it is the case that:
\begin{enumerate}
    \item \eval{[a/x]\varphi}{p}
    \item \evalCost{[a/x]M}{c}{v} for some \isVal{v}{[a/x]B}
    \item \eval{[a/x,v/y]\varrho}{p'}
    \item $\toNat{p} \ge c + \toNat{p'}$
\end{enumerate}

\end{itemize}



%
\section{Hypothetical Judgments}

\begin{center}
\begin{tabularx}{1.0\textwidth}{lXl}
    \toprule
 Judgment & Presupposition & Meaning\\ \midrule
  \rowcolor[gray]{0.7}
 \isCtx{\cdot} & --- & ---\\
  \rowcolor[gray]{0.7}
 \isCtx{\isOf{x}{A},\Gamma} & --- & \isTypeComp{\star}{A}, and for all \isVal{a}{A}, \isCtx{[a/x]\Gamma}\\
  \isSub{\cdot}{\cdot} & --- & --- \\
 \isSub{x\mapsto M,\gamma}{\isOf{x}{A},\Gamma} & --- & \isVal{M}{A} and \isSub{\gamma}{[M/x]\Gamma}\\
 \rowcolor[gray]{0.7}
 \gammaToTypeComp{\cdot}{\star}{B} & --- & \isTypeComp{\star}{B}\\
  \rowcolor[gray]{0.7}
 \gammaToTypeComp{\isOf{x}{A},\Gamma}{\star}{B} & \isCtx{\isOf{x}{A},\Gamma} & for all \isVal{a}{A}, $[a/x]\left(\gammaToTypeComp{\Gamma}{\star}{B}\right)$ \\ 
 \gammaToVal{\cdot}{v}{B} & --- & \isVal{v}{B} \\
 \gammaToVal{\isOf{x}{A},\Gamma}{v}{B} & \isCtx{\isOf{x}{A},\Gamma} & for all \isVal{a}{A}, 
    $[a/x]\left(\gammaToVal{\Gamma}{v}{B}\right)$\\
\rowcolor[gray]{0.7}
 \gammaToPot{\cdot}{\varphi} & --- & \isPot{\varphi} \\ 
 \rowcolor[gray]{0.7}
 \gammaToPot{\isOf{x}{A},\Gamma}{\varphi} & \isCtx{\isOf{x}{A},\Gamma} & for all \isVal{a}{A},                       $[a/x]\left(\gammaToPot{\Gamma}{\varphi}\right)$\\
 \gammaToComp{\cdot}{\varphi}{M}{\isOf{y}{B}}{\varrho} & \isTypeComp{\star}{B}, \isPot{\varphi}, and \gammaToPot{\isOf{y}{B}}{\varrho}
    & \makecell[tl]{\evalCost{M}{c}{v},\\ \isVal{v}{B},\\ \eval{\varphi}{p},\\ \eval{[v/y]\varrho}{p'},\\ and $\toNat{p} \ge c + \toNat{p'}$}\\
  \gammaToComp{\isOf{x}{A},\Gamma}{\varphi}{M}{\isOf{y}{B}}{\varrho} & 
  \makecell[tl]{\isCtx{\isOf{x}{A},\Gamma},\\
  \gammaToTypeComp{\isOf{x}{A},\Gamma}{\star}{B},\\
  \gammaToPot{\isOf{x}{A},\Gamma}{\varphi},\\ 
  \gammaToPot{\isOf{x}{A},\Gamma,\isOf{y}{B}}{\varrho}}
  &\makecell[tl]{for all \isVal{a}{A},\\ $[a/x]\left(\gammaToComp{\Gamma}{\varphi}{M}{\isOf{y}{B}}{\varrho}\right)$}\\
    \bottomrule
\end{tabularx}
\end{center}

%
\section{Lemmas}
Define:
\begin{gather*}
    \zp \triangleq \ret{\zero}\\
    \plus \triangleq \lam{x}{a}{\ret{\lam{y}{b}{\rec{x}{\ret{y}}{\_}{f}{\ret{\suc{f}}}}}}\\
    \mult \triangleq \lam{x}{a}{\ret{\lam{y}{b}{\rec{x}{\ret{y}}{\_}{f}{\plus\; y\; f}}}}\\
    \sub \triangleq \lam{x}{a}{\rec{x}{\ret{\lam{y}{b}{\ret{\zero}}}}{z}{f}{\ret{\lam{y}{b}{\rec{y}{\ret{\suc{p}}}{y'}{\_}{f y'}}}}}\\
    \pmin \triangleq \lam{x}{a}{\ret{\lam{y}{b}{\seq{\sub\; x\; y}{d}{\rec{d}{\ret{x}}{\_}{\_}{\ret{y}}}}}}
\end{gather*}

\begin{lemma}\textnormal{(basis)}\label{lemma:basic}
\isPot{\zp}, and for all \isVal{x}{\ret{\nat}} and \isVal{y}{\ret{\nat}}, \isPot{\plus\; x\; y}, 
 \isPot{\mult\; x\; y}, \isPot{\sub\; x\; y} and \isPot{\pmin\; x\; y}.
\end{lemma}

Notation:
\begin{gather*}
\bind{M_1}{x}{M_2} \triangleq \seq{\thunk{M_1}}{x}{M_2}\\
f\;v_1\;\dots\;v_n \triangleq \bind{f\;v_1}{r_1}{\bind{r_1\;v_2}{r_2}{\dots,\bind{r_{n-2}\;v_{n-1}}{r_{n-1}}{r_{n-1}\;v_n}}}\\
\varphi^+ \triangleq \bind{\varphi}{v}{\ret{\suc{v}}}\\
\end{gather*}

\begin{lemma}\textnormal{(Recursor)}\label{lemma:recursor}
Given 
\begin{gather}
\gammaToPot{\isOf{x}{\ret{\nat}}}{\varphi_0},\\
\gammaToPot{\isOf{x}{\ret{\nat}},\isOf{z}{\ret{\nat}}, \isOf{g}{\comp{\nat}}}{\varphi_1},\\
\gammaToTypeComp{\isOf{x}{\ret{\nat}}}{\star}{B},\\
\gammaToPot{\isOf{x}{\ret{\nat}},\isOf{y}{B}}{\varrho_0},\\
\gammaToPot{\isOf{x}{\ret{\nat}}, \isOf{z}{\ret{\nat}}, \isOf{g}{\comp{\nat}}, \isOf{y}{B}}{\varrho_1},\\
\gammaToComp{\cdot}{[\zero/x]\varphi_0}{M_0}{\isOf{y}{[\zero/x]B}}{[\zero/x]\varrho_0},\\
\gammaToComp{\isOf{z}{\ret{\nat}}}{[\suc{z}/x]\lsub{\rec{z}{\varphi_0}{z}{g}{\varphi_1}}{g}{\varphi_1}}
{\lsub{\rec{z}{M_0}{z}{f}{M_1}}{f}{M_1}\\}{\isOf{y}{[\suc{z}/x]B}}{[\suc{z}/x]\lsub{\rec{z}{\varrho_0}{z}{g}{\varrho_1}}{g}{\varrho_1}}
\end{gather}
Then 
\gammaToComp{\isOf{x}{\ret{\nat}}}{\rec{x}{\varphi_0}{z}{g}{\varphi_1}^+}{\rec{x}{M_0}{z}{f}{M_1}}{\isOf{y}{B}}{
\rec{x}{\varrho_0}{z}{g}{\varrho_1}}
\end{lemma}

\begin{proof}
Presuppositions: assumed by given. Let \isVal{n}{\ret{\nat}}. STS \gammaToComp{\cdot}{[n/x]\rec{x}{\varphi_0}{z}{g}{\varphi_1}^+}{\rec{n}{M_0}{z}{f}{M_1}}{\isOf{y}{[n/x]B}}{
[n/x]\rec{x}{\varrho_0}{z}{g}{\varrho_1}}
. Presuppositions: 
\begin{enumerate}
    \item \isTypeComp{[n/x]B}. By instantiating (2) with $n$ for $x$. 
    \item \isPot{[n/x]\rec{x}{\varphi_0}{z}{g}{\varphi_1}^+}. By instantiating (1) with $n$ for $x$. 
    \item \gammaToPot{\isOf{y}{[n/x]B}}{[n/x]\rec{x}{\varrho_0}{z}{g}{\varrho_1}}
\end{enumerate}
Next, induction on $n$: 
\begin{itemize}
    \item $n$ is $\zero$:\\
    Then \step{\rec{\zero}{M_0}{z}{f}{M_1}}{M_0}. By (6), we know \evalCost{M_0}{c}{u} for some \isVal{u}{[\zero/x]B}.
    By head expansion, \evalCost{\rec{\zero}{M_0}{z}{f}{M_1}}{c+1}{u}. 
    Furthermore, we know \eval{[\zero/x]\varphi_0}{p}, \eval{[\zero/x,u/y]\varrho_0}{p'}, and $\toNat{p} \ge c + \toNat{p'}$. Furthermore, we know that 
    \begin{align*}
        &[\zero/x]\rec{x}{\varphi_0}{z}{g}{\varphi_1}^+\\ 
        \mapsto & [\zero/x]\varphi_0^+\\
        \mapsto^* &\seq{\thunk{\ret{p}}}{v}{\ret{\suc{v}}}\\
        \mapsto& \ret{\suc{p}}\\
    \end{align*}. Since $\toNat{\suc{p}} = \toNat{p} + 1$, STS that\\
    \eval{[\zero/x,u/y]\rec{x}{\varrho_0}{z}{g}{\varrho_1}}{p''}
    and that $p' \ge p''$. Evidently this is the case since
    \step{[\zero/x,u/y]\rec{x}{\varrho_0}{z}{g}{\varrho_1}}{\eval{[\zero/x,u/y]\varrho_0}{p'}}.
    \item $n$ is $\suc{n'}$ for some $n'$:\\
    Then \step{\rec{\suc{n'}}{M_0}{z}{f}{M_1}}{\lsub{\rec{n'}{M_0}{z}{f}{M_1}}{f}{[n'/z]M_1}}. 
    Instantiating (7) with $[n'/z]$, we know that \evalCost{\lsub{\rec{n'}{M_0}{z}{f}{M_1}}{f}{[n'/z]M_1}}{c}{u}, 
    and by head expansion, \evalCost{\rec{\suc{n'}}{M_0}{z}{f}{M_1}}{c+1}{u}. Furthermore, we know
    \begin{align*}
        &\rec{\suc{n'}}{[\suc{n'}/x]\varphi_0}{z}{g}{[\suc{n'}/x]\varphi_1}^+\\
        \mapsto& [\suc{z}/x]\lsub{\rec{z}{\varphi_0}{z}{g}{\varphi_1}}{g}{\varphi_1}^+\\
        \mapsto^*& \ret{p}^+\\
        \mapsto& \ret{\suc{p}}
    \end{align*} and that 
    \begin{align*}
        &\rec{\suc{n'}}{[\suc{n'}/x]\varrho_0}{z}{g}{[\suc{n'}/x]\varrho_1}\\
        \mapsto &[\suc{z}/x]\lsub{\rec{z}{\varrho_0}{z}{g}{\varrho_1}}{g}{\varrho_1}\\
        \mapsto^* & \ret{p'}
    \end{align*}
    such that $\toNat{p} \ge c + \toNat{p'}$, which suffices for 
    $\toNat{\suc{p}} \ge c+1 + \toNat{p'}$.
\end{itemize}
\end{proof}

\newpage
\begin{lemma}\textnormal{(Recursor')}\label{lemma:recursor'}
Given 
\begin{gather}
\gammaToPot{\cdot}{\varphi_0},\\
\gammaToPot{\isOf{z}{\ret{\nat}}}{\varphi_1},\\
\gammaToTypeComp{\isOf{x}{\ret{\nat}}}{\star}{B},\\
\gammaToPot{\cdot}{\varrho_0},\\
\gammaToPot{\isOf{z}{\ret{\nat}}}{\varrho_1},\\
\gammaToComp{\cdot}{\varphi_0}{M_0}{\isOf{y}{[\zero/x]B}}{\varrho_0},\\
\gammaToComp{\isOf{z}{\ret{\nat}},\isOf{f}{\ret{\ccomp{\zp}{\isOf{w}{[z/x]B}}{\zp}}}}{\varphi_1}
{M_1}{\isOf{y}{[\suc{z}/x]B}}{\varrho_1},\\
\gammaToVal{\cdot}{n}{\ret{\nat}}
\end{gather}
Then 
\gammaToComp{\cdot}{\rec{n}{\varphi_0^+}{z}{g}{\bind{\varphi_1}{\phi_1}{\plus\; g\; \phi_1}^+}}{\rec{n}{M_0}{z}{f}{M_1}}{\isOf{y}{[n/x]B}}{
\rec{n}{\varrho_0}{z}{g}{\bind{\varrho_1}{\rho_1}{\plus\;\rho_1\; g}}}
\end{lemma}

\begin{proof}
Presuppositions: 
\begin{enumerate}
    \item \isTypeComp{[n/x]B}. By instantiating (11) with $n$ for $x$. 
    \item \isPot{\rec{n}{\varphi_0^+}{z}{g}{\bind{\varphi_1}{\phi_1}{\plus\;\phi_1\;g}^+}}. TODO
    \item \isPot{\rec{n}{\varrho_0}{z}{g}{\bind{\varrho_1}{\rho_1}{\plus\;\rho_1\; g}}}. TODO
\end{enumerate}
Next, induction on $n$: 
\begin{itemize}
    \item $n$ is $\zero$:\\
    Then \step{\rec{\zero}{M_0}{z}{f}{M_1}}{M_0}. By (14), we know \evalCost{M_0}{c}{u} 
    for some \isVal{u}{[\zero/x]B}.
    By head expansion, \evalCost{\rec{\zero}{M_0}{z}{f}{M_1}}{c+1}{u}. 
    Furthermore, we know \eval{\varphi_0}{p}, \eval{[u/y]\varrho_0}{p'}, and $\toNat{p} \ge c + \toNat{p'}$. Furthermore, we know that 
    \begin{align*}
        &\rec{\zero}{\varphi_0^+}{z}{g}{\bind{\varphi_1}{\phi_1}{\plus\;\phi_1\;g}^+}\\ 
        \mapsto & \varphi_0^+\\
        \mapsto^* &\seq{\thunk{\ret{p}}}{v}{\ret{\suc{v}}}\\
        \mapsto& \ret{\suc{p}}
    \end{align*}
    Since $\toNat{\suc{p}} = \toNat{p} + 1$, STS that
    \eval{[u/y]\rec{\zero}{\varrho_0}{z}{g}{\bind{\varrho_1}{\rho_1}{\plus\;\rho_1\;g}}}{p''}
    and that $p' \ge p''$. Evidently this is the case since
    \step{[u/y]\rec{\zero}{\varrho_0}{z}{g}{\bind{\varrho_1}{\rho_1}{\plus\; \rho_1\; g}}}{\eval{[u/y]\varrho_0}{p'}}.
    \item $n$ is $\suc{n'}$ for some $n'$:\\
    Then \step{\rec{\suc{n'}}{M_0}{z}{f}{M_1}}{\lsub{\rec{n'}{M_0}{z}{f}{M_1}}{f}{[n'/p]M_1}}. 
    By induction, we know that 
    \gammaToComp{\cdot}{\rec{n'}{\varphi_0^+}{z}{g}{\bind{\varphi_1}{\phi_1}{\plus\;\phi_1\;g}^+}}
    {\rec{n'}{M_0}{z}{f}{M_1}}{\isOf{y}{[n'/x]B}}
    {\rec{n'}{\varrho_0}{z}{g}{\plus\;g\;\varrho_1}}. This implies that 
    \begin{gather}
    \evalCost{\rec{n'}{M_0}{z}{f}{M_1}}{c}{u}, \\
    \eval{\rec{n'}{\varphi_0^+}{z}{g}{\bind{\varphi_1}{\phi_1}{\plus\;\phi_1\;g}^+}}{p},\\
    \eval{\rec{n'}{\varrho_0}{z}{g}{\bind{\varrho_1}{\rho_1}{\plus\;\rho_1\;g}}}{p'}, \text{ and that}\\
    \toNat{p} \ge c + \toNat{p'}
    \end{gather}
    Instantiating (15) with $[n'/p, \thunk{\ret{u}}/f]$, we know that
    \evalCost{\lsub{\ret{u}}{f}{[n'/p]M_1}}{c'}{u'},
    and by head expansion, \evalCost{\rec{\suc{n'}}{M_0}{z}{f}{M_1}}{c+1+c'}{u'}. Furthermore, we know
    \begin{align}
        &\eval{[n'/z]\varphi_1}{q},\\
        &\eval{[n'/z]\varrho_1}{q'}, \text{ and that }\\
        &q \ge c' + q'
    \end{align}
    Expanding our current potential, we have:
    \begin{align*}
        &\rec{\suc{n'}}{\varphi_0^+}{z}{g}{\bind{\varphi_1}{\phi_1}{\plus\;\phi_1\;g}^+}\\
        \mapsto& \lsub{\rec{n'}{\varphi_0^+}{z}{g}{\bind{\varphi_1}{\phi_1}{\plus\;\phi_1\;g}^+}}{g}{[n'/p]\bind{\varphi_1}{\phi_1}{\plus\;\phi_1\;g}^+}\\
        \mapsto^*& \lsub{\ret{p}}{g}{\bind{[n'/z]\varphi_1}{\phi_1}{\plus\;\phi_1\;g}^+}\\
        \mapsto& \bind{[n'/z]\varphi_1}{\phi_1}{\plus\;\phi_1\;p}^+\\
        \mapsto^*& \bind{\ret{q}}{\phi_1}{\plus\;\phi_1\;p}^+\\
        \mapsto& \plus\; q \; p ^+
    \end{align*} and that 
    \begin{align*}
        &\rec{\suc{n'}}{\varrho_0}{z}{g}{\bind{\varrho_1}{\rho_1}{\plus\;\rho_1\; g}}\\
        \mapsto& \lsub{\rec{n'}{\varrho_0}{z}{g}{\bind{\varrho_1}{\rho_1}{\plus\;\rho_1\; g}}}{g}{[n'/z]\bind{\varrho_1}{\rho_1}{\plus\;\rho_1\; g}}\\
        \mapsto^*& \lsub{\ret{p'}}{g}{\bind{[n'/z]\varrho_1}{\rho_1}{\plus\;\rho_1\; g}}\\
        \mapsto& \bind{[n'/z]\varrho_1}{\rho_1}{\plus\;\rho_1\; p'}\\
        \mapsto^*& \bind{\ret{q'}}{\rho_1}{\plus\;\rho_1\; p'}\\
        \mapsto& \plus\;q'\; p'
    \end{align*}
    Now it suffices to show $\toNat{\plus\; q \; p ^+} \ge c + 1 + c' + \toNat{\plus\;q'\; p'}$,
    which holds by (20) and (23).
\end{itemize}
\end{proof}

\begin{lemma}\textnormal{(Return)}\label{lemma:ret}
If \gammaToVal{\Gamma}{v}{B}, then \gammaToComp{\Gamma}{\zp}{\ret{v}}{\isOf{y}{B}}{\zp}
\end{lemma}

\begin{proof}
Induction on the length of $\Gamma$.
\begin{itemize}
    \item $\Gamma = \cdot$\\
    Suppose \gammaToVal{\cdot}{v}{B}. By definition, this means \isVal{v}{B}. STS 
    \gammaToComp{\cdot}{\zp}{\ret{v}}{\isOf{y}{B}}{\zp}. We know that 
    $\evalCost{\ret{v}}{0}{v}$, $\eval{\zp}{\zero}$, $\eval{[v/y]\zp}{\zero}$, and 
    $\toNat{\zero} \ge 0 + \toNat{\zero}$.
    \item $\Gamma = \isOf{x}{A},\Gamma'$\\
    Suppose \gammaToVal{\isOf{x}{A},\Gamma'}{v}{B}. This means 
    \fact{$[a/x]\left(\gammaToVal{\Gamma}{v}{B}\right)$ for all \isVal{a}{A}}{1}. 
    NTS \gammaToComp{\isOf{x}{A},\Gamma'}{\zp}{\ret{v}}{\isOf{y}{B}}{\zp}. Let
    \isVal{a}{A}. STS $[a/x]\left(\gammaToComp{\Gamma}{\varphi}{\ret{v}}{\isOf{y}{B}}{\varrho}\right)$, which 
    holds by IH when instantiated with \applyAss{1}.
\end{itemize}
\end{proof}

\begin{lemma}\textnormal{(Context)}\label{lemma:ctx}
If \isCtx{\Gamma[\isOf{x}{A}]\Gamma'}, then for all \isSub{\gamma}{\Gamma}, \isTypeComp{\subst{\gamma}{A}}.
\end{lemma}

\begin{proof}
Induction on length of $\Gamma$.
\begin{itemize}
    \item $\Gamma = \cdot$\\
    Suppose \isCtx{[\isOf{x}{A}]\Gamma'} and \isSub{\gamma}{\cdot}. The former implies \isTypeComp{A},
    together with the latter which implies $\gamma = \cdot$, we have \isTypeComp{A\gamma}.
    \item $\Gamma = \isOf{x'}{A'},\Gamma''$\\
    \isCtx{\isOf{x'}{A'},\Gamma''[\isOf{x}{A}]\Gamma'} and \isSub{\gamma}{\isOf{x'}{A'},\Gamma''}. 
    Then $\gamma$ must 
    be $x' \mapsto M, \gamma'$ for some \isVal{M}{A'} and \isSub{\gamma'}{[M/x']\Gamma''}. 
    Further, \isCtx{[M/x'](\Gamma''[\isOf{x}{A}]\Gamma')}. By IH,
    \isTypeComp{[\gamma'][M/x']A}. This suffices since $[\gamma]A = [\gamma'][M/x']A$.
\end{itemize}
\end{proof}

\begin{lemma}\textnormal{(Abstraction)}\label{lemma:abs}
If \gammaToComp{\Gamma[\isOf{x}{A}]}{\varphi}{M}{\isOf{y}{B}}{\varrho}, then 
\gammaToVal{\Gamma}{\lam{x}{A}{M}}{\ret{\arr{\isOf{x}{A}}{\varphi}{\isOf{y}{B}}{\varrho}}}
\end{lemma}

\begin{proof}
Induction on length of $\Gamma$.
\begin{itemize}
    \item $\Gamma = \cdot$\\
    Suppose \gammaToComp{\isOf{x}{A}}{\varphi}{M}{\isOf{y}{B}}{\varrho}. This means
     \fact{\gammaToComp{\cdot}{[a/x]\varphi}{[a/x]M}{\isOf{y}{[a/x]B}}{[a/x]\varrho} for all \isVal{a}{A}}{1}.
     NTS \gammaToVal{\cdot}{\lam{x}{A}{M}}{\ret{\arr{A}{\varphi}{\isOf{y}{B}}{\varrho}}}. STS
     \isVal{\lam{x}{A}{M}}{\ret{\arr{A}{\varphi}{\isOf{y}{B}}{\varrho}}}. First, NTS 
     \isTypeComp{\arr{\isOf{x}{A}}{\varphi}{\isOf{y}{B}}{\varrho}}:
     \begin{enumerate}
         \item \isTypeComp{A} by presupposition with Lemma~\ref{lemma:ctx}
         \item NTS \gammaToTypeCompFixed{\isOf{x}{A}}{1}{B}. Let \isVal{a}{A}. STS 
         \isTypeComp{[a/x]B}. By presupposition, \gammaToTypeComp{\isOf{x}{A},\cdot}{\star}{B}. This implies
         \gammaToTypeComp{\cdot}{\star}{[a/x]B}, which means \isTypeComp{[a/x]B}.
         \item NTS \gammaToPotFixed{\isOf{x}{A}}{1}{\varphi}. Let \isVal{a}{A}. STS \isPot{[a/x]\varphi}. 
         By presupposition, \gammaToPot{\isOf{x}{A},\cdot}{\varphi}.
         This implies \gammaToPot{\cdot}{[a/x]\varphi}, which means \isPot{[a/x]\varphi}.
         \item NTS \gammaToPotFixed{\isOf{x}{A},\isOf{y}{B}}{1}{\varrho}. 
         Let \isVal{a}{A} and \isVal{b}{[a/x]B}. STS \isPot{[a/x,b/y]\varrho}. By presupposition, 
         \gammaToPot{\isOf{x}{A},\cdot[\isOf{y}{B}]}{\varrho}. This implies 
         \gammaToPot{\cdot[\isOf{y}{[a/x]B}]}{[a/x]\varrho}, which means \gammaToPot{\isOf{y}{[a/x]B}}{[a/x]\varrho}.
         Next, this implies \gammaToPot{\cdot}{[b/y][a/x]\varrho}, which means \isPot{[b/y][a/x]\varrho}. 
     \end{enumerate}
     Next, NTS \gammaToCompFixed{\isOf{x}{A}}{\varphi}{1}{M}{\isOf{y}{B}}{\varrho}.
     Let \isVal{a}{A}. NTS 
     \begin{enumerate}
         \item \eval{[a/x]\varphi}{p}. By presupposition on \applyAss{1}, \gammaToPot{\cdot}{[a/x]\varphi}. This 
         means \isPot{[a/x]\varphi}, which ensures a return value. 
         \item \evalCost{[a/x]M}{c}{v} for some \isVal{v}{[a/x]B}. By definition of \applyAss{1}, 
         \evalCost{[a/x]M}{c}{v} and \isVal{v}{[a/x]B}
         \item \eval{[a/x,v/y]\varrho}{p'}. By presupposition on \applyAss{1}, 
         \gammaToPot{\isOf{y}{[a/x]B}}{[a/x]\varrho}. This 
         means \isPot{[a/x,b/y]\varrho} for all \isVal{b}{B}. Thus take $b$ to be $v$, and we are guaranteed a $p'$.
         \item $\toNat{p} \ge c + \toNat{p'}$. Follows from definition of \applyAss{1}.
     \end{enumerate}
    \item $\Gamma = \isOf{x'}{A'},\Gamma'$\\
    Suppose \gammaToComp{\isOf{x'}{A'},\Gamma'[\isOf{x}{A}]}{\varphi}{M}{\isOf{y}{B}}{\varrho}. This means
    \fact{$[a'/x']\left(\gammaToComp{\Gamma'[\isOf{x}{A}]}{\varphi}{M}{\isOf{y}{B}}{\varrho}\right)$ for all \isVal{a'}{A'}}{2}. 
    NTS \gammaToVal{\isOf{x'}{A'},\Gamma'}{\lam{x}{A}{M}}{\ret{\arr{\isOf{x}{A}}{\varphi}{\isOf{y}{B}}{\varrho}}}. 
    This means 
    $[a'/x']\left(\gammaToVal{\Gamma'}{\lam{x}{A}{M}}{\ret{\arr{\isOf{x}{A}}{\varphi}{\isOf{y}{B}}{\varrho}}}\right)$
    for all \isVal{a'}{A'}. Suppose \isVal{a'}{A'}. Then the result holds by \applyAss{2} on IH.
\end{itemize}
\end{proof}

\begin{lemma}\textnormal{(Application)}
If \isVal{f}{\ret{\arr{\isOf{x}{A}}{\varphi}{\isOf{y}{B}}{\varrho}}} and \isVal{a}{A}, then 
\gammaToComp{\cdot}{[a/x]\varphi^+}{f\;v}{\isOf{y}{[a/x]B}}{[a/x]\varrho}.
\end{lemma}

\begin{proof}
Given \isVal{f}{\ret{\arr{\isOf{x}{A}}{\varphi}{\isOf{y}{B}}{\varrho}}} and \isVal{a}{A}, 
we know that $f = \lam{x}{a}{M}$ s.t. 
\[
    \gammaToCompFixed{\isOf{x}{A}}{\varphi}{1}{M}{\isOf{y}{B}}{\varrho}
\]
Instantiating this with $[a/x]$, we know
\begin{enumerate}
    \item \eval{[a/x]\varphi}{p}
    \item \evalCost{[a/x]M}{c}{v} for some \isVal{v}{[a/x]B}
    \item \eval{[a/x,v/y]\varrho}{p'}
    \item $\toNat{p} \ge c + \toNat{p'}$
\end{enumerate}

Since \step{\lam{x}{a}{M}\; a}{[a/x]M}, it suffices to show 
$\step{[a/x]\varphi^+}{q}$ s.t $\toNat{q} \ge \toNat{p} + 1$, which 
holds since $\eval{[a/x]\varphi^+}{\suc{p}}$. 

\end{proof}

\begin{lemma}\textnormal{(Sequence)}\label{lemma:seq}
If \gammaToComp{\cdot}{\varphi}{M_1}{\isOf{x}{A}}{\varrho} and 
\gammaToComp{\isOf{x}{A}}{\varrho}{M_2}{\isOf{y}{B}}{\varsigma}, then 
\gammaToComp{\isOf{x}{A}}{\varphi^+}{\seq{\thunk{M_1}}{x}{M_2}}{\isOf{y}{B}}{\varsigma}.
\end{lemma}

\begin{proof}

\end{proof}

\begin{corollary}\textnormal{(Potential Sequence)}\label{cor:potseq}
If \isPot{\varphi} and
\gammaToPot{\isOf{x}{\ret{\nat}}}{\varrho}, then 
\isPot{\seq{\thunk{\varphi}}{x}{\varrho}}.
\end{corollary}

\begin{lemma}\textnormal{(Pair)}\label{lemma:pair}
If \gammaToVal{\cdot}{v}{A}, \gammaToTypeComp{\isOf{x}{A}}{\star}{B}, and \gammaToVal{\cdot}{w}{[v/x]B},
then \gammaToVal{\cdot}{\pair{v}{w}}{\ret{\dprod{\isOf{x}{A}}{B}}}. 
\end{lemma}



%
\section{Arithmetic}

\begin{lemma}(Plus)
\isVal{\plus}
{\ret{\arr{\isOf{x}{\ret{\nat}}}{\zp}{\ret{\arr{\ret{\nat}}{\varphi}{\ret{\nat}}{\varrho}}}{\zp}}},
where 
\begin{gather*}
    \varphi_0 = \zp\\
    \varphi_1 = \ret{\suc{\zero}}\\
    \varphi= \rec{x}{\varphi_0^+}{z}{g}{\bind{\varphi_1}{\phi_1}{\plus\;\phi_1\;g}^+}\\
    \varrho_0 = \zp\\
    \varrho_1 = \zp\\
    \varrho= \rec{x}{\varrho_0}{z}{g}{\bind{\varrho_1}{\rho_1}{\plus\;\rho_1\;g}}\\
\end{gather*}
\end{lemma}

\begin{proof}
By definition, it suffices to show\\
\gammaToCompFixed{\isOf{x}{\ret{\nat}}}{\zp}{1}{\ret{\lam{y}{b}{\rec{x}{\ret{y}}{\_}{f}{\ret{\suc{f}}}}}}{\ret{\arr{\ret{\nat}}{\varphi}{\ret{\nat}}{\varrho}}}{\zp}\\
Let \isVal{n}{\ret{\nat}}. 
Then by Lemma~\ref{lemma:ret}, it suffices to show
\[
\isVal{\lam{y}{b}{\rec{n}{\ret{y}}{\_}{f}{\ret{\suc{f}}}}}{\ret{\arr{\ret{\nat}}{[n/x]\varphi}{\ret{\nat}}{\varrho}}}\]

By definition, we need to show 
\[
\gammaToCompFixed{\isOf{y}{\ret{\nat}}}{[n/x]\varphi}{1}{\rec{n}{\ret{y}}{\_}{f}{\ret{\suc{f}}}}{\ret{\nat}}{\varrho}
\]

Let \isVal{m}{\ret{\nat}} and unfold with $[m/y]$.  
By Lemma~\ref{lemma:recursor}, it suffices to show
\begin{enumerate}
    \item \gammaToComp{\cdot}{\zp}{\ret{m}}{\isOf{y}{\ret{\nat}}}{\zp}: follows from Lemma~\ref{lemma:ret}.
    \item \gammaToComp{\isOf{z}{\ret{\nat}},\isOf{f}{\ccomp{\zp}{\isOf{w}{\ret{\nat}}}{\zp}} }{\varphi_1}{\ret{\suc{f}}}{\ret{\nat}}{\varrho_1}.\\
    Let \isVal{z}{\ret{\nat}}. STS 
    \gammaToComp{\isOf{f}{\ccomp{\zp}{\isOf{w}{\ret{\nat}}}{\zp}} }{\varphi_1}{\ret{\suc{f}}}{\ret{\nat}}{\varrho_1}.
    Let \isVal{f}{\ccomp{\zp}{\isOf{w}{\ret{\nat}}}{\zp}}. 
    Then $f$ is $\thunk{F}$ for some $F$ s.t. \isComp{\zp}{F}{\isOf{w}{\ret{\nat}}}{\zp}. 
    This implies that $\evalCost{F}{0}{u}$, and hence $F = \ret{u}$. 
         Thus,
        \begin{align*}
            &\lsub{F}{f}{[\zero/z]\ret{\suc{f}}} = \lsub{\ret{u}}{f}{\ret{\suc{f}}} \mapsto \ret{\suc{u}}
        \end{align*} 
        Further, 
        \begin{align*}
           &\eval{\varphi_1 = \ret{\suc{\zero}}}{\suc{\zero}}
        \end{align*}
        Similarly, \eval{\varrho_1}{\zero}.
        The result holds since $\toNat{\suc{\zero}} \ge 1 + \toNat{\zero}$.
\end{enumerate}

Simplified recurrence for $\plus$: $\Phi(\plus)(x) = 2x + 1$.
\end{proof}

Recall $\mult \triangleq \lam{x}{a}{\ret{\lam{y}{b}{\rec{x}{\ret{\zero}}{\_}{f}{\plus\; y\; f}}}}
$. 
\begin{lemma}(Mult)\label{lemma:mult}
\isVal{\mult}{\ret{\arr{\isOf{x}{\ret{\nat}}}{\zp}{\ret{\arr{\isOf{y}{\ret{\nat}}}{\varphi}{\ret{\nat}}{\varrho}}}{\zp}}},
where 
\begin{gather*}
    \varphi_0 = \zp\\
    \varphi_1 = \bind{\mult\;\toNum{2}\;y}{r}{\plus\;r\;\toNum{5}}\\
    \varphi= \rec{x}{\varphi_0^+}{z}{g}{\bind{\varphi_1}{\phi_1}{\plus\;\phi_1\;g}^+}\\
    \varrho_0 = \zp\\
    \varrho_1 = \zp\\
    \varrho= \rec{x}{\varrho_0}{z}{g}{\bind{\varrho_1}{\rho_1}{\plus\;\rho_1\;g}}\\
\end{gather*}
\end{lemma}

\begin{proof}
It suffices to show 

\gammaToCompFixed{\isOf{x}{\ret{\nat}}}{\zp}{1}{\ret{\lam{y}{b}{\rec{x}{\ret{\zero}}{\_}{f}{\plus\; y\; f}}}}{\ret{\arr{\isOf{y}{\ret{\nat}}}{\varphi}{\ret{\nat}}{\varrho}}}{\zp}

Let \isVal{n}{\ret{\nat}}. Instantiate with $[n/x]$, by Lemma~\ref{lemma:ret}, STS
\isVal{\lam{y}{b}{\rec{n}{\ret{\zero}}{\_}{f}{\plus\; y\; f}}}{\ret{\arr{\isOf{y}{\ret{\nat}}}{\varphi}{\ret{\nat}}{\varrho}}}. Again,
NTS
\gammaToCompFixed{\isOf{y}{\ret{\nat}}}{\varphi}{1}{\rec{n}{\ret{\zero}}{\_}{f}{\plus\; y\; f}}{\ret{\nat}}{\varrho}. Let \isVal{m}{\ret{\nat}}. Instantiating with $[m/y]$, STS

\[
\gammaToComp{\cdot}{\varphi}{\rec{n}{\ret{\zero}}{\_}{f}{\plus\; m\; f}}{\ret{\nat}}{\varrho}
\]
By Lemma~\ref{lemma:recursor'}, STS
\begin{enumerate}
    \item \gammaToComp{\cdot}{\zp}{\ret{\zero}}{\isOf{y}{\ret{\nat}}}{\zp}: follows from
    Lemma~\ref{lemma:ret}
    \item \gammaToComp{\isOf{z}{\ret{\nat}},\isOf{f}{\ccomp{\zp}{\isOf{w}{\ret{\nat}}}{\zp}} }{\varphi_1}{\plus\; y\; f}{\ret{\nat}}{\varrho_1}.\\
    Let \isVal{z}{\ret{\nat}} and \isVal{f}{\ccomp{\zp}{\isOf{w}{\ret{\nat}}}{\zp}}. 
    Then $f = \thunk{F}$ s.t. \isComp{\zp}{F}{\isOf{w}{\ret{\nat}}}{\zp}. Hence,
    $F = \ret{u}$ for some \isVal{u}{\ret{\nat}}. Thus,
    \begin{align*}
    &\lsub{\ret{u}}{f}{\plus\; y\; f}\\
    \mapsto& \plus\; y\; u = \bind{\plus\; y}{r}{r\; u}\\
    \mapsto& \bind{\lam{y'}{b}{\rec{y}{\ret{y'}}{\_}{f}{\ret{\suc{f}}}}}{r}{r\; u}\\
    \mapsto& \lam{y'}{b}{\rec{y}{\ret{y'}}{\_}{f}{\ret{\suc{f}}}}\; u\\
    \mapsto& \rec{y}{\ret{u}}{\_}{f}{\ret{\suc{f}}}\\
    \mapsto^{2\toNat{y}+1}& \ret{v}
    \end{align*}
    Hence \evalCost{\lsub{\ret{u}}{f}{\plus\; y\; f}}{2\toNat{y}+5}{v}.
    Furthermore, \eval{\varphi_1 = \bind{\mult\;\toNum{2}\;y}{r}{\plus\;r\;\toNum{5}}}{\toNum{2\toNat{y}+5}},
    and \eval{\varrho_1}{\zero}. The result follows since 
    $\toNat{\toNum{2\toNat{y}+5}} \ge 2\toNat{y}+5 + \toNat{\zero}$.
\end{enumerate}

Simplified recurrence for $\mult$: $\Phi(\mult)(x)(y) = x^2y + (6-y)x + 1$.



\end{proof}



%\newcommand{\leaf}{\ensuremath{\mathtt{leaf}}}
\newcommand{\single}[1]{\ensuremath{\mathtt{single}(#1)}}
\newcommand{\ttwo}[2]{\ensuremath{\mathtt{t2}(#1,#2)}}
\newcommand{\tthree}[3]{\ensuremath{\mathtt{t3}(#1,#2,#3)}}
\newcommand{\ttrec}[8]{\mathtt{23TREC}(#1)(#2 \mid #3.#4 \mid #5.#6 \mid #7.#8)}
\newcommand{\tttree}[3]{\ensuremath{\mathtt{23tree}(#1,#2,#3)}}
\newcommand{\word}{\mathtt{word}}
\newcommand{\mw}[1]{\mathtt{mw}(#1)}
\newcommand{\wrec}[4]{\mathtt{WREC}(#1)(#2; #3.#4)}
\newcommand{\join}{\mathsf{join}}
\newcommand{\cmp}{\mathsf{cmp}}
\newcommand{\ifnat}[3]{\mathsf{if}(#1;#2;#3)}
\newcommand{\const}[1]{\lam{\_}{\_}{#1}}
\newcommand{\eq}{\mathsf{eq}}
\newcommand{\macro}{@\!\!=}
\newcommand{\wtonat}[1]{\mathsf{nat}(#1)}
\section{Trees}

Augment the language with the following values and commands:

\begin{align*}
    \mathsf{Node} \quad n &::= 
         \leaf
    \mid \single{v}
    \mid \ttwo{v_l}{v_r}
    \mid \tthree{v_l}{v_m}{v_r}\\
    \mathsf{Val} \quad v &::= \dots
    \mid \tttree{A}{v}{w}
    \mid (\isOf{n}{\tttree{A}{v}{w}})
    \mid \word
    \mid \mw{0} \dots \mw{2^{64}-1}\\
    \mathsf{Comp} \quad M &::= 
    \dots
    \mid \wrec{v}{M_0}{f}{M_1}\\
    &\mid \ttrec{v}{M_0}{a}{M_1}{l,r,f_l,f_r}{M_2}{l,m,r,f_l,f_m,f_r}{M_3}\\
\end{align*}

The semantics is extended with the computation rules:

\begin{mathpar}
\inferrule{
}{
  \step{\wrec{\mw{0}}{M_0}{f}{M_1}}{M_0}
}

\inferrule{
}{
  \step{\wtonat{\mw{w}}}{\toNum{w}}
}

\inferrule{
}{
 \step{\ttrec{\leaf}{M_0}{a}{M_1}{l,r,f_l,f_r}{M_2}{l,m,r,f_l,f_m,f_r}{M_3}}{M_0}
}

\inferrule{
}{
 \step{\ttrec{\single{v}}{M_0}{a}{M_1}{l,r,f_l,f_r}{M_2}{l,m,r,f_l,f_m,f_r}{M_3}}{[v/a]M_1}
}

\inferrule{
  t = \isOf{\ttwo{l}{r}}{\tttree{A}{s}{d}}\\
  \eval{d > \mw{0}}{\suc{\zero}}\\
  \eval{s > \mw{0}}{\suc{\zero}}\\
  v_l = \isOf{l}{\tttree{A}{S_l}{D_l}}\\
  v_r = \isOf{r}{\tttree{A}{S_r}{D_r}}\\
  \eval{d_l+1}{d}\\
  \eval{d_r+1}{d}\\
  \eval{s_l+s_r}{s}\\\\
  F_l \triangleq \thunk{\ttrec{v_l}{M_0}{a}{M_1}{l,r,f_l,f_r}{M_2}{l,m,r,f_l,f_m,f_r}{M_3}}\\
  F_r \triangleq \thunk{\ttrec{v_r}{M_0}{a}{M_1}{l,r,f_l,f_r}{M_2}{l,m,r,f_l,f_m,f_r}{M_3}}\\
}{
  \ttrec{t}{M_0}{a}{M_1}{l,r,f_l,f_r}{M_2}{l,m,r,f_l,f_m,f_r}{M_3} \mapsto
  [v_l/l,v_r/r,F_l/f_l,F_r/f_r]M_2
}

\inferrule{
  t = \isOf{\tthree{l}{m}{r}}{\tttree{A}{s}{d}}\\
  \eval{d > \mw{0}}{\suc{\zero}}\\
  \eval{s > \mw{0}}{\suc{\zero}}\\
  v_l = \isOf{l}{\tttree{A}{S_l}{D_l}}\\
  v_m = \isOf{m}{\tttree{A}{S_m}{D_m}}\\
  v_r = \isOf{r}{\tttree{A}{S_r}{D_r}}\\
  \eval{d_l+1}{d}\\
  \eval{d_m+1}{d}\\
  \eval{d_r+1}{d}\\
  \eval{s_l+s_m+s_r}{s}\\\\
  F_l \triangleq \thunk{\ttrec{n_l}{M_0}{a}{M_1}{l,r,f_l,f_r}{M_2}{l,m,r,f_l,f_m,f_r}{M_3}}\\
  F_m \triangleq \thunk{\ttrec{n_m}{M_0}{a}{M_1}{l,r,f_l,f_r}{M_2}{l,m,r,f_l,f_m,f_r}{M_3}}\\
  F_r \triangleq \thunk{\ttrec{n_r}{M_0}{a}{M_1}{l,r,f_l,f_r}{M_2}{l,m,r,f_l,f_m,f_r}{M_3}}\\
}{
 \ttrec{t}{M_0}{a}{M_1}{l,r,f_l,f_r}{M_2}{l,m,r,f_l,f_m,f_r}{M_3} \mapsto\\
  \ret{\thunk{[v_l/l,v_m/m,v_r/r,F_l/f_l,F_m/f_m,F_r/f_r]M_3}}
}

\end{mathpar}

Augment the types:
\begin{itemize}
\item \isType{\tttree{A}{s}{d}} 
  \begin{enumerate}
  \item \isType{A}
  \item \isVal{s}{\word}
  \item \isVal{d}{\word}
  \end{enumerate}

\item \isVal{\isOf{\leaf}{\tttree{A}{\mw{0}}{\mw{0}}}}{\tttree{A}{\mw{0}}{\mw{0}}}
\item \isVal{\isOf{\single{a}}{\tttree{A}{\mw{1}}{\mw{1}}}}{\tttree{A}{\mw{1}}{\mw{1}}}
  \begin{enumerate}
  \item \isVal{a}{A}
  \end{enumerate}
\item \isVal{\isOf{\ttwo{l}{r}}{\tttree{A}{s}{d}}}{\tttree{A}{s}{d}}
  \begin{enumerate}
  \item \isVal{l}{\tttree{A}{s_l}{d}}
  \item \isVal{r}{\tttree{A}{s_r}{d}}
  \item \eval{s_l+s_r}{s}, \eval{d_l+\mw{1}}{d}, \eval{d_r+\mw{1}}{d}
  \end{enumerate}

\item \isVal{\isOf{\tthree{l}{m}{r}}{\tttree{A}{s}{d}}}{\tttree{A}{s}{d}}
  \begin{enumerate}
  \item \isVal{l}{\tttree{A}{s_l}{d}}
  \item \isVal{m}{\tttree{A}{s_m}{d}}
  \item \isVal{r}{\tttree{A}{s_r}{d}}
  \item \eval{s_l+s_m+s_r}{s}, \eval{d_l+\mw{1}}{d}, \eval{d_m+\mw{1}}{d}, \eval{d_r+\mw{1}}{d}
  \end{enumerate} 
\end{itemize}

\begin{verbatim}
map := (\f.ret(\t.
  trec(t){
    ret(leaf:(0,0));
  | a.
    r <- f a; // phi_f(a) + 2
    ret(single(r):(1,1));
  | l.r.fl.fr.
    seq(fl; Fl. // 1
      seq(fr; Fr. // 1
        sl <- size Fl; // 3
        sr <- size Fr; // 3
        d <- depth Fl; // 3
        s' <- sl + sr; // 2
        d' <- d + 1; // 2
        ret(t2(Fl,Fr):(s',d'));
      )
    )
  | l.m.r.fl.fm.fr.
  seq(fl; Fl. // 1
    seq(fm; Fm. // 1
      seq(fr; Fr. // 1
        sl <- size Fl; // 3
        sm <- size Fm; // 3
        sr <- size Fr; // 3
        d <- depth Fl; //3 
        s' <- sl + sm; // 2
        s'' <- s' + sr; // 2
        d' <- d + 1; // 2
        ret(t3(Fl,Fm,Fr):(s'',d'));
      )
    )
  )
  }
));

num2 := (\t. trec(t){
    ret(0)
  | a.ret(0)
  | l.r.fl.fr.
    seq(fl; Fl. // 1
      seq(fr; Fr. // 1
        n <- Fl + Fr;
        n' <- n + 1;
        ret(n');
      )
    )
  | l.m.r.fl.fm.fr.
  seq(fl; Fl. // 1
    seq(fm; Fm. // 1
      seq(fr; Fr. // 1
        n <- Fl + Fm;
        n' <- n + Fr;
        ret(n');
      )
    )
  )
});

num3 := (\t. trec(t){
    ret(0)
  | a.ret(0)
  | l.r.fl.fr.
    seq(fl; Fl. // 1
      seq(fr; Fr. // 1
        n <- Fl + Fr;
        ret(n);
      )
    )
  | l.m.r.fl.fm.fr.
  seq(fl; Fl. // 1
    seq(fm; Fm. // 1
      seq(fr; Fr. // 1
        n <- Fl + Fm;
        n' <- n + Fr;
        n'' <- n' + 1;
        ret(n'');
      )
    )
  )
});
\end{verbatim}

\begin{lemma}(Map)\label{lemma:map}
  $\isVal{\mathtt{map}}{\arr{\arr{\isOf{a}{A}}{\varphi_1}{\isOf{b}{B}}{\varrho_1}}{\zp}{\arr{\isOf{t}{\tttree{A}{s}{d}}}{\varphi}{\tttree{B}{s}{d}}{\varrho}}{\zp}}$
  where $\varphi \triangleq \ttrec{t}{\zp^+}{a}{\bind{\varphi_1}{p}{\plus\;p\;\toNum{3}}}{l,r,f_l,f_r}{\bind{f_l}{F_l}{\bind{f_r}{F_r}{\bind{\plus\;F_l\;F_r}{p}{\plus\;p\;\toNum{16}}}}}{l,m,r,f_l,f_m,f_r}
    {\bind{f_l}{F_l}{\bind{f_m}{F_m}{\bind{f_r}{F_r}{\bind{\plus\;F_l\;F_m}{p}{\bind{\plus\;p\;F_r}{p'}{\plus\;p'\;\toNum{22}}}}}}}$.
  and $\varrho \triangleq \ttrec{t}{\zp}{a}{\varrho_1}{l,r,f_l,f_r}{\bind{f_l}{F_l}{\bind{f_r}{F_r}{\plus\;F_l\;F_r}}}{l,m,r,f_l,f_m,f_r}
    {\bind{f_l}{F_l}{\bind{f_m}{F_m}{\bind{f_r}{F_r}{\bind{\plus\;F_l\;F_m}{p}{\plus\;p\;F_r}}}}}$.\\
    Informally:
    $\Phi(map)(t) = \begin{cases}
      1 \text{ if } t = \leaf\\
      \varphi_1 + 3 \text{ if } t = \single{a}\\ 
      \Phi(map)(l) + \Phi(map)(r) + 16 \text{ if } t = \ttwo{l}{r}\\
      \Phi(map)(l) + \Phi(map)(m) + \Phi(map)(r) + 22 \text{ if } t = \tthree{l}{m}{r}\\
    \end{cases}$  
\end{lemma}
  Instantiate with \isVal{f}{\arr{\isOf{a}{A}}{\varphi_1}{\isOf{b}{B}}{\varrho_1}}. By Lemma~\ref{lemma:ret}, suffices to show 
  \isVal{\lam{t}{\_}{\dots}}{\arr{\tttree{A}{s}{d}}{\varphi}{\tttree{B}{s}{d}}{\varrho}}. Instantiate with 
  \isVal{t}{\tttree{A}{s}{d}}, and suffices to show 
  \gammaToComp{\cdot}{\varphi}{\mathtt{trec}(\dots)}{\tttree{B}{s}{d}}{\zp}.
\begin{proof}
Induction on $t$.
  \begin{itemize}
    \item $t = \leaf$: then \evalCost{\texttt{trec(t){leaf => ret(leaf)\dots}}}{1}{\texttt{ret(leaf)}}. 
      The potential $\zp^+$ suffices. 
    \item $t = \single{a}$: then 
      \begin{align*}
        &\texttt{trec(t){r <- f a; ret(single(r));\dots}}\\
        \mapsto& \texttt{r <- f a; ret(single(r));}\\
        \mapsto& \texttt{r <- (...); ret(single(r));}\\
        \mapsto^c& \texttt{r <- ret(v); ret(single(r));}\\
        \mapsto& \texttt{ret(single(v));}
      \end{align*}
      and \eval{\varphi}{p}, \eval{\varrho}{p'}, such that $\toNat{p} \ge c + \toNat{p'}$.
      Thus \eval{\bind{\varphi_1}{p}{\plus\;p\;\toNum{3}}}{\toNat{p}+3} suffices to cover the cost and
      leaves the same the remaining potential $\varrho$.
    \item $t = \ttwo{l}{r}$, for $\isVal{l}{\tttree{A}{s_1}{d'}}$ and $\isVal{r}{\tttree{A}{s_2}{d'}}$. Then 
      \begin{align*}
        &\texttt{trec(t){\dots}} \\
        \mapsto& \texttt{seq(fl; Fl.\dots)}\\
        \mapsto^{c_l}& \texttt{seq(thunk(ret(v1))); Fl.\dots}\\
        \mapsto& \texttt{seq(fr; Fr.\dots)}\\
        \mapsto^{c_r}& \texttt{seq(thunk(ret(v2)); Fr.\dots)}\\
        \mapsto& \texttt{sl <- size Fl;\dots}\\
        \mapsto^3& \texttt{sr <- size Fr;\dots}\\
        \mapsto^3& \texttt{d <- depth Fl;\dots}\\
        \mapsto^3& \texttt{s' <- sl + sr;\dots}\\
        \mapsto^2& \texttt{d' <- d + 1;\dots}\\
        \mapsto^2& \texttt{ret(t2(Fl,Fr):(s',d'));\dots}\\
      \end{align*}
      Hence the cost is $c = c_l+c_r + 16$.
      Let $F_l,F_r$ be the recursive results of $\varphi$, and $G_l,G_r$ of $\varrho$.
      . By induction, 
      $\toNat{F_l} \ge c_l + \toNat{G_l}$ and $\toNat{F_r} \ge c_r + \toNat{G_r}$.
      Thus the potential \eval{\varphi}{\toNat{F_l} + \toNat{F_r} + 16} is sufficient 
      to cover the cost and remaining potential.
    \item $t = \tthree{l}{m}{r}$: similar to above. 
  \end{itemize}
\end{proof}

\begin{lemma}
  Let \isVal{t}{\tttree{A}{s}{d}}, \eval{\texttt{size(t)}}{n}, \eval{\texttt{num2(t)}}{n_2}, and \eval{\texttt{num3(t)}}{n_3}.
  Given \isVal{f}{\arr{\isOf{\_}{A}}{\varphi_1}{B}{\varrho_1}},
  if \eval{\Phi(map_f)(t)}{v}, then \eval{\phi(map_f)(n,n_2,n_3)}{v}, where
  $\phi(map_f)(n,n_2,n_3) \triangleq 1 + (3 + \varphi_1)\cdot n + 16n_2 + 22n_3$.
\end{lemma}

\begin{proof}
  Induction on $t$.
  \begin{itemize}
    \item $t = \leaf$: then \eval{\Phi(map_f)(t)}{1}, and \eval{\texttt{size(t)}}{0}, \eval{\texttt{num2(t)}}{0}, and \eval{\texttt{num3(t)}}{0}.
      Thus \eval{\phi(map_f)(0,0,0)}{1}. 
    \item $t = \single{a}$:  then \step{\Phi(map_f)(t)}{3 + \varphi_1},  and \eval{\texttt{size(t)}}{1}, \eval{\texttt{num2(t)}}{0}, and \eval{\texttt{num3(t)}}{0}.
      Thus \step{\phi(map_f)(1,0,0)}{3 + \varphi_1}. 
    \item $t = \ttwo{l}{r}$: then \step{\Phi(map_f)(t)}{\Phi(map_f)(l) + \Phi(map_f)(r) + 16}, and \step{\texttt{size(t)}}{size(l) + size(r)}, 
      \step{\texttt{num2(t)}}{1 + num2(l) + num2(r)}, and \step{\texttt{num3(t)}}{num3(l) + num3(r)}.
      Thus 
      \begin{align*}
        & \phi(map_f)(size(l) + size(r),1 + num2(l) + num2(r),num3(l) + num3(r))\\
        &\mapsto 1 + (3 + \varphi)(size(l) + size(r)) + 16(1 + num2(l) + num2(r)) + 22(num3(l) + num3(r))\\
        &\mapsto \left(1 + (3 + \varphi)size(l) + 16(num2(l)) + 22(num3(l))\right) + \\
                 &\left(1 + (3 + \varphi)size(r) + 16(num2(r)) + 22(num3(r))\right) + 16\\
        &= \Phi(map_f)(l) + \Phi(map_f)(r) + 16 \tag{I.H.}
      \end{align*}
      \item $t = \tthree{l}{m}{r}$: symmetric.
  \end{itemize}
\end{proof}

\begin{verbatim}
foldl := (\f. ret(\t. trec(t){
    ret(\init. ret(init))
  | a.ret(\init. r <- f(init,a); ret(r);)
  | l.r.fl.fr.
    ret(\init.
    seq(fl; Fl. // 1
      bl <- Fl init; // ?
      seq(fr; Fr. // 1
        br <- Fr bl;
        ret(br);
      );
      )
    )
  | l.m.r.fl.fm.fr.
  ret(\init.
  seq(fl; Fl. // 1
    bl <- Fl init; // ?
    seq(fm; Fm. // 1
      bm <- Fm bl;
      seq(fr; Fr. // 1
        br <- Fr bm;
        ret(br);
      );
    );
  ))
}));
\end{verbatim}

\begin{lemma}(Fold?)
  \isVal{\texttt{foldl}}{\arr{\arr{\isOf{\_}{\dprod{A}{B}}}{\varphi_1}{B}{\varrho_1}}{\zp}{\arr{\isOf{t}{\tttree{A}{s}{d}}}{\zp}{\arr{\isOf{init}{B}}{\varphi}{B}{\zp}}{\zp}}{\zp}}
  where \\
  $\varphi(\texttt{foldl}\;f\;init)(t) = \begin{cases}
    1 \text{ if } t = \leaf\\
    \varphi_1 + 2 \text{ if } t = \single{a} \\
    \varphi(\texttt{foldl}\;f\;init)(l) + \varphi(\texttt{foldl}\;f\;b)(r) + 8 \text{ if } t = \ttwo{l}{r}\\
    \varphi(\texttt{foldl}\;f\;init)(l) + \varphi(\texttt{foldl}\;f\;bl)(m) + \varphi(\texttt{foldl}\;f\;br)(r) + 11\text{ if } t = \tthree{l}{m}{r}
  \end{cases}$
\end{lemma}

\iffalse

\begin{lemma}(Tree Recursor)\label{lemma:trec}
  Given 
  \begin{gather*}
    \gammaToVal{\cdot}{t}{\tttree{A}{s}{d}}\\
    \gammaToComp{\cdot}{\varphi_0}{M_0}{[\leaf/x]B}{\varrho_0}\\
    \gammaToComp{\isOf{a}{A}}{\varphi_1}{M_1}{[\single{a}/x]B}{\varrho_0}\\
    \gammaToComp{\isOf{l}{\ret{\tttree{A}{s_1}{d}}},\isOf{r}{\ret{\tttree{A}{s_2}{d}}},\isOf{f_l}{\ccomp{\zp}{[l/x]B}{\zp}},\isOf{f_r}{\ccomp{\zp}{[r/x]B}{\zp}}}
      {\varphi_2\\}{M_2}{[\ttwo{l}{r}/x]B}{\varrho_2} \tag{*}\\
    \gammaToComp{\isOf{l}{\ret{\tttree{A}{s_1}{d}}},\isOf{m}{\ret{\tttree{A}{s_2}{d}}},\isOf{r}{\ret{\tttree{A}{s_3}{d}}},\isOf{f_l}{\ccomp{\zp}{[l/x]B}{\zp}},\\
      \isOf{f_m}{\ccomp{\zp}{[m/x]B}{\zp}},\isOf{f_r}{\ccomp{\zp}{[r/x]B}{\zp}}}
      {\varphi_3\\}{M_3}{[\tthree{l}{m}{r}/x]B}{\varrho_3}
  \end{gather*}
  Then \gammaToComp{\cdot}{\varphi}{\ttrec{t}{M_0}{a}{M_1}{l,r,f_l,f_r}{M_2}{l,m,r,f_l,f_m,f_r}{M_3}}{[t/x]B}{\varrho} where
  \begin{gather*}
  \varphi \triangleq \ttrec{t}{\varphi_0^+}{a}{\varphi_1^+}{l,r,f_l,f_r}{?\varphi_2^+}{l,m,r}{?\varphi_3^+}\\
  \varrho \triangleq \ttrec{t}{\varrho_0}{a}{\varrho_1}{l,r,f_l,f_r}{\varrho_2}{l,m,r,f_l,f_m,f_r}{\varrho_3}
  \end{gather*}
\end{lemma}

\begin{proof}
Induction on structure of $t$:
  \begin{itemize}
    \item $t = \leaf$, $t = \single{a}$: similar to Lemma~\ref{lemma:recursor'}
    \item $t = \ttwo{l}{r}$, for some \isOf{l}{\tttree{A}{s_1}{d}}, \isOf{r}{\tttree{A}{s_2}{d}}:
      Then \step{\ttrec{t}{M_0}{a}{M_1}{l,r,f_l,f_r}{M_2}{l,m,r,f_l,f_m,f_r}{M_3}}{[v_l/l,v_r/r,F_l/f_l,F_r/f_r]M_2}. 
      where $F_l \triangleq \thunk{\ret{\thunk{\ttrec{v_l}{M_0}{a}{M_1}{l,r,f_l,f_r}{M_2}{l,m,r,f_l,f_m,f_r}{M_3}}}}$ 
      and $F_r \triangleq \thunk{\ret{\thunk{\ttrec{v_r}{M_0}{a}{M_1}{l,r,f_l,f_r}{M_2}{l,m,r,f_l,f_m,f_r}{M_3}}}}$. 
      By Lemma~\ref{lemma:ret}, $\isOf{F_l}{\ccomp{\zp}{[l/x]B}{\zp}}$,
      Instantiating * gives \evalCost{[v_l/l,v_r/r,F_l/f_l,F_r/f_r]M_2}{c}{}

  \end{itemize}
\end{proof}

\fi

\begin{verbatim}

to_nat = \d. WREC(d){zero; p.f. r <- f 1; ret(suc(r))}

if(v;M_t;M_f) @= rec(v){M_f; _,_.M_t}

size := (\t. tmatch(t){
    s.d.ret(s)
  | s.d.a.ret(s)
  | s.d.l.r.ret(s)
  | s.d.l.m.r.ret(s)
});

depth := (\t. tmatch(t){
    s.d.ret(d)
  | s.d.a.ret(d)
  | s.d.l.r.ret(d)
  | s.d.l.m.r.ret(d)
});

T := (t2(single(0):(1,1),single(1):(1,1)):(2,2));
T1 := (t3(T,T,T):(6,3));

balance2 := (\T1. ret(\T2. tmatch(T1){
    s.d.ret(0);
  | s.d.a.ret(0);
  | s.d.l.r.
    s2 <- size T2; 
    s' <- s + s2;
    ret(t3(l,r,T2):(s',d));
  | s.d.l.m.r.
    sl <- size l; // 3
    sm <- size m; // 3
    sl' <- sl + sm; //2
    sr <- size r; //3
    s2 <- size T2; //3
    sr' <- sr + s2; //2
    d' <- d + 1; //2
    s' <- sl' + sr'; //a 2
    ret(t2(t2(l,m):(sl',d), t2(r,T2):(sr',d)):(s',d'));
}));

balance3 := (\T1. ret(\T2. ret(\T3.
  tmatch(T1){
    s.d.ret(0);
  | s.d.a.ret(0);
  | s.d.l.r.
    s2 <- size T2;
    s3 <- size T3;
    s23 <- s2 + s3;
    s' <- s + s23;
    d' <- d + 1;
    ret(t2(t2(l,r):(s,d),t2(T2,T3):(s23,d)):(s',d'));
  | s.d.l.m.r.
    sl <- size l; // 3
    sm <- size m; // 3
    sl' <- sl + sm; // 2
    sr <- size r; // 3
    s2 <- size T2; // 3
    s3 <- size T3; // 3
    s23 <- s2 + s3; // 2
    d' <- d + 1; // 2
    s' <- s + s23; // 2
    ret(t2(t3(l,m,r):(s,d), t2(T2,T3):(s23,d)):(s',d'));
})));

insertleft := (\x.ret(\t. 
  trec(t){
     ret(thunk(ret(single(x):(1,1))));
   | a.ret(thunk(ret(t2(single(x):(1,1), single(a):(1,1)):(2,2))));
   | l.r.fl.fr.
      ret(thunk(
        seq(fl; Fl.
        d <- depth Fl;
        dl <- depth l;
        b <- d > dl;
        if(b)
        then{
          balance2 Fl r
          }
        else{
          sl <- size Fl;
          sr <- size r;
          s' <- sl + sr;
          d' <- dl + 1;
          ret(t2(Fl,r):(s',d'));
          };)
      ));

   | l.m.r.fl.fm.fr. 
      ret(thunk(
        seq(fl; Fl. 
        d <- depth Fl;
        dl <- depth l;
        b <- d > dl;
        if(b) 
        then{
          balance3 Fl m r
          }
        else{
          sl <- size Fl;
          sm <- size m;
          sr <- size r;
          s' <- sl + sm;
          s'' <- s' + sr;
          d' <- dl + 1;
          ret(t3(Fl,m,r):(s'',d'));
          };)
      ));
  };
));

// 6 defs
T' <- insertleft 5 T1;
seq(T'; t'.ret(t'));
\end{verbatim}

\iffalse
full_left = \t.n. 
  23TREC(t){
    leaf => ret(zero);
    single a => ret(zero);
    t2(l,r,sl,sr,dl,dr,fl,fr) => 
      d <- dl + 1;
      b <- n == d;
      if(b; ret(zero); ret(fl));
    t3(l,m,r,sl,sm,sr,dl,dm,dr,fl,fm,fr) => 
      d <- dl + 1;
      b <- n == d;
      if(b; ret(suc(zero)); ret(fl));
  }


mk2 = \a.b. 
  case a of
    | inl l =>
      case b of
      | inl r =>
        d1 <- depth l;
        d2 <- depth r;
        b <- d1 < d2;
        if(b;
          23TREC(r){
            leaf => ret(!);
            single a => ret(inl(single(a)));
            t2(rl,rr,_,_) => ret(inl(t3(l,rl,rr)));
            t3(rl,rm,rr) => ret(inl(t2(t2(l,rl),t2(rm,rr))));
          };
          b <- d1 > d2;
          if(b;
            23TREC(l){
              leaf => ret(!);
              single a => ret(inl(single(a)));
              t2(ll,lr,_,_) => ret(inl(t3(ll,lr,r)));
              t3(ll,lm,lr) => ret(inl(t2(t2(ll,lm),t2(lr,r))));
            };
            ret(inl(t2(l,r)))
          )
        )
      | inr (r1,r2) => 
        d1 <- depth l;
        d2 <- depth r1;
        b <- d1 < d2;
        if(b;
          23TREC(r1){
            leaf => ret(!);
            single a => ret(inl(t2(r1,r2)));
            t2(r1l,r1r,_,_) => ret(inl(t2(t3(l,r1l,r1r),r2)));
            t3(r1l,r1m,r1r) => ret(inl(t3(t2(l,r1l),t2(r1m,r1r),r2)));
          };
          b <- d1 > d2;
          if(b;
            23TREC(l){
              leaf => ret(!);
              single a => ret(inl(single(a)));
              t2(ll,lr,_,_) => ret(inl(t2(l,t2(r1,r2))));
              t3(ll,lm,lr) => ret(inl(t2(t3(ll,lm,lr),t2(r1,r2))));
            };
            ret(inl(t3(l,r1,r2)))
          )
        )
    | inr (l1,l2) => 
      case b of
      | inl r =>
        d1 <- depth l1;
        d2 <- depth r;
        b <- d1 < d2;
        if(b;
          ret(inl(t2(t2(l1,l2),r)));
          b <- d1 > d2;
          if(b;
            23TREC(l2){
              leaf => ret(!);
              single a => ret(inl(t2(l1,single(a))));
              t2(rl,rr,_,_) => ret(inl(t2(l1,t3(rl,rr,r))));
              t3(rl,rm,rr) => ret(inl(t3(l,t2(rl,rm), t2(rr,r))));
            };
            ret(inl(t3(l1,l2,r)))
          )
        )
      | inr(r1,r2) => 
        d1 <- depth l1;
        d2 <- depth r1;
        b <- d1 < d2;
        if(b;
          ret(inl(t3(t2(l1,l2),r1,r2)));
          b <- d1 > d2;
          if(b;
            ret(inl(t3(l1,l2,t2(r1,r2))));
            ret(inl(t2(t2(l1,l2),t2(r1,r2))));
          )
        )

join' = \tr1.
  23TREC(tr1){
    leaf => \tr2. ret(inl(tr2));
    single a => \tr2.
      23TREC(tr2){
        leaf => ret(inl(tr1));
        single b => ret(inr(single a, single b));
        t2(l,r,fl,fr) => mk2(fl,r);
        t3(l,m,r,fl,fm,fr) => mk3(fl,m,r);
      }
    t2(l,r,fl,fr) => \tr2.
      d1 <- depth tr1;
      d2 <- depth tr2;
      b <- d1 < d2;
      if(b;
        23TREC(tr2){
         leaf => ret(!);
         single a => ret(!);
         t2(l,r,gl,gr) => mk2(gl,r);
         t3(l,m,r,gl,gm,gr) => mk3(gl,m,r);
        };
        b <- d1 > d2;
        if(b;
          r' <- fr tr2;
          mk2(l,r');;
          mk2(tr1,tr2)
        )
      )
    t3(l,m,r,fl,fm,fr) => \tr2.
      d1 <- depth tr1;
      d2 <- depth tr2;
      b <- d1 < d2;
      if(b;
        23TREC(tr2){
         leaf => ret(!);
         single a => ret(!);
         t2(l,r,gl,gr) => mk2(gl,r);
         t3(l,m,r,gl,gm,gr) => mk3(gl,m,r);
        };
        b <- d1 > d2;
        if(b;
          r' <- fr tr2;
          mk3(l,m,r');;
          mk2(tr1,tr2)
        )
      )
  }
\fi

\iffalse
  \left
  \mathsf{full\_left} \triangleq 
    \lam{t}{\_}{
      \lam{n}{\_}{
      \ttrec{t}{\const{\zero}}{\_}{\const{\zero}}
        {\_,\_,s_l,d_l,\_,\_,f_l,\_}\\{
          \bind{d_l+1}{d}{\bind{\eq\;n\;d}{b}{\ifnat{b}{\ret{b}}{\ret{f_l}}}}}\\
        {\_,\_,\_,s_l,d_l,\_,\_,\_,\_,f_l,\_,\_}{
          \bind{d_l+1}{d}{\bind{\eq\;n\;d}{b}{\ifnat{b}{\ret{\zero}}{\ret{f_l}}}}}
        }
        }\\
  \mathsf{full\_right} \triangleq 
    \lam{t}{}{
      \ttrec{t}{\const{\zero}}{\_}{\const{\zero}}
        {\_}{\lam{n}{\_}{\eq\;n\;d}}
          {\_,\_,f_l,f_r}{\bind{n-1}{n'}{f_r\; n'}}
        }\\
\join \triangleq 
  \lam{t_1}{\tttree{A}{s_1}{d_1}}{
    \lam{t_2}{\tttree{A}{s_2}{d_2}}{
    }}
\fi


\section{Construction}

Notation:

\begin{enumerate}
  \item $A \sim A' \downarrow \alpha \in \tau$ when \eval{A}{v}, \eval{A'}{v'}, and 
  $\tau(v,v',\alpha)$.
  \item $a : \alpha \rhd B \sim B' \downarrow \beta \in \tau$ when 
  for all $v,v'$ s.t. $\alpha(v,v')$, \sameType{[v/a]B}{[v'/a]B'}{\beta_{v,v'}}{\tau}.
\item \sameCComp{}{M}{M'}{\alpha}{P}{a.Q} when \evalCost{M}{c}{v}, \evalCost{M'}{c'}{v'}, 
  $\alpha(v,v')$. 
    Furthermore, \eval{P}{\bar{p}}, \eval{[v/a]Q}{\bar{q}}, and 
    $p \ge c + q$ and $p \ge c' + q$.
  \item \sameComp{}{M}{M'}{\alpha} when \sameCComp{}{M}{M'}{\alpha}{P}{a.Q} for some 
    $P$ and $a.Q$. 
  \item $\sameCComp{\isOf{a}{\alpha}}{M}{M'}{\beta}{P}{b.Q}$ when 
  for all $v,v'$ s.t. $\alpha(v,v')$, 
    \evalCost{[v/a]M}{c}{u}, \evalCost{[v'/a]M'}{c'}{u'}, and 
    $\beta_{v,v'}(u,u')$.
    Furthermore, \eval{[v/a]P}{\bar{p}}, \eval{[v/a,u/b]Q}{\bar{q}}, and 
    $p \ge c + q$ and $p \ge c' + q$.
\item \sameComp{\isOf{a}{\alpha}, \isOf{b}{\beta}}{M}{M'}{\gamma} when 
  for all $v,v'$ s.t. $\alpha(v,v')$, 
  for all $u,u'$ s.t. $\beta_{v,v'}(u,u')$, 
  \eval{[v/a,u/b]M}{w}, \eval{[v'/a,u'/b]M'}{w'}, and $\gamma_{{v,v}_{u,u'}}(w,w')$.
  \item $\omega = \mu \alpha.\, \{(\zero,\zero)\} \cup \{(\suc{v},\suc{v'}) \mid \alpha(v,v')\}$
\end{enumerate}

Construct the type systems $\tau_n$ as the least fixed-point(s) of the 
monotone function defined by the various clauses: 

\begin{align*}
  \text{Nat} &= \{(\nat,\nat,\phi) \mid \phi = \omega\}\\
  \iffalse
  \text{Pot}(\tau,\pi) &= 
    \{(\relpotty{A}{a}{B}, \relpotty{A'}{a}{B'}, \phi) \mid &\\
    &\exists \alpha.\, \sameType{A}{A'}{\alpha}{\tau} \\
    &\land\exists \beta.\, \sameTypeOne{a}{\alpha}{B}{B'}{\beta}{\pi} \\
    &\land \phi = \kappa(\alpha,a.\beta)
    \}\\
    \fi
  \text{Fun}(\tau, \pi) &=
  \{(\arrabt{A}{a.B}{a.P,a.b.Q}, \arrabt{A'}{a.B'}{a.P', a.b.Q'}, \phi) 
  \mid & \\
  &\exists \alpha.\, \sameType{A}{A'}{\alpha}{\tau} \\
  &\land\exists \beta.\, \sameTypeOne{a}{\alpha}{B}{B'}{\beta}{\pi} \\
  &\land\sameComp{\isOf{a}{\alpha}}{P}{P'}{\omega} \\
  &\land\sameComp{\isOf{a}{\alpha},\isOf{b}{\beta}}{Q}{Q'}{\omega} \\
  \land\phi &= \{(\lam{a}{\_}{M}, \lam{a}{}{M'}) \mid
  \sameCComp{\isOf{a}{\alpha}}{M}{M'}{\beta}{P}{b.Q}
   \}
  \}\\
  \text{Comp}(\tau) &= \{
  (\ccomp{P}{\isOf{a}{A}}{Q}, \ccomp{P'}{\isOf{a}{A'}}{Q'}, \phi) \mid \\
  &\exists \alpha.\, \sameType{A}{A'}{\alpha}{\tau}\\
  &\land \sameComp{}{P}{P'}{\omega}\\ 
  &\land \sameComp{\isOf{a}{\alpha}}{Q}{Q'}{\omega}\\
  \land\phi &= \{(\thunk{M}, \thunk{M'}) \mid \sameCComp{}{M}{M'}{\alpha}{P}{a.Q} \}
  \}\\
  \text{Eq}(\tau) &= \{
    (\eqty{A}{M}{N}, \eqty{A'}{M'}{N'}, \phi) \mid \\
  &\exists \alpha.\, \sameType{A}{A'}{\alpha}{\tau}\\
  &\land\sameComp{}{M}{M'}{\alpha}\\
  &\land\sameComp{}{N}{N'}{\alpha}\\
  \land\phi &= \{(\triv, \triv) \mid \sameComp{}{M}{N}{\alpha}
  \}
  \}\\
  \text{Intersection}(\tau) &= \{
    (\intty{\isOf{x}{A}}{B}, \intty{\isOf{x}{A'}}{B'},\phi) \mid \\
    &\exists \alpha.\, \sameType{A}{A'}{\alpha}{\tau} \\
    &\land\exists \beta.\, \sameTypeOne{a}{\alpha}{B}{B'}{\beta}{\tau} \\
    \land\phi &= \{(u,u') \mid \forall \alpha(v,v').\, \beta_{v,v'}(u,u')\}
  \}\\
  \text{Subset}(\tau) &= \{
    (\subsetty{\isOf{x}{A}}{B}, \subsetty{\isOf{x}{A'}}{B'},\phi) \mid \\
    &\exists \alpha.\, \sameType{A}{A'}{\alpha}{\tau} \\
    &\land\exists \beta.\, \sameTypeOne{a}{\alpha}{B}{B'}{\beta}{\tau} \\
    \land\phi &= \{(v,v') \mid \alpha(v,v') \land \exists u,u'.\beta_{v,v'}(u,u')\}
  \}\\
  \text{Type}(\tau) &= \{(\type{i}, \type{i}, \phi) \mid \exists \phi.\, \tau(\type{i}, \type{i}, \phi)\}\\
  v_n &= \{(\type{i}, \type{i}, \phi) \mid i < n \land \phi = \{(A,B) 
  \mid \exists \alpha.\, \sameType{A}{B}{\alpha}{\tau_i}\}\}\\
  \text{Types}(v,\tau) &= \text{Fun}(\tau,\tau) \cup \text{Pot}(\tau,\tau) \cup \text{Nat} \cup \text{Comp}(\tau) \cup \text{Type}(v)\\
  \tau_n &= \mu \tau.\, \text{Types}(v_n,\tau)
\end{align*}

Let the set of possible type systems be $\mathcal{D} = 
\mathcal{P}(\term \times \term \times \mathcal{P}(\term \times \term))$.
Since the various clauses have disjoint images, it suffices to show that each is monotone.

% TODO: monotone 
\iffalse
\begin{lemma}\label{lemma:monotone}
For all $v \in \mathcal{D}$, $f : \mathcal{D} \to \mathcal{D}, \tau \mapsto \text{Types}(v,\tau)$ is a monotone function.
\end{lemma}

\begin{proof}

\begin{itemize}
\item $\text{Fun}$:\\
Suppose $\tau \subseteq \tau'$. NTS $\text{Fun}(\tau,\tau) \subseteq \text{Fun}(\tau',\tau')$. 
Suppose $(\arr{A}{\varphi}{B}{\varrho}, \arr{A'}{\varphi'}{B'}{\varrho'}, \phi)  \in \text{Fun}(\tau,\tau)$. 
Unrolling the definition, we know:
\begin{enumerate}
  \item there is a $\alpha$ s.t. \sameType{A}{A'}{\alpha}{\tau}. Then by assumption, \sameType{A}{A'}{\alpha}{\tau'}. 
  \item there is a $\beta$ s.t. \fact{\sameTypeOne{a}{\alpha}{B}{B'}{\beta}{\tau}}{1}. We show
\sameTypeOne{a}{\alpha}{B}{B'}{\beta}{\tau'}. Let $v,v'$ s.t. $\alpha(v,v')$, and show 
\sameType{[v/a]B}{[v'/a]B'}{\beta_{v,v'}}{\tau'}. But this holds by \applyAss{1}.
\end{enumerate}
Since the next three constraints are determined by $\alpha$ and $\beta$ which did not change, we conclude 
$(\arr{A}{\varphi}{B}{\varrho}, \arr{A'}{\varphi'}{B'}{\varrho'}, \phi)  \in \text{Fun}(\tau',\tau')$.
\end{itemize}
\end{proof}
\fi

A type system is a possible type system $\tau$ which has the following properties:
\begin{enumerate}
\item Unicity: $\tau(A,B,\phi)$ and $\tau(A,B,\phi')$ implies $\phi = \phi'$.
\item PER valuation: $\tau(A,B,\phi)$ implies $\phi$ is a PER.
\item Symmetry: $\tau(A,B,\phi)$ implies $\tau(B,A,\phi)$. 
\item Transitivity: $\tau(A,B,\phi)$ and $\tau(B,C,\phi)$ implies $\tau(A,C,\phi)$.
\end{enumerate}

\begin{lemma}(TYPES)\label{lemma:types}
If $v$ is a type system, then $\tau^* = \mu \tau.\, \text{Types}(v,\tau)$ is a type system.
\end{lemma}


\begin{lemma}
$\tau_n$ is a type system for all $n$.
\end{lemma}

\begin{proof}
  Induction on $n$ with Lemma~\ref{lemma:types}
\end{proof}

%\section{Stack Machine}

\begin{mathpar}
\inferrule{}{
  \final{\retState{\cdot}{\ret{v}}}
}

\inferrule{}{
  \doState{k}{\ret{v}} \mapsto \retState{k}{\ret{v}}
}

\inferrule{}{
  \retState{k;\seq{\thunkcst{\_}}{x}{M}}{\ret{v}} \mapsto \doState{k}{[v/x]M}
}

\inferrule{}{
  \doState{k}{\dbr{dec}{base}{ind}{n}{s}{r}} \mapsto \\\\
  \doState{k; \seq{\thunkcst{\_}}{x}{\case{x}{p}{base\;n\;s\;r\;p}{\_}{ind\;n\;s\;\;r\;(\lam{t,r'}{}{\dbr{dec}{base}{ind}{\suc{n}}{s\oplus_n t}{r'}})}
}}{dec\,n\,s}
}

\inferrule{}{
  \doState{k}{\dbrp{dec}{base}{ind}{n}{s}{r}} \mapsto \\\\
  \doState{k; \seq{\thunkcst{\_}}{x}{\case{x}{p}{base\;n\;s\;r\;p}{\_}{ind\;n\;s\;\;r\;(\lam{t,r'}{}{\dbrp{dec}{base}{ind}{\suc{n}}{s\oplus_n t}{r'}})}
}}{dec\,n\,s}
}

\end{mathpar}

Recall the two spread laws:
\begin{gather*}
\isVal{S_1}{\pityc{\isOf{n}{\lift{\nat}}}{
\pityc{\isOf{s}{\lift{\nseq{n}}}}{R(n,s) \tolr \sigmatyc{\isOf{m}{\lift{\nat}}}{R(\suc{n},s \oplus_n m)}}}}\\
\isVal{S_2}{\pityc{\isOf{n}{\lift{\nat}}}{
  \pityc{\isOf{s}{\lift{\nseq{n}}}}{R(n,s) \tolr 
  \pityc{\isOf{m}{\lift{\nat}}}{m \le n \tolr R(m,s \restriction_m)}}}}
\end{gather*}

Given a spread $R$ and an admissible prefix \isVal{s}{\nseq{n}} with \isVal{r}{\squash{R(n,s)}}, 
we can extend it to an infinite sequence. 
\begin{align*}
  extend(s,r) &\triangleq \lam{k}{}{
  \ifthen{k <_b n}{s}{
  \bind{
  \rec{k+1-n}{(s,r)}{p}{f\\}{
  \bind{\fst{f}\\}{&s'}{
    \bind{\snd{f}\\}{&r'}{
      \bind{S_1\,(p+n)\,s'\,r'}{&x}{
        \bind{\fst{x}}{m}{
          \bind{\snd{x}}{r''}{
            (s' \oplus_{p+n} m, r'')
          }
        }
      }
      }
    }
  }\\
}{(s',r')}{&s', r'}
  }
}\\
  \extSeq{s}{r} &\triangleq \lam{k}{}{\bind{extend(s,r)\,k}{f}{
    \bind{\fst{f}}{s}{s(k)}}}
\end{align*}

Let \isVal{m}{\lift{\nat}}. We need to show $R(m, \extSeq{s}{r}\restriction_m)$.
If $m \le n$, then it follows from the second spread law. 
Otherwise, \eval{extend(s,r)\,(m-1)}{(s',r')}, which is a pair of 
length $m$ extension of $s$ and proof that $R(m, s')$. 
This suffices since $s' = \extSeq{s}{r}\restriction_m$.

\begin{lemma}
If
\begin{align*}
    &\isVal{R}{\pityc{\isOf{n}{\lift{\nat}}}{\embed\nseq{n} \tolr \embed\type{}}}\\
    &\isVal{B}{\pityc{\isOf{n}{\lift{\nat}}}{\embed\nseq{n} \tolr \embed\type{}}}\\
    &\isVal{P}{\pityc{\isOf{n}{\lift{\nat}}}{\embed\nseq{n} \tolr \embed\type{}}}\\
    &\isVal{bar}{\pityc{\isOf{s}{\lift{\baire}}}{\intty{\isOf{m}{\lift{\nat}}}{\squash{R(m,s)}} 
      \tolr \squash{\lift{\sigmaty{\isOf{n}{\lift{\nat}}}{B(n,s)}}}}}\\
    &\isVal{ind}{\pityc{\isOf{n}{\lift{\nat}}}{\pityc{\isOf{s}{\lift{\nseq{n}}}}
      {\\
      &\qquad\squash{R(n,s)} \tolr (\pityc{\isOf{m}{\lift{\nat}}}{R(\suc{n}, s \oplus_n m) 
        \tolr P(\suc{n}, s \oplus_n m)}) \tolr P(n,s)}}}\\
    &\isVal{dec}{\pityc{\isOf{n}{\lift{\nat}}}{\pityc{\isOf{s}{\lift{\nseq{n}}}}{\lift{B(n,s) + \neg B(n,s)}}}}\\
    &\isVal{base}{\pityc{\isOf{n}{\lift{\nat}}}{\pityc{\isOf{s}{\lift{\nseq{n}}}}{\squash{R(n,s)} \tolr 
      B(n,s) \tolr P(n,s)}}}\\
    &\isVal{r}{R(0,\empseq)}
  \end{align*}

Then $\doState{k}{\dbrp{dec}{base}{ind}{0}{\empseq}{r}} \mapsto^* \retState{k}{\ret{v}}$ and
\isVal{v}{P(0,\empseq)}. 
Suppose the inductive argument is fully applied in the body of $ind$, i.e. the 4th argument is 
fully applied. If 
\doState{k}{\dbrp{dec}{base}{ind}{i}{s}{r}} occurs as a 
in the computation for some $i,s,r$, 
then it is the $i$th distinguished $\underline{\mathtt{dbr}}$.
Let \doState{k}{\dbrp{dec}{base}{ind}{n}{s}{r}} be the last such state.
Then $n$ is bounded by the bar for the extended sequence \extSeq{s}{r}:
$n \le_b b$, where \eval{\bind{bar\,(\extSeq{s}{r})\,\triv}{x}{\fst{x}}}{b}.
Write $term(\extSeq{s}{r})$ for b.
\end{lemma}

\begin{proof}
  The first part follows from Lemma~\ref{lemma:dbr} and ~\ref{lemma:stack}.
  Note 
  \begin{gather*}
  \doState{k}{\dbrp{dec}{base}{ind}{n}{s}{r}} \mapsto \\
  \doState{k; \seq{\thunkcst{\_}}{x}{\case{x}{p}{base\;n\;s\;r\;p}{\_}{ind\;n\;s\;\;r\;(\lam{t,r'}{}{\dbrp{dec}{base}{ind}{\suc{n}}{s\oplus_n t}{r'}})}
}}{dec\,n\,s}
  \end{gather*}
  For the second part, proceed by induction on $n$. 
  
  If $n = 0$, 
  then \doState{k}{\dbrp{dec}{base}{ind}{0}{\empseq}{r}} is the last 
  $\underline{\mathtt{dbr}}$. This implies \eval{dec\,0\,\empseq}{\inl{p}} 
  for some \isVal{p}{\lift{B(0,\empseq)}}, 
  since otherwise it would not be the last such state. 
  Then clearly there are exactly 1 such state and 
  $0 \le_b term(\extSeq{s}{r})$.

  If $n = \suc{n'}$, then
  \doState{k}{\dbrp{dec}{base}{ind}{\suc{n'}}{s}{r}} is the last 
  $\underline{\mathtt{dbr}}$. Suppose $n >_b term(\extSeq{s}{r})$.
  This means there exists $m < n$ s.t. $B(m,s\restriction_m)$ is inhabited.
  But this is a contradiction since $dec\,m\,(s\restriction_m)$ would have 
  returned $\inl{p}$ and 
  \doState{k'}{\dbrp{dec}{base}{ind}{m}{s\restriction_m}{r}} would have been
  the last. Hence $n \le_b term(\extSeq{s}{r})$. 
\end{proof}



\section{Lemmas}
Define:
\begin{gather*}
    \zp \triangleq \ret{\zero}\\
    \plus \triangleq \lam{x}{a}{\ret{\lam{y}{b}{\rec{x}{\ret{y}}{\_}{f}{\ret{\suc{f}}}}}}\\
    \mult \triangleq \lam{x}{a}{\ret{\lam{y}{b}{\rec{x}{\ret{y}}{\_}{f}{\plus\; y\; f}}}}\\
    \sub \triangleq \lam{x}{a}{\rec{x}{\ret{\lam{y}{b}{\ret{\zero}}}}{z}{f}{\ret{\lam{y}{b}{\rec{y}{\ret{\suc{p}}}{y'}{\_}{f y'}}}}}\\
    \pmin \triangleq \lam{x}{a}{\ret{\lam{y}{b}{\seq{\sub\; x\; y}{d}{\rec{d}{\ret{x}}{\_}{\_}{\ret{y}}}}}}
\end{gather*}

\begin{lemma}\textnormal{(basis)}\label{lemma:basic}
\isPot{\zp}, and for all \isVal{x}{\ret{\nat}} and \isVal{y}{\ret{\nat}}, \isPot{\plus\; x\; y}, 
 \isPot{\mult\; x\; y}, \isPot{\sub\; x\; y} and \isPot{\pmin\; x\; y}.
\end{lemma}

Notation:
\begin{gather*}
\bind{M_1}{x}{M_2} \triangleq \seq{\thunk{M_1}}{x}{M_2}\\
f\;v_1\;\dots\;v_n \triangleq \bind{f\;v_1}{r_1}{\bind{r_1\;v_2}{r_2}{\dots,\bind{r_{n-2}\;v_{n-1}}{r_{n-1}}{r_{n-1}\;v_n}}}\\
\varphi^+ \triangleq \bind{\varphi}{v}{\ret{\suc{v}}}\\
\end{gather*}

\begin{lemma}\textnormal{(Recursor)}\label{lemma:recursor}
Given 
\begin{gather}
\gammaToPot{\isOf{x}{\ret{\nat}}}{\varphi_0},\\
\gammaToPot{\isOf{x}{\ret{\nat}},\isOf{z}{\ret{\nat}}, \isOf{g}{\comp{\nat}}}{\varphi_1},\\
\gammaToTypeComp{\isOf{x}{\ret{\nat}}}{\star}{B},\\
\gammaToPot{\isOf{x}{\ret{\nat}},\isOf{y}{B}}{\varrho_0},\\
\gammaToPot{\isOf{x}{\ret{\nat}}, \isOf{z}{\ret{\nat}}, \isOf{g}{\comp{\nat}}, \isOf{y}{B}}{\varrho_1},\\
\gammaToComp{\cdot}{[\zero/x]\varphi_0}{M_0}{\isOf{y}{[\zero/x]B}}{[\zero/x]\varrho_0},\\
\gammaToComp{\isOf{z}{\ret{\nat}}}{[\suc{z}/x]\lsub{\rec{z}{\varphi_0}{z}{g}{\varphi_1}}{g}{\varphi_1}}
{\lsub{\rec{z}{M_0}{z}{f}{M_1}}{f}{M_1}\\}{\isOf{y}{[\suc{z}/x]B}}{[\suc{z}/x]\lsub{\rec{z}{\varrho_0}{z}{g}{\varrho_1}}{g}{\varrho_1}}
\end{gather}
Then 
\gammaToComp{\isOf{x}{\ret{\nat}}}{\rec{x}{\varphi_0}{z}{g}{\varphi_1}^+}{\rec{x}{M_0}{z}{f}{M_1}}{\isOf{y}{B}}{
\rec{x}{\varrho_0}{z}{g}{\varrho_1}}
\end{lemma}

\begin{proof}
Presuppositions: assumed by given. Let \isVal{n}{\ret{\nat}}. STS \gammaToComp{\cdot}{[n/x]\rec{x}{\varphi_0}{z}{g}{\varphi_1}^+}{\rec{n}{M_0}{z}{f}{M_1}}{\isOf{y}{[n/x]B}}{
[n/x]\rec{x}{\varrho_0}{z}{g}{\varrho_1}}
. Presuppositions: 
\begin{enumerate}
    \item \isTypeComp{\star}{[n/x]B}. By instantiating (2) with $n$ for $x$. 
    \item \isPot{[n/x]\rec{x}{\varphi_0}{z}{g}{\varphi_1}^+}. By instantiating (1) with $n$ for $x$. 
    \item \gammaToPot{\isOf{y}{[n/x]B}}{[n/x]\rec{x}{\varrho_0}{z}{g}{\varrho_1}}
\end{enumerate}
Next, induction on $n$: 
\begin{itemize}
    \item $n$ is $\zero$:\\
    Then \step{\rec{\zero}{M_0}{z}{f}{M_1}}{M_0}. By (6), we know \evalCost{M_0}{c}{u} for some \isVal{u}{[\zero/x]B}.
    By head expansion, \evalCost{\rec{\zero}{M_0}{z}{f}{M_1}}{c+1}{u}. 
    Furthermore, we know \eval{[\zero/x]\varphi_0}{p}, \eval{[\zero/x,u/y]\varrho_0}{p'}, and $\toNat{p} \ge c + \toNat{p'}$. Furthermore, we know that 
    \begin{align*}
        &[\zero/x]\rec{x}{\varphi_0}{z}{g}{\varphi_1}^+\\ 
        \mapsto & [\zero/x]\varphi_0^+\\
        \mapsto^* &\seq{\thunk{\ret{p}}}{v}{\ret{\suc{v}}}\\
        \mapsto& \ret{\suc{p}}\\
    \end{align*}. Since $\toNat{\suc{p}} = \toNat{p} + 1$, STS that\\
    \eval{[\zero/x,u/y]\rec{x}{\varrho_0}{z}{g}{\varrho_1}}{p''}
    and that $p' \ge p''$. Evidently this is the case since
    \step{[\zero/x,u/y]\rec{x}{\varrho_0}{z}{g}{\varrho_1}}{\eval{[\zero/x,u/y]\varrho_0}{p'}}.
    \item $n$ is $\suc{n'}$ for some $n'$:\\
    Then \step{\rec{\suc{n'}}{M_0}{z}{f}{M_1}}{\lsub{\rec{n'}{M_0}{z}{f}{M_1}}{f}{[n'/z]M_1}}. 
    Instantiating (7) with $[n'/z]$, we know that \evalCost{\lsub{\rec{n'}{M_0}{z}{f}{M_1}}{f}{[n'/z]M_1}}{c}{u}, 
    and by head expansion, \evalCost{\rec{\suc{n'}}{M_0}{z}{f}{M_1}}{c+1}{u}. Furthermore, we know
    \begin{align*}
        &\rec{\suc{n'}}{[\suc{n'}/x]\varphi_0}{z}{g}{[\suc{n'}/x]\varphi_1}^+\\
        \mapsto& [\suc{z}/x]\lsub{\rec{z}{\varphi_0}{z}{g}{\varphi_1}}{g}{\varphi_1}^+\\
        \mapsto^*& \ret{p}^+\\
        \mapsto& \ret{\suc{p}}
    \end{align*} and that 
    \begin{align*}
        &\rec{\suc{n'}}{[\suc{n'}/x]\varrho_0}{z}{g}{[\suc{n'}/x]\varrho_1}\\
        \mapsto &[\suc{z}/x]\lsub{\rec{z}{\varrho_0}{z}{g}{\varrho_1}}{g}{\varrho_1}\\
        \mapsto^* & \ret{p'}
    \end{align*}
    such that $\toNat{p} \ge c + \toNat{p'}$, which suffices for 
    $\toNat{\suc{p}} \ge c+1 + \toNat{p'}$.
\end{itemize}
\end{proof}

\newpage
\begin{lemma}\textnormal{(Recursor')}\label{lemma:recursor'}
Given 
\begin{gather}
\gammaToPot{\cdot}{\varphi_0},\\
\gammaToPot{\isOf{z}{\ret{\nat}}}{\varphi_1},\\
\gammaToTypeComp{\isOf{x}{\ret{\nat}}}{\star}{B},\\
\gammaToPot{\cdot}{\varrho_0},\\
\gammaToPot{\isOf{z}{\ret{\nat}}}{\varrho_1},\\
\gammaToComp{\cdot}{\varphi_0}{M_0}{\isOf{y}{[\zero/x]B}}{\varrho_0},\\
\gammaToComp{\isOf{z}{\ret{\nat}},\isOf{f}{\ret{\ccomp{\zp}{\isOf{w}{[z/x]B}}{\zp}}}}{\varphi_1}
{M_1}{\isOf{y}{[\suc{z}/x]B}}{\varrho_1},\\
\gammaToVal{\cdot}{n}{\ret{\nat}}
\end{gather}
Then 
\gammaToComp{\cdot}{\rec{n}{\varphi_0^+}{z}{g}{\bind{\varphi_1}{\phi_1}{\plus\; g\; \phi_1}^+}}{\rec{n}{M_0}{z}{f}{M_1}}{\isOf{y}{[n/x]B}}{
\rec{n}{\varrho_0}{z}{g}{\bind{\varrho_1}{\rho_1}{\plus\;\rho_1\; g}}}
\end{lemma}

\begin{proof}
Presuppositions: 
\begin{enumerate}
    \item \isTypeComp{\star}{[n/x]B}. By instantiating (11) with $n$ for $x$. 
    \item \isPot{\rec{n}{\varphi_0^+}{z}{g}{\bind{\varphi_1}{\phi_1}{\plus\;\phi_1\;g}^+}}. TODO
    \item \isPot{\rec{n}{\varrho_0}{z}{g}{\bind{\varrho_1}{\rho_1}{\plus\;\rho_1\; g}}}. TODO
\end{enumerate}
Next, induction on $n$: 
\begin{itemize}
    \item $n$ is $\zero$:\\
    Then \step{\rec{\zero}{M_0}{z}{f}{M_1}}{M_0}. By (14), we know \evalCost{M_0}{c}{u} 
    for some \isVal{u}{[\zero/x]B}.
    By head expansion, \evalCost{\rec{\zero}{M_0}{z}{f}{M_1}}{c+1}{u}. 
    Furthermore, we know \eval{\varphi_0}{p}, \eval{[u/y]\varrho_0}{p'}, and $\toNat{p} \ge c + \toNat{p'}$. Furthermore, we know that 
    \begin{align*}
        &\rec{\zero}{\varphi_0^+}{z}{g}{\bind{\varphi_1}{\phi_1}{\plus\;\phi_1\;g}^+}\\ 
        \mapsto & \varphi_0^+\\
        \mapsto^* &\seq{\thunk{\ret{p}}}{v}{\ret{\suc{v}}}\\
        \mapsto& \ret{\suc{p}}
    \end{align*}
    Since $\toNat{\suc{p}} = \toNat{p} + 1$, STS that
    \eval{[u/y]\rec{\zero}{\varrho_0}{z}{g}{\bind{\varrho_1}{\rho_1}{\plus\;\rho_1\;g}}}{p''}
    and that $p' \ge p''$. Evidently this is the case since
    \step{[u/y]\rec{\zero}{\varrho_0}{z}{g}{\bind{\varrho_1}{\rho_1}{\plus\; \rho_1\; g}}}{\eval{[u/y]\varrho_0}{p'}}.
    \item $n$ is $\suc{n'}$ for some $n'$:\\
    Then \step{\rec{\suc{n'}}{M_0}{z}{f}{M_1}}{\lsub{\rec{n'}{M_0}{z}{f}{M_1}}{f}{[n'/p]M_1}}. 
    By induction, we know that 
    \gammaToComp{\cdot}{\rec{n'}{\varphi_0^+}{z}{g}{\bind{\varphi_1}{\phi_1}{\plus\;\phi_1\;g}^+}}
    {\rec{n'}{M_0}{z}{f}{M_1}}{\isOf{y}{[n'/x]B}}
    {\rec{n'}{\varrho_0}{z}{g}{\plus\;g\;\varrho_1}}. This implies that 
    \begin{gather}
    \evalCost{\rec{n'}{M_0}{z}{f}{M_1}}{c}{u}, \\
    \eval{\rec{n'}{\varphi_0^+}{z}{g}{\bind{\varphi_1}{\phi_1}{\plus\;\phi_1\;g}^+}}{p},\\
    \eval{\rec{n'}{\varrho_0}{z}{g}{\bind{\varrho_1}{\rho_1}{\plus\;\rho_1\;g}}}{p'}, \text{ and that}\\
    \toNat{p} \ge c + \toNat{p'}
    \end{gather}
    Instantiating (15) with $[n'/p, \thunk{\ret{u}}/f]$, we know that
    \evalCost{\lsub{\ret{u}}{f}{[n'/p]M_1}}{c'}{u'},
    and by head expansion, \evalCost{\rec{\suc{n'}}{M_0}{z}{f}{M_1}}{c+1+c'}{u'}. Furthermore, we know
    \begin{align}
        &\eval{[n'/z]\varphi_1}{q},\\
        &\eval{[n'/z]\varrho_1}{q'}, \text{ and that }\\
        &q \ge c' + q'
    \end{align}
    Expanding our current potential, we have:
    \begin{align*}
        &\rec{\suc{n'}}{\varphi_0^+}{z}{g}{\bind{\varphi_1}{\phi_1}{\plus\;\phi_1\;g}^+}\\
        \mapsto& \lsub{\rec{n'}{\varphi_0^+}{z}{g}{\bind{\varphi_1}{\phi_1}{\plus\;\phi_1\;g}^+}}{g}{[n'/p]\bind{\varphi_1}{\phi_1}{\plus\;\phi_1\;g}^+}\\
        \mapsto^*& \lsub{\ret{p}}{g}{\bind{[n'/z]\varphi_1}{\phi_1}{\plus\;\phi_1\;g}^+}\\
        \mapsto& \bind{[n'/z]\varphi_1}{\phi_1}{\plus\;\phi_1\;p}^+\\
        \mapsto^*& \bind{\ret{q}}{\phi_1}{\plus\;\phi_1\;p}^+\\
        \mapsto& \plus\; q \; p ^+
    \end{align*} and that 
    \begin{align*}
        &\rec{\suc{n'}}{\varrho_0}{z}{g}{\bind{\varrho_1}{\rho_1}{\plus\;\rho_1\; g}}\\
        \mapsto& \lsub{\rec{n'}{\varrho_0}{z}{g}{\bind{\varrho_1}{\rho_1}{\plus\;\rho_1\; g}}}{g}{[n'/z]\bind{\varrho_1}{\rho_1}{\plus\;\rho_1\; g}}\\
        \mapsto^*& \lsub{\ret{p'}}{g}{\bind{[n'/z]\varrho_1}{\rho_1}{\plus\;\rho_1\; g}}\\
        \mapsto& \bind{[n'/z]\varrho_1}{\rho_1}{\plus\;\rho_1\; p'}\\
        \mapsto^*& \bind{\ret{q'}}{\rho_1}{\plus\;\rho_1\; p'}\\
        \mapsto& \plus\;q'\; p'
    \end{align*}
    Now it suffices to show $\toNat{\plus\; q \; p ^+} \ge c + 1 + c' + \toNat{\plus\;q'\; p'}$,
    which holds by (20) and (23).
\end{itemize}
\end{proof}

\begin{lemma}\textnormal{(Return)}\label{lemma:ret}
If \gammaToVal{\Gamma}{v}{B}, then \gammaToComp{\Gamma}{\zp}{\ret{v}}{\isOf{y}{B}}{\zp}
\end{lemma}

\begin{proof}
Induction on the length of $\Gamma$.
\begin{itemize}
    \item $\Gamma = \cdot$\\
    Suppose \gammaToVal{\cdot}{v}{B}. By definition, this means \isVal{v}{B}. STS 
    \gammaToComp{\cdot}{\zp}{\ret{v}}{\isOf{y}{B}}{\zp}. We know that 
    $\evalCost{\ret{v}}{0}{v}$, $\eval{\zp}{\zero}$, $\eval{[v/y]\zp}{\zero}$, and 
    $\toNat{\zero} \ge 0 + \toNat{\zero}$.
    \item $\Gamma = \isOf{x}{A},\Gamma'$\\
    Suppose \gammaToVal{\isOf{x}{A},\Gamma'}{v}{B}. This means 
    \fact{$[a/x]\left(\gammaToVal{\Gamma}{v}{B}\right)$ for all \isVal{a}{A}}{1}. 
    NTS \gammaToComp{\isOf{x}{A},\Gamma'}{\zp}{\ret{v}}{\isOf{y}{B}}{\zp}. Let
    \isVal{a}{A}. STS $[a/x]\left(\gammaToComp{\Gamma}{\varphi}{\ret{v}}{\isOf{y}{B}}{\varrho}\right)$, which 
    holds by IH when instantiated with \applyAss{1}.
\end{itemize}
\end{proof}

\begin{lemma}\textnormal{(Context)}\label{lemma:ctx}
If \isCtx{\Gamma[\isOf{x}{A}]\Gamma'}, then for all \isSub{\gamma}{\Gamma}, \isTypeComp{\star}{\subst{\gamma}{A}}.
\end{lemma}

\begin{proof}
Induction on length of $\Gamma$.
\begin{itemize}
    \item $\Gamma = \cdot$\\
    Suppose \isCtx{[\isOf{x}{A}]\Gamma'} and \isSub{\gamma}{\cdot}. The former implies \isTypeComp{\star}{A},
    together with the latter which implies $\gamma = \cdot$, we have \isTypeComp{\star}{A\gamma}.
    \item $\Gamma = \isOf{x'}{A'},\Gamma''$\\
    \isCtx{\isOf{x'}{A'},\Gamma''[\isOf{x}{A}]\Gamma'} and \isSub{\gamma}{\isOf{x'}{A'},\Gamma''}. 
    Then $\gamma$ must 
    be $x' \mapsto M, \gamma'$ for some \isVal{M}{A'} and \isSub{\gamma'}{[M/x']\Gamma''}. 
    Further, \isCtx{[M/x'](\Gamma''[\isOf{x}{A}]\Gamma')}. By IH,
    \isTypeComp{\star}{[\gamma'][M/x']A}. This suffices since $[\gamma]A = [\gamma'][M/x']A$.
\end{itemize}
\end{proof}

\begin{lemma}\textnormal{(Abstraction)}\label{lemma:abs}
If \gammaToComp{\Gamma[\isOf{x}{A}]}{\varphi}{M}{\isOf{y}{B}}{\varrho}, then 
\gammaToVal{\Gamma}{\lam{x}{A}{M}}{\ret{\arr{\isOf{x}{A}}{\varphi}{\isOf{y}{B}}{\varrho}}}
\end{lemma}

\begin{proof}
Induction on length of $\Gamma$.
\begin{itemize}
    \item $\Gamma = \cdot$\\
    Suppose \gammaToComp{\isOf{x}{A}}{\varphi}{M}{\isOf{y}{B}}{\varrho}. This means
     \fact{\gammaToComp{\cdot}{[a/x]\varphi}{[a/x]M}{\isOf{y}{[a/x]B}}{[a/x]\varrho} for all \isVal{a}{A}}{1}.
     NTS \gammaToVal{\cdot}{\lam{x}{A}{M}}{\ret{\arr{A}{\varphi}{\isOf{y}{B}}{\varrho}}}. STS
     \isVal{\lam{x}{A}{M}}{\ret{\arr{A}{\varphi}{\isOf{y}{B}}{\varrho}}}. First, NTS 
     \isTypeComp{\star}{\arr{\isOf{x}{A}}{\varphi}{\isOf{y}{B}}{\varrho}}:
     \begin{enumerate}
         \item \isTypeComp{\star}{A} by presupposition with Lemma~\ref{lemma:ctx}
         \item NTS \gammaToTypeCompFixed{\isOf{x}{A}}{1}{B}. Let \isVal{a}{A}. STS 
         \isTypeComp{\star}{[a/x]B}. By presupposition, \gammaToTypeComp{\isOf{x}{A},\cdot}{\star}{B}. This implies
         \gammaToTypeComp{\cdot}{\star}{[a/x]B}, which means \isTypeComp{\star}{[a/x]B}.
         \item NTS \gammaToPotFixed{\isOf{x}{A}}{1}{\varphi}. Let \isVal{a}{A}. STS \isPot{[a/x]\varphi}. 
         By presupposition, \gammaToPot{\isOf{x}{A},\cdot}{\varphi}.
         This implies \gammaToPot{\cdot}{[a/x]\varphi}, which means \isPot{[a/x]\varphi}.
         \item NTS \gammaToPotFixed{\isOf{x}{A},\isOf{y}{B}}{1}{\varrho}. 
         Let \isVal{a}{A} and \isVal{b}{[a/x]B}. STS \isPot{[a/x,b/y]\varrho}. By presupposition, 
         \gammaToPot{\isOf{x}{A},\cdot[\isOf{y}{B}]}{\varrho}. This implies 
         \gammaToPot{\cdot[\isOf{y}{[a/x]B}]}{[a/x]\varrho}, which means \gammaToPot{\isOf{y}{[a/x]B}}{[a/x]\varrho}.
         Next, this implies \gammaToPot{\cdot}{[b/y][a/x]\varrho}, which means \isPot{[b/y][a/x]\varrho}. 
     \end{enumerate}
     Next, NTS \gammaToCompFixed{\isOf{x}{A}}{\varphi}{1}{M}{\isOf{y}{B}}{\varrho}.
     Let \isVal{a}{A}. NTS 
     \begin{enumerate}
         \item \eval{[a/x]\varphi}{p}. By presupposition on \applyAss{1}, \gammaToPot{\cdot}{[a/x]\varphi}. This 
         means \isPot{[a/x]\varphi}, which ensures a return value. 
         \item \evalCost{[a/x]M}{c}{v} for some \isVal{v}{[a/x]B}. By definition of \applyAss{1}, 
         \evalCost{[a/x]M}{c}{v} and \isVal{v}{[a/x]B}
         \item \eval{[a/x,v/y]\varrho}{p'}. By presupposition on \applyAss{1}, 
         \gammaToPot{\isOf{y}{[a/x]B}}{[a/x]\varrho}. This 
         means \isPot{[a/x,b/y]\varrho} for all \isVal{b}{B}. Thus take $b$ to be $v$, and we are guaranteed a $p'$.
         \item $\toNat{p} \ge c + \toNat{p'}$. Follows from definition of \applyAss{1}.
     \end{enumerate}
    \item $\Gamma = \isOf{x'}{A'},\Gamma'$\\
    Suppose \gammaToComp{\isOf{x'}{A'},\Gamma'[\isOf{x}{A}]}{\varphi}{M}{\isOf{y}{B}}{\varrho}. This means
    \fact{$[a'/x']\left(\gammaToComp{\Gamma'[\isOf{x}{A}]}{\varphi}{M}{\isOf{y}{B}}{\varrho}\right)$ for all \isVal{a'}{A'}}{2}. 
    NTS \gammaToVal{\isOf{x'}{A'},\Gamma'}{\lam{x}{A}{M}}{\ret{\arr{\isOf{x}{A}}{\varphi}{\isOf{y}{B}}{\varrho}}}. 
    This means 
    $[a'/x']\left(\gammaToVal{\Gamma'}{\lam{x}{A}{M}}{\ret{\arr{\isOf{x}{A}}{\varphi}{\isOf{y}{B}}{\varrho}}}\right)$
    for all \isVal{a'}{A'}. Suppose \isVal{a'}{A'}. Then the result holds by \applyAss{2} on IH.
\end{itemize}
\end{proof}

\begin{lemma}\textnormal{(Application)}
If \isVal{f}{\ret{\arr{\isOf{x}{A}}{\varphi}{\isOf{y}{B}}{\varrho}}} and \isVal{a}{A}, then 
\gammaToComp{\cdot}{[a/x]\varphi^+}{f\;v}{\isOf{y}{[a/x]B}}{[a/x]\varrho}.
\end{lemma}

\begin{proof}
Given \isVal{f}{\ret{\arr{\isOf{x}{A}}{\varphi}{\isOf{y}{B}}{\varrho}}} and \isVal{a}{A}, 
we know that $f = \lam{x}{a}{M}$ s.t. 
\[
    \gammaToCompFixed{\isOf{x}{A}}{\varphi}{1}{M}{\isOf{y}{B}}{\varrho}
\]
Instantiating this with $[a/x]$, we know
\begin{enumerate}
    \item \eval{[a/x]\varphi}{p}
    \item \evalCost{[a/x]M}{c}{v} for some \isVal{v}{[a/x]B}
    \item \eval{[a/x,v/y]\varrho}{p'}
    \item $\toNat{p} \ge c + \toNat{p'}$
\end{enumerate}

Since \step{\lam{x}{a}{M}\; a}{[a/x]M}, it suffices to show 
$\step{[a/x]\varphi^+}{q}$ s.t $\toNat{q} \ge \toNat{p} + 1$, which 
holds since $\eval{[a/x]\varphi^+}{\suc{p}}$. 

\end{proof}

\begin{lemma}\textnormal{(Sequence)}\label{lemma:seq}
If \gammaToComp{\cdot}{\varphi}{M_1}{\isOf{x}{A}}{\varrho} and 
\gammaToComp{\isOf{x}{A}}{\varrho}{M_2}{\isOf{y}{B}}{\varsigma}, then 
\gammaToComp{\isOf{x}{A}}{\varphi^+}{\seq{\thunk{M_1}}{x}{M_2}}{\isOf{y}{B}}{\varsigma}.
\end{lemma}

\begin{proof}

\end{proof}

\begin{corollary}\textnormal{(Potential Sequence)}\label{cor:potseq}
If \isPot{\varphi} and
\gammaToPot{\isOf{x}{\ret{\nat}}}{\varrho}, then 
\isPot{\seq{\thunk{\varphi}}{x}{\varrho}}.
\end{corollary}

\begin{lemma}\textnormal{(Pair)}\label{lemma:pair}
If \gammaToVal{\cdot}{v}{A}, \gammaToTypeComp{\isOf{x}{A}}{\star}{B}, and \gammaToVal{\cdot}{w}{[v/x]B},
then \gammaToVal{\cdot}{\pair{v}{w}}{\ret{\dprod{\isOf{x}{A}}{B}}}. 
\end{lemma}



\section{Fold}

Now we can characterize the complexity of some familiar functions. 
First, add lists to the language: 

\begin{align*}
    \mathsf{Val} \quad v &::= \listty{A}
    \mid \nil
    \mid \cons{u}{v}\\
    \mathsf{Comp} \quad M &::= \lrec{v}{M_0}{x,xs,f}{M_1}
\end{align*}

\begin{mathpar}

\inferrule{
}{
  \lrec{\nil}{M_0}{x,xs,f}{M_1} \mapsto M_0
}

\inferrule{
}{
  \lrec{\cons{u}{v}}{M_0}{x,xs,f}{M_1} \mapsto 
  \seq{\thunk{\lrec{v}{M_0}{x,xs,f}{M_1}}}{f}{[u/x,v/xs]M_1}
}
\end{mathpar}

For convenience, we define a resource monoid on pairs of natural numbers (from 
Section 3.3.1 in~\cite{Hoffmann11}: 
\[
(p,q) \cdot (p',q') \triangleq \begin{cases}
(p + p' - q, q') \quad\text{ if } $p' > q$\\
(p, q - p' + q') \quad\text{ else }
\end{cases}
\]

Then we have some properties:

\begin{lemma}(Resource Monoid)
\begin{enumerate}
\item If $p \ge c + q$ and $p' \ge c' + q'$ then $r \ge c+c' + s$ where 
$(r,s) = (p,q) \cdot (p',q')$
\item If $p > q$, then $(p,q)^n = (np - (n-1)q, q)$. Otherwise, 
$(p,q)^n = (p, nq - (n-1)p)$. 
\end{enumerate}
\end{lemma}

We can lift the monoid to the computation layer:

\begin{verbatim}
ressum = \(p,q),(p',q') ->
  if p' > q then ret((p+p'-q,q'))
  else ret((p, q+q'-p'))
\end{verbatim}

Suppose we define fold as follows:

\[
fold \triangleq \lam{f,b,l}{}{\lrec{l}{\ret{b}}{x,xs,z}{f(x,z)}}
\]

Then we have the following:

\begin{lemma}(Fold)\label{lemma:fold}
Given
\begin{gather*}
\isVal{f}{\arrabtc{\dprodc{A}{B}}{B}{x.P,x.y.Q}}\\
\isVal{b}{B}\\
\isVal{l}{\listtyc{A}}
\end{gather*}
let 
\begin{verbatim}
T = fold (\(a,(v,p,q)) ->
  v' <- f(a,v);
  p' <- [(a,v)/x]P;
  q' <- [(a,v)/x,v'/y]Q;
  ressum (p,q) (p',q')
) (0,0) l
\end{verbatim}

then \isCComp{fold\,f\,b\,l}{B}{\fst{T} + 3n+2}{\snd{T}}
\end{lemma}

\begin{proof}
Let $l = [a_0,...,a_{n-1}]$. Unfolding the computation of fold:
\begin{verbatim}
fold f b l 
==>^3 lrec{l}(...) 
==> v1 <- lrec{[a1,...,a{n-})]}(...);
    f(a0,v1)
==> v1 <- 
      v2 <- lrec{[a2,...a{n-1}]}(...);
      f(a1,v2);
    f(a0,v1)
... // n-2 steps
==> v1 <- 
      v2 <- 
        ...
        vn <- lrec{[]}(...);
        f(a{n-1},vn);
        ...
      f(a1,v2);
    f(a0,v1)
==> v1 <- 
      v2 <- 
        ...
        vn <- ret(b);
        f(a{n-1},vn);
        ...
      f(a1,v2);
    f(a0,v1)
==> v1 <- 
      v2 <- 
        ...
        v{n-1} <- f(a{n-1},b);
        ...
      f(a1,v2);
    f(a0,v1)
\end{verbatim}
So essentially we have to compute 
\[
f(a_0,f(a_1,...,f(a_{n-1},b)...))
\]
Let each \evalCost{f(a_i,...)}{c_i}{v_i}, and $v_n \triangleq b$. We know by assumption that 
\begin{gather*}
\eval{[(a_i,v_{i+1})/x]P}{\bar{p_i}},\\
\eval{[(a_i,v_{i+1})/x,v_i/y]Q}{\bar{q_i}},\\
p_i \ge c_i + q_i
\end{gather*}

Thus the total cost for the function calls is $c = sum_i(c_i)$, which is bounded by
the sum in the resource monoid: 
\[
(p_{i-1}, q_{i-1}) \cdot (p_{i-2}, q_{i-2}) \cdot \dots \cdot (p_0,q_0) 
\]

This can be computed as another fold:

\begin{verbatim}
t = fold (\(a,(v,p,q)) ->
  v' <- f(a,v);
  p' <- [(a,v)/x]P;
  q' <- [(a,v)/x,v'/y]Q;
  ressum (p,q) (p',q')
) (0,0) l

p,q = t.1, t.2
\end{verbatim}
Where $p \ge c + q$.

Furthermore, we need 1 unit for each function application, and 1 unit for each sequence, 
which accounts for another $2n-1$ units.

Hence in total we need $3 + n + c + 2n-1 = c + 3n+2$, and is bounded by the given cost specification.
\end{proof}

We can define map in terms of fold: 

\[
map \triangleq \lam{f,l}{}{fold\,(\lam{(a,b)}{}{\bind{f(a)}{v}{\ret{\cons{v}{b}}}})\,\nil\,l}
\]

In particular, assuming \isVal{f}{\arrabtc{A}{B}{\_.\ret{p},\_.\_.\ret{q}}}, we can deduce that 
\[
\isVal{f'}{\arrabtc{\dprodc{A}{B}}{\listtyc{B}}{\_.\ret{p+2},\_.\_.\ret{q}}}
\]

where $f'$ is the function argument to fold.
Then the total function cost can be bounded by $(p+2,q)^n = (n(p+2) - (n-1)q,q) = (n(p-q+2) + q,q)$,
i.e. it suffices to start map with $n(p-q+2)$ units, where $n = |l|$.


\section{Cbv vs. Cbn}

In our setup, all computations are performed ``call by value''. In particular, this means
the recursor schedules the recursive calls to complete before the body. This leads to
predictable generic cost behavior given the cost of the body. However, the call by value
protocol leads to wasted computations in case when the body short-circuits in some way.
Consider the gcd as usually defined:

\begin{verbatim}
gcd = 
  \m.rec(m){
    0 => \x,y -> 0
    s(m'),f => \x,y -> if x = 0 then y else f y (mod x y)
  }
\end{verbatim}

Where \texttt{m} is a structural argument that witnesses termination. Morally, 
it represents the amount of ``fuel'' the computation needs before the recursion stops.
A naive analysis shows that the sum of relevant inputs suffices, since it decreases
by at least one each recursive call.
We can trace how \texttt{gcd 100} computes:

\begin{verbatim}
gcd 100 = rec(100){...}
==> f1 <- rec(99){...}; 
    \x,y.if x = 0 then y else f1 y (mod x y)
==> f1 <- 
      f2 <- rec(98){...};
      \x,y.if x = 0 then y else f2 y (mod x y);
    \x,y.if x = 0 then y else f1 y (mod x y)
...
==> f1 <- 
      f2 <- 
        ...
        f99 <- rec(0){...};
        \x,y.if x = 0 then y else f99 y (mod x y)
        ... 
      \x,y.if x = 0 then y else f2 y (mod x y);
    \x,y.if x = 0 then y else f1 y (mod x y)
==>  f1 <- 
      f2 <- 
        ...
        f99 <- 0;
        \x,y.if x = 0 then y else f99 y (mod x y)
        ... 
      \x,y.if x = 0 then y else f2 y (mod x y);
    \x,y.if x = 0 then y else f1 y (mod x y)
==>  f1 <- 
      f2 <- 
        ...
        f99 <- \x,y -> 0;
        \x,y.if x = 0 then y else f99 y (mod x y)
        ... 
      \x,y.if x = 0 then y else f2 y (mod x y);
    \x,y.if x = 0 then y else f1 y (mod x y)
...
==> \x,y.if x = 0 then y else 
       (\x,y. if x = 0 then y else 
         (\x,y. if x = 0 then y else
           ...
           (\x,y. if x = 0 then y else
             (
              \x,y. 0
             ) y (mod x y)
           )
           ...
         ) y (mod x y)
        
        ) y (mod x y)
\end{verbatim}

Note that \emph{all} recursive calls are scheduled regardless whether the body actually 
uses them. Consider the input pair \texttt{(70,30)}. Ordinarily, one would expect the
call graph to look like 

\[
\texttt{gcd(70,30)} \to \texttt{gcd(30,10)} \to \texttt{gcd(10,0)} \to \texttt{gcd(0,10)} 
\]

And so we have \text{gcd(70,30) = 10}. Indeed, this is what happens if recursive calls 
are scheduled lazily. However, with CBV, all 100 recursive calls are 
computed before we even get to look at the body. 

It seems that lazy evaluation is advantageous in this sense. However, this comes at the 
cost of non-uniformity, as the unevaluated thunks can be arbitrarily duplicated within
the body, making any general statements about the complexity of a recursive function 
impossible. 

The issue further complicated by tree-like call graphs engenered by inductive types with
multiple recursive constructors. In some cases, such as node insertion in a binary tree,
lazy evaluation is the only way to achieve the expected complexity if all we have 
are the corresponding elimination principles. Again, we have no way of recording this 
information in the type theory if recursive calls are lazily evaluated. We choose to 
value expressiveness over performance, since all is moot if we cannot write anything 
down. 

The notion of the structural argument also deserves attention. In the gcd example, we 
naively chose the sum of the inputs as the measure of termination. A closer analysis 
was given by Gabriel Lam\'{e} in 1844, where he showed that the number of recursive calls
$n$ satisfied the relation $F_{n+2} \ge \max{x,y}$. In essense, this means $n$ is
proportional to the log of the inputs. With this fact, one can define a ``better '' 
algorithm computing the gcd using only the recursor. 

We also discovered another way to incorporate this observation in an efficient
implementation of Euclid's algorithm, which uses the notion of bar induction.
The benefit of doing this is that the termination proof is not 
directly ``baked in'' to the algorithm itself. The analogy would be 
bar induction is to ``while-loop'' as recursion is to ``for-loop''.
This modularity enables the programmer to implement the algorithm, and gradually 
refine the descriptive complexity as needed 
(we were able to extend  the mere existence of 
the bar with the above bound without rewriting the code itself). However, as mentioned
earlier, this does not yet enable us to give a cost bound to the resulting bar 
recursion, as we discuss later.

For the case of gcd, we first encode the patterns of recursive calls as the spread of 
admissible sequences. In this case, an admissible sequence $s$ of length $n$ must
begin with the inputs i.e. $s_0 = x, s_1 = y$, and furthermore satisfying the relation
 $mod\,s_{n-j-2}\,s_{n-j-1} = s_{n-j}$ for all $1 \le j \le n-2$. 
In other words, it is the sequence of 
arguments to gcd laid out linearly. 
We call a sequence barred if at some point it becomes 0.
Next, we prove the fact that each of these 
sequences are barred, i.e. the recursive calls terminate. 
Lastly, we construct a predicate $P$ on sequences which is 1) implied by the bar,
and 2) inductive, which corresponds to the fact that if each 
admissible extension of a sequence satisfy $P$, then so does that sequence.
All of these ingredients suffice to define a ``while-loop'' version of the gcd, in which
termination is not evident in the code written, but instead implied by the 
bar condition. In the next section, we give the detailed construction of this
description.







\section{Bar Induction}

\begin{definition}(Admissibility of Bar Induction)\label{lemma:birule}
  If 
  \begin{align*}
    &\text{(wfb)}\quad \openTypeComp{\Gamma, \isOf{n}{\lift{\nat}}, \isOf{s}{\lift{\nseq{n}}}}{B}\\
    &\text{(wfs)}\quad \openTypeComp{\Gamma, \isOf{n}{\lift{\nat}}, \isOf{s}{\lift{\nseq{n}}}}{R}\\
    &\text{(init)}\quad \openComp{\Gamma}{M_0}{R(\zero,\empseq)}\\
    &\text{(bar)}\quad \openComp{\Gamma,\isOf{s}{\lift{\baire}}, 
      \isOf{z}{\lift{\intty{\isOf{m}{\lift{\nat}}}{\squash{R(m,s)}}}}}{M_1}
      {\squash{\lift{\sigmaty{\isOf{n}{\lift{\nat}}}{B(n,s)}}}}\\ 
    &\text{(base)}\quad \openComp{\Gamma,\isOf{n}{\lift{\nat}}, \isOf{s}{\lift{\nseq{n}}},
    \isOf{z}{\squash{R(n,s)}}, \isOf{b}{B(n,s)}}{M_2}{P(n,s)}\\
    &\text{(ind)} \quad \openComp{\Gamma,\isOf{n}{\lift{\nat}}, \isOf{s}{\lift{\nseq{n}}},
    \isOf{z}{\squash{R(n,s)}},\\
    &\qquad\qquad\isOf{i}{\lift{\pityc{\isOf{m}{\lift{\nat}}}{
      R(\suc{n}, s \oplus_n m) \tolr P(\suc{n}, s \oplus_n m)}}}}{M_2}{P(n,s)}
  \end{align*}
  Then 
  \[
  \openComp{\Gamma}{\_}{\squash{P(0,\empseq)}}
  \]
\end{definition}

\begin{proof}
  By LEM, we get $\lift{\neg} \squash{P(0,\empseq)}$, which gives $\lift{\neg} P(0,\empseq)$.
  Taking the contrapositive of (ind), we have a function $f$ inhabiting
  \begin{align*}
    &\pityc{\isOf{n}{\lift{\nat}}}{
    \pityc{\isOf{s}{\lift{\nseq{n}}}}{
      \pityc{\isOf{z}{\squash{R(n,s)}}}{\\
        &\lift{\neg}P(n,s) 
      \tolr \sigmatyc{\isOf{m}{\lift{\nat}}}{R(\suc{n}, s \oplus_n m) 
      \lift{\times} \lift{\neg} P(\suc{n}, s \oplus_n m)}}}}
  \end{align*}
  Since $\lift{\neg}P(0,\empseq)$ and $R(0,\empseq)$, we can instantiate $f$ to obtain 
  \isVal{m}{\lift{\nat}} s.t. $R(1,\empseq \oplus_0 m)$ and $\lift{\neg}P(1,\empseq \oplus_0 m)$.
  Iterating this gives us a sequence \isVal{\alpha}{\lift{\baire}} s.t. 
  \inttyc{\isOf{m}{\lift{\nat}}}{\squash{R(m,\alpha)}} is inhabited. 
  It also shows that for all \isVal{n}{\lift{\nat}}, $\lift{\neg}P(n,\alpha)$.
  By (bar), we know it's true that there is \isVal{n}{\lift{\nat}} s.t. $B(n,\alpha)$.
  But by (base), we know that $P(n,\alpha)$, which is a contradiction.
\end{proof}

\begin{definition}(DBR)
  \begin{mathpar}
  \inferrule{}{
    \dbr{dec}{base}{ind}{n}{s}{r} \mapsto \\
  \seq{\thunk{dec\;n\;s}}{x}{
    \case{x}{p}{base\;n\;s\;r\;p}{\_}{ind\;n\;s\;\;r\;(\lam{t,r'}{}{\dbr{dec}{base}{ind}{\suc{n}}{s\oplus_n t}{r'}})}
}}
  \end{mathpar}
\end{definition}

\begin{lemma}\label{lemma:dbr}
  $\mathtt{dbr}$ implements the following induction principle: 
  \begin{align*}
    \mathsf{BID} = 
    &\ulcorner\pityc{\isOf{R,B,P}{\pityc{\isOf{n}{\lift{\nat}}}{(\embed\nseq{n}) \tolr (\embed\type{})}}}{\\
          &\pityc{\isOf{s}{\lift{\baire}}}{\intty{\isOf{m}{\lift{\nat}}}{\squash{R(m,s)}} 
      \tolr \squash{\lift{\sigmaty{\isOf{n}{\lift{\nat}}}{B(n,s)}}}}\\
      \tolr &(\pityc{\isOf{n}{\lift{\nat}}}{\pityc{\isOf{s}{\lift{\nseq{n}}}}
      {\\
      & \squash{R(n,s)} \tolr (\pityc{\isOf{m}{\lift{\nat}}}{R(\suc{n}, s \oplus_n m) 
        \tolr P(\suc{n}, s \oplus_n m)}) \tolr P(n,s)}})\\
      \tolr &(\pityc{\isOf{n}{\lift{\nat}}}{\pityc{\isOf{s}{\lift{\nseq{n}}}}{\lift{B(n,s) + \neg B(n,s)}}}) \\
      \tolr &(\pityc{\isOf{n}{\lift{\nat}}}{\pityc{\isOf{s}{\lift{\nseq{n}}}}{\squash{R(n,s)} \tolr 
      B(n,s) \tolr P(n,s)}})\\
      \tolr & R(0,\empseq)\\
      \tolr & P(0,\empseq)\urcorner
    }
  \end{align*}
\end{lemma}

\begin{proof}
  Need to show that 
  \isVal{\lam{R,B,P,bar,ind,dec,base,r}{}{\dbr{dec}{base}{ind}{\zero}{\empseq}{r}}}{\mathsf{BID}}.
  We have assumptions:
  \begin{align*}
    &\isVal{R}{\pityc{\isOf{n}{\lift{\nat}}}{\embed\nseq{n} \tolr \embed\type{}}}\\
    &\isVal{B}{\pityc{\isOf{n}{\lift{\nat}}}{\embed\nseq{n} \tolr \embed\type{}}}\\
    &\isVal{P}{\pityc{\isOf{n}{\lift{\nat}}}{\embed\nseq{n} \tolr \embed\type{}}}\\
    &\isVal{bar}{\pityc{\isOf{s}{\lift{\baire}}}{\intty{\isOf{m}{\lift{\nat}}}{\squash{R(m,s)}} 
      \tolr \squash{\lift{\sigmaty{\isOf{n}{\lift{\nat}}}{B(n,s)}}}}}\\
    &\isVal{ind}{\pityc{\isOf{n}{\lift{\nat}}}{\pityc{\isOf{s}{\lift{\nseq{n}}}}
      {\\
      &\qquad\squash{R(n,s)} \tolr (\pityc{\isOf{m}{\lift{\nat}}}{R(\suc{n}, s \oplus_n m) 
        \tolr P(\suc{n}, s \oplus_n m)}) \tolr P(n,s)}}}\\
    &\isVal{dec}{\pityc{\isOf{n}{\lift{\nat}}}{\pityc{\isOf{s}{\lift{\nseq{n}}}}{\lift{B(n,s) + \neg B(n,s)}}}}\\
    &\isVal{base}{\pityc{\isOf{n}{\lift{\nat}}}{\pityc{\isOf{s}{\lift{\nseq{n}}}}{\squash{R(n,s)} \tolr 
      B(n,s) \tolr P(n,s)}}}\\
    &\isVal{r}{R(0,\empseq)}
  \end{align*}
  We need to find a term $M$ s.t. \isComp{M}{P(\zero,\empseq)}.
  Define \[
    Q \triangleq \lam{n,s}{}{\lift{\eqty{P(n,s)}{\dbr{dec}{base}{ind}{n}{s}{r}}{\dbr{dec}{base}{ind}{n}{s}{r}}}}\]. 
  By Rule~\ref{lemma:birule}, we know that $\squash{Q(\zero,\empseq)}$ is inhabited.
  This means that \isVal{\triv}{Q(\zero,\empseq)}, which implies that\\
  \eqComp{\dbr{dec}{base}{ind}{\zero}{\empseq}{r}}{\dbr{dec}{base}{ind}{\zero}{\empseq}{r}}{P(\zero,\empseq)}.
  Hence it suffices to show that the hypotheses of Rule~\ref{lemma:birule} holds. 
  Premises (wfb), (wfs), (init), and (bar) follow directly from assumptions $B,R,r,$ and $bar$.
  For (base), let \isVal{n}{\lift{\nat}}, \isVal{s}{\lift{\nseq{n}}}, \isVal{z}{\squash{R(n,s)}},
  and \isVal{b}{B(n,s)}. We need to show 
  \begin{align*}
    Q(n,s) &\iff \eqty{P(n,s)}{\dbr{dec}{base}{ind}{n}{s}{r}}{\dbr{dec}{base}{ind}{n}{s}{r}} \\
           &\iff \isComp{\dbr{dec}{base}{ind}{n}{s}{r}}{P(n,s)}
  \end{align*}
  We case on the result of $dec\;n\;s$.
  \begin{itemize}
    \item $\eqComp{dec\;n\;s}{\ret{\inl{p}}}{\lift{B(n,s) + \neg B(n,s)}}$ and \isVal{p}{B(n,s)}:\\
      Then $\dbr{dec}{base}{ind}{n}{s}{r} \mapsto^* base\;n\;s\;r\;p$. By assumption,
      \isComp{base\;n\;s\;r\;p}{P(n,s)}, and by Lemma~\ref{lemma:headexp},  
      \isComp{\dbr{dec}{base}{ind}{n}{s}{r}}{P(n,s)}.
    \item $\eqComp{dec\;n\;s}{\ret{\inr{p}}}{\lift{B(n,s) + \neg B(n,s)}}$ and \isVal{p}{\lift{\neg} B(n,s)}:\\
      Then \isComp{p\;b}{\bot}, which is a contradiction.
  \end{itemize}
  Hence the base case holds.
  For (ind), let \isVal{n}{\lift{\nat}}, \isVal{s}{\lift{\nseq{n}}},
    \isVal{z}{\squash{R(n,s)}},\\
    \isVal{i}{\lift{\pityc{\isOf{m}{\lift{\nat}}}{R(\suc{n}, s \oplus_n m) \tolr Q(\suc{n}, s \oplus_n m)}}}.
    We need to show $Q(n,s) \iff \isComp{\dbr{dec}{base}{ind}{n}{s}{r}}{P(n,s)}$.
    Again we case on $dec\;n\;s$. In case $B(n,s)$, we proceed as before. Otherwise, we know 
    \eqComp{dec\;n\;s}{\ret{\inr{p}}}{\lift{B(n,s) + \neg B(n,s)}} and \isVal{p}{\lift{\neg} B(n,s)}.
    Then  $\dbr{dec}{base}{ind}{n}{s}{r} \mapsto^* ind\;n\;s\;\;r\;(\lam{t,r'}{}{\dbr{dec}{base}{ind}{\suc{n}}{s\oplus_n t}{r'}})$.
    By definition, 
    \isComp{ind\;n\;s\;\;r\;(\lam{t,r'}{}{\dbr{dec}{base}{ind}{\suc{n}}{s\oplus_n t}{r'}})}{P(n,s)}, 
    given that we can show 
    \[ \isVal{\lam{t,r'}{}{\dbr{dec}{base}{ind}{\suc{n}}{s\oplus_n t}{r'}}}{
      \pityc{\isOf{m}{\lift{\nat}}}{R(\suc{n}, s \oplus_n m) \tolr P(\suc{n}, s \oplus_n m)}}\]
    Hence let \isVal{t}{\lift{\nat}} and \isVal{r'}{R(\suc{n},s \oplus_n m)}. Suffices to show 
    \isComp{\dbr{dec}{base}{ind}{\suc{n}}{s\oplus_n t}{r'}}{P(\suc{n}, s \oplus_n m)}.
    By assumption, we know \isComp{i\;t\;r'}{Q(\suc{n},s \oplus_n m)},
    which implies \isComp{\dbr{dec}{base}{ind}{\suc{n}}{s \oplus_n m}{r'}}{P(\suc{n},s \oplus_n m)}.
    Hence by Lemma~\ref{lemma:headexp}, we have shown that 
    \isComp{\dbr{dec}{base}{ind}{n}{s}{r}}{P(n,s)}.
  \end{proof}

\section{GCD}

Notation: let $M =_{\nat} N$ when \eqtyc{\lift{\nat}}{M}{N}.
We can use $\mathtt{dbr}$ to define gcd:
\[
  gcdProp(x,y) \triangleq \sigmatyc{\isOf{d,l,k}{\lift{\nat}}}{\ret{d} =_{\nat} kx + ly}
\]

Define the spread $R$, bar $B$, and $P$: 

\begin{align*}
  R(n,s) \triangleq &
  \ifthen{eq_b\;n\;0}{\ret{\top}\\}{
    &\ifthen{eq_b\;n\;1}{s(0) =_{\nat} \ret{x}\\}{
      &\ifthen{eq_b\;n\;2}{
        s(0) =_{\nat} \ret{x} \lift{\times} s(1) =_{\nat} \ret{y}\\
      }{
        &s(0) =_{\nat} \ret{x} \lift{\times} s(1) =_{\nat} \ret{y}
        \lift{\times} mod\, s(n-3)\, s(n-2) =_{\nat} s(n-1)
      }
    }
  }\\
  B(n,s) \triangleq & n > 1 \lift{\times} (s(n-1) =_{\nat} \ret{\zero}) \\
  P(n,s) \triangleq &\bind{n \le_b 2\\}{x}{
    &\ifthen{x}{gcdProp(x,y)}
  {\bind{s(n-1)}{s_1}{\bind{s(n-2)}{s_2}{gcdProp(s_2,s_1)}}}}
\end{align*}

The predicates $R,B,P$ are well-formed by the various formation rules. 
Additionally, we need to show the following: 
\begin{enumerate}
  \item \isVal{bar}{\pityc{\isOf{s}{\lift{\baire}}}{\intty{\isOf{m}{\lift{\nat}}}{\squash{R(m,s)}} 
      \tolr \squash{\lift{\sigmaty{\isOf{n}{\lift{\nat}}}{B(n,s)}}}}}:\\
    Strengthen the property by considering any value of the sequence: 
    \begin{align*}
      &\pityc{\isOf{i,j}{\lift{\nat}}}{
        \pityc{\isOf{s}{\lift{\baire}}}{\intty{\isOf{m}{\lift{\nat}}}{\squash{R(m,s)}}
      \tolr \bind{s(j)}{v}{v \le i}
      \tolr\\ &\squash{\lift{\sigmaty{\isOf{n}{\lift{\nat}}}{
      B(n-j,s) \lift{\times} s(j+1) \ge F_{n-j-1} \lift{\times} s(j+2) \ge F_{n-j-2}}}}}}
    \end{align*}
    Let \isVal{j}{\lift{\nat}},
    \isVal{s}{\lift{\baire}}, \isVal{a}{\inttyc{\isOf{m}{\lift{\nat}}}{\squash{R(m,s)}}}, and 
    \isVal{p}{\bind{s(j)}{v}{v \le i}}.
    Proceed by induction:
    \begin{itemize}
      \item $i = \zero$: 
        Induction on $j$. Case $j = 0$:
        Need to find a \isVal{n}{\lift{\nat}} s.t. $B(n,s)$, $s(1) \ge F_{n-1}$, and 
        $s(2) \ge F_{n-2}$. By assumption, $s(0) \le 0$, which means $s(0) = 0$.
        Then $n = 1$ suffices, since $F_{1-1} = F_{1-2} = 0$.
        For $j = j' + 1$, suppose $s(j'+1) \le 0$. Now need to find $n$ s.t. 
        $B(n- j' -1,s)$, $s(j'+2) \ge F_{n -j' - 2}$, and $s(j'+3) \ge F_{n-j'-3}$. 
    Let \eval{s(j)}{v}. \isVal{p}{\bind{s(j)}{v}{v \le \zero}} implies that 
    $\ret{v} =_{\nat} \ret{\zero}$, which implies $s(j) =_{\nat} \ret{\zero}$.
    If $j = \zero$, by the spread law we know $s(\suc{\suc{j}}) =_{\nat} \ret{\zero}$,
    and we know \isComp{\ret{(\triv, \triv)}}{B(\suc{\suc{\suc{j}}},s)}.
    Otherwise, \isComp{\ret{(\triv, \triv)}}{B(\suc{j},s)} since $\suc{j} > 1$.
      \item $i = \suc{i'}$ for some \isVal{i'}{\lift{\nat}}:
    By induction we have a term $f$ inhabiting
        \begin{align*}
  \compc{
    &\pityc{\isOf{j}{\lift{\nat}}}{\\
          &\pityc{\isOf{s}{\lift{\baire}}}{\intty{\isOf{m}{\lift{\nat}}}{\squash{R(m,s)}} 
        \tolr \bind{s(j)}{v}{v \le i'}
        \tolr \squash{\lift{\sigmaty{\isOf{n}{\lift{\nat}}}{B(n,s)}}}}}}
        \end{align*}
    By the spread law, we know:
        \begin{gather*}
          mod\, s(j) \, s(\suc{j}) =_{\nat} s(\suc{\suc{j}})\\
        \end{gather*}
        Case $s(j+1) < s(j)$: if $s(j+1) =_{\nat} \ret{\zero}$, then we can take
        \isComp{\ret{(\triv, \triv)}}{B(\suc{\suc{j}},s)}.
        Else, we know $s(j+2) < s(j)$ by Lemma~\ref{mod:prop}, and we can apply the 
        induction hypothesis: \isComp{f\,(j+2)\,s\,a\,\triv}
        {\squash{\lift{\sigmaty{\isOf{n}{\lift{\nat}}}{B(n,s)}}}}.
        In the case that $s(j+1) = s(j)$, we know $s(j+2) = 0$ by the spread law.

        We case on $s(\suc{j})$ and $s(\suc{\suc{j}})$. If \eval{s(\suc{j})}{\zero}, 
        again \isComp{\ret{(\triv, \triv)}}{B(\suc{\suc{j}},s)}.
        Otherwise, we know $u < v \le \suc{i'}$, and hence $u \le i'$. 
        Then \isComp{f\,\suc{\suc{j}}\, s\,a\,\triv}{\squash{\lift{\sigmaty{\isOf{n}{\lift{\nat}}}{B(n,s)}}}}
        .
    \end{itemize}
  \item \isVal{ind}{\pityc{\isOf{n}{\lift{\nat}}}{\pityc{\isOf{s}{\lift{\nseq{n}}}} {\\
      \qquad\squash{R(n,s)} \tolr (\pityc{\isOf{m}{\lift{\nat}}}{R(\suc{n}, s \oplus_n m) 
        \tolr P(\suc{n}, s \oplus_n m)}) \tolr P(n,s)}}}\\
  Given \isVal{n}{\lift{\nat}}, \isVal{s}{\lift{\nseq{n}}}, 
    \isVal{r}{\squash{R(n,s)}}, 
    \isVal{i}{\pityc{\isOf{m}{\lift{\nat}}}{R(\suc{n}, s \oplus_n m)
        \tolr P(\suc{n}, s \oplus_n m)}},
    need to show $P(n,s)$. Proceed with induction on $n$.
    \begin{itemize}
      \item $n = \zero$:
        Suffices to show $P(0,s) \mapsto^* gcdProp(x,y)$.
        Note that \isComp{i\, x\, \triv}{P(1,[x])}, since 
        $R(1,[x]) \mapsto^* s(0) =_{\nat} \ret{x} \mapsto^* 
        \ret{x} =_{\nat} \ret{x}$. Thus 
        \isComp{i\, x\, \triv}{gcdProp(x,y)} since $P(0,s) \mapsto^* gcdProp(x,y)$.
      \item $n = \suc{n'}$ for \isVal{n'}{\lift{\nat}}:
        By induction, we have a term 
        \begin{gather*}
        \isVal{f}{\pityc{\isOf{s'}{\lift{\nseq{n'}}}} {
          \squash{R(n',s')} \tolr (\pityc{\isOf{m}{\lift{\nat}}}{R(\suc{n'}, s' \oplus_{n'} m) \tolr P(\suc{n'}, s' \oplus_{n'} m)})\\ \tolr P(n',s')}}
        \end{gather*}
        Suffices to show a term inhabiting $P(\suc{n'}, s)$.
        Now case on $n'$: 
        \begin{itemize}
          \item $n' = \zero$: then again it suffices to show 
            $P(\suc{n'}, s) \mapsto^* gcdProp(x,y)$.
            Since \isVal{r}{\squash{R(\suc{\zero},s)}}, we know
            $s(0) =_{\nat} \ret{x}$.
            Note that \isComp{i\, y\, (\triv,\triv)}{P(\suc{\suc{\zero}},[x,y])}, since 
            \[R(\suc{\suc{\zero}}, [x,y]) \mapsto^* s(0) =_{\nat} \ret{x} \lift{\times}
            s(1) =_{\nat} \ret{y}\]. Done since 
            $P(\suc{\suc{\zero}},[x,y]) \mapsto^* gcdProp(x,y)$.
          \item $n' = \suc{\zero}$: suffices to show
            $P(\suc{\suc{n'}}, s) \mapsto^* gcdProp(x,y)$.
            Since \isVal{r}{\squash{R(\suc{\suc{\zero}},s)}}, we know
            $s(0) =_{\nat} \ret{x}$ and $s(1) =_{\nat} \ret{y}$.
            Let \eval{x-y}{v}.
            Note that \isComp{\bind{x-y}{d}{i\, d\, (\triv,\triv,\triv)}}{
              \bind{x-y}{d}{P(3,[x,y,d])}}, since 
            \[R(3,s\oplus_3 v) \mapsto^* \ret{x} =_{\nat} \ret{x} \lift{\times} 
            \ret{y} =_{\nat} \ret{y} \lift{\times} \ret{v} =_{\nat} v\].
            Furthermore, note that 
            \begin{align*} 
              &\bind{x-y}{d}{P(3,[x,y,d])} \\
              \mapsto &gcdProp(y,v)
            \end{align*}
            Hence \isComp{\bind{x-y}{d}{\bind{i\, d\, (\triv,\triv,\triv)}{r}{
              \fst{\snd{r}}}}}{gcdProp(y,v)}.
            So we know the type $gcdProp(y,v) = 
            \sigmatyc{\isOf{d,k,l}{\lift{\z}}}{d =_{\z} k x + l v}$ is inhabited.
            To obtain $gcdProp(x,y)$, we case on $x \ge_b y$. If this is the case,
            then: 
            \begin{align*}
              d &=_{\z} k y + l v\\
                &= k y + l (x - y)\\
                & = k y + l x - l y\\
                & = k y - l y + l x\\
                & = (k - l)y + l x\\
                & = l x + (k-l)y
            \end{align*}
            Similarly for $x <_b y$, we know $d = (-l) x + (k+l)y$.
            So given \isVal{g}{gcdProp(y,v)}, we know 
            \begin{align*}
              \isComp{\bind{\fst{g}\\}{&d}{
                \bind{\fst{\snd{g}}\\}{&k}{
                  \bind{\fst{\snd{\snd{g}}}\\}{&l}{
                    &\ifthen{x \ge_b y}{(d,l,k-l,\triv)}{(d,-l,k+l,\triv)}}
            }}}{gcdProp(x,y)}
            \end{align*}
            Hence 
            \begin{align*}
              \isComp{
                \bind{\bind{x-y}{d}{i\, d\, (\triv,\triv,\triv)\\}}{&g}{
              \bind{\fst{g}\\}{&d}{
                \bind{\fst{\snd{g}}\\}{&k}{
                  \bind{\fst{\snd{\snd{g}}}\\}{&l}{
                    &\ifthen{x \ge_b y}{(d,l,k-l,\triv)}{(d,-l,k+l,\triv)}}}
            }}}{gcdProp(x,y)}
            \end{align*}
          \item $n' > \suc{\zero}$:
            Let \eval{s(n-1)}{s_1} and \eval{s(n-2)}{s_2}.
            Now we need show 
            \begin{align*}
              P(\suc{n'},s) \mapsto^* gcdProp(s_2,s_1)
            \end{align*}
            is inhabited. Then we have 
            \isComp{\bind{|s_2 - s_1|}{d}{i\, d\, (\triv,\triv,\triv)}}{
              P(\suc{\suc{n'}}, s \oplus_{\suc{n'}} d)}. 
            Note that $P(\suc{\suc{n'}}, s \oplus_{\suc{n'}} d) \mapsto^* 
            gcdProp(s_1,v)$, where \eval{|s_2 - s_1|}{v}.
            Similar to above, we case on $s_2 \ge s_1$, and derive the following:
            \begin{align*}
              \isComp{
                \bind{\bind{|s_2-s_1|}{d}{i\, d\, (\triv,\triv,\triv)\\}}{&g}{
              \bind{\fst{g}\\}{&d}{
                \bind{\fst{\snd{g}}\\}{&k}{
                  \bind{\fst{\snd{\snd{g}}}\\}{&l}{
                    &\ifthen{s_2 \ge_b s_1}{(d,l,k-l,\triv)}{(d,-l,k+l,\triv)}}}
            }}}{gcdProp(s_2,s_1)}
            \end{align*}
        \end{itemize}
        Thus realizer can be defined as 
        \begin{align*}
          ind \triangleq &\lam{n,s,r,i}{}{\\
            \ifthen{&eq\,n\,0}{i\,x\,\triv\\}{
              \ifthen{&eq\,n\,1}{i\,y\,(\triv,\triv)\\}{
                \ifthen{&eq\,n\,2\\}{
          \bind{\bind{x-y}{d}{i\, d\, (\triv,\triv,\triv)\\}}{&g}{
              \bind{\fst{g}\\}{&d}{
                \bind{\fst{\snd{g}}\\}{&k}{
                  \bind{\fst{\snd{\snd{g}}}\\}{&l}{
                    &\ifthen{x \ge_b y}{(d,l,k-l,\triv)}{(d,-l,k+l,\triv)}}}
            }\\}
          }{
            \bind{\bind{|s_2-s_1|}{d}{i\, d\, (\triv,\triv,\triv)\\}}{&g}{
              \bind{\fst{g}\\}{&d}{
                \bind{\fst{\snd{g}}\\}{&k}{
                  \bind{\fst{\snd{\snd{g}}}\\}{&l}{
                    &\ifthen{s_2 \ge_b s_1}{(d,l,k-l,\triv)}{(d,-l,k+l,\triv)}}}
            }}
          }
              }
            }
          }
        \end{align*}
    \end{itemize}
  \item \isVal{dec}{\pityc{\isOf{n}{\lift{\nat}}}{\pityc{\isOf{s}{\lift{\nseq{n}}}}{\lift{B(n,s) + \lift{\neg} B(n,s)}}}}:
  Let \isVal{n}{\lift{\nat}} and \isVal{s}{\lift{\nseq{n}}}.
  Case on $n$:
  \begin{itemize}
    \item $n = \zero, \suc{\zero}$:
      Prove the right disjunct:
      $B(n,s) \mapsto^* \bot \lift{\times} \_$, so 
      \isVal{\lam{b}{}{\bind{\fst{b}}{v}{absurd\,v}}}{\lift{\neg} B(n,s)}.
    \item $n \ge_b 2$:
      Let \eval{s(n-1)}{v}. If $v = \zero$, we can prove the left disjunct:
      \isVal{(\triv, \triv)}{B(n,s)}. Otherwise, we can prove 
      \isVal{\lam{b}{}{\triv}}{\lift{\neg}B(n,s)}.
  \end{itemize}
  The realizer for dec is 
  \begin{align*}
    dec &\triangleq \lam{n,s}{}{
      \ifthen{n \le_b 1}{\lam{b}{}{\bind{\fst{b}}{v}{absurd\,v}}}
      {
        \lam{b}{}{\triv}
      }
    }
  \end{align*}

  \item \isVal{base}{\pityc{\isOf{n}{\lift{\nat}}}{\pityc{\isOf{s}{\lift{\nseq{n}}}}{\squash{R(n,s)} \tolr 
      B(n,s) \tolr P(n,s)}}}:
    Let \isVal{n}{\lift{\nat}}, \isVal{s}{\lift{\nseq{n}}}, 
    \isVal{r}{\squash{R(n,s)}}, and \isVal{b}{B(n,s)}. We need to show 
    $P(n,s)$. Case on $n$: 
    \begin{itemize}
      \item $n = \zero,\suc{\zero}$:
        Then $B(n,s) \mapsto^* \bot \lift{\times}  (s(n-1) =_{\nat} \ret{\zero})$, so 
        \isComp{\bind{\fst{b}}{v}{absurd\,v}}{P(n,s)}.
      \item $n = \suc{\suc{\zero}}$: 
        Then $B(n,s) \mapsto^* \top \lift{\times}  (s(n-1) =_{\nat} \ret{\zero})$.
        We need to show 
        $P(n,s) \mapsto^* gcdProp(x,y)$. By $\squash{R(n,s)}$, we know 
        $s(0) =_{\nat} \ret{x}$ and $s(1) =_{\nat} \ret{y}$. Hence we know 
        $\ret{y} =_{\nat} \ret{\zero}$. Since $x = 1 \cdot x + 0 \cdot 0$,  
        \isComp{\ret{(x,\suc{\zero},\zero)}}{gcdProp(x,\zero)}.
      \item $n \ge_b 3$: 
        Then $B(n,s) \mapsto^* \top \lift{\times}  (s(n-1) =_{\nat} \ret{\zero})$.
        Let \eval{s(n-2)}{s_2}. We need to show 
        $P(n,s) \mapsto^* gcdProp(s_2,\zero)$.
        Since $s_2 = 1 \cdot s_2 + 0 \cdot 0$,  
        \isComp{\ret{(s_2,\suc{\zero},\zero)}}{gcdProp(s_2,\zero)}.
    \end{itemize}
    Hence the realizer for base is:
    \begin{align*}
      base &\triangleq \lam{n,s,r,b}{}{\\
        \ifthen{&n \le_b 1}{\bind{\fst{b}}{v}{absurd\,v}\\}{
          \ifthen{&eq\,n\,2}{\ret{(x,\suc{\zero},\zero)}\\}{
            &\bind{s(n-2)}{s_2}{(s_2,\suc{\zero},\zero)}
          }
        }}
    \end{align*}
  \item \isVal{r}{R(0,\empseq)}:
    Since $R(0,\empseq) \mapsto^* \top$, $r \triangleq \triv$ suffices.
\end{enumerate}

Finally, we can define gcd as
\[
  gcd(x,y) \triangleq \dbr{dec}{base}{ind}{0}{\empseq}{r}
\]



\section{GCD}

Notation: let $M =_{\nat} N$ when \eqtyc{\lift{\nat}}{M}{N}.
We can use $\mathtt{dbr}$ to define gcd:
\[
  gcdProp(x,y) \triangleq \sigmatyc{\isOf{d}{\lift{\nat}}}{
    \sigmatyc{\isOf{l,k}{\lift{\z}}}{\ret{\pos{d}} =_{\z} kx + ly}}
\]
(Clearly, we want the minimum such $d$. However, for the sake of brevity, let us work with this 
``truncated'' GCD, which demonstrates the main ideas).

Define the spread $R$, bar $B$, and $P$: 
\begin{align*}
  R(n,s) \triangleq &
  \ifthen{eq_b\;n\;0}{\ret{\top}\\}{
    &\ifthen{eq_b\;n\;1}{s(0) =_{\nat} \ret{max(x,y)}\\}{
      &\ifthen{eq_b\;n\;2}{
        s(0) =_{\nat} \ret{max(x,y)} \lift{\times} s(1) =_{\nat} \ret{min(x,y)} \\
      }{
        &s(0) =_{\nat} \ret{max(x,y)} \lift{\times} s(1) =_{\nat} \ret{min(x,y)}\\
        &\lift{\times} \pityc{\isOf{j}{\lift{\nat}}}{
          {1 \le j \le n-2 \to mod\, s(n-j-2)\, s(n-j-1) =_{\nat} s(n-j)}
        }
      }
    }
  }\\
  B(n,s) \triangleq & n \ge 1 \lift{\times} (s(n-1) =_{\nat} \ret{\zero}) \\
  P(n,s) \triangleq &\bind{n \le_b 1\\}{b}{
    &\ifthen{b}{gcdProp(max(x,y),min(x,y))}
  {\bind{s(n-1)}{s_1}{\bind{s(n-2)}{s_2}{gcdProp(s_2,s_1)}}}}
\end{align*}


The predicates $R,B,P$ are well-formed by the various formation rules. 
Additionally, we need to show the following: 
\begin{enumerate}
  \item \isVal{bar}{\pityc{\isOf{s}{\lift{\baire}}}{\intty{\isOf{m}{\lift{\nat}}}{\squash{R(m,s)}} 
      \tolr \squash{\lift{\sigmaty{\isOf{n}{\lift{\nat}}}{B(n,s)}}}}}:\\
    Strengthen the property by considering any value of the sequence: 
    \begin{align*}
      &\pityc{\isOf{i,j}{\lift{\nat}}}{
        \pityc{\isOf{s}{\lift{\baire}}}{\intty{\isOf{m}{\lift{\nat}}}{\squash{R(m,s)}}
      \tolr \bind{s(j)}{v}{v \le i}
      \tolr\\ &\squash{\lift{\sigmaty{\isOf{n}{\lift{\nat}}}{
        B(n,s) \lift{\times} G(n,j,s)
    }}}}}
    \end{align*}
    where 
    \[
      G(n,j,s) \triangleq 
      n > j \lift{\times}
      \ifthen{s(j) >_b s(j+1)}{
        s(j) \ge F_{n-j} \lift{\times} s(j+1) \ge F_{n-j-1}
      }{
        s(j+1) \ge F_{n-j} \lift{\times} s(j) \ge F_{n-j-1}
      }
    \]
    
And $F_n$ is defined as the Fibonacci sequence prepended with a 0, i.e.
$F = 0,0,1,1,2,3,5...$.
    Let \isVal{j}{\lift{\nat}},
    \isVal{s}{\lift{\baire}}, \isVal{a}{\inttyc{\isOf{m}{\lift{\nat}}}{\squash{R(m,s)}}}, and 
    \isVal{p}{\bind{s(j)}{v}{v \le i}}.
    Proceed by induction:
    \begin{itemize}
      \item $i = \zero$: 
        $p$ implies that $s(j) = 0$. We need to show $n$ s.t. $B(n,s)$ and 
        $G(n,j,s)$. Taking $n = j+1$ suffices since $F_0 = F_1 = 0$.
      \item $i = \suc{i'}$ for some \isVal{i'}{\lift{\nat}}:
    By induction we have a term $f$ inhabiting
        \begin{align*}
  \compc{
    &\pityc{\isOf{j}{\lift{\nat}}}{\\
          &\pityc{\isOf{s}{\lift{\baire}}}{\intty{\isOf{m}{\lift{\nat}}}{\squash{R(m,s)}} 
        \tolr\\ 
          &\bind{s(j)}{v}{v \le i'}
          \tolr \squash{\lift{\sigmaty{\isOf{n}{\lift{\nat}}}{B(n,s) \lift{\times} G(n,s)
          }}}}}}
        \end{align*}
    By the spread law, we know:
        \begin{gather*}
          mod\, s(j) \, s(\suc{j}) =_{\nat} s(\suc{\suc{j}})\\
        \end{gather*}
        Case $s(j+1) < s(j)$: if $s(j+1) =_{\nat} \ret{\zero}$, then we can take
        \isComp{\ret{(\triv, \triv)}}{B(\suc{\suc{j}},s)}. Furthermore, 
        we need to show $G(j+2,j,s) \mapsto^*
        j + 2 > j \lift{\times} s(j) \ge F_2 \lift{\times} s(j+1) \ge F_1$. This holds since 
        \[ s(j) > 0 \implies s(j) \ge 1 = F_2 \text{ and } s(j+1) = 0 = F_1\]
        Else, we know $s(j+2) < s(j+1) < s(j)$ by Lemma~\ref{mod:prop}, 
        and we can apply the 
        induction hypothesis: \isComp{f\,(j+2)\,s\,a\,\triv}
        {\squash{\lift{\sigmaty{\isOf{n}{\lift{\nat}}}{B(n,s) \lift{\times} G(n,j+2,s)}}}}.
        If $s(j+2) > s(j+3)$, we know $G(n,j+2,s) \mapsto^* n > j+2 \lift{\times}
        s(j+2) \ge F_{n-j-2} \lift{\times} s(j+3) \ge F_{n-j-3}$. 
        We can take $n$ directly, and then it suffices to show
        $n > j$ and $s(j) \ge F_{n-j} \lift{\times} s(j+1) \ge F_{n-j-1}$.
        Note that 
        \begin{align*}
          s(j+1) - s(j+2) &\ge mod\,(s(j+1))\,(s(j+2)) \\
          & = s(j+3)\\
          &\ge F_{n-j-3}\\
          &\implies s(j+1) \ge s(j+2) + F_{n-j-3}\\
          &\implies s(j+1) \ge F_{n-j-2} + F_{n-j-3}\\
          &\implies s(j+1) \ge F_{n-j-1}
        \end{align*}
        Similary, 
        \begin{align*}
          s(j) - s(j+1) &\ge s(j+2) \\
                        &\ge F_{n-j-2}\\
                        &\implies s(j) \ge s(j+1) + F_{n-j-2}\\
                        &\implies s(j) \ge F_{n-j-1} + F_{n-j-2}\\
                        &\implies s(j) \ge F_{n-j}
        \end{align*}
        So $G(n,j,s)$ holds.

        If $s(j+2) \le s(j+3)$, by Lemma~\ref{lemma:mod}, it must be the case that 
        $s(j+2) = 0$. The condition $s(j+2) \ge F_{n-j-2}$ implies that 
        $F_{n-j-2} = 0$ which means $n = j+3$. Hence it suffices to show 
        $s(j) \ge F_3$ and $s(j+1) \ge F_2$. The former holds since 
        \[s(j) > s(j+1) > s(j+2) \implies s(j) \ge 2 \ge 1 = F_3\]. The latter since
        \[s(j+1) > s(j+2) \implies s(j+1) \ge 1 = F_2\].

        Now consider the case $s(j) = s(j+1)$. By the spread law, this means $s(j+2) = 0$.
        If $s(j) = s(j+1) = 0$, we can take $n = j+1$. Clearly, $j+1 > j$, so we just 
        need to show $s(j+1) \ge F_{1} \lift{\times} s(j) \ge F_{0}$, which holds since
        $F_1 = F_0 = 0$. Else, we know $s(j) = s(j+1) \ge 1$, 
        and we can take $n = j+3$, and need to show $s(j+1) \ge F_{3} \lift{\times} s(j) \ge F_{2}$. Again, this holds since $F_3 = F_2 = 1$.

        Lastly, consider $s(j) < s(j+1)$. By the spread law, this means $s(j+2) = s(j)$. 
        If $s(j) = 0$, we can take $n = j+1$ and show that $s(j+1) \ge F_1$ and 
        $s(j) \ge F_0$ since $F_1 = F_0 = 0$. Else, we have $j = 0$, since otherwise 
        $mod\,(s(j-1))\, (s(j)) = s(j+1)$, which is impossible. But then this is
        a contradiction since the spread law requires $s(0) \ge s(1)$.
    \end{itemize}
  \item \isVal{ind}{\pityc{\isOf{n}{\lift{\nat}}}{\pityc{\isOf{s}{\lift{\nseq{n}}}} {\\
      \qquad\squash{R(n,s)} \tolr (\pityc{\isOf{m}{\lift{\nat}}}{R(\suc{n}, s \oplus_n m) 
        \tolr P(\suc{n}, s \oplus_n m)}) \tolr P(n,s)}}}:\\

  Given \isVal{n}{\lift{\nat}}, \isVal{s}{\lift{\nseq{n}}}, 
    \isVal{r}{\squash{R(n,s)}}, 
    \isVal{i}{\pityc{\isOf{m}{\lift{\nat}}}{R(\suc{n}, s \oplus_n m)
        \tolr P(\suc{n}, s \oplus_n m)}},
    need to show $P(n,s)$. Proceed with induction on $n$.
            Let $a = max(x,y)$ and $b = min(x,y)$.
    \begin{itemize}
      \item $n = \zero$:
        Suffices to show $P(0,s) \mapsto^* gcdProp(a,b)$.
        Note that \isComp{i\, a\, \triv}{P(1,[a])}, since 
        $R(1,[a]) \mapsto^* s(0) =_{\nat} \ret{a} \mapsto^* 
        \ret{a} =_{\nat} \ret{a}$, and so \isVal{\triv}{R(1,[a])}. Thus 
        \isComp{i\, a\, \triv}{gcdProp(a,b)}.
      \item $n = \suc{\zero}$: then again it suffices to show 
            $P(\suc{n'}, s) \mapsto^* gcdProp(a,b)$.
            Since \isVal{r}{\squash{R(\suc{\zero},s)}}, we know
            $s(0) =_{\nat} \ret{a}$.
            Note that \isComp{i\, b\, (\triv,\triv)}
            {P(\suc{\suc{\zero}},[a,b])}, since 
            \begin{gather*}
              R(\suc{\suc{\zero}}, [a,b]) \\
              \mapsto^* \ret{a} =_{\nat} \ret{a} \lift{\times}
              \ret{b} =_{\nat} \ret{b}
            \end{gather*}
            Done since 
            $P(\suc{\suc{\zero}},[a,b]) \mapsto^* gcdProp(a,b)$.
      \item $n > \suc{\zero}$:
            Let \eval{s(n-1)}{s_1} and \eval{s(n-2)}{s_2}.
            Now we to need show 
            \begin{align*}
              P(\suc{n'},s) \mapsto^* gcdProp(s_2,s_1)
            \end{align*}
            is inhabited.
            Since \isVal{r}{\squash{R(n,s)}}, we know
            $s(0) =_{\nat} \ret{a}$, $s(1) =_{\nat} \ret{b}$, 
            and \pityc{\isOf{j}{\lift{\nat}}}{
          {1 \le j \le n-2 \to mod\, s(n-j-2)\, s(n-j-1) =_{\nat} s(n-j)}
        }.
            Let \eval{mod\,s_2\,s_1}{v}.
            To apply $i\,v$, we need to show $R(\suc{n}, s \oplus_{n} v)$. 
            The first two conditions carry from $R(n,s)$. For the last, we 
            need to show  
            \[\pityc{\isOf{j}{\lift{\nat}}}{
              {1 \le j \le n-1 \to mod\, (s\oplus_n v)(n-j-1)\, (s\oplus_n v)(n-j) 
              =_{\nat} (s\oplus_n v)(n+1-j)} }\]
            For $2 \le j \le n-1$, this is equivalent to
            \[\pityc{\isOf{j'}{\lift{\nat}}}{
              {1 \le j' \le n-2 \to mod\, s(n-j'-2)\, s(n-j'-1) 
              =_{\nat} s(n-j')} }\]
            , which holds from above. For $j = 1$, we need to show 
            \[
              mod\, (s\oplus_n v)(n-2)\, (s\oplus_n v)(n-1) =_{\nat} (s\oplus_n v)(n)
            \]
            or that 
            \[
              mod\, s_2\, s_1 =_{\nat} v
            \]
            which is true as by construction.
            Thus we have 
            \[
            \isComp{\bind{mod\,s_2\,s_1}{d}{i\, d\, (\triv,\triv,\triv)}}{
              P(\suc{n}, s \oplus_{n} d)}\]
            Note that $P(\suc{n}, s \oplus_{n} d) \mapsto^* 
            gcdProp(s_1,v)$.
            So we know the type $gcdProp(s_1,v) =
            \sigmatyc{\isOf{d,k,l}{\lift{\z}}}{d =_{\z} k s_1  + l v}$ is inhabited.
            To obtain $gcdProp(s_2,s_1)$, rearrange the coefficients:
            \begin{align*}
              d &=_{\z} k s_1 + l v\\
                &= k s_1 + l (mod\,s_2\, s_1)\\
                & = k s_1 + l (s_2 - qs_1) 
                \tag{ where  $s_2 = q s_1 + v$ for some $q$ by Lemma~{}}\\
                & = k s_1 + l s_2 - l qs_1 \\
                & = k s_1 - l qs_1 + l s_2\\
                & = (k - lq)s_1 + l s_2\\
                & = l s_2 + (k - lq)s_1
            \end{align*}\todo{what's the lemma?}
            Hence
            \begin{align*}
              \isComp{
                \bind{&divmod\,s_2\,s_1\\}{&q,r}{
                \bind{i\, r\, (\triv,\triv,\triv)\\}{&g}{
              \bind{\fst{g}\\}{&d}{
                \bind{\fst{\snd{g}}\\}{&k}{
                  \bind{\fst{\snd{\snd{g}}}\\}{&l}{
                \ret{(d,l,k-lq,\triv)}\\}}
              }}}}{&gcdProp(s_2,s_1)}
            \end{align*}
        \end{itemize}
        Thus realizer can be defined as 
        \begin{align*}
          ind \triangleq &\lam{n,s,r,i}{}{\\
            \ifthen{&eq_b\,n\,0}{\bind{max(x,y)}{a}{i\,a\,\triv}\\}{
                \ifthen{&eq_b\,n\,1}{\bind{min(x,y)}{b}{i\,b\,(\triv,\triv)}\\}{
                    \bind{&s(n-1)}{s_1\\}{
                    \bind{&s(n-2)}{s_2\\}{
                    \bind{&divmod(s_2,s_1)}{q,r\\}{
                        \bind{&i\, r\, (\triv,\triv,\triv)}{g\\}{
                          \bind{&\fst{g}}{d\\}{
                            \bind{&\fst{\snd{g}}}{k\\}{
                              \bind{&\fst{\snd{\snd{g}}}}{l\\}{
                    &\ret{(d,l,k-lq,\triv)}}}
          }}}}}
              }
            }
          }
        \end{align*}
  \item \isVal{dec}{\pityc{\isOf{n}{\lift{\nat}}}{\pityc{\isOf{s}{\lift{\nseq{n}}}}{\lift{B(n,s) + \lift{\neg} B(n,s)}}}}:
  Let \isVal{n}{\lift{\nat}} and \isVal{s}{\lift{\nseq{n}}}.
  Case on $n$:
  \begin{itemize}
    \item $n = \zero$:
      Prove the right disjunct:
      $B(n,s) \mapsto^* \bot \lift{\times} \_$, so 
      \isVal{\lam{b}{}{\triv}}{\lift{\neg} B(n,s)}.
    \item $n \ge_b 1$:
      Let \eval{s(n-1)}{v}. If $v = \zero$, we can prove the left disjunct:
      \isVal{(\triv, \triv)}{B(n,s)}. Otherwise, we have a proof of
      $eq_b\,v\,\zero = \mathtt{ff}$, and proof the right side. Let
      \isVal{b}{B(n,s)}. We know $B(n,s) = n \ge 1 \lift{\times} s(n-1) =_{\nat} \zero$.
      This means $eq_b\,v\,\zero = \mathtt{tt}$, which gives us $\mathtt{tt} = \mathtt{ff}$.
      Since this is a contradiction, we are done.
      Hence we have \isVal{\lam{b}{}{\triv}}{\lift{\neg}B(n,s)}.
  \end{itemize}
  The realizer for dec is 
  \begin{align*}
    dec &\triangleq \lam{n,s}{}{
      \ifthen{eq_b\,n\,\zero}{\ret{\inr{\lam{b}{}{\triv}}}}
      {
        \ifthen{eq_b\,s(n-1)\,\zero}{
            \ret{\inl{(\triv,\triv)}}
            }{\ret{\inr{\lam{b}{}{\triv}}}
        }
      }
    }
  \end{align*}

  \item \isVal{base}{\pityc{\isOf{n}{\lift{\nat}}}{\pityc{\isOf{s}{\lift{\nseq{n}}}}{\squash{R(n,s)} \tolr 
      B(n,s) \tolr P(n,s)}}}:
    Let \isVal{n}{\lift{\nat}}, \isVal{s}{\lift{\nseq{n}}}, 
    \isVal{r}{\squash{R(n,s)}}, and \isVal{b}{B(n,s)}. We need to show 
    $P(n,s)$. Case on $n$: 
    \begin{itemize}
      \item $n = \zero$:
        Then $B(n,s) \mapsto^* \bot \lift{\times}  (s(n-1) =_{\nat} \ret{\zero})$,
        which is a contradiction so \isVal{\triv}{P(n,s)}.
      \item $n = \suc{\zero}$: 
        Then $B(n,s) \mapsto^* \top \lift{\times}  (s(n-1) =_{\nat} \ret{\zero})$.
        Let \eval{max(x,y)}{a} and \eval{min(x,y)}{b}.
        We need to show 
        $P(n,s) \mapsto^* gcdProp(a,b)$. By $\squash{R(n,s)}$, we know 
        $s(0) =_{\nat} \ret{a}$. Hence we know 
        $\ret{x} =_{\nat} \ret{\zero}$ and
        $\ret{y} =_{\nat} \ret{\zero}$. 
        We need to show $gcdProp(0,0)$.
        Since $0 = 0 \cdot 0 + 0 \cdot 0$,  
        \isComp{\ret{(0,0,0)}}{gcdProp(0,0)}.
      \item $n \ge_b 2$: 
        Then $B(n,s) \mapsto^* \top \lift{\times} (s(n-1) =_{\nat} \ret{\zero})$.
        Let \eval{s(n-2)}{s_2}. We need to show 
        $P(n,s) \mapsto^* gcdProp(s_2,\zero)$.
        Since $s_2 = 1 \cdot s_2 + 0 \cdot 0$,  
        \isComp{\ret{(s_2,\suc{\zero},\zero)}}{gcdProp(s_2,\zero)}.
    \end{itemize}
    Hence the realizer for base is:
    \begin{align*}
      base &\triangleq \lam{n,s,r,b}{}{\\
        \ifthen{&eq_b\,n\,\zero}{\ret{\triv}\\}{
          \ifthen{&eq\,n\,1}{\ret{(\zero,\zero,\zero)}\\}{
            &\bind{s(n-2)}{s_2}{\ret{(s_2,\suc{\zero},\zero)}}
          }
        }}
    \end{align*}
  \item \isVal{r}{R(0,\empseq)}:
    Since $R(0,\empseq) \mapsto^* \top$, $r \triangleq \triv$ suffices.
\end{enumerate}

Finally, we can define gcd as
\[
  gcd(x,y) \triangleq \dbr{dec}{base}{ind}{0}{\empseq}{r}
\]

Without further analysis, the most we can state in the type theory is the 
fact that $gcd$ has the 
correct functional behavior. In fact, the number of $\mathtt{dbr}$ calls made is 
bounded by the witness $n$ to the bar condition. However, due to the lazy nature of
bar recursion, this cannot be expressed in the theory. A wanting result would be a
lemma that states $\sigmatyc{\isOf{n}{\nat}}{B(n,s)}$ implies number of recursive 
calls is bounded by $n$. Together with the constraint on $n$ we have shown in the 
bar condition, this would suffices to bound the number of recursive calls to the 
inverse Fibonacci of $\max{(x,y)}$. However, to achieve this, we would need to know 
that the recursive call is only made once each time, which is 
a property of the \emph{presentation} of the code, not of its \emph{behavior}.
Perhaps there is a behavioral condition which implies the former, but we have not 
found such a formulation yet.

All is not lost, since we \emph{can} analyze the code further. To keep the analysis 
manageable, let us assume the existence of primitive arithmetic operators with unit
cost:

\begin{mathpar}
  \isVal{divmod}{\arrabtc{\lift{\nat}}{\arrabtc{\lift{\nat}}{\lift{\nat}}
  {\_.\constp{1}, \_.\_.\zp}}{\_.\zp, \_.\_.\zp}}

  \isVal{plus}{\arrabtc{\lift{\nat}}{\arrabtc{\lift{\nat}}{\lift{\nat}}
  {\_.\constp{1}, \_.\_.\zp}}{\_.\zp, \_.\_.\zp}}

  \isVal{minus}{\arrabtc{\lift{\nat}}{\arrabtc{\lift{\nat}}{\lift{\nat}}
  {\_.\constp{1}, \_.\_.\zp}}{\_.\zp, \_.\_.\zp}}

  \isVal{mult}{\arrabtc{\lift{\nat}}{\arrabtc{\lift{\nat}}{\lift{\nat}}
  {\_.\constp{1}, \_.\_.\zp}}{\_.\zp, \_.\_.\zp}}

  \isVal{eq_b}{\arrabtc{\lift{\nat}}{\arrabtc{\lift{\nat}}{\lift{\nat}}
  {\_.\constp{1}, \_.\_.\zp}}{\_.\zp, \_.\_.\zp}}

  \isVal{\le_b}{\arrabtc{\lift{\nat}}{\arrabtc{\lift{\nat}}{\lift{\nat}}
  {\_.\constp{1}, \_.\_.\zp}}{\_.\zp, \_.\_.\zp}}
\end{mathpar}

Furthermore, we need to specifiy the cost of the sequences. Let 
\[\nseqc{n}{a.P}{a.b.Q} \triangleq \arrabt{\lift{\nat}_n}{\lift{\nat}}{a.P,a.b.Q}\].
So far, the implementation of sequence extension was not relevant, since we were only concerned
with its functional specification. But now, we need to state exactly how the operator 
extends a sequence with a given cost specification. For this, we implement the operator as follows:
\begin{align*}
  \oplus_n \triangleq \lam{s,t,i}{}{
    \ifthen{eq_b\,i\,n}{\ret{t}}{s(i)}
  }
\end{align*}

The sequence extension operator itself has a constant cost for all \isVal{n}{\lift{\nat}}, since 
the body is immediately abstracted. Furthermore, it delivers an extended sequence such that
it is constant cost to retrieve the last element:
\[
  \isVal{\oplus_n}{\arrabtc{\nseqc{n}{a.P}{a.b.Q}}{\arrabtc{\lift{\nat}}{\nseqc{\suc{n}}{a.P'}{a.b.Q'}}
  {\_.\zp, \_.\_.\zp}}{\_.\zp, \_.\_.\zp}}
\]

where
\begin{gather*}
  P' \triangleq \ifthen{eq_b\,a\,n}{\ret{2}}{\seq{P}{p}{p+3}}\\
  Q' \triangleq \ifthen{eq_b\,a\,n}{\ret{0}}{Q}\\
\end{gather*}

Now we analyze the components of the induction:

\textbf{dec}:
\begin{align*}
dec &\triangleq \lam{n,s}{}{
  \ifthen{eq_b(n,\zero)}{\ret{\inr{\lam{b}{}{\triv}}}}
      {
        \ifthen{eq_b\,s(n-1)\,\zero}{
            \ret{\inl{(\triv,\triv)}}
            }{\ret{\inr{\lam{b}{}{\triv}}}
        }
      }
    }
\end{align*}

We already know that 
\isVal{dec}{\pityc{\isOf{n}{\lift{\nat}}}{\pityc{\isOf{s}{\lift{\nseq{n}}}}{\lift{B(n,s) + \lift{\neg} B(n,s)}}}}, so it remains to count the number of steps the branches take.

Let \isVal{n}{\lift{\nat}} and \isVal{s}{\lift{\nseqc{n}{a.P}{a.b.Q}}}.
\begin{enumerate}
  \item $n = \zero$: 
    \begin{align*}
      &\ifthen{eq_b(n,\zero)}{\ret{\inr{\lam{b}{}{\triv}}}}
      {
        \ifthen{eq_b(s(n-1),\zero)}{
          \ret{\inl{(\triv,\triv)} }
        }{\ret{\inr{\lam{b}{}{\triv}}}
        }
      }\\
      \mapsto^2& 
      \ifthen{\ret{\ttcst}}{\ret{\inr{\lam{b}{}{\triv}}}}
      {
        \ifthen{eq_b(s(n-1),\zero)}{
          \ret{\inl{(\triv,\triv)}}  
        }{\ret{\inr{\lam{b}{}{\triv}}}
        }
      }\\
      \mapsto&\lam{b}{}{\triv}
    \end{align*}
    Hence 3 steps in this case.
  \item $n \ne \zero$: 
    \begin{align*}
      &\ifthen{eq_b(n,\zero)}{\ret{\inr{\lam{b}{}{\triv}}}}
      {
        \ifthen{eq_b(s(n-1),\zero)}{
          \ret{\inl{(\triv,\triv)}}
        }{\ret{\inr{\lam{b}{}{\triv}}}
        }
      }\\
      \mapsto^2& 
      \ifthen{\ret{\ffcst}}{\ret{\inr{\lam{b}{}{\triv}}}}
      {
        \ifthen{eq_b(s(n-1),\zero)}{
          \ret{\inl{(\triv,\triv)}}  
        }{\ret{\inr{\lam{b}{}{\triv}}}}
        }
      \\
      \mapsto&
      \ifthen{eq_b(s(n-1),\zero)}{
          \ret{\inl{(\triv,\triv)}}  
        }{\ret{\lam{b}{}{\triv}
        }}
    \end{align*}
    So far 3 steps. Now case on $eq_b(s(n-1), \zero)$. Let \evalCost{s(n-1)}{c}{v}.
    \begin{itemize}
      \item $v = \zero$:
        \begin{align*}
          &\ifthen{eq_b(s(n-1),\zero)}{
            \ret{\inl{(\triv,\triv)}}
          }{\ret{\inr{\lam{b}{}{\triv}}}
        }\\
          \mapsto^2&
          \ifthen{eq_b(s (\ret{\toNum{n-1}}),\zero)}{
            \ret{\inl{(\triv,\triv)}}
          }{\ret{\inr{\lam{b}{}{\triv}}}
        }\\
          \mapsto&
          \ifthen{eq_b(s (\toNum{n-1}),\zero)}{
            \ret{\inl{(\triv,\triv)}}
          }{\ret{\inr{\lam{b}{}{\triv}}}
        }\\
        \mapsto^{1+c}&
          \ifthen{eq_b(\zero,\zero)}{
            \ret{\inl{(\triv,\triv)}}
          }{\ret{\inr{\lam{b}{}{\triv}}}
        }\\
          \mapsto^2& 
          \ifthen{\ret{\ttcst}}{
            \ret{\inl{(\triv,\triv)}}
          }{\ret{\inr{\lam{b}{}{\triv}}}
        }\\
          \mapsto& \ret{\inl{(\triv,\triv)}}
        \end{align*}
        Furthermore, we know \eval{[\toNum{n-1}/a]P}{p}, \eval{[\toNum{n-1}/a,\zero/b]Q}{q}, and
        $p \ge c + q$.
      \item  $v \ne \zero$:
        \begin{align*}
          &\ifthen{eq_b(s(n-1),\zero)}{
            \ret{\inl{(\triv,\triv)}}
          }{\ret{\inr{\lam{b}{}{\triv}}}
        }\\
          \mapsto^2&
          \ifthen{eq_b(s (\ret{\toNum{n-1}}),\zero)}{
            \ret{\inl{(\triv,\triv)}}
          }{\ret{\inr{\lam{b}{}{\triv}}}
        }\\
          \mapsto&
          \ifthen{eq_b(s (\toNum{n-1}),\zero)}{
            \ret{\inl{(\triv,\triv)}}
          }{\ret{\inr{\lam{b}{}{\triv}}}
        }\\
        \mapsto^{1+c}&
          \ifthen{eq_b(v,\zero)}{
            \ret{\inl{(\triv,\triv)}}
          }{\ret{\inr{\lam{b}{}{\triv}}}
        }\\
          \mapsto^2& 
          \ifthen{\ret{\ffcst}}{
            \ret{\inl{(\triv,\triv)}}
          }{\ret{\inr{\lam{b}{}{\triv}}}
        }\\
          \mapsto& \ret{\inr{\lam{b}{}{\triv}}}
        \end{align*}
        Again, we know \eval{[\toNum{n-1}/a]P}{p}, \eval{[\toNum{n-1}/a,v/b]Q}{q'}, and
        $p \ge c + q'$.
    \end{itemize}
    In any case we need $p+7$ steps.
\end{enumerate}

Hence $p+10$ steps will be enough in all cases:
\[
  \isVal{dec}{\arrabtc{\lift{\nat}}
  {n.\arrabtc{\lift{\nseqc{n}{a.P}{a.b.Q}}}{s.\lift{B(n,s) + \lift{\neg} B(n,s)}}
  {\_.\seq{[\toNum{n-1}/a]P}{p}{p + 10}, \_.\_.\zp}}{\_.\zp,\_.\_.\zp}}
\]

\textbf{base}:
 \begin{align*}
      base &\triangleq \lam{n,s,r,b}{}{\\
        \ifthen{&eq_b(n,\zero)}{\ret{\triv}\\}{
            \ifthen{&eq_b(n,1)}{\ret{(\zero,\zero,\zero)}\\}{
            &\bind{s(n-2)}{s_2}{\ret{(s_2,\suc{\zero},\zero)}}
          }
        }}
 \end{align*}

 Let \isVal{n}{\lift{\nat}}, \isVal{s}{\nseqc{n}{a.S}{a.b.T}}, \isVal{r}{\squash{R(n,s)}},
 and \isVal{b}{B(n,s)}.
The longest path in $base$ is when we compute $s(n-2)$, after 6 steps through
 the conditional. Let \evalCost{s(\toNum{n-2})}{c}{v}:

 \begin{align*}
&\bind{s(n-2)}{s_2}{\ret{(s_2,\suc{\zero},\zero)}}\\
   \mapsto^2 &\bind{s (\ret{\toNum{n-2}})}{s_2}{\ret{(s_2,\suc{\zero},\zero)}}\\
   \mapsto &\bind{s (\toNum{n-2})}{s_2}{\ret{(s_2,\suc{\zero},\zero)}}\\
   \mapsto^{1+c} &\bind{\ret{v}}{s_2}{\ret{(s_2,\suc{\zero},\zero)}}\\
   \mapsto &\ret{(v,\suc{\zero},\zero)}\\
 \end{align*}

 Where \eval{[\toNum{n-2}/a]/S}{p}, \eval{[\toNum{n-2}/a, v/b]T}{t} s.t. $p \ge c + t$.
Hence $p + 5$ steps suffices. 
\begin{align*}
\isVal{base}{\pityc{\isOf{n}{\lift{\nat}}}
  {\pityc{\isOf{s}{\lift{\nseqc{n}{a.S}{a.b.T}}}}{
  \squash{R(n,s)} \tolr \arrabtc{B(n,s)}{P(n,s)}
  {\_.\seq{[\toNum{n-2}/a]S}{p}{p + 11},\_.\_.\zp}}}}
\end{align*}

\begin{lemma}
  Let \isVal{n}{\lift{\nat}}, \isVal{s}{\nseqc{n}{a.P_n}{a.b.Q}}, and \isVal{r}{\squash{R(n,s)}},
  where $P_n = \ifthen{eq_b(a,n-1)}{\ret{2}}{P'}$ for some $P'$. Then
    \begin{itemize}
      \item If $n = \zero$:\\
        \stepIn{\dbr{dec}{base}{ind}{n}{s}{r}}{c}{\dbr{dec}{base}{ind}{\suc{n}}{s'}{r'}}, and
        \isVal{s'}{\nseqc{\suc{n}}{a.P_{\suc{n}}}{a.b.Q}}, \isVal{r'}{\squash{R(\suc{n},s)}}.
      \item If \eval{s(n-1)}{\zero}:\\
        \stepIn{\dbr{dec}{base}{ind}{n}{s}{r}}{c}{\ret{d,k,l}},
        \isVal{d}{\lift{\nat}}, \isVal{k}{\lift{\z}}, \isVal{l}{\lift{\z}}, and
        \isVal{s'}{\nseqc{\suc{n}}{a.P_{\suc{n}}}{a.b.Q}}, \isVal{r'}{\squash{R(\suc{n},s)}}.
      \item Else, if \eval{s(n-1)}{v}, $v \ne \zero$:
        \begin{enumerate}
      \item If $n = 0,1$:
        \stepIn{\dbr{dec}{base}{ind}{n}{s}{r}}{c}{\dbr{dec}{base}{ind}{\suc{n}}{s'}{r'}}, and
        \isVal{s'}{\nseqc{\suc{n}}{a.P_{\suc{n}}}{a.b.Q}}, \isVal{r'}{\squash{R(\suc{n},s)}}.
      \item If $n > 1$:
        \stepIn{\dbr{dec}{base}{ind}{n}{s}{r}}{c}{
          \bind{\dbr{dec}{base}{ind}{\suc{n}}{s'}{r'}}{g}{M}
        },
        \isVal{s'}{\nseqc{\suc{n}}{a.P_{\suc{n}}}{a.b.Q}}, \isVal{r'}{\squash{R(n,s)}},\\
        and \openCComp{\isOf{g}{\lift{\nat} \lift{\times} \lift{\z} \lift{\times} \lift{\z}}}{M}{
      \lift{\nat} \lift{\times} \lift{\z} \lift{\times} \lift{\z}}{c_1}{c_2}
  \end{enumerate}
    \end{itemize}
\end{lemma}

So we know that there are a constant number of steps between each extension up the spread. 
Since the spread is linear (no branches), it suffices to find the length of the 
initial segment of non-zero entries.

\begin{lemma}
  Let \isVal{n}{\lift{\nat}}, \isVal{s}{\lift{\nseq{n}}}, and \isVal{r}{\squash{R(n,s)}}.
  Then the first $\zero$ in $s$ occurs at index $i$, such that:
  \begin{cases}
    x \ge F_i \text{ and } y \ge F_{i-1} \text{ if $x > y$ }\\
    y \ge F_i \text{ and } x \ge F_{i-1} \text{ o.w.}\\
  \end{cases}
\end{lemma}

Hence $max(x,y) \ge F_i$, and $i \le F^{-1}(max(x,y))$. We know the computation of gcd 
starts from the empty sequence, so:

\begin{verbatim}
dbr(dec,base,ind,0,[],r)
==>^c1 dbr(dec,base,ind,1,[max(x,y)],r) 
==>^c2 dbr(dec,base,ind,2,[max(x,y),min(x,y)],r) 
==>^c3 do dbr(dec,base,ind,3,[max(x,y),min(x,y),s2],r) as g in M
==>^c3 do 
        do 
          dbr(dec,base,ind,4,[max(x,y),min(x,y),s2,s3],r) 
        as g in M
       as g in M
...
==>^c3 do 
        do 
        ...
          do
            dbr(dec,base,ind,i+1,s,r) 
          as g in M
        as g in M
       as g in M
==>^c4 do 
        do 
        ...
          do
            ret(d,k,l)
          as g in M
        as g in M
       as g in M
...
==> ret(d',k',l')
\end{verbatim}

Where we first build up a $i$-layer stack of \texttt{do} statements and then 
pop out $i$ times again. The dominating term is $c\cdot i$, for some constant $c$.


\begin{verbatim}
gcd(x,y) = dbr(dec,base,ind,0,[],r) 
==> do dec(0,[]) as d in 
    case d of
    inl(p) => base 0 [] r p 
    inr(p) => ind 0 [] r (\t,r'. do [] ++ [t] as s in dbr(dec,base,ind,1,s,r')) 
==>^(3+12) do ret(inr(\b.*)) as d in 
    case d of
    inl(p) => base 0 [] r p 
    inr(p) => ind 0 [] r (\t,r'. do [] ++ [t] as s in dbr(dec,base,ind,1,s,r')) 
==> case inr(\b.*) of
    inl(p) => base 0 [] r p 
    inr(p) => ind 0 [] r (\t,r'. do [] ++ [t] as s in dbr(dec,base,ind,1,s,r')) 
==> ind 0 [] r (\t,r'. do [] ++ [t] as s in dbr(dec,base,ind,1,s, r')) 
==>^7 if(eq 0 0; (\t,r'. do [] ++ [t] as s in dbr(dec,base,ind,1,s,r')) x * ; ...)
==> if(ret(tt); (\t,r'. do [] ++ [t] as s in dbr(dec,base,ind,1,s,r')) x * ; ...)
==> (\t,r'. do [] ++ [t] as s in dbr(dec,base,ind,1,s,r')) x * 
==>^2 do [] ++ [x] as s in dbr(dec,base,ind,1,s,*)
==> do ret([x]) as s in dbr(dec,base,ind,1,s,*)
==> dbr(dec,base,ind,1,ret([x]),*)
==> do dec(1,[x]) as b in 
    case b of
    inl(p) => base 1 [x] r p
    inr(p) => ind 1 [x] r (\t,r'. do [x] ++ [t] as s in dbr(dec,base,ind,2,s,r')) 
==> 
\end{verbatim}

If $x = 0$:
\begin{verbatim}
==>^(7+2) do ret(inl(*,*)) as b in 
    case b of
    inl(p) => base 1 [x] r p
    inr(p) => ind 1 [x] r (\t,r'. do [x] ++ [t] as s in dbr(dec,base,ind,2,s,r')) 
==> case inl(*,*) of 
    inl(p) => base 1 [x] r p
    inr(p) => ind 1 [x] r (\t,r'. do [x] ++ [t] as s in dbr(dec,base,ind,2,s,r')) 
==> base 1 [x] r (*,*)
==>^(4+4) ret(0,0,0)
\end{verbatim}

TODO: CHANGE X AND Y TO MAX(x,y) and MIN(x,y)!!!!
Otherwise:
\begin{verbatim}
==>^(7+2) do ret(inr(\_.*)) as b in 
    case b of
    inl(p) => base 1 [x] r p
    inr(p) => ind 1 [x] r (\t,r'. do [x] ++ [t] as s in dbr(dec,base,ind,2,s,r')) 
==> case inr(\_.*) of 
    inl(p) => base 1 [x] r p
    inr(p) => ind 1 [x] r (\t,r'. do [x] ++ [t] as s in dbr(dec,base,ind,2,s,r')) 
==> ind 1 [x] r (\t,r'. do [x] ++ [t] as s in dbr(dec,base,ind,2,s,r')) 
==>^4 if(eq 1 0; ... ; 
        if(eq 1 1; (\t,r'. do [x] ++ [t] as s in dbr(dec,base,ind,2,s,r')) y (*,*); ...)) 
==> if(ret(ff); ... ; 
        if(eq 1 1; (\t,r'. do [x] ++ [t] as s in dbr(dec,base,ind,2,s,r')) y (*,*); ...)) 
==> if(eq 1 1; (\t,r'. do [x] ++ [t] as s in dbr(dec,base,ind,2,s,r')) y (*,*); ...) 
==> if(ret(tt); (\t,r'. do [x] ++ [t] as s in dbr(dec,base,ind,2,s,r')) y (*,*); ...) 
==> (\t,r'. do [x] ++ [t] as s in dbr(dec,base,ind,2,s,r')) y (*,*)
==>^2 do [x] ++ [t] as s in dbr(dec,base,ind,2,s,(*,*))
==> do ret([x,t]) as s in dbr(dec,base,ind,2,s,(*,*))
==> dbr(dec,base,ind,2,[x,t],(*,*))
==> do dec(2,[x]) as b in 
    case b of
    inl(p) => base 1 [x] r p
    inr(p) => ind 1 [x] r (\t,r'. do [x] ++ [t] as s in dbr(dec,base,ind,2,s,r')) 
==> 


\end{verbatim}




\section{Appendix}

\subsection{Proof of Lemma~\ref{lemma:types}}
\begin{proof}
blah
\begin{enumerate}
\item Unicity:\\
Since $\tau^*$ is the least (pre)-fixed point, for all $\tau$, if $\tau$ is a pre-fixed point 
($\text{Types}(v,\tau) \subseteq \tau$), $\tau^* \subseteq \tau$. Thus define a $\tau$ such that all elements 
satisfy unicity w.r.t $\tau^*$: 
\[\tau = \{(A,B,\phi) \mid \forall \phi'.\, \tau^*(A,B,\phi') \implies \phi = \phi'\}\]
and show that $\tau$ is a pre-fixed point, which suffices to show that $\tau^*$ satisfies unicity.

Let $(S,T,\phi) \in \text{Types}(v,\tau)$. NTS $(S,T,\phi) \in \tau$.
\begin{itemize}
  \item $(S,T,\phi) = (\arrabt{A}{a.B}{a.P,a.b.Q}, \arrabt{A'}{a.B'}{a.P',a.b.Q'}, \phi)$:\\
  Suppose $\tau^*(S,T,\phi')$ for some $\phi'$. STS $\phi = \phi'$.
  Since $\tau^*$ is a fixed point, we have $\text{Types}(v,\tau^*)(S,T,\phi')$.
  Then, by the definition of $\text{Fun}(\tau^*,\tau^*)$, we know:
  \begin{enumerate}
  \item $\exists \alpha^*.\, \sameType{A}{A'}{\alpha^*}{\tau^*}$. This means 
  \eval{A}{A_0}, \eval{A'}{A'_0}, and $\tau^*(A_0,A'_0,\alpha^*)$.
  \item $\exists \beta^*.\, \sameTypeOne{a}{\alpha^*}{B}{B'}{\beta^*}{\tau^*}$. This means
  for all $v,v'$ s.t. $\alpha^*(v,v')$, \eval{[v/a]B}{B_0}, \eval{[v'/a]B'}{B'_0}, and $\tau^*(B_0,B'_0,\beta^*_{v,v'})$.
  \item \sameComp{\isOf{a}{\alpha^*}}{P}{P'}{\omega}
  \item \sameComp{\isOf{a}{\alpha^*},\isOf{b}{\beta^*}}{Q}{Q'}{\omega}
  \end{enumerate}
  Furthermore, by assumption and the definition of $\text{Fun}(\tau,\tau)$, we know:
  \begin{enumerate}
  \item $\exists \alpha.\, \sameType{A}{A'}{\alpha}{\tau}$. This means 
  \eval{A}{A_0}, \eval{A'}{A'_0}, and $\tau(A_0,A'_0,\alpha)$. By definition of $\tau$, $\alpha = \alpha^*$.
  \item $\exists \beta.\, \sameTypeOne{a}{\alpha}{B}{B'}{\beta}{\tau}$. This means
  for all $v,v'$ s.t. $\alpha(v,v')$, \eval{[v/a]B}{B_0}, \eval{[v'/a]B'}{B'_0}, and $\tau(B_0,B'_0,\beta_{v,v'})$.
  Let $v,v'$ s.t. $\alpha(v,v')$. By previous, $\alpha^*(v,v')$. Furthermore, we have
  \eval{[v/a]B}{B_0}, \eval{[v'/a]B'}{B'_0}, and $\tau^*(B_0,B'_0,\beta^*_{v,v'})$.
  By definition of $\tau$, we know $\beta_{v,v'} = \beta^*_{v,v'}$. Hence we know $\beta = \beta^*$.
  \end{enumerate}
  Since $\phi'$ is determined by $a.P,a.b.Q$, $\alpha^* = \alpha$ and $\beta^* = \beta$ and so is $\phi$, we conclude $\phi = \phi'$.
\end{itemize}
\item PER valuation, Symmetry, Transitivity:\\
Define $\tau$ where all elements have the above properties:
\begin{align*}
  \tau(A,B,\phi) &\iff \phi \text{ is a PER}\\
& \land \tau^*(A,B,\phi) \land \tau^*(B,A,\phi)\\
& \land \forall C,\phi'.\, \tau^*(B,C,\phi') \implies \tau^*(A,C,\phi') \land \phi = \phi'\\
& \land \forall C,\phi'.\, \tau^*(C,A,\phi') \implies \tau^*(C,B,\phi') \land \phi = \phi'
\end{align*}

Let $(S,T,\phi) \in \text{Types}(v,\tau)$. NTS $(S,T,\phi) \in \tau$.
\begin{itemize}
  \item $(S,T,\phi) = (\ccomp{P}{\isOf{a}{A}}{Q}, \ccomp{P'}{\isOf{a}{A'}}{Q'}, \phi)$\\
  By assumption, we know:
    \begin{enumerate}[label=\textbf{\Alph*)}]
     \item $\exists \alpha.\, \sameType{A}{A'}{\alpha}{\tau}$. This means 
  \eval{A}{A_0}, \eval{A'}{A'_0}, and $\tau(A_0,A'_0,\alpha)$. By definition of $\tau$, we know: 
  \begin{enumerate}[label=\textbf{A1.\arabic*}]
    \item $\alpha$ is a PER 
    \item $\tau^*(A_0,A'_0,\alpha)$ and $\tau^*(A'_0,A_0,\alpha)$ \label{fact:alphaper}
    \item $\forall C,\alpha'.\, \tau^*(A'_0,C,\alpha') \implies \tau^*(A_0,C,\alpha') \land \alpha = \alpha'$ \label{fact:propA}
    \item $\forall C,\alpha'.\, \tau^*(C,A_0,\alpha') \implies \tau^*(C,A'_0,\alpha') \land \alpha = \alpha'$
   \end{enumerate}
     \item \sameComp{}{P}{P'}{\omega} \label{fact:peq}
     \item \sameComp{\isOf{a}{\alpha}}{Q}{Q'}{\omega} \label{fact:qeq}
     \item $\phi = \{(\thunk{M}, \thunk{M'}) \mid
       \sameCComp{}{M}{M'}{\alpha}{P}{a.Q} \}$
    \end{enumerate}
    We NTS:
    \begin{itemize}
      \item $\phi$ is a \textbf{PER}.\\
        \textbf{For symmetry}, let $\phi(\thunk{M}, \thunk{M'})$. 
        NTS $\phi(\thunk{M'}, \thunk{M})$. 
        First, \sameComp{}{M'}{M}{\alpha} holds by assumption and the fact that 
        $\alpha$ is a PER by \textbf{A1.1}. Next, given 
        \begin{gather*}
          \evalCost{M}{c}{v}, \evalCost{M'}{c'}{v'}, \eval{P}{\bar{p}},\\
          \eval{[v/a]Q}{\bar{q}},\\
          p \ge c + q \land p \ge c' + q 
        \end{gather*}
        we need to show $p \ge c'  + q$ and $p \ge c + q$, which is true.

        \textbf{For transitivity}, let $\phi(\thunk{M}, \thunk{M'})$ and 
        $\phi(\thunk{M'}, \thunk{M''})$. NTS $\phi(\thunk{M}, \thunk{M''})$. Again,
        \sameComp{}{M}{M''}{\alpha} holds by assumption and the fact that $\alpha$ is 
        a PER. Next, given 
        \begin{gather*}
          \land \evalCost{M}{c}{v}, \evalCost{M'}{c'}{v'}, \evalCost{M''}{c''}{v''},\\
          \eval{P}{\bar{p}},
          \eval{[v/a]Q}{\bar{q}},
            \eval{[v'/a]Q}{\bar{q'}}, \eval{[v''/a]Q}{\bar{q''}}\\
          p \ge c + q \land p \ge c' + q\\
          p \ge c' + q' \land p \ge c'' + q'
        \end{gather*}
        we need to show $p \ge c  + q$ and $p \ge c'' + q$, which is the case 
        since $q = q'$.
      \item \textbf{}$\tau^*(\ccomp{P}{\isOf{a}{A}}{Q}, \ccomp{P'}{\isOf{a}{A'}}{Q'}, \phi)$ and
        $\tau^*(\ccomp{P'}{\isOf{a}{A'}}{Q'}, \ccomp{P}{\isOf{a}{A}}{Q}, \phi)$.
        For the former, suffices to show:
        \begin{itemize}
           \item $\exists \alpha^*.\, \sameType{A}{A'}{\alpha^*}{\tau^*}$. This holds by 
             taking $\alpha^* = \alpha$. 
           \item \sameComp{}{P}{P'}{\omega}. This holds by Fact~\ref{fact:peq}.
           \item \sameComp{\isOf{a}{\alpha}}{Q}{Q'}{\omega}. This holds by 
             Fact~\ref{fact:qeq}.
        \end{itemize}
        Then since $\phi$ is determined by $P$,$Q$, and $\alpha^* = \alpha$, we get the 
        result since $\tau^*$ is the least fixed point (and hence closed under Comp).
        For the latter:
        \begin{itemize}
           \item $\exists \alpha^*.\, \sameType{A'}{A}{\alpha^*}{\tau^*}$. This holds by 
             taking $\alpha^* = \alpha$. 
           \item \sameComp{}{P'}{P}{\omega}. This holds by Fact~\ref{fact:peq} and 
             the fact that $\omega$ is a PER.
           \item \sameComp{\isOf{a}{\alpha^*}}{Q'}{Q}{\omega}. 
             Let $\alpha(v,v')$. Suffices to show that 
             \eval{[v/a]Q'}{u}, \eval{[v'/a]Q}{u'}, and $\omega(u,u')$ for some $u,u'$. 
             By Fact~\ref{fact:qeq}, we know \eval{[v'/a]Q}{Q_{v'}}, 
             \eval{[v/a]Q'}{Q'_{v}}, and $\omega(Q_{v'}, Q'_v)$. Thus suffices to take 
             $u = Q_{v'}, u' = Q'_v$.
        \end{itemize}
        Similar to above. Only need to show in addition that
        \[\sameCComp{}{M}{M'}{\alpha}{P}{a.Q} \iff \sameCComp{}{M}{M'}{\alpha}{P'}{a.Q'}\]
        which holds by Lemma~\ref{lemma:samecomp}.
      \item Let $C, \phi'$ s.t. $\tau^*(\ccomp{P'}{\isOf{a}{A'}}{Q'},C,\phi')$. NTS
        $\tau^*(\ccomp{P}{\isOf{a}{A}}{Q},C,\phi')$ and $\phi = \phi'$.
        From the assumption, we know:
        \begin{enumerate}
          \item $C = \ccomp{P''}{\isOf{a}{A''}}{Q''}$ for some $P'',A'',Q''$.
          \item $\exists \alpha^*.\, \sameType{A'}{A''}{\alpha^*}{\tau^*}$. This means  
            \eval{A'}{A'_0}, \eval{A''}{A''_0}, and $\tau^*(A'_0,A''_0,\alpha^*)$. 
            \label{fact:factA}
          \item \sameComp{}{P'}{P''}{\omega} \label{fact:peq1}
          \item \sameComp{\isOf{a}{\alpha^*}}{Q'}{Q''}{\omega} \label{fact:qeq1}
          \item $\phi' = \{(\thunk{M}, \thunk{M'}) \mid 
            \sameCComp{}{M}{M'}{\alpha^*}{P'}{a.Q'}
            \}$ \label{fact:phiprime}
        \end{enumerate}
        For $\tau^*(\ccomp{P}{\isOf{a}{A}}{Q}, \ccomp{P''}{\isOf{a}{A''}}{Q''},\phi')$, 
        STS:
        \begin{itemize}
          \item $\exists \alpha'.\, \sameType{A}{A''}{\alpha'}{\tau^*}$. 
            Since $\Downarrow$ is deterministic, this means showing 
            \eval{A}{A_0}, \eval{A''}{A''_0}, and $\tau^*(A_0,A''_0,\alpha')$.
            By Fact~\ref{fact:propA} and~\ref{fact:factA}, we know 
            $\tau^*(A_0,A''_0,\alpha^*)$ and $\alpha = \alpha^*$.
            Hence the result holds with $\alpha' = \alpha^*$.
          \item \sameComp{}{P}{P''}{\omega}. Since $\omega$ is a PER, this follows from
            Fact~\ref{fact:peq} and~\ref{fact:peq1}.
          \item \sameComp{\isOf{a}{\alpha^*}}{Q}{Q''}{\omega}. By applying 
            Lemma~\ref{omega:per} with Fact~\ref{fact:qeq} and~\ref{fact:qeq1}.
          \item $\phi' = \{(\thunk{M}, \thunk{M'}) \mid 
            \sameCComp{}{M}{M'}{\alpha'}{P}{a.Q}
            \}$.

            This holds by applying Lemma~\ref{lemma:samecomp}.
        \end{itemize}
    \end{itemize}
\item $(S,T,\phi) =(\arrabt{A}{a.B}{R}, \arrabt{A'}{a.B'}{R'}, \phi)$\\
    From the assumption, \fact{we know}{1} (some conditions omitted): 
  \begin{enumerate}
  \item $\exists \alpha.\, \sameType{A}{A'}{\alpha}{\tau}$. This means 
  \eval{A}{A_0}, \eval{A'}{A'_0}, and $\tau(A_0,A'_0,\alpha)$. By definition of $\tau$, we know: 
  \begin{enumerate}[label=\textbf{A.\arabic*}]
    \item $\alpha$ is a PER
    \item $\tau^*(A_0,A'_0,\alpha)$ and $\tau^*(A'_0,A_0,\alpha)$ 
    \item $\forall C,\alpha'.\, \tau^*(A'_0,C,\alpha') \implies \tau^*(A_0,C,\alpha') \land \alpha = \alpha'$ \label{fact:1}
    \item $\forall C,\alpha'.\, \tau^*(C,A_0,\alpha') \implies \tau^*(C,A'_0,\alpha') \land \alpha = \alpha'$
   \end{enumerate}
 \item $\exists \beta.\, \sameTypeOne{a}{\alpha}{B}{B'}{\beta}{\tau}$. \label{fact:10}
  Let $v,v'$ s.t. $\alpha(v,v')$.
  By definition, \eval{[v/a]B}{B_0}, \eval{[v'/a]B'}{B'_0}, and $\tau(B_0,B'_0,\beta_{v,v'})$.
  By definition of $\tau$, we know:
  \begin{enumerate}[label=\textbf{B.\arabic*}]
    \item $\beta_{v,v'}$ is a PER \label{fact:7}
    \item $\tau^*(B_0,B'_0,\beta_{v,v'})$ and $\tau^*(B'_0,B_0,\beta_{v,v'})$ \label{fact:11}
    \item $\forall C,\beta'.\, \tau^*(B'_0,C,\beta') \implies \tau^*(B_0,C,\beta') \land \beta_{v,v'} = \beta'$ \label{fact:4}
    \item $\forall C,\beta'.\, \tau^*(C,B_0,\beta') \implies \tau^*(C,B'_0,\beta') \land \beta_{v,v'} = \beta'$ \label{fact:5}
   \end{enumerate}
   \item $R \sim R' \in \kappa(\alpha, a.\beta)$
  \item $\phi = \{(\lam{a}{\_}{M}, \lam{a}{}{M'}) \mid\\
  \sameComp{\isOf{a}{\alpha}}{M}{M'}{\beta}\\
  \land\fequiv{\cost{a.M}}{\cost{a.M'}}\\
  \land\eval{R}{\relpot{a}{P}{a.b}{Q}}, \eval{R'}{\relpot{a}{P'}{a.b}{Q'}}\\
  \land\widehat{a.P} \succeq \cost{a.M} + \{(v, \widehat{a.Q}(v,\widehat{a.M}(v))) \mid \forall v.\, \alpha(v,v)\}\\
  \land\widehat{a.P'} \succeq \cost{a.M'} + \{(v, \widehat{a.Q'}(v,\widehat{a.M'}(v))) \mid \forall v.\, \alpha(v,v)\}$
  \end{enumerate}
  Recall to show $(S,T,\phi) \in \tau$, STS:
  \begin{enumerate}
    \item $\phi \text{ is a PER}$. For symmetry, assume $(\lam{a}{\_}{M}, \lam{a}{}{M'}) \in \phi$.
      STS $(\lam{a}{\_}{M'}, \lam{a}{}{M}) \in \phi$:
      \begin{itemize}
        \item \sameComp{\isOf{a}{\alpha}}{M'}{M}{\beta}: let $\alpha(v,v')$. NTS 
          \eval{[v/a]M'}{u}, \eval{[v'/a]M}{u'}, and $\beta_{v,v'}(u,u')$ for some $u,u'$. 
          Since $\alpha$ is a PER, $\alpha(v',v)$.
          By definition of $\phi$, we know \eval{[v/a]M'}{M'_0}, \eval{[v'/a]M}{M_0}, and $\beta_{v',v}(M_0,M'_0)$.
          If we set $u = M_0$, $u' = M'_0$, it remains to show $\beta_{v,v'} = \beta_{v',v}$.
          Instantiating Fact~\ref{fact:10} with $v',v'$, we have  
          \eval{[v'/a]B}{B_{v'}}, \eval{[v'/a]B'}{B'_{v'}}, and $\tau(B_{v'},B'_{v'},\beta_{v',v'})$.
          By Fact~\ref{fact:11}$_{v',v'}$, we know $\tau^*(B'_{v'},B_{v'},\beta_{v',v'})$.
          Now instantiating Fact~\ref{fact:10} with $v,v'$, we have 
          \eval{[v/a]B}{B_{v}}, \eval{[v'/a]B'}{B'_{v'}}, and $\tau(B_{v},B'_{v'},\beta_{v,v'})$.
          Applying Fact~\ref{fact:4}$_{v,v'}$ with 
          $C = B_{v'}$ and $\beta' = \beta_{v',v'}$, we know $\tau^*(B_v, B_{v'}, \beta_{v',v'})$ and 
          $\beta_{v,v'} = \beta_{v',v'}$. 
          Lastly, instantiate Fact~\ref{fact:10} with $v',v$, we have 
          \eval{[v'/a]B}{B_{v'}}, \eval{[v/a]B'}{B'_{v}}, and $\tau(B_{v'},B'_{v},\beta_{v',v})$.
          By Fact~\ref{fact:11}$_{v',v}$, we know $\tau^*(B'_{v},B_{v'},\beta_{v',v})$.
          Applying Fact~\ref{fact:5}$_{v',v'}$ with 
          $C = B'_{v}$ and $\beta' = \beta_{v',v}$, we know $\tau^*(B'_v, B_{v'}, \beta_{v',v})$ and 
          $\beta_{v',v'} = \beta_{v',v}$. 
          Hence $\beta_{v,v'} = \beta_{v',v}$ as required. We can instantiate Fact~\ref{fact:10} with 
          $v',v'$ and $v',v$ since $\alpha$ is a PER and we assumed $\alpha(v,v')$.
        \item \fequiv{\cost{a.M}}{\cost{a.M'}}: Lemma~\ref{lemma:asymp}
        \item $\widehat{a.P} \succeq \cost{a.M'} + \{(v, \widehat{a.Q}(v,\widehat{a.M'}(v))) \mid \forall v.\, \alpha(v,v)\}$: Lemma~\ref{lemma:dominance}
        \item $\widehat{a.P'} \succeq \cost{a.M} + \{(v, \widehat{a.Q'}(v,\widehat{a.M}(v))) \mid \forall v.\, \alpha(v,v)\}$: Lemma~\ref{lemma:dominance}
      \end{itemize}
      For transitivity, let $(\lam{a}{\_}{M}, \lam{a}{}{M'}) \in \phi$ and 
      $(\lam{a}{\_}{M'}, \lam{a}{}{M''}) \in \phi$ 
      STS $(\lam{a}{\_}{M}, \lam{a}{}{M''}) \in \phi$:
      \begin{itemize}
        \item \sameComp{\isOf{a}{\alpha}}{M}{M''}{\beta}: let $\alpha(v,v')$. NTS 
        \eval{[v/a]M}{u}, \eval{[v'/a]M''}{u'}, and $\beta_{v,v'}(u,u')$ for some $u,u'$. 
        By assumption, \sameComp{\isOf{a}{\alpha}}{M}{M'}{\beta}, and we know 
          \eval{[v/a]M}{M_v}, \eval{[v'/a]M'}{M'_{v'}}, and $\beta_{v,v'}(M_v,M'_{v'})$.
          Since $\alpha$ is a PER, $\alpha(v',v')$. 
          By assumption, \sameComp{\isOf{a}{\alpha}}{M'}{M''}{\beta}, and we know
          \eval{[v'/a]M'}{M_{v'}}, \eval{[v'/a]M''}{M''_{v'}}, and $\beta_{v',v'}(M'_{v'},M''_{v'})$.
          By the previous argument, we know $\beta_{v,v'} = \beta_{v',v'}$, and thus
          $\beta_{v,v'}(M'_{v'},M''_{v'})$.
          Since $\beta_{v,v'}$ is a PER, $\beta_{v,v'}(M_{v},M''_{v'})$, as required.
        \item \fequiv{\cost{a.M}}{\cost{a.M''}}: Lemma~\ref{lemma:asymp}
        \item $\widehat{a.P} \succeq \cost{a.M''} + \{(v, \widehat{a.Q}(v,\widehat{a.M''}(v))) \mid \forall v.\, \alpha(v,v)\}$: Lemma~\ref{lemma:dominance}
        \item $\widehat{a.P'} \succeq \cost{a.M} + \{(v, \widehat{a.Q'}(v,\widehat{a.M}(v))) \mid \forall v.\, \alpha(v,v)\}$: Lemma~\ref{lemma:dominance}
      \end{itemize}

    \item $\tau^*(S,T,\phi) \land \tau^*(T,S,\phi)$: the former follows directly from \applyAss{1} and the conditions entailed. 
      For the latter, since $\alpha$ is a PER and $\beta$ is a family of PERs, 
      the first two conditions follow. The rest folllow exactly from the \applyAss{1}
      in conjunction with Lemma~\ref{lemma:asymp}.
  \item $\forall C,\phi'.\, \tau^*(T,C,\phi') \implies \tau^*(S,C,\phi') \land \phi = \phi'$:\\
   Suppose $\tau^*(T,C,\phi')$ for some $\phi'$. This means that: 
  \begin{enumerate}[label=\textbf{C.\arabic*}]
  \item $C = \arrabt{\isOf{a}{A''}}{B''}{R''}$ for some $A'',B'',R''$
  \item $\exists \alpha^*.\, \sameType{A'}{A''}{\alpha^*}{\tau^*}$. This means 
    \eval{A'}{A'_0}, \eval{A''}{A''_0}, and $\tau^*(A'_0,A''_0,\alpha^*)$. \label{fact:2}
  \item $\exists \beta^*.\, \sameTypeOne{a}{\alpha^*}{B'}{B''}{\beta^*}{\tau^*}$. This means
  for all $v,v'$ s.t. $\alpha^*(v,v')$, \eval{[v/a]B'}{B'_0}, \eval{[v'/a]B''}{B''_0}, and $\tau^*(B'_0,B''_0,\beta^*_{v,v'})$. \label{fact:3}
  \item $R' \sim R'' \in \kappa(\alpha^*, a.\beta^*)$
  \item $\phi' = \{(\lam{a}{\_}{M}, \lam{a}{}{M'}) \mid\\
  \sameComp{\isOf{a}{\alpha^*}}{M}{M'}{\beta^*}\\
  \land\fequiv{\cost{a.M}}{\cost{a.M'}}\\
  \land\eval{R'}{\relpot{a}{P'}{a.b}{Q'}}, \eval{R''}{\relpot{a}{P''}{a.b}{Q''}}\\
  \land\widehat{a.P'} \succeq \cost{a.M} + \{(v, \widehat{a.Q'}(v,\widehat{a.M}(v))) \mid \forall v.\, \alpha^*(v,v)\}\\
  \land\widehat{a.P''} \succeq \cost{a.M'} + \{(v, \widehat{a.Q''}(v,\widehat{a.M'}(v))) \mid \forall v.\, \alpha^*(v,v)\}$
  \end{enumerate}
  Now STS $\tau^*(S,C,\phi') \land \phi = \phi'$. For the former, STS:
      \begin{itemize}
        \item $\exists \alpha'.\, \sameType{A}{A''}{\alpha'}{\tau^*}$. Take $\alpha' = \alpha^*$. Since $\Downarrow$ deterministic,
          \eval{A}{A_0}, \eval{A''}{A''_0}, and STS $\tau^*(A_0,A''_0,\alpha^*)$. By Fact~\ref{fact:2}, we know  $\tau^*(A'_0,A''_0,\alpha^*)$, 
          and applying Fact~\ref{fact:1} gives $\tau^*(A_0,A''_0,\alpha^*)$ and $\alpha = \alpha^* \triangleq \alpha'$. 
        \item $\exists \beta'.\, \sameTypeOne{a}{\alpha'}{B}{B''}{\beta'}{\tau^*}$. Take $\beta' = \beta^*$. Let $v,v'$ s.t. $\alpha^*(v,v')$. Then STS
          \eval{[v/a]B}{u}, \eval{[v'/a]B''}{u'}, and $\tau^*(u,u',\beta^*_{v,v'})$ for some $u,u'$. 
          Since $\alpha(v',v')$, apply Fact~\ref{fact:3}, and we know 
          \eval{[v'/a]B'}{B'_{v'}}, \eval{[v'/a]B''}{B''_{v'}}, and
          $\tau^*(B'_{v'},B''_{v'},\beta^*_{v',v'})$.
          Now, applying Fact~\ref{fact:10} with $v,v'$, we know
          \eval{[v/a]B}{B_v}, \eval{[v'/a]B'}{B'_{v'}}, and $\tau(B_v,B'_{v'},\beta_{v,v'})$. 
          Further, applying Fact~\ref{fact:4}$_{v,v'}$ with
          $C = B''_{v'}$ and $\beta' = \beta^*_{v',v'}$, 
          we know $\tau^*(B_v,B''_{v'},\beta^*_{v',v'})$ and $\beta_{v,v'} = \beta^*_{v',v'}$.
          Hence the result holds by taking $u = B_v$ and $u' = B''_{v'}$ and showing
          $\beta^*_{v',v'} = \beta^*_{v,v'}$. Note that we have shown $\beta_{v,v'} = \beta_{v',v}$, 
          so STS $\beta_{v',v} = \beta^*_{v,v'}$. 
          So, applying Fact~\ref{fact:10} with $v',v$, we know
          \eval{[v'/a]B}{B_{v'}}, \eval{[v/a]B'}{B'_{v}}, and $\tau(B_{v'},B'_{v},\beta_{v',v})$. 
          By Fact~\ref{fact:3}$_{v,v'}$, we know 
          \eval{[v/a]B'}{B'_{v}}, \eval{[v'/a]B''}{B''_{v'}}, and $\tau^*(B'_{v},B''_{v'},\beta^*_{v,v'})$. 
          Now by Fact~\ref{fact:4}$_{v',v}$ with $C =  B''_{v'}$ and $\beta' = \beta^*_{v,v'}$, 
          we know $\tau^*(B_{v'}, B''_{v'}, \beta^*_{v,v'})$ and $\beta_{v',v} = \beta^*_{v,v'}$, as required.
        \item $R \sim R'' \in \kappa(\alpha',a.\beta')$. Since $\alpha' \triangleq \alpha^* = \alpha$, 
          to apply Lemma~\ref{lemma:kappa}, STS $\beta' \triangleq \beta^* = \beta$, which was shown above.
        \item \begin{gather*}
      \phi' = \{(\lam{a}{\_}{M}, \lam{a}{}{M'}) \mid 
          \sameComp{\isOf{a}{\alpha^*}}{M}{M'}{\beta^*}\}\\
      \end{gather*}
          Thus STS
          \begin{gather*}
            \widehat{a.P} \succeq \cost{a.M} + \{(v, \widehat{a.Q}(v,\widehat{a.M}(v))) \mid \forall v.\, \alpha^*(v,v)\}\\
          \iff \widehat{a.P'} \succeq \cost{a.M} + \{(v, \widehat{a.Q'}(v,\widehat{a.M}(v))) \mid \forall v.\, \alpha^*(v,v)\} 
          \end{gather*}
          Which holds by Lemma~\ref{lemma:dominance}.
      \end{itemize}
      Lastly, we NTS $\phi = \phi'$. Since the value PERs are determined by $\alpha^* = \alpha'$, $\beta^* = \beta'$, 
      and 
          \begin{gather*}
            \widehat{a.P''} \succeq \cost{a.M'} + \{(v, \widehat{a.Q''}(v,\widehat{a.M}(v))) \mid \forall v.\, \alpha^*(v,v)\}\\
          \iff \widehat{a.P'} \succeq \cost{a.M'} + \{(v, \widehat{a.Q'}(v,\widehat{a.M}(v))) \mid \forall v.\, \alpha^*(v,v)\} 
          \end{gather*}
          by Lemma~\ref{lemma:asympsum},
      we conclude $\phi^* = \phi'$.
  \item $\forall C,\phi'.\, \tau^*(C,S,\phi') \implies \tau^*(C,T,\phi') \land \phi = \phi'$. Similar to above.
  \end{enumerate}

\end{itemize}
\end{enumerate}
\end{proof}


\bibliography{bib}{}
\bibliographystyle{plain}
\end{document}
