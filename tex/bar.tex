


\section{Bar Induction}

\begin{definition}(Admissibility of Bar Induction)\label{lemma:birule}
  If 
  \begin{align*}
    &\text{(wfb)}\quad \openTypeComp{\Gamma, \isOf{n}{\lift{\nat}}, \isOf{s}{\lift{\nseq{n}}}}{B}\\
    &\text{(wfs)}\quad \openTypeComp{\Gamma, \isOf{n}{\lift{\nat}}, \isOf{s}{\lift{\nseq{n}}}}{R}\\
    &\text{(init)}\quad \openComp{\Gamma}{M_0}{R(\zero,\empseq)}\\
    &\text{(bar)}\quad \openComp{\Gamma,\isOf{s}{\lift{\baire}}, 
      \isOf{z}{\lift{\intty{\isOf{m}{\lift{\nat}}}{\squash{R(m,s)}}}}}{M_1}
      {\squash{\lift{\sigmaty{\isOf{n}{\lift{\nat}}}{B(n,s)}}}}\\ 
    &\text{(base)}\quad \openComp{\Gamma,\isOf{n}{\lift{\nat}}, \isOf{s}{\lift{\nseq{n}}},
    \isOf{z}{\squash{R(n,s)}}, \isOf{b}{B(n,s)}}{M_2}{P(n,s)}\\
    &\text{(ind)} \quad \openComp{\Gamma,\isOf{n}{\lift{\nat}}, \isOf{s}{\lift{\nseq{n}}},
    \isOf{z}{\squash{R(n,s)}},\\
    &\qquad\qquad\isOf{i}{\lift{\pityc{\isOf{m}{\lift{\nat}}}{
      R(\suc{n}, s \oplus_n m) \tolr P(\suc{n}, s \oplus_n m)}}}}{M_2}{P(n,s)}
  \end{align*}
  Then 
  \[
  \openComp{\Gamma}{\_}{\squash{P(0,\empseq)}}
  \]
\end{definition}

\begin{proof}
  By LEM, we get $\lift{\neg} \squash{P(0,\empseq)}$, which gives $\lift{\neg} P(0,\empseq)$.
  Taking the contrapositive of (ind), we have a function $f$ inhabiting
  \begin{align*}
    &\pityc{\isOf{n}{\lift{\nat}}}{
    \pityc{\isOf{s}{\lift{\nseq{n}}}}{
      \pityc{\isOf{z}{\squash{R(n,s)}}}{\\
        &\lift{\neg}P(n,s) 
      \tolr \sigmatyc{\isOf{m}{\lift{\nat}}}{R(\suc{n}, s \oplus_n m) 
      \lift{\times} \lift{\neg} P(\suc{n}, s \oplus_n m)}}}}
  \end{align*}
  Since $\lift{\neg}P(0,\empseq)$ and $R(0,\empseq)$, we can instantiate $f$ to obtain 
  \isVal{m}{\lift{\nat}} s.t. $R(1,\empseq \oplus_0 m)$ and $\lift{\neg}P(1,\empseq \oplus_0 m)$.
  Iterating this gives us a sequence \isVal{\alpha}{\lift{\baire}} s.t. 
  \inttyc{\isOf{m}{\lift{\nat}}}{\squash{R(m,\alpha)}} is inhabited. 
  It also shows that for all \isVal{n}{\lift{\nat}}, $\lift{\neg}P(n,\alpha)$.
  By (bar), we know it's true that there is \isVal{n}{\lift{\nat}} s.t. $B(n,\alpha)$.
  But by (base), we know that $P(n,\alpha)$, which is a contradiction.
\end{proof}

\begin{definition}(DBR)
  \begin{mathpar}
  \inferrule{}{
    \dbr{dec}{base}{ind}{n}{s}{r} \mapsto \\
  \seq{\thunk{dec\;n\;s}}{x}{
    \case{x}{p}{base\;n\;s\;r\;p}{\_}{ind\;n\;s\;\;r\;(\lam{t,r'}{}{\dbr{dec}{base}{ind}{\suc{n}}{s\oplus_n t}{r'}})}
}}
  \end{mathpar}
\end{definition}

\begin{lemma}\label{lemma:dbr}
  $\mathtt{dbr}$ implements the following induction principle: 
  \begin{align*}
    \mathsf{BID} = 
    &\ulcorner\pityc{\isOf{R,B,P}{\pityc{\isOf{n}{\lift{\nat}}}{(\embed\nseq{n}) \tolr (\embed\type{})}}}{\\
          &\pityc{\isOf{s}{\lift{\baire}}}{\intty{\isOf{m}{\lift{\nat}}}{\squash{R(m,s)}} 
      \tolr \squash{\lift{\sigmaty{\isOf{n}{\lift{\nat}}}{B(n,s)}}}}\\
      \tolr &(\pityc{\isOf{n}{\lift{\nat}}}{\pityc{\isOf{s}{\lift{\nseq{n}}}}
      {\\
      & \squash{R(n,s)} \tolr (\pityc{\isOf{m}{\lift{\nat}}}{R(\suc{n}, s \oplus_n m) 
        \tolr P(\suc{n}, s \oplus_n m)}) \tolr P(n,s)}})\\
      \tolr &(\pityc{\isOf{n}{\lift{\nat}}}{\pityc{\isOf{s}{\lift{\nseq{n}}}}{\lift{B(n,s) + \neg B(n,s)}}}) \\
      \tolr &(\pityc{\isOf{n}{\lift{\nat}}}{\pityc{\isOf{s}{\lift{\nseq{n}}}}{\squash{R(n,s)} \tolr 
      B(n,s) \tolr P(n,s)}})\\
      \tolr & R(0,\empseq)\\
      \tolr & P(0,\empseq)\urcorner
    }
  \end{align*}
\end{lemma}

\begin{proof}
  Need to show that 
  \isVal{\lam{R,B,P,bar,ind,dec,base,r}{}{\dbr{dec}{base}{ind}{\zero}{\empseq}{r}}}{\mathsf{BID}}.
  We have assumptions:
  \begin{align*}
    &\isVal{R}{\pityc{\isOf{n}{\lift{\nat}}}{\embed\nseq{n} \tolr \embed\type{}}}\\
    &\isVal{B}{\pityc{\isOf{n}{\lift{\nat}}}{\embed\nseq{n} \tolr \embed\type{}}}\\
    &\isVal{P}{\pityc{\isOf{n}{\lift{\nat}}}{\embed\nseq{n} \tolr \embed\type{}}}\\
    &\isVal{bar}{\pityc{\isOf{s}{\lift{\baire}}}{\intty{\isOf{m}{\lift{\nat}}}{\squash{R(m,s)}} 
      \tolr \squash{\lift{\sigmaty{\isOf{n}{\lift{\nat}}}{B(n,s)}}}}}\\
    &\isVal{ind}{\pityc{\isOf{n}{\lift{\nat}}}{\pityc{\isOf{s}{\lift{\nseq{n}}}}
      {\\
      &\qquad\squash{R(n,s)} \tolr (\pityc{\isOf{m}{\lift{\nat}}}{R(\suc{n}, s \oplus_n m) 
        \tolr P(\suc{n}, s \oplus_n m)}) \tolr P(n,s)}}}\\
    &\isVal{dec}{\pityc{\isOf{n}{\lift{\nat}}}{\pityc{\isOf{s}{\lift{\nseq{n}}}}{\lift{B(n,s) + \neg B(n,s)}}}}\\
    &\isVal{base}{\pityc{\isOf{n}{\lift{\nat}}}{\pityc{\isOf{s}{\lift{\nseq{n}}}}{\squash{R(n,s)} \tolr 
      B(n,s) \tolr P(n,s)}}}\\
    &\isVal{r}{R(0,\empseq)}
  \end{align*}
  We need to find a term $M$ s.t. \isComp{M}{P(\zero,\empseq)}.
  Define \[
    Q \triangleq \lam{n,s}{}{\lift{\eqty{P(n,s)}{\dbr{dec}{base}{ind}{n}{s}{r}}{\dbr{dec}{base}{ind}{n}{s}{r}}}}\]. 
  By Rule~\ref{lemma:birule}, we know that $\squash{Q(\zero,\empseq)}$ is inhabited.
  This means that \isVal{\triv}{Q(\zero,\empseq)}, which implies that\\
  \eqComp{\dbr{dec}{base}{ind}{\zero}{\empseq}{r}}{\dbr{dec}{base}{ind}{\zero}{\empseq}{r}}{P(\zero,\empseq)}.
  Hence it suffices to show that the hypotheses of Rule~\ref{lemma:birule} holds. 
  Premises (wfb), (wfs), (init), and (bar) follow directly from assumptions $B,R,r,$ and $bar$.
  For (base), let \isVal{n}{\lift{\nat}}, \isVal{s}{\lift{\nseq{n}}}, \isVal{z}{\squash{R(n,s)}},
  and \isVal{b}{B(n,s)}. We need to show 
  \begin{align*}
    Q(n,s) &\iff \eqty{P(n,s)}{\dbr{dec}{base}{ind}{n}{s}{r}}{\dbr{dec}{base}{ind}{n}{s}{r}} \\
           &\iff \isComp{\dbr{dec}{base}{ind}{n}{s}{r}}{P(n,s)}
  \end{align*}
  We case on the result of $dec\;n\;s$.
  \begin{itemize}
    \item $\eqComp{dec\;n\;s}{\ret{\inl{p}}}{\lift{B(n,s) + \neg B(n,s)}}$ and \isVal{p}{B(n,s)}:\\
      Then $\dbr{dec}{base}{ind}{n}{s}{r} \mapsto^* base\;n\;s\;r\;p$. By assumption,
      \isComp{base\;n\;s\;r\;p}{P(n,s)}, and by Lemma~\ref{lemma:headexp},  
      \isComp{\dbr{dec}{base}{ind}{n}{s}{r}}{P(n,s)}.
    \item $\eqComp{dec\;n\;s}{\ret{\inr{p}}}{\lift{B(n,s) + \neg B(n,s)}}$ and \isVal{p}{\lift{\neg} B(n,s)}:\\
      Then \isComp{p\;b}{\bot}, which is a contradiction.
  \end{itemize}
  Hence the base case holds.
  For (ind), let \isVal{n}{\lift{\nat}}, \isVal{s}{\lift{\nseq{n}}},
    \isVal{z}{\squash{R(n,s)}},\\
    \isVal{i}{\lift{\pityc{\isOf{m}{\lift{\nat}}}{R(\suc{n}, s \oplus_n m) \tolr Q(\suc{n}, s \oplus_n m)}}}.
    We need to show $Q(n,s) \iff \isComp{\dbr{dec}{base}{ind}{n}{s}{r}}{P(n,s)}$.
    Again we case on $dec\;n\;s$. In case $B(n,s)$, we proceed as before. Otherwise, we know 
    \eqComp{dec\;n\;s}{\ret{\inr{p}}}{\lift{B(n,s) + \neg B(n,s)}} and \isVal{p}{\lift{\neg} B(n,s)}.
    Then  $\dbr{dec}{base}{ind}{n}{s}{r} \mapsto^* ind\;n\;s\;\;r\;(\lam{t,r'}{}{\dbr{dec}{base}{ind}{\suc{n}}{s\oplus_n t}{r'}})$.
    By definition, 
    \isComp{ind\;n\;s\;\;r\;(\lam{t,r'}{}{\dbr{dec}{base}{ind}{\suc{n}}{s\oplus_n t}{r'}})}{P(n,s)}, 
    given that we can show 
    \[ \isVal{\lam{t,r'}{}{\dbr{dec}{base}{ind}{\suc{n}}{s\oplus_n t}{r'}}}{
      \pityc{\isOf{m}{\lift{\nat}}}{R(\suc{n}, s \oplus_n m) \tolr P(\suc{n}, s \oplus_n m)}}\]
    Hence let \isVal{t}{\lift{\nat}} and \isVal{r'}{R(\suc{n},s \oplus_n m)}. Suffices to show 
    \isComp{\dbr{dec}{base}{ind}{\suc{n}}{s\oplus_n t}{r'}}{P(\suc{n}, s \oplus_n m)}.
    By assumption, we know \isComp{i\;t\;r'}{Q(\suc{n},s \oplus_n m)},
    which implies \isComp{\dbr{dec}{base}{ind}{\suc{n}}{s \oplus_n m}{r'}}{P(\suc{n},s \oplus_n m)}.
    Hence by Lemma~\ref{lemma:headexp}, we have shown that 
    \isComp{\dbr{dec}{base}{ind}{n}{s}{r}}{P(n,s)}.
  \end{proof}

\section{GCD}

Notation: let $M =_{\nat} N$ when \eqtyc{\lift{\nat}}{M}{N}.
We can use $\mathtt{dbr}$ to define gcd:
\[
  gcdProp(x,y) \triangleq \sigmatyc{\isOf{d,l,k}{\lift{\nat}}}{\ret{d} =_{\nat} kx + ly}
\]

Define the spread $R$, bar $B$, and $P$: 

\begin{align*}
  R(n,s) \triangleq &
  \ifthen{eq_b\;n\;0}{\ret{\top}\\}{
    &\ifthen{eq_b\;n\;1}{s(0) =_{\nat} \ret{x}\\}{
      &\ifthen{eq_b\;n\;2}{
        s(0) =_{\nat} \ret{x} \lift{\times} s(1) =_{\nat} \ret{y}\\
      }{
        &s(0) =_{\nat} \ret{x} \lift{\times} s(1) =_{\nat} \ret{y}
        \lift{\times} mod\, s(n-3)\, s(n-2) =_{\nat} s(n-1)
      }
    }
  }\\
  B(n,s) \triangleq & n > 1 \lift{\times} (s(n-1) =_{\nat} \ret{\zero}) \\
  P(n,s) \triangleq &\bind{n \le_b 2\\}{x}{
    &\ifthen{x}{gcdProp(x,y)}
  {\bind{s(n-1)}{s_1}{\bind{s(n-2)}{s_2}{gcdProp(s_2,s_1)}}}}
\end{align*}

The predicates $R,B,P$ are well-formed by the various formation rules. 
Additionally, we need to show the following: 
\begin{enumerate}
  \item \isVal{bar}{\pityc{\isOf{s}{\lift{\baire}}}{\intty{\isOf{m}{\lift{\nat}}}{\squash{R(m,s)}} 
      \tolr \squash{\lift{\sigmaty{\isOf{n}{\lift{\nat}}}{B(n,s)}}}}}:\\
    Strengthen the property by considering any value of the sequence: 
    \begin{align*}
      &\pityc{\isOf{i,j}{\lift{\nat}}}{
        \pityc{\isOf{s}{\lift{\baire}}}{\intty{\isOf{m}{\lift{\nat}}}{\squash{R(m,s)}}
      \tolr \bind{s(j)}{v}{v \le i}
      \tolr\\ &\squash{\lift{\sigmaty{\isOf{n}{\lift{\nat}}}{
      B(n-j,s) \lift{\times} s(j+1) \ge F_{n-j-1} \lift{\times} s(j+2) \ge F_{n-j-2}}}}}}
    \end{align*}
    Let \isVal{j}{\lift{\nat}},
    \isVal{s}{\lift{\baire}}, \isVal{a}{\inttyc{\isOf{m}{\lift{\nat}}}{\squash{R(m,s)}}}, and 
    \isVal{p}{\bind{s(j)}{v}{v \le i}}.
    Proceed by induction:
    \begin{itemize}
      \item $i = \zero$: 
        Induction on $j$. Case $j = 0$:
        Need to find a \isVal{n}{\lift{\nat}} s.t. $B(n,s)$, $s(1) \ge F_{n-1}$, and 
        $s(2) \ge F_{n-2}$. By assumption, $s(0) \le 0$, which means $s(0) = 0$.
        Then $n = 1$ suffices, since $F_{1-1} = F_{1-2} = 0$.
        For $j = j' + 1$, suppose $s(j'+1) \le 0$. Now need to find $n$ s.t. 
        $B(n- j' -1,s)$, $s(j'+2) \ge F_{n -j' - 2}$, and $s(j'+3) \ge F_{n-j'-3}$. 
    Let \eval{s(j)}{v}. \isVal{p}{\bind{s(j)}{v}{v \le \zero}} implies that 
    $\ret{v} =_{\nat} \ret{\zero}$, which implies $s(j) =_{\nat} \ret{\zero}$.
    If $j = \zero$, by the spread law we know $s(\suc{\suc{j}}) =_{\nat} \ret{\zero}$,
    and we know \isComp{\ret{(\triv, \triv)}}{B(\suc{\suc{\suc{j}}},s)}.
    Otherwise, \isComp{\ret{(\triv, \triv)}}{B(\suc{j},s)} since $\suc{j} > 1$.
      \item $i = \suc{i'}$ for some \isVal{i'}{\lift{\nat}}:
    By induction we have a term $f$ inhabiting
        \begin{align*}
  \compc{
    &\pityc{\isOf{j}{\lift{\nat}}}{\\
          &\pityc{\isOf{s}{\lift{\baire}}}{\intty{\isOf{m}{\lift{\nat}}}{\squash{R(m,s)}} 
        \tolr \bind{s(j)}{v}{v \le i'}
        \tolr \squash{\lift{\sigmaty{\isOf{n}{\lift{\nat}}}{B(n,s)}}}}}}
        \end{align*}
    By the spread law, we know:
        \begin{gather*}
          mod\, s(j) \, s(\suc{j}) =_{\nat} s(\suc{\suc{j}})\\
        \end{gather*}
        Case $s(j+1) < s(j)$: if $s(j+1) =_{\nat} \ret{\zero}$, then we can take
        \isComp{\ret{(\triv, \triv)}}{B(\suc{\suc{j}},s)}.
        Else, we know $s(j+2) < s(j)$ by Lemma~\ref{mod:prop}, and we can apply the 
        induction hypothesis: \isComp{f\,(j+2)\,s\,a\,\triv}
        {\squash{\lift{\sigmaty{\isOf{n}{\lift{\nat}}}{B(n,s)}}}}.
        In the case that $s(j+1) = s(j)$, we know $s(j+2) = 0$ by the spread law.

        We case on $s(\suc{j})$ and $s(\suc{\suc{j}})$. If \eval{s(\suc{j})}{\zero}, 
        again \isComp{\ret{(\triv, \triv)}}{B(\suc{\suc{j}},s)}.
        Otherwise, we know $u < v \le \suc{i'}$, and hence $u \le i'$. 
        Then \isComp{f\,\suc{\suc{j}}\, s\,a\,\triv}{\squash{\lift{\sigmaty{\isOf{n}{\lift{\nat}}}{B(n,s)}}}}
        .
    \end{itemize}
  \item \isVal{ind}{\pityc{\isOf{n}{\lift{\nat}}}{\pityc{\isOf{s}{\lift{\nseq{n}}}} {\\
      \qquad\squash{R(n,s)} \tolr (\pityc{\isOf{m}{\lift{\nat}}}{R(\suc{n}, s \oplus_n m) 
        \tolr P(\suc{n}, s \oplus_n m)}) \tolr P(n,s)}}}\\
  Given \isVal{n}{\lift{\nat}}, \isVal{s}{\lift{\nseq{n}}}, 
    \isVal{r}{\squash{R(n,s)}}, 
    \isVal{i}{\pityc{\isOf{m}{\lift{\nat}}}{R(\suc{n}, s \oplus_n m)
        \tolr P(\suc{n}, s \oplus_n m)}},
    need to show $P(n,s)$. Proceed with induction on $n$.
    \begin{itemize}
      \item $n = \zero$:
        Suffices to show $P(0,s) \mapsto^* gcdProp(x,y)$.
        Note that \isComp{i\, x\, \triv}{P(1,[x])}, since 
        $R(1,[x]) \mapsto^* s(0) =_{\nat} \ret{x} \mapsto^* 
        \ret{x} =_{\nat} \ret{x}$. Thus 
        \isComp{i\, x\, \triv}{gcdProp(x,y)} since $P(0,s) \mapsto^* gcdProp(x,y)$.
      \item $n = \suc{n'}$ for \isVal{n'}{\lift{\nat}}:
        By induction, we have a term 
        \begin{gather*}
        \isVal{f}{\pityc{\isOf{s'}{\lift{\nseq{n'}}}} {
          \squash{R(n',s')} \tolr (\pityc{\isOf{m}{\lift{\nat}}}{R(\suc{n'}, s' \oplus_{n'} m) \tolr P(\suc{n'}, s' \oplus_{n'} m)})\\ \tolr P(n',s')}}
        \end{gather*}
        Suffices to show a term inhabiting $P(\suc{n'}, s)$.
        Now case on $n'$: 
        \begin{itemize}
          \item $n' = \zero$: then again it suffices to show 
            $P(\suc{n'}, s) \mapsto^* gcdProp(x,y)$.
            Since \isVal{r}{\squash{R(\suc{\zero},s)}}, we know
            $s(0) =_{\nat} \ret{x}$.
            Note that \isComp{i\, y\, (\triv,\triv)}{P(\suc{\suc{\zero}},[x,y])}, since 
            \[R(\suc{\suc{\zero}}, [x,y]) \mapsto^* s(0) =_{\nat} \ret{x} \lift{\times}
            s(1) =_{\nat} \ret{y}\]. Done since 
            $P(\suc{\suc{\zero}},[x,y]) \mapsto^* gcdProp(x,y)$.
          \item $n' = \suc{\zero}$: suffices to show
            $P(\suc{\suc{n'}}, s) \mapsto^* gcdProp(x,y)$.
            Since \isVal{r}{\squash{R(\suc{\suc{\zero}},s)}}, we know
            $s(0) =_{\nat} \ret{x}$ and $s(1) =_{\nat} \ret{y}$.
            Let \eval{x-y}{v}.
            Note that \isComp{\bind{x-y}{d}{i\, d\, (\triv,\triv,\triv)}}{
              \bind{x-y}{d}{P(3,[x,y,d])}}, since 
            \[R(3,s\oplus_3 v) \mapsto^* \ret{x} =_{\nat} \ret{x} \lift{\times} 
            \ret{y} =_{\nat} \ret{y} \lift{\times} \ret{v} =_{\nat} v\].
            Furthermore, note that 
            \begin{align*} 
              &\bind{x-y}{d}{P(3,[x,y,d])} \\
              \mapsto &gcdProp(y,v)
            \end{align*}
            Hence \isComp{\bind{x-y}{d}{\bind{i\, d\, (\triv,\triv,\triv)}{r}{
              \fst{\snd{r}}}}}{gcdProp(y,v)}.
            So we know the type $gcdProp(y,v) = 
            \sigmatyc{\isOf{d,k,l}{\lift{\z}}}{d =_{\z} k x + l v}$ is inhabited.
            To obtain $gcdProp(x,y)$, we case on $x \ge_b y$. If this is the case,
            then: 
            \begin{align*}
              d &=_{\z} k y + l v\\
                &= k y + l (x - y)\\
                & = k y + l x - l y\\
                & = k y - l y + l x\\
                & = (k - l)y + l x\\
                & = l x + (k-l)y
            \end{align*}
            Similarly for $x <_b y$, we know $d = (-l) x + (k+l)y$.
            So given \isVal{g}{gcdProp(y,v)}, we know 
            \begin{align*}
              \isComp{\bind{\fst{g}\\}{&d}{
                \bind{\fst{\snd{g}}\\}{&k}{
                  \bind{\fst{\snd{\snd{g}}}\\}{&l}{
                    &\ifthen{x \ge_b y}{(d,l,k-l,\triv)}{(d,-l,k+l,\triv)}}
            }}}{gcdProp(x,y)}
            \end{align*}
            Hence 
            \begin{align*}
              \isComp{
                \bind{\bind{x-y}{d}{i\, d\, (\triv,\triv,\triv)\\}}{&g}{
              \bind{\fst{g}\\}{&d}{
                \bind{\fst{\snd{g}}\\}{&k}{
                  \bind{\fst{\snd{\snd{g}}}\\}{&l}{
                    &\ifthen{x \ge_b y}{(d,l,k-l,\triv)}{(d,-l,k+l,\triv)}}}
            }}}{gcdProp(x,y)}
            \end{align*}
          \item $n' > \suc{\zero}$:
            Let \eval{s(n-1)}{s_1} and \eval{s(n-2)}{s_2}.
            Now we need show 
            \begin{align*}
              P(\suc{n'},s) \mapsto^* gcdProp(s_2,s_1)
            \end{align*}
            is inhabited. Then we have 
            \isComp{\bind{|s_2 - s_1|}{d}{i\, d\, (\triv,\triv,\triv)}}{
              P(\suc{\suc{n'}}, s \oplus_{\suc{n'}} d)}. 
            Note that $P(\suc{\suc{n'}}, s \oplus_{\suc{n'}} d) \mapsto^* 
            gcdProp(s_1,v)$, where \eval{|s_2 - s_1|}{v}.
            Similar to above, we case on $s_2 \ge s_1$, and derive the following:
            \begin{align*}
              \isComp{
                \bind{\bind{|s_2-s_1|}{d}{i\, d\, (\triv,\triv,\triv)\\}}{&g}{
              \bind{\fst{g}\\}{&d}{
                \bind{\fst{\snd{g}}\\}{&k}{
                  \bind{\fst{\snd{\snd{g}}}\\}{&l}{
                    &\ifthen{s_2 \ge_b s_1}{(d,l,k-l,\triv)}{(d,-l,k+l,\triv)}}}
            }}}{gcdProp(s_2,s_1)}
            \end{align*}
        \end{itemize}
        Thus realizer can be defined as 
        \begin{align*}
          ind \triangleq &\lam{n,s,r,i}{}{\\
            \ifthen{&eq\,n\,0}{i\,x\,\triv\\}{
              \ifthen{&eq\,n\,1}{i\,y\,(\triv,\triv)\\}{
                \ifthen{&eq\,n\,2\\}{
          \bind{\bind{x-y}{d}{i\, d\, (\triv,\triv,\triv)\\}}{&g}{
              \bind{\fst{g}\\}{&d}{
                \bind{\fst{\snd{g}}\\}{&k}{
                  \bind{\fst{\snd{\snd{g}}}\\}{&l}{
                    &\ifthen{x \ge_b y}{(d,l,k-l,\triv)}{(d,-l,k+l,\triv)}}}
            }\\}
          }{
            \bind{\bind{|s_2-s_1|}{d}{i\, d\, (\triv,\triv,\triv)\\}}{&g}{
              \bind{\fst{g}\\}{&d}{
                \bind{\fst{\snd{g}}\\}{&k}{
                  \bind{\fst{\snd{\snd{g}}}\\}{&l}{
                    &\ifthen{s_2 \ge_b s_1}{(d,l,k-l,\triv)}{(d,-l,k+l,\triv)}}}
            }}
          }
              }
            }
          }
        \end{align*}
    \end{itemize}
  \item \isVal{dec}{\pityc{\isOf{n}{\lift{\nat}}}{\pityc{\isOf{s}{\lift{\nseq{n}}}}{\lift{B(n,s) + \lift{\neg} B(n,s)}}}}:
  Let \isVal{n}{\lift{\nat}} and \isVal{s}{\lift{\nseq{n}}}.
  Case on $n$:
  \begin{itemize}
    \item $n = \zero, \suc{\zero}$:
      Prove the right disjunct:
      $B(n,s) \mapsto^* \bot \lift{\times} \_$, so 
      \isVal{\lam{b}{}{\bind{\fst{b}}{v}{absurd\,v}}}{\lift{\neg} B(n,s)}.
    \item $n \ge_b 2$:
      Let \eval{s(n-1)}{v}. If $v = \zero$, we can prove the left disjunct:
      \isVal{(\triv, \triv)}{B(n,s)}. Otherwise, we can prove 
      \isVal{\lam{b}{}{\triv}}{\lift{\neg}B(n,s)}.
  \end{itemize}
  The realizer for dec is 
  \begin{align*}
    dec &\triangleq \lam{n,s}{}{
      \ifthen{n \le_b 1}{\lam{b}{}{\bind{\fst{b}}{v}{absurd\,v}}}
      {
        \lam{b}{}{\triv}
      }
    }
  \end{align*}

  \item \isVal{base}{\pityc{\isOf{n}{\lift{\nat}}}{\pityc{\isOf{s}{\lift{\nseq{n}}}}{\squash{R(n,s)} \tolr 
      B(n,s) \tolr P(n,s)}}}:
    Let \isVal{n}{\lift{\nat}}, \isVal{s}{\lift{\nseq{n}}}, 
    \isVal{r}{\squash{R(n,s)}}, and \isVal{b}{B(n,s)}. We need to show 
    $P(n,s)$. Case on $n$: 
    \begin{itemize}
      \item $n = \zero,\suc{\zero}$:
        Then $B(n,s) \mapsto^* \bot \lift{\times}  (s(n-1) =_{\nat} \ret{\zero})$, so 
        \isComp{\bind{\fst{b}}{v}{absurd\,v}}{P(n,s)}.
      \item $n = \suc{\suc{\zero}}$: 
        Then $B(n,s) \mapsto^* \top \lift{\times}  (s(n-1) =_{\nat} \ret{\zero})$.
        We need to show 
        $P(n,s) \mapsto^* gcdProp(x,y)$. By $\squash{R(n,s)}$, we know 
        $s(0) =_{\nat} \ret{x}$ and $s(1) =_{\nat} \ret{y}$. Hence we know 
        $\ret{y} =_{\nat} \ret{\zero}$. Since $x = 1 \cdot x + 0 \cdot 0$,  
        \isComp{\ret{(x,\suc{\zero},\zero)}}{gcdProp(x,\zero)}.
      \item $n \ge_b 3$: 
        Then $B(n,s) \mapsto^* \top \lift{\times}  (s(n-1) =_{\nat} \ret{\zero})$.
        Let \eval{s(n-2)}{s_2}. We need to show 
        $P(n,s) \mapsto^* gcdProp(s_2,\zero)$.
        Since $s_2 = 1 \cdot s_2 + 0 \cdot 0$,  
        \isComp{\ret{(s_2,\suc{\zero},\zero)}}{gcdProp(s_2,\zero)}.
    \end{itemize}
    Hence the realizer for base is:
    \begin{align*}
      base &\triangleq \lam{n,s,r,b}{}{\\
        \ifthen{&n \le_b 1}{\bind{\fst{b}}{v}{absurd\,v}\\}{
          \ifthen{&eq\,n\,2}{\ret{(x,\suc{\zero},\zero)}\\}{
            &\bind{s(n-2)}{s_2}{(s_2,\suc{\zero},\zero)}
          }
        }}
    \end{align*}
  \item \isVal{r}{R(0,\empseq)}:
    Since $R(0,\empseq) \mapsto^* \top$, $r \triangleq \triv$ suffices.
\end{enumerate}

Finally, we can define gcd as
\[
  gcd(x,y) \triangleq \dbr{dec}{base}{ind}{0}{\empseq}{r}
\]

