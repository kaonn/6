
\section{GCD}

Notation: let $M =_{\nat} N$ when \eqtyc{\lift{\nat}}{M}{N}.
We can use $\mathtt{dbr}$ to define gcd:
\[
  gcdProp(x,y) \triangleq \sigmatyc{\isOf{d,l,k}{\lift{\nat}}}{\ret{d} =_{\nat} kx + ly}
\]

Define the spread $R$, bar $B$, and $P$: 

\begin{align*}
  R(n,s) \triangleq &
  \ifthen{eq_b\;n\;0}{\ret{\top}\\}{
    &\ifthen{eq_b\;n\;1}{s(0) =_{\nat} \ret{max(x,y)}\\}{
      &\ifthen{eq_b\;n\;2}{
        s(0) =_{\nat} \ret{max(x,y)} \lift{\times} s(1) =_{\nat} \ret{min(x,y)} \\
      }{
        &s(0) =_{\nat} \ret{max(x,y)} \lift{\times} s(1) =_{\nat} \ret{min(x,y)}\\
        &\lift{\times} \pityc{\isOf{j}{\lift{\nat}}}{
          {1 \le j \le n-2 \to mod\, s(n-3)\, s(n-2) =_{\nat} s(n-1)}
        }
      }
    }
  }\\
  B(n,s) \triangleq & n \ge 1 \lift{\times} (s(n-1) =_{\nat} \ret{\zero}) \\
  P(n,s) \triangleq &\bind{n \le_b 2\\}{b}{
    &\ifthen{b}{gcdProp(max(x,y),min(x,y))}
  {\bind{s(n-1)}{s_1}{\bind{s(n-2)}{s_2}{gcdProp(s_2,s_1)}}}}
\end{align*}


The predicates $R,B,P$ are well-formed by the various formation rules. 
Additionally, we need to show the following: 
\begin{enumerate}
  \item \isVal{bar}{\pityc{\isOf{s}{\lift{\baire}}}{\intty{\isOf{m}{\lift{\nat}}}{\squash{R(m,s)}} 
      \tolr \squash{\lift{\sigmaty{\isOf{n}{\lift{\nat}}}{B(n,s)}}}}}:\\
    Strengthen the property by considering any value of the sequence: 
    \begin{align*}
      &\pityc{\isOf{i,j}{\lift{\nat}}}{
        \pityc{\isOf{s}{\lift{\baire}}}{\intty{\isOf{m}{\lift{\nat}}}{\squash{R(m,s)}}
      \tolr \bind{s(j)}{v}{v \le i}
      \tolr\\ &\squash{\lift{\sigmaty{\isOf{n}{\lift{\nat}}}{
        B(n,s) \lift{\times} G(n,j,s)
    }}}}}
    \end{align*}
    where 
    \[
      G(n,j,s) \triangleq 
      n > j \lift{\times}
      \ifthen{s(j) >_b s(j+1)}{
        s(j) \ge F_{n-j} \lift{\times} s(j+1) \ge F_{n-j-1}
      }{
        s(j+1) \ge F_{n-j} \lift{\times} s(j) \ge F_{n-j-1}
      }
    \]
    
And $F_n$ is defined as the Fibonacci sequence prepended with a 0, i.e.
$F = 0,0,1,1,2,3,5...$.
    Let \isVal{j}{\lift{\nat}},
    \isVal{s}{\lift{\baire}}, \isVal{a}{\inttyc{\isOf{m}{\lift{\nat}}}{\squash{R(m,s)}}}, and 
    \isVal{p}{\bind{s(j)}{v}{v \le i}}.
    Proceed by induction:
    \begin{itemize}
      \item $i = \zero$: 
        $p$ implies that $s(j) = 0$. We need to show $n$ s.t. $B(n,s)$ and 
        $G(n,j,s)$. Taking $n = j+1$ suffices since $F_0 = F_1 = 0$.
      \item $i = \suc{i'}$ for some \isVal{i'}{\lift{\nat}}:
    By induction we have a term $f$ inhabiting
        \begin{align*}
  \compc{
    &\pityc{\isOf{j}{\lift{\nat}}}{\\
          &\pityc{\isOf{s}{\lift{\baire}}}{\intty{\isOf{m}{\lift{\nat}}}{\squash{R(m,s)}} 
        \tolr\\ 
          &\bind{s(j)}{v}{v \le i'}
          \tolr \squash{\lift{\sigmaty{\isOf{n}{\lift{\nat}}}{B(n,s) \lift{\times} G(n,s)
          }}}}}}
        \end{align*}
    By the spread law, we know:
        \begin{gather*}
          mod\, s(j) \, s(\suc{j}) =_{\nat} s(\suc{\suc{j}})\\
        \end{gather*}
        Case $s(j+1) < s(j)$: if $s(j+1) =_{\nat} \ret{\zero}$, then we can take
        \isComp{\ret{(\triv, \triv)}}{B(\suc{\suc{j}},s)}. Furthermore, 
        we need to show $G(j+2,j,s) \mapsto^*
        j + 2 > j \lift{\times} s(j) \ge F_2 \lift{\times} s(j+1) \ge F_1$. This holds since 
        \[ s(j) > 0 \implies s(j) \ge 1 = F_2 \text{ and } s(j+1) = 0 = F_1\]
        Else, we know $s(j+2) < s(j+1) < s(j)$ by Lemma~\ref{mod:prop}, 
        and we can apply the 
        induction hypothesis: \isComp{f\,(j+2)\,s\,a\,\triv}
        {\squash{\lift{\sigmaty{\isOf{n}{\lift{\nat}}}{B(n,s) \lift{\times} G(n,j+2,s)}}}}.
        If $s(j+2) > s(j+3)$, we know $G(n,j+2,s) \mapsto^* n > j+2 \lift{\times}
        s(j+2) \ge F_{n-j-2} \lift{\times} s(j+3) \ge F_{n-j-3}$. 
        We can take $n$ directly, and then it suffices to show
        $n > j$ and $s(j) \ge F_{n-j} \lift{\times} s(j+1) \ge F_{n-j-1}$.
        Note that 
        \begin{align*}
          s(j+1) - s(j+2) &\ge mod\,(s(j+1))\,(s(j+2)) \\
          & = s(j+3)\\
          &\ge F_{n-j-3}\\
          &\implies s(j+1) \ge s(j+2) + F_{n-j-3}\\
          &\implies s(j+1) \ge F_{n-j-2} + F_{n-j-3}\\
          &\implies s(j+1) \ge F_{n-j-1}
        \end{align*}
        Similary, 
        \begin{align*}
          s(j) - s(j+1) &\ge s(j+2) \\
                        &\ge F_{n-j-2}\\
                        &\implies s(j) \ge s(j+1) + F_{n-j-2}\\
                        &\implies s(j) \ge F_{n-j-1} + F_{n-j-2}\\
                        &\implies s(j) \ge F_{n-j}
        \end{align*}
        So $G(n,j,s)$ holds.

        If $s(j+2) \le s(j+3)$, by Lemma~\ref{lemma:mod}, it must be the case that 
        $s(j+2) = 0$. The condition $s(j+2) \ge F_{n-j-2}$ implies that 
        $F_{n-j-2} = 0$ which means $n = j+3$. Hence it suffices to show 
        $s(j) \ge F_3$ and $s(j+1) \ge F_2$. The former holds since 
        \[s(j) > s(j+1) > s(j+2) \implies s(j) \ge 2 \ge 1 = F_3\]. The latter since
        \[s(j+1) > s(j+2) \implies s(j+1) \ge 1 = F_2\].

        Now consider the case $s(j) = s(j+1)$. By the spread law, this means $s(j+2) = 0$.
        If $s(j) = s(j+1) = 0$, we can take $n = j+1$. Clearly, $j+1 > j$, so we just 
        need to show $s(j+1) \ge F_{1} \lift{\times} s(j) \ge F_{0}$, which holds since
        $F_1 = F_0 = 0$. Else, we know $s(j) = s(j+1) \ge 1$, 
        and we can take $n = j+3$, and need to show $s(j+1) \ge F_{3} \lift{\times} s(j) \ge F_{2}$. Again, this holds since $F_3 = F_2 = 1$.

        Lastly, consider $s(j) < s(j+1)$. By the spread law, this means $s(j+2) = s(j)$. 
        If $s(j) = 0$, we can take $n = j+1$ and show that $s(j+1) \ge F_1$ and 
        $s(j) \ge F_0$ since $F_1 = F_0 = 0$. Else, we have $j = 0$, since otherwise 
        $mod\,(s(j-1))\, (s(j)) = s(j+1)$, which is impossible. But then this is
        a contradiction since the spread law requires $s(0) \ge s(1)$.
    \end{itemize}
  \item \isVal{ind}{\pityc{\isOf{n}{\lift{\nat}}}{\pityc{\isOf{s}{\lift{\nseq{n}}}} {\\
      \qquad\squash{R(n,s)} \tolr (\pityc{\isOf{m}{\lift{\nat}}}{R(\suc{n}, s \oplus_n m) 
        \tolr P(\suc{n}, s \oplus_n m)}) \tolr P(n,s)}}}\\
  Given \isVal{n}{\lift{\nat}}, \isVal{s}{\lift{\nseq{n}}}, 
    \isVal{r}{\squash{R(n,s)}}, 
    \isVal{i}{\pityc{\isOf{m}{\lift{\nat}}}{R(\suc{n}, s \oplus_n m)
        \tolr P(\suc{n}, s \oplus_n m)}},
    need to show $P(n,s)$. Proceed with induction on $n$.
            Let $a = max(x,y)$ and $b = min(x,y)$.
    \begin{itemize}
      \item $n = \zero$:
        Suffices to show $P(0,s) \mapsto^* gcdProp(a,b)$.
        Note that \isComp{i\, a\, \triv}{P(1,[a])}, since 
        $R(1,[a]) \mapsto^* s(0) =_{\nat} \ret{a} \mapsto^* 
        \ret{a} =_{\nat} \ret{a}$. Thus 
        \isComp{i\, a\, \triv}{gcdProp(a,b)} since 
        $P(0,s) \mapsto^* gcdProp(a,b)$.
      \item $n = \suc{n'}$ for \isVal{n'}{\lift{\nat}}:
        Suffices to show a term inhabiting $P(\suc{n'}, s)$.
        Now case on $n'$: 
          \item $n' = \zero$: then again it suffices to show 
            $P(\suc{n'}, s) \mapsto^* gcdProp(a,b)$.
            Since \isVal{r}{\squash{R(\suc{\zero},s)}}, we know
            $s(0) =_{\nat} \ret{a}$.
            Note that \isComp{i\, b\, (\triv,\triv)}
            {P(\suc{\suc{\zero}},[a,b])}, since 
            \begin{gather*}
              R(\suc{\suc{\zero}}, [a,b]) \\
              \mapsto^* \ret{a} =_{\nat} \ret{a} \lift{\times}
              \ret{b} =_{\nat} \ret{b}
            \end{gather*}
            Done since 
            $P(\suc{\suc{\zero}},[a,b]) \mapsto^* gcdProp(a,b)$.
          \item $n' = \suc{\zero}$: suffices to show
            $P(\suc{\suc{n'}}, s) \mapsto^* gcdProp(x,y)$.
            Since \isVal{r}{\squash{R(\suc{\suc{\zero}},s)}}, we know
            $s(0) =_{\nat} \ret{a}$ and $s(1) =_{\nat} \ret{b}$.
            Let \eval{mod\,a\,b}{v}.
            Note that \isComp{\bind{mod\,a\,b}{d}{i\, d\, (\triv,\triv,\triv)}}{
              \bind{mod\,a\,b}{d}{P(3,[a,b,d])}}, since 
            \[R(3,s\oplus_3 v) \mapsto^* \ret{a} =_{\nat} \ret{a} \lift{\times} 
            \ret{b} =_{\nat} \ret{b} \lift{\times} \ret{v} =_{\nat} \ret{v}\].
            Furthermore, note that 
            \begin{align*} 
              &\bind{mod\,a\,b}{d}{P(3,[a,b,d])} \\
              \mapsto &gcdProp(b,v)
            \end{align*}
            Hence \isComp{\bind{mod\,a\,b}{d}{\bind{i\, d\, (\triv,\triv,\triv)}{r}{
              \fst{\snd{r}}}}}{gcdProp(b,v)}.
            So we know the type $gcdProp(b,v) = 
            \sigmatyc{\isOf{d,k,l}{\lift{\z}}}{d =_{\z} k b + l v}$ is inhabited.
            To obtain $gcdProp(a,b)$, we case on $x \ge_b y$. If this is the case,
            then: 
            \begin{align*}
              d &=_{\z} k b + l v\\
                &=_{\z} k (min(x,y)) + l v\\
                &=_{\z} k y + l v\\
                &= k y + l (mod\,x\, y)\\
                & = k y + l (x - qy) \tag{ where  $x = qy + r$}\\
                & = k y + l x - l qy \\
                & = k y - l qy + l x\\
                & = (k - lq)y + l x\\
                & = l x + (k - lq)y
            \end{align*}
            Similarly for $x <_b y$, we know $d = l y + (k - lq)$ where
            $y = qx + r$.
            So given \isVal{g}{gcdProp(y,v)}, we know 
            \begin{align*}
              \isComp{\bind{\fst{g}\\}{&d}{
                \bind{\fst{\snd{g}}\\}{&k}{
                  \bind{\fst{\snd{\snd{g}}}\\}{&l}{
                    &(d,l,k-lq,\triv)}
            }}}{gcdProp(a,b)}
            \end{align*}
            Hence 
            \begin{align*}
              \isComp{
                \bind{divmod\,a\,b\\}{&q,r}{
                \bind{i\, r\, (\triv,\triv,\triv)\\}{&g}{
              \bind{\fst{g}\\}{&d}{
                \bind{\fst{\snd{g}}\\}{&k}{
                  \bind{\fst{\snd{\snd{g}}}\\}{&l}{
                    &(d,l,k-lq,\triv)}}
              }}}}{gcdProp(a,b)}
            \end{align*}
          \item $n' > \suc{\zero}$:
            Let \eval{s(n-1)}{s_1} and \eval{s(n-2)}{s_2}.
            Now we to need show 
            \begin{align*}
              P(\suc{n'},s) \mapsto^* gcdProp(s_2,s_1)
            \end{align*}
            is inhabited. Then we have 
            \isComp{\bind{mod\,s_2\,s_1}{d}{i\, d\, (\triv,\triv,\triv)}}{
              P(\suc{\suc{n'}}, s \oplus_{\suc{n'}} d)}. 
            Note that $P(\suc{\suc{n'}}, s \oplus_{\suc{n'}} d) \mapsto^* 
            gcdProp(s_1,v)$, where \eval{mod\,s_2\,s_1}{v}.
            Similar to above, we derive the following:
            \begin{align*}
              \isComp{
                \bind{divmod\,s_2\,s_1\\}{&q,r}{
                \bind{i\, r\, (\triv,\triv,\triv)\\}{&g}{
              \bind{\fst{g}\\}{&d}{
                \bind{\fst{\snd{g}}\\}{&k}{
                  \bind{\fst{\snd{\snd{g}}}\\}{&l}{
                    &(d,l,k-lq,\triv)}}
              }}}}{gcdProp(s_2,s_1)}
            \end{align*}
        \end{itemize}
        Thus realizer can be defined as 
        \begin{align*}
          ind \triangleq &\lam{n,s,r,i}{}{\\
            \ifthen{&eq\,n\,0}{i\,x\,\triv\\}{
              \ifthen{&eq\,n\,1}{i\,y\,(\triv,\triv)\\}{
                \ifthen{&eq\,n\,2\\}{
                  \bind{&divmod\,a\,b}{q,r\\}{
                    \bind{&i\, r\, (\triv,\triv,\triv)}{g\\}{
                      \bind{&\fst{g}}{d\\}{
                        \bind{&\fst{\snd{g}}}{k\\}{
                          \bind{&\fst{\snd{\snd{g}}}}{l\\}{
                    &(d,l,k-lq,\triv)}}\\
              }}}
                    }{
                      \bind{&divmod\,s_2\,s_1}{q,r\\}{
                        \bind{&i\, r\, (\triv,\triv,\triv)}{g\\}{
                          \bind{&\fst{g}}{d\\}{
                            \bind{&\fst{\snd{g}}}{k\\}{
                              \bind{&fst{\snd{\snd{g}}}}{l\\}{
                    &(d,l,k-lq,\triv)}}
              }}}
          }
              }
            }
          }
        \end{align*}
  \item \isVal{dec}{\pityc{\isOf{n}{\lift{\nat}}}{\pityc{\isOf{s}{\lift{\nseq{n}}}}{\lift{B(n,s) + \lift{\neg} B(n,s)}}}}:
  Let \isVal{n}{\lift{\nat}} and \isVal{s}{\lift{\nseq{n}}}.
  Case on $n$:
  \begin{itemize}
    \item $n = \zero$:
      Prove the right disjunct:
      $B(n,s) \mapsto^* \bot \lift{\times} \_$, so 
      \isVal{\lam{b}{}{\bind{\fst{b}}{v}{absurd\,v}}}{\lift{\neg} B(n,s)}.
    \item $n \ge_b 1$:
      Let \eval{s(n-1)}{v}. If $v = \zero$, we can prove the left disjunct:
      \isVal{(\triv, \triv)}{B(n,s)}. Otherwise, we can prove 
      \isVal{\lam{b}{}{\triv}}{\lift{\neg}B(n,s)}.
  \end{itemize}
  The realizer for dec is 
  \begin{align*}
    dec &\triangleq \lam{n,s}{}{
      \ifthen{n \le_b 1}{\lam{b}{}{\bind{\fst{b}}{v}{absurd\,v}}}
      {
        \lam{b}{}{\triv}
      }
    }
  \end{align*}

  \item \isVal{base}{\pityc{\isOf{n}{\lift{\nat}}}{\pityc{\isOf{s}{\lift{\nseq{n}}}}{\squash{R(n,s)} \tolr 
      B(n,s) \tolr P(n,s)}}}:
    Let \isVal{n}{\lift{\nat}}, \isVal{s}{\lift{\nseq{n}}}, 
    \isVal{r}{\squash{R(n,s)}}, and \isVal{b}{B(n,s)}. We need to show 
    $P(n,s)$. Case on $n$: 
    \begin{itemize}
      \item $n = \zero$:
        Then $B(n,s) \mapsto^* \bot \lift{\times}  (s(n-1) =_{\nat} \ret{\zero})$, so 
        \isComp{\bind{\fst{b}}{v}{absurd\,v}}{P(n,s)}.
      \item $n = \suc{\zero}, \suc{\suc{\zero}}$: 
        Then $B(n,s) \mapsto^* \top \lift{\times}  (s(n-1) =_{\nat} \ret{\zero})$.
        We need to show 
        $P(n,s) \mapsto^* gcdProp(x,y)$. By $\squash{R(n,s)}$, we know 
        $s(0) =_{\nat} \ret{x}$. Hence we know 
        $\ret{x} =_{\nat} \ret{\zero}$. 
        If $y \le x$, then $y = 0$, and 
        \isComp{\ret{(0,0,0)}}{gcdProp(0,0)}.
        Else, we need to show $gcdProp(y,0)$.
        Since $y = 1 \cdot y + 0 \cdot 0$,  
        \isComp{\ret{(y,1,0)}}{gcdProp(y,0)}.
      \item $n \ge_b 3$: 
        Then $B(n,s) \mapsto^* \top \lift{\times}  (s(n-1) =_{\nat} \ret{\zero})$.
        Let \eval{s(n-2)}{s_2}. We need to show 
        $P(n,s) \mapsto^* gcdProp(s_2,\zero)$.
        Since $s_2 = 1 \cdot s_2 + 0 \cdot 0$,  
        \isComp{\ret{(s_2,\suc{\zero},\zero)}}{gcdProp(s_2,\zero)}.
    \end{itemize}
    Hence the realizer for base is:
    \begin{align*}
      base &\triangleq \lam{n,s,r,b}{}{\\
        \ifthen{&n \le_b 1}{\bind{\fst{b}}{v}{absurd\,v}\\}{
          \ifthen{&eq\,n\,2}{\ret{(x,\suc{\zero},\zero)}\\}{
            &\bind{s(n-2)}{s_2}{(s_2,\suc{\zero},\zero)}
          }
        }}
    \end{align*}
  \item \isVal{r}{R(0,\empseq)}:
    Since $R(0,\empseq) \mapsto^* \top$, $r \triangleq \triv$ suffices.
\end{enumerate}

Finally, we can define gcd as
\[
  gcd(x,y) \triangleq \dbr{dec}{base}{ind}{0}{\empseq}{r}
\]

Again, note that the most we can state in the type theory is the fact that $gcd$ has the 
correct functional behavior. In fact, the number of $\mathtt{dbr}$ calls made is 
bounded by the witness $n$ to the bar condition. However, due to the lazy nature of
bar recursion, this cannot be expressed in the theory. A wanting result would be a
lemma that states $\sigmatyc{\isOf{n}{\nat}}{B(n,s)}$ implies number of recursive 
calls is bounded by $n$. Together with the constraint on $n$ we have shown in the 
bar condition, this would suffices to bound the number of recursive calls to the 
inverse Fibonacci of $\max{(x,y)}$. However, to achieve this, we would need to know 
that the recursive call is only made once each time, which is 
a property of the \emph{presentation} of the code, not of its \emph{behavior}.
Perhaps there is a behavioral condition which implies the former, but we have not 
found such a formulation yet.

