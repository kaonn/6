
\section{GCD}

Notation: let $M =_{\nat} N$ when \eqtyc{\lift{\nat}}{M}{N}.
We can use $\mathtt{dbr}$ to define gcd:
\[
  gcdProp(x,y) \triangleq \sigmatyc{\isOf{d}{\lift{\nat}}}{
    \sigmatyc{\isOf{l,k}{\lift{\z}}}{\ret{\pos{d}} =_{\z} kx + ly}}
\]
(Clearly, we want the minimum such $d$. However, for the sake of brevity, let us work with this 
``truncated'' GCD, which demonstrates the main ideas).

Define the spread $R$, bar $B$, and $P$: 
\begin{align*}
  R(n,s) \triangleq &
  \ifthen{eq_b\;n\;0}{\ret{\top}\\}{
    &\ifthen{eq_b\;n\;1}{s(0) =_{\nat} \ret{max(x,y)}\\}{
      &\ifthen{eq_b\;n\;2}{
        s(0) =_{\nat} \ret{max(x,y)} \lift{\times} s(1) =_{\nat} \ret{min(x,y)} \\
      }{
        &s(0) =_{\nat} \ret{max(x,y)} \lift{\times} s(1) =_{\nat} \ret{min(x,y)}\\
        &\lift{\times} \pityc{\isOf{j}{\lift{\nat}}}{
          {1 \le j \le n-2 \to mod\, s(n-j-2)\, s(n-j-1) =_{\nat} s(n-j)}
        }
      }
    }
  }\\
  B(n,s) \triangleq & n \ge 1 \lift{\times} (s(n-1) =_{\nat} \ret{\zero}) \\
  P(n,s) \triangleq &\bind{n \le_b 1\\}{b}{
    &\ifthen{b}{gcdProp(max(x,y),min(x,y))}
  {\bind{s(n-1)}{s_1}{\bind{s(n-2)}{s_2}{gcdProp(s_2,s_1)}}}}
\end{align*}


The predicates $R,B,P$ are well-formed by the various formation rules. 
Additionally, we need to show the following: 
\begin{enumerate}
  \item \isVal{bar}{\pityc{\isOf{s}{\lift{\baire}}}{\intty{\isOf{m}{\lift{\nat}}}{\squash{R(m,s)}} 
      \tolr \squash{\lift{\sigmaty{\isOf{n}{\lift{\nat}}}{B(n,s)}}}}}:\\
    Strengthen the property by considering any value of the sequence: 
    \begin{align*}
      &\pityc{\isOf{i,j}{\lift{\nat}}}{
        \pityc{\isOf{s}{\lift{\baire}}}{\intty{\isOf{m}{\lift{\nat}}}{\squash{R(m,s)}}
      \tolr \bind{s(j)}{v}{v \le i}
      \tolr\\ &\squash{\lift{\sigmaty{\isOf{n}{\lift{\nat}}}{
        B(n,s) \lift{\times} G(n,j,s)
    }}}}}
    \end{align*}
    where 
    \[
      G(n,j,s) \triangleq 
      n > j \lift{\times}
      \ifthen{s(j) >_b s(j+1)}{
        s(j) \ge F_{n-j} \lift{\times} s(j+1) \ge F_{n-j-1}
      }{
        s(j+1) \ge F_{n-j} \lift{\times} s(j) \ge F_{n-j-1}
      }
    \]
    
And $F_n$ is defined as the Fibonacci sequence prepended with a 0, i.e.
$F = 0,0,1,1,2,3,5...$.
    Let \isVal{j}{\lift{\nat}},
    \isVal{s}{\lift{\baire}}, \isVal{a}{\inttyc{\isOf{m}{\lift{\nat}}}{\squash{R(m,s)}}}, and 
    \isVal{p}{\bind{s(j)}{v}{v \le i}}.
    Proceed by induction:
    \begin{itemize}
      \item $i = \zero$: 
        $p$ implies that $s(j) = 0$. We need to show $n$ s.t. $B(n,s)$ and 
        $G(n,j,s)$. Taking $n = j+1$ suffices since $F_0 = F_1 = 0$.
      \item $i = \suc{i'}$ for some \isVal{i'}{\lift{\nat}}:
    By induction we have a term $f$ inhabiting
        \begin{align*}
  \compc{
    &\pityc{\isOf{j}{\lift{\nat}}}{\\
          &\pityc{\isOf{s}{\lift{\baire}}}{\intty{\isOf{m}{\lift{\nat}}}{\squash{R(m,s)}} 
        \tolr\\ 
          &\bind{s(j)}{v}{v \le i'}
          \tolr \squash{\lift{\sigmaty{\isOf{n}{\lift{\nat}}}{B(n,s) \lift{\times} G(n,s)
          }}}}}}
        \end{align*}
    By the spread law, we know:
        \begin{gather*}
          mod\, s(j) \, s(\suc{j}) =_{\nat} s(\suc{\suc{j}})\\
        \end{gather*}
        Case $s(j+1) < s(j)$: if $s(j+1) =_{\nat} \ret{\zero}$, then we can take
        \isComp{\ret{(\triv, \triv)}}{B(\suc{\suc{j}},s)}. Furthermore, 
        we need to show $G(j+2,j,s) \mapsto^*
        j + 2 > j \lift{\times} s(j) \ge F_2 \lift{\times} s(j+1) \ge F_1$. This holds since 
        \[ s(j) > 0 \implies s(j) \ge 1 = F_2 \text{ and } s(j+1) = 0 = F_1\]
        Else, we know $s(j+2) < s(j+1) < s(j)$ by Lemma~\ref{mod:prop}, 
        and we can apply the 
        induction hypothesis: \isComp{f\,(j+2)\,s\,a\,\triv}
        {\squash{\lift{\sigmaty{\isOf{n}{\lift{\nat}}}{B(n,s) \lift{\times} G(n,j+2,s)}}}}.
        If $s(j+2) > s(j+3)$, we know $G(n,j+2,s) \mapsto^* n > j+2 \lift{\times}
        s(j+2) \ge F_{n-j-2} \lift{\times} s(j+3) \ge F_{n-j-3}$. 
        We can take $n$ directly, and then it suffices to show
        $n > j$ and $s(j) \ge F_{n-j} \lift{\times} s(j+1) \ge F_{n-j-1}$.
        Note that 
        \begin{align*}
          s(j+1) - s(j+2) &\ge mod\,(s(j+1))\,(s(j+2)) \\
          & = s(j+3)\\
          &\ge F_{n-j-3}\\
          &\implies s(j+1) \ge s(j+2) + F_{n-j-3}\\
          &\implies s(j+1) \ge F_{n-j-2} + F_{n-j-3}\\
          &\implies s(j+1) \ge F_{n-j-1}
        \end{align*}
        Similary, 
        \begin{align*}
          s(j) - s(j+1) &\ge s(j+2) \\
                        &\ge F_{n-j-2}\\
                        &\implies s(j) \ge s(j+1) + F_{n-j-2}\\
                        &\implies s(j) \ge F_{n-j-1} + F_{n-j-2}\\
                        &\implies s(j) \ge F_{n-j}
        \end{align*}
        So $G(n,j,s)$ holds.

        If $s(j+2) \le s(j+3)$, by Lemma~\ref{lemma:mod}, it must be the case that 
        $s(j+2) = 0$. The condition $s(j+2) \ge F_{n-j-2}$ implies that 
        $F_{n-j-2} = 0$ which means $n = j+3$. Hence it suffices to show 
        $s(j) \ge F_3$ and $s(j+1) \ge F_2$. The former holds since 
        \[s(j) > s(j+1) > s(j+2) \implies s(j) \ge 2 \ge 1 = F_3\]. The latter since
        \[s(j+1) > s(j+2) \implies s(j+1) \ge 1 = F_2\].

        Now consider the case $s(j) = s(j+1)$. By the spread law, this means $s(j+2) = 0$.
        If $s(j) = s(j+1) = 0$, we can take $n = j+1$. Clearly, $j+1 > j$, so we just 
        need to show $s(j+1) \ge F_{1} \lift{\times} s(j) \ge F_{0}$, which holds since
        $F_1 = F_0 = 0$. Else, we know $s(j) = s(j+1) \ge 1$, 
        and we can take $n = j+3$, and need to show $s(j+1) \ge F_{3} \lift{\times} s(j) \ge F_{2}$. Again, this holds since $F_3 = F_2 = 1$.

        Lastly, consider $s(j) < s(j+1)$. By the spread law, this means $s(j+2) = s(j)$. 
        If $s(j) = 0$, we can take $n = j+1$ and show that $s(j+1) \ge F_1$ and 
        $s(j) \ge F_0$ since $F_1 = F_0 = 0$. Else, we have $j = 0$, since otherwise 
        $mod\,(s(j-1))\, (s(j)) = s(j+1)$, which is impossible. But then this is
        a contradiction since the spread law requires $s(0) \ge s(1)$.
    \end{itemize}
  \item \isVal{ind}{\pityc{\isOf{n}{\lift{\nat}}}{\pityc{\isOf{s}{\lift{\nseq{n}}}} {\\
      \qquad\squash{R(n,s)} \tolr (\pityc{\isOf{m}{\lift{\nat}}}{R(\suc{n}, s \oplus_n m) 
        \tolr P(\suc{n}, s \oplus_n m)}) \tolr P(n,s)}}}:\\

  Given \isVal{n}{\lift{\nat}}, \isVal{s}{\lift{\nseq{n}}}, 
    \isVal{r}{\squash{R(n,s)}}, 
    \isVal{i}{\pityc{\isOf{m}{\lift{\nat}}}{R(\suc{n}, s \oplus_n m)
        \tolr P(\suc{n}, s \oplus_n m)}},
    need to show $P(n,s)$. Proceed with induction on $n$.
            Let $a = max(x,y)$ and $b = min(x,y)$.
    \begin{itemize}
      \item $n = \zero$:
        Suffices to show $P(0,s) \mapsto^* gcdProp(a,b)$.
        Note that \isComp{i\, a\, \triv}{P(1,[a])}, since 
        $R(1,[a]) \mapsto^* s(0) =_{\nat} \ret{a} \mapsto^* 
        \ret{a} =_{\nat} \ret{a}$, and so \isVal{\triv}{R(1,[a])}. Thus 
        \isComp{i\, a\, \triv}{gcdProp(a,b)}.
      \item $n = \suc{\zero}$: then again it suffices to show 
            $P(\suc{n'}, s) \mapsto^* gcdProp(a,b)$.
            Since \isVal{r}{\squash{R(\suc{\zero},s)}}, we know
            $s(0) =_{\nat} \ret{a}$.
            Note that \isComp{i\, b\, (\triv,\triv)}
            {P(\suc{\suc{\zero}},[a,b])}, since 
            \begin{gather*}
              R(\suc{\suc{\zero}}, [a,b]) \\
              \mapsto^* \ret{a} =_{\nat} \ret{a} \lift{\times}
              \ret{b} =_{\nat} \ret{b}
            \end{gather*}
            Done since 
            $P(\suc{\suc{\zero}},[a,b]) \mapsto^* gcdProp(a,b)$.
      \item $n > \suc{\zero}$:
            Let \eval{s(n-1)}{s_1} and \eval{s(n-2)}{s_2}.
            Now we to need show 
            \begin{align*}
              P(\suc{n'},s) \mapsto^* gcdProp(s_2,s_1)
            \end{align*}
            is inhabited.
            Since \isVal{r}{\squash{R(n,s)}}, we know
            $s(0) =_{\nat} \ret{a}$, $s(1) =_{\nat} \ret{b}$, 
            and \pityc{\isOf{j}{\lift{\nat}}}{
          {1 \le j \le n-2 \to mod\, s(n-j-2)\, s(n-j-1) =_{\nat} s(n-j)}
        }.
            Let \eval{mod\,s_2\,s_1}{v}.
            To apply $i\,v$, we need to show $R(\suc{n}, s \oplus_{n} v)$. 
            The first two conditions carry from $R(n,s)$. For the last, we 
            need to show  
            \[\pityc{\isOf{j}{\lift{\nat}}}{
              {1 \le j \le n-1 \to mod\, (s\oplus_n v)(n-j-1)\, (s\oplus_n v)(n-j) 
              =_{\nat} (s\oplus_n v)(n+1-j)} }\]
            For $2 \le j \le n-1$, this is equivalent to
            \[\pityc{\isOf{j'}{\lift{\nat}}}{
              {1 \le j' \le n-2 \to mod\, s(n-j'-2)\, s(n-j'-1) 
              =_{\nat} s(n-j')} }\]
            , which holds from above. For $j = 1$, we need to show 
            \[
              mod\, (s\oplus_n v)(n-2)\, (s\oplus_n v)(n-1) =_{\nat} (s\oplus_n v)(n)
            \]
            or that 
            \[
              mod\, s_2\, s_1 =_{\nat} v
            \]
            which is true as by construction.
            Thus we have 
            \[
            \isComp{\bind{mod\,s_2\,s_1}{d}{i\, d\, (\triv,\triv,\triv)}}{
              P(\suc{n}, s \oplus_{n} d)}\]
            Note that $P(\suc{n}, s \oplus_{n} d) \mapsto^* 
            gcdProp(s_1,v)$.
            So we know the type $gcdProp(s_1,v) =
            \sigmatyc{\isOf{d,k,l}{\lift{\z}}}{d =_{\z} k s_1  + l v}$ is inhabited.
            To obtain $gcdProp(s_2,s_1)$, rearrange the coefficients:
            \begin{align*}
              d &=_{\z} k s_1 + l v\\
                &= k s_1 + l (mod\,s_2\, s_1)\\
                & = k s_1 + l (s_2 - qs_1) 
                \tag{ where  $s_2 = q s_1 + v$ for some $q$ by Lemma~{}}\\
                & = k s_1 + l s_2 - l qs_1 \\
                & = k s_1 - l qs_1 + l s_2\\
                & = (k - lq)s_1 + l s_2\\
                & = l s_2 + (k - lq)s_1
            \end{align*}\todo{what's the lemma?}
            Hence
            \begin{align*}
              \isComp{
                \bind{&divmod\,s_2\,s_1\\}{&q,r}{
                \bind{i\, r\, (\triv,\triv,\triv)\\}{&g}{
              \bind{\fst{g}\\}{&d}{
                \bind{\fst{\snd{g}}\\}{&k}{
                  \bind{\fst{\snd{\snd{g}}}\\}{&l}{
                \ret{(d,l,k-lq,\triv)}\\}}
              }}}}{&gcdProp(s_2,s_1)}
            \end{align*}
        \end{itemize}
        Thus realizer can be defined as 
        \begin{align*}
          ind \triangleq &\lam{n,s,r,i}{}{\\
            \ifthen{&eq_b\,n\,0}{\bind{max(x,y)}{a}{i\,a\,\triv}\\}{
                \ifthen{&eq_b\,n\,1}{\bind{min(x,y)}{b}{i\,b\,(\triv,\triv)}\\}{
                    \bind{&s(n-1)}{s_1\\}{
                    \bind{&s(n-2)}{s_2\\}{
                    \bind{&divmod(s_2,s_1)}{q,r\\}{
                        \bind{&i\, r\, (\triv,\triv,\triv)}{g\\}{
                          \bind{&\fst{g}}{d\\}{
                            \bind{&\fst{\snd{g}}}{k\\}{
                              \bind{&\fst{\snd{\snd{g}}}}{l\\}{
                    &\ret{(d,l,k-lq,\triv)}}}
          }}}}}
              }
            }
          }
        \end{align*}
  \item \isVal{dec}{\pityc{\isOf{n}{\lift{\nat}}}{\pityc{\isOf{s}{\lift{\nseq{n}}}}{\lift{B(n,s) + \lift{\neg} B(n,s)}}}}:
  Let \isVal{n}{\lift{\nat}} and \isVal{s}{\lift{\nseq{n}}}.
  Case on $n$:
  \begin{itemize}
    \item $n = \zero$:
      Prove the right disjunct:
      $B(n,s) \mapsto^* \bot \lift{\times} \_$, so 
      \isVal{\lam{b}{}{\triv}}{\lift{\neg} B(n,s)}.
    \item $n \ge_b 1$:
      Let \eval{s(n-1)}{v}. If $v = \zero$, we can prove the left disjunct:
      \isVal{(\triv, \triv)}{B(n,s)}. Otherwise, we have a proof of
      $eq_b\,v\,\zero = \mathtt{ff}$, and proof the right side. Let
      \isVal{b}{B(n,s)}. We know $B(n,s) = n \ge 1 \lift{\times} s(n-1) =_{\nat} \zero$.
      This means $eq_b\,v\,\zero = \mathtt{tt}$, which gives us $\mathtt{tt} = \mathtt{ff}$.
      Since this is a contradiction, we are done.
      Hence we have \isVal{\lam{b}{}{\triv}}{\lift{\neg}B(n,s)}.
  \end{itemize}
  The realizer for dec is 
  \begin{align*}
    dec &\triangleq \lam{n,s}{}{
      \ifthen{eq_b\,n\,\zero}{\ret{\inr{\lam{b}{}{\triv}}}}
      {
        \ifthen{eq_b\,s(n-1)\,\zero}{
            \ret{\inl{(\triv,\triv)}}
            }{\ret{\inr{\lam{b}{}{\triv}}}
        }
      }
    }
  \end{align*}

  \item \isVal{base}{\pityc{\isOf{n}{\lift{\nat}}}{\pityc{\isOf{s}{\lift{\nseq{n}}}}{\squash{R(n,s)} \tolr 
      B(n,s) \tolr P(n,s)}}}:
    Let \isVal{n}{\lift{\nat}}, \isVal{s}{\lift{\nseq{n}}}, 
    \isVal{r}{\squash{R(n,s)}}, and \isVal{b}{B(n,s)}. We need to show 
    $P(n,s)$. Case on $n$: 
    \begin{itemize}
      \item $n = \zero$:
        Then $B(n,s) \mapsto^* \bot \lift{\times}  (s(n-1) =_{\nat} \ret{\zero})$,
        which is a contradiction so \isVal{\triv}{P(n,s)}.
      \item $n = \suc{\zero}$: 
        Then $B(n,s) \mapsto^* \top \lift{\times}  (s(n-1) =_{\nat} \ret{\zero})$.
        Let \eval{max(x,y)}{a} and \eval{min(x,y)}{b}.
        We need to show 
        $P(n,s) \mapsto^* gcdProp(a,b)$. By $\squash{R(n,s)}$, we know 
        $s(0) =_{\nat} \ret{a}$. Hence we know 
        $\ret{x} =_{\nat} \ret{\zero}$ and
        $\ret{y} =_{\nat} \ret{\zero}$. 
        We need to show $gcdProp(0,0)$.
        Since $0 = 0 \cdot 0 + 0 \cdot 0$,  
        \isComp{\ret{(0,0,0)}}{gcdProp(0,0)}.
      \item $n \ge_b 2$: 
        Then $B(n,s) \mapsto^* \top \lift{\times} (s(n-1) =_{\nat} \ret{\zero})$.
        Let \eval{s(n-2)}{s_2}. We need to show 
        $P(n,s) \mapsto^* gcdProp(s_2,\zero)$.
        Since $s_2 = 1 \cdot s_2 + 0 \cdot 0$,  
        \isComp{\ret{(s_2,\suc{\zero},\zero)}}{gcdProp(s_2,\zero)}.
    \end{itemize}
    Hence the realizer for base is:
    \begin{align*}
      base &\triangleq \lam{n,s,r,b}{}{\\
        \ifthen{&eq_b\,n\,\zero}{\ret{\triv}\\}{
          \ifthen{&eq\,n\,1}{\ret{(\zero,\zero,\zero)}\\}{
            &\bind{s(n-2)}{s_2}{\ret{(s_2,\suc{\zero},\zero)}}
          }
        }}
    \end{align*}
  \item \isVal{r}{R(0,\empseq)}:
    Since $R(0,\empseq) \mapsto^* \top$, $r \triangleq \triv$ suffices.
\end{enumerate}

Finally, we can define gcd as
\[
  gcd(x,y) \triangleq \dbr{dec}{base}{ind}{0}{\empseq}{r}
\]

Without further analysis, the most we can state in the type theory is the 
fact that $gcd$ has the 
correct functional behavior. In fact, the number of $\mathtt{dbr}$ calls made is 
bounded by the witness $n$ to the bar condition. However, due to the lazy nature of
bar recursion, this cannot be expressed in the theory. A wanting result would be a
lemma that states $\sigmatyc{\isOf{n}{\nat}}{B(n,s)}$ implies number of recursive 
calls is bounded by $n$. Together with the constraint on $n$ we have shown in the 
bar condition, this would suffices to bound the number of recursive calls to the 
inverse Fibonacci of $\max{(x,y)}$. However, to achieve this, we would need to know 
that the recursive call is only made once each time, which is 
a property of the \emph{presentation} of the code, not of its \emph{behavior}.
Perhaps there is a behavioral condition which implies the former, but we have not 
found such a formulation yet.

All is not lost, since we \emph{can} analyze the code further. To keep the analysis 
manageable, let us assume the existence of primitive arithmetic operators with unit
cost:

\begin{mathpar}
  \isVal{divmod}{\arrabtc{\lift{\nat}}{\arrabtc{\lift{\nat}}{\lift{\nat}}
  {\_.\constp{1}, \_.\_.\zp}}{\_.\zp, \_.\_.\zp}}

  \isVal{plus}{\arrabtc{\lift{\nat}}{\arrabtc{\lift{\nat}}{\lift{\nat}}
  {\_.\constp{1}, \_.\_.\zp}}{\_.\zp, \_.\_.\zp}}

  \isVal{minus}{\arrabtc{\lift{\nat}}{\arrabtc{\lift{\nat}}{\lift{\nat}}
  {\_.\constp{1}, \_.\_.\zp}}{\_.\zp, \_.\_.\zp}}

  \isVal{mult}{\arrabtc{\lift{\nat}}{\arrabtc{\lift{\nat}}{\lift{\nat}}
  {\_.\constp{1}, \_.\_.\zp}}{\_.\zp, \_.\_.\zp}}

  \isVal{eq_b}{\arrabtc{\lift{\nat}}{\arrabtc{\lift{\nat}}{\lift{\nat}}
  {\_.\constp{1}, \_.\_.\zp}}{\_.\zp, \_.\_.\zp}}

  \isVal{\le_b}{\arrabtc{\lift{\nat}}{\arrabtc{\lift{\nat}}{\lift{\nat}}
  {\_.\constp{1}, \_.\_.\zp}}{\_.\zp, \_.\_.\zp}}
\end{mathpar}

Furthermore, we need to specifiy the cost of the sequences. Let 
\[\nseqc{n}{a.P}{a.b.Q} \triangleq \arrabt{\lift{\nat}_n}{\lift{\nat}}{a.P,a.b.Q}\].
So far, the implementation of sequence extension was not relevant, since we were only concerned
with its functional specification. But now, we need to state exactly how the operator 
extends a sequence with a given cost specification. For this, we implement the operator as follows:
\begin{align*}
  \oplus_n \triangleq \lam{s,t,i}{}{
    \ifthen{eq_b\,i\,n}{\ret{t}}{s(i)}
  }
\end{align*}

The sequence extension operator itself has a constant cost for all \isVal{n}{\lift{\nat}}, since 
the body is immediately abstracted. Furthermore, it delivers an extended sequence such that
it is constant cost to retrieve the last element:
\[
  \isVal{\oplus_n}{\arrabtc{\nseqc{n}{a.P}{a.b.Q}}{\arrabtc{\lift{\nat}}{\nseqc{\suc{n}}{a.P'}{a.b.Q'}}
  {\_.\zp, \_.\_.\zp}}{\_.\zp, \_.\_.\zp}}
\]

where
\begin{gather*}
  P' \triangleq \ifthen{eq_b\,a\,n}{\ret{3}}{\seq{P}{p}{p+4}}\\
  Q' \triangleq \ifthen{eq_b\,a\,n}{\ret{0}}{Q}\\
\end{gather*}

Now we analyze the components of the induction:

\textbf{dec}:
\begin{align*}
dec &\triangleq \lam{n,s}{}{
  \ifthen{eq_b(n,\zero)}{\ret{\inr{\lam{b}{}{\triv}}}}
      {
        \ifthen{eq_b\,s(n-1)\,\zero}{
            \ret{\inl{(\triv,\triv)}}
            }{\ret{\inr{\lam{b}{}{\triv}}}
        }
      }
    }
\end{align*}

We already know that 
\isVal{dec}{\pityc{\isOf{n}{\lift{\nat}}}{\pityc{\isOf{s}{\lift{\nseq{n}}}}{\lift{B(n,s) + \lift{\neg} B(n,s)}}}}, so it remains to count the number of steps the branches take.

Let \isVal{n}{\lift{\nat}} and \isVal{s}{\lift{\nseqc{n}{a.P}{a.b.Q}}}.
\begin{enumerate}
  \item $n = \zero$: 
    \begin{align*}
      &\ifthen{eq_b(n,\zero)}{\ret{\inr{\lam{b}{}{\triv}}}}
      {
        \ifthen{eq_b(s(n-1),\zero)}{
          \ret{\inl{(\triv,\triv)} }
        }{\ret{\inr{\lam{b}{}{\triv}}}
        }
      }\\
      \mapsto^2& 
      \ifthen{\ret{\ttcst}}{\ret{\inr{\lam{b}{}{\triv}}}}
      {
        \ifthen{eq_b(s(n-1),\zero)}{
          \ret{\inl{(\triv,\triv)}}  
        }{\ret{\inr{\lam{b}{}{\triv}}}
        }
      }\\
      \mapsto&\lam{b}{}{\triv}
    \end{align*}
    Hence 3 steps in this case.
  \item $n \ne \zero$: 
    \begin{align*}
      &\ifthen{eq_b(n,\zero)}{\ret{\inr{\lam{b}{}{\triv}}}}
      {
        \ifthen{eq_b(s(n-1),\zero)}{
          \ret{\inl{(\triv,\triv)}}
        }{\ret{\inr{\lam{b}{}{\triv}}}
        }
      }\\
      \mapsto^2& 
      \ifthen{\ret{\ffcst}}{\ret{\inr{\lam{b}{}{\triv}}}}
      {
        \ifthen{eq_b(s(n-1),\zero)}{
          \ret{\inl{(\triv,\triv)}}  
        }{\ret{\inr{\lam{b}{}{\triv}}}}
        }
      \\
      \mapsto&
      \ifthen{eq_b(s(n-1),\zero)}{
          \ret{\inl{(\triv,\triv)}}  
        }{\ret{\lam{b}{}{\triv}
        }}
    \end{align*}
    So far 3 steps. Now case on $eq_b(s(n-1), \zero)$. Let \evalCost{s(n-1)}{c}{v}.
    \begin{itemize}
      \item $v = \zero$:
        \begin{align*}
          &\ifthen{eq_b(s(n-1),\zero)}{
            \ret{\inl{(\triv,\triv)}}
          }{\ret{\inr{\lam{b}{}{\triv}}}
        }\\
          \mapsto^2&
          \ifthen{eq_b(s (\ret{\toNum{n-1}}),\zero)}{
            \ret{\inl{(\triv,\triv)}}
          }{\ret{\inr{\lam{b}{}{\triv}}}
        }\\
          \mapsto&
          \ifthen{eq_b(s (\toNum{n-1}),\zero)}{
            \ret{\inl{(\triv,\triv)}}
          }{\ret{\inr{\lam{b}{}{\triv}}}
        }\\
        \mapsto^{1+c}&
          \ifthen{eq_b(\zero,\zero)}{
            \ret{\inl{(\triv,\triv)}}
          }{\ret{\inr{\lam{b}{}{\triv}}}
        }\\
          \mapsto^2& 
          \ifthen{\ret{\ttcst}}{
            \ret{\inl{(\triv,\triv)}}
          }{\ret{\inr{\lam{b}{}{\triv}}}
        }\\
          \mapsto& \ret{\inl{(\triv,\triv)}}
        \end{align*}
        Furthermore, we know \eval{[\toNum{n-1}/a]P}{p}, \eval{[\toNum{n-1}/a,\zero/b]Q}{q}, and
        $p \ge c + q$.
      \item  $v \ne \zero$:
        \begin{align*}
          &\ifthen{eq_b(s(n-1),\zero)}{
            \ret{\inl{(\triv,\triv)}}
          }{\ret{\inr{\lam{b}{}{\triv}}}
        }\\
          \mapsto^2&
          \ifthen{eq_b(s (\ret{\toNum{n-1}}),\zero)}{
            \ret{\inl{(\triv,\triv)}}
          }{\ret{\inr{\lam{b}{}{\triv}}}
        }\\
          \mapsto&
          \ifthen{eq_b(s (\toNum{n-1}),\zero)}{
            \ret{\inl{(\triv,\triv)}}
          }{\ret{\inr{\lam{b}{}{\triv}}}
        }\\
        \mapsto^{1+c}&
          \ifthen{eq_b(v,\zero)}{
            \ret{\inl{(\triv,\triv)}}
          }{\ret{\inr{\lam{b}{}{\triv}}}
        }\\
          \mapsto^2& 
          \ifthen{\ret{\ffcst}}{
            \ret{\inl{(\triv,\triv)}}
          }{\ret{\inr{\lam{b}{}{\triv}}}
        }\\
          \mapsto& \ret{\inr{\lam{b}{}{\triv}}}
        \end{align*}
        Again, we know \eval{[\toNum{n-1}/a]P}{p}, \eval{[\toNum{n-1}/a,v/b]Q}{q'}, and
        $p \ge c + q'$.
    \end{itemize}
    In any case we need $p+7$ steps.
\end{enumerate}

Hence $p+10$ steps will be enough in all cases:
\[
  \isVal{dec}{\arrabtc{\lift{\nat}}
  {n.\arrabtc{\lift{\nseqc{n}{a.P}{a.b.Q}}}{s.\lift{B(n,s) + \lift{\neg} B(n,s)}}
  {\_.\seq{[\toNum{n-1}/a]P}{p}{p + 10}, \_.\_.\zp}}{\_.\zp,\_.\_.\zp}}
\]

\textbf{base}:
 \begin{align*}
      base &\triangleq \lam{n,s,r,b}{}{\\
        \ifthen{&eq_b(n,\zero)}{\ret{\triv}\\}{
            \ifthen{&eq_b(n,1)}{\ret{(\zero,\zero,\zero)}\\}{
            &\bind{s(n-2)}{s_2}{\ret{(s_2,\suc{\zero},\zero)}}
          }
        }}
 \end{align*}

 Let \isVal{n}{\lift{\nat}}, \isVal{s}{\nseqc{n}{a.S}{a.b.T}}, \isVal{r}{\squash{R(n,s)}},
 and \isVal{b}{B(n,s)}.
The longest path in $base$ is when we compute $s(n-2)$, after 6 steps through
 the conditional. Let \evalCost{s(\toNum{n-2})}{c}{v}:

 \begin{align*}
&\bind{s(n-2)}{s_2}{\ret{(s_2,\suc{\zero},\zero)}}\\
   \mapsto^2 &\bind{s (\ret{\toNum{n-2}})}{s_2}{\ret{(s_2,\suc{\zero},\zero)}}\\
   \mapsto &\bind{s (\toNum{n-2})}{s_2}{\ret{(s_2,\suc{\zero},\zero)}}\\
   \mapsto^{1+c} &\bind{\ret{v}}{s_2}{\ret{(s_2,\suc{\zero},\zero)}}\\
   \mapsto &\ret{(v,\suc{\zero},\zero)}\\
 \end{align*}

 Where \eval{[\toNum{n-2}/a]/S}{p}, \eval{[\toNum{n-2}/a, v/b]T}{t} s.t. $p \ge c + t$.
Hence $p + 5$ steps suffices. 
\begin{align*}
\isVal{base}{\pityc{\isOf{n}{\lift{\nat}}}
  {\pityc{\isOf{s}{\lift{\nseqc{n}{a.S}{a.b.T}}}}{
  \squash{R(n,s)} \tolr \arrabtc{B(n,s)}{P(n,s)}
  {\_.\seq{[\toNum{n-2}/a]S}{p}{p + 11},\_.\_.\zp}}}}
\end{align*}

\begin{lemma}
  Let \isVal{n}{\lift{\nat}}, \isVal{s}{\nseqc{n}{a.P_n}{a.b.Q}}, and \isVal{r}{\squash{R(n,s)}},
  where $P_n = \ifthen{eq_b(a,n-1)}{\ret{3}}{P'}$ for some $P'$. Then
    \begin{itemize}
      \item If $n = \zero$:\\
        \stepIn{\dbr{dec}{base}{ind}{n}{s}{r}}{38}{\dbr{dec}{base}{ind}{\suc{n}}{s'}{r'}}, 
        \isVal{s'}{\nseqc{\suc{n}}{a.P_{\suc{n}}}{a.b.\zp}}, and \isVal{r'}{\squash{R(\suc{n},s')}}.
      \item Else, if \eval{s(n-1)}{\zero}:\\
        \stepIn{\dbr{dec}{base}{ind}{n}{s}{r}}{40}{\ret{v}} and \isVal{v}{P(n,s)}.
      \item Else, if \eval{s(n-1)}{v}, $v \ne \zero$:
        \begin{enumerate}
      \item If $n = 1$:
        \stepIn{\dbr{dec}{base}{ind}{n}{s}{r}}{41}{\dbr{dec}{base}{ind}{\suc{n}}{s'}{r'}},
        \isVal{s'}{\nseqc{\suc{n}}{a.P_{\suc{n}}}{a.b.\zp}}, and \isVal{r'}{\squash{R(\suc{n},s')}}.
      \item If $n > 1$:
        \stepIn{\dbr{dec}{base}{ind}{n}{s}{r}}{61}{
          \bind{\dbr{dec}{base}{ind}{\suc{n}}{s'}{r'}}{g}{M}
        },
        \isVal{s'}{\nseqc{\suc{n}}{a.P_{\suc{n}}}{a.b.\zp}}, \isVal{r'}{\squash{R(n,s')}},
        and \openCComp{\isOf{g}{P(\suc{n},s')}}{M}{P(n,s)}{\constp{18}}{\zp}
  \end{enumerate}
    \end{itemize}
\end{lemma}

\begin{proof}
  Let \eval{max(x,y)}{a} and \eval{min(x,y)}{b}. Below, \texttt{dec} and \texttt{base} take a 
  uniform constant amount of work at
13 and 14 steps respectively, since the extension operator maintains that it takes constant work 
to retrieve the last element, and that's all we need for both functions.

\paragraph{\textbf{Case} $n = \zero$:}
\begin{verbatim}
dbr(dec,base,ind,0,[],r) 
==> do dec(0,[]) as d in 
    case d of
    inl(p) => base 0 [] r p 
    inr(p) => ind 0 [] r (\t,r'. do [] ++ [t] as s in dbr(dec,base,ind,1,s,r')) 
==>^(3+13) do ret(inr(\b.*)) as d in 
    case d of
    inl(p) => base 0 [] r p 
    inr(p) => ind 0 [] r (\t,r'. do [] ++ [t] as s in dbr(dec,base,ind,1,s,r')) 
==> case inr(\b.*) of
    inl(p) => base 0 [] r p 
    inr(p) => ind 0 [] r (\t,r'. do [] ++ [t] as s in dbr(dec,base,ind,1,s,r')) 
==> ind 0 [] r (\t,r'. do [] ++ [t] as s in dbr(dec,base,ind,1,s, r')) 
==>^7 if(eq(0,0); do max(x,y) as a in (\t,r'. do [] ++ [t] as s in dbr(dec,base,ind,1,s,r')) a * ; ...)
==>^2 if(ret(tt); do max(x,y) as a in (\t,r'. do [] ++ [t] as s in dbr(dec,base,ind,1,s,r')) a * ; ...)
==> do max(x,y) as a in (\t,r'. do [] ++ [t] as s in dbr(dec,base,ind,1,s,r')) a * 
==>^2 do ret(a) as a in (\t,r'. do [] ++ [t] as s in dbr(dec,base,ind,1,s,r')) a * 
==> (\t,r'. do [] ++ [t] as s in dbr(dec,base,ind,1,s,r')) a * 
==>^3 do [] ++ [a] as s in dbr(dec,base,ind,1,s,*)
==>^2 do ret([a]) as s in dbr(dec,base,ind,1,s,*)
==> dbr(dec,base,ind,1,ret([a]),*)
\end{verbatim}

Hence we see that it takes 38 steps to reach the second recursive call.
Since \isVal{\empseq}{\lift{\nseqc{\zero}{\_.\zp}{\_.\_.\zp}}} and \isVal{r}{\squash{R(\zero,\empseq)}},
by construction, we know \isVal{[a]}
{\lift{\nseqc{\zero}{\_.P_{\suc{\zero}}}{\_.\_.\zp}}} with $P' = \seq{\zp}{p}{p+4}$.
The remaining potential is still $\zp$ since $\ifthen{eq_b(a,n)}{\ret{0}}{\ret{0}}$ is 0 for all 
  \isVal{a}{\lift{\nat_{\suc{\zero}}}}. Furthermore, $[a]$ is part of the spread: 
  \isVal{\triv}{\squash{R(\suc{\zero},[a])}}.

\paragraph{\textbf{Case} \eval{s(n-1)}{\zero}:}
\begin{verbatim}
dbr(dec,base,ind,n,s,r) 
==> do dec(n,s) as d in 
    case d of
    inl(p) => base n s r p 
    inr(p) => ind n s r (\t,r'. do s ++ [t] as s' in dbr(dec,base,ind,n+1,s',r')) 
==>^(3+13) do ret(inl(*,*)) as d in 
    case d of
    inl(p) => base n s r p
    inr(p) => ind n s r (\t,r'. do [x] ++ [t] as s' in dbr(dec,base,ind,2,s',r')) 
==> case inl(*,*) of 
    inl(p) => base n s r p
    inr(p) => ind n s r (\t,r'. do [x] ++ [t] as s' in dbr(dec,base,ind,2,s',r')) 
==> base n s r (*,*)
==>^(7+14) ret(v)
\end{verbatim}

Number of steps is 40. \isVal{v}{P(n,s)} by construction of \texttt{base}. 

\paragraph{\textbf{Case} \eval{s(n-1)}{v}, $v \ne \zero$, $n = 1$:}
\begin{verbatim}
dbr(dec,base,ind,1,s,r) 
==> do dec(1,s) as b in 
    case b of
    inl(p) => base 1 s r p
    inr(p) => ind 1 s r (\t,r'. do s ++ [t] as s' in dbr(dec,base,ind,2,s',r')) 
==>^(3+13) do ret(inr(\b.*)) as d in 
    case d of
    inl(p) => base 1 s r p 
    inr(p) => ind 1 s r (\t,r'. do s ++ [t] as s' in dbr(dec,base,ind,2,s',r')) 
==> case inr(\b.*) of
    inl(p) => base 1 s r p 
    inr(p) => ind 1 s r (\t,r'. do s ++ [t] as s' in dbr(dec,base,ind,2,s',r')) 
==> ind 1 s r (\t,r'. do s ++ [t] as s' in dbr(dec,base,ind,2,s',r')) 
==>^7 if(eq(1,0); ...; if(eq 1 1; ...; ...))
==>^2 if(ret(ff); ...; if(eq 1 1; ...; ...))
==> if(eq 1 1; do min(x,y) as b in (\t,r'. do s ++ [t] as s' in dbr(dec,base,ind,2,s',r')) b (*,*))
==>^2 if(ret(tt); do min(x,y) as b in (\t,r'. do s ++ [t] as s' in dbr(dec,base,ind,2,s',r')) b (*,*))
==> do min(x,y) as b in (\t,r'. do s ++ [t] as s' in dbr(dec,base,ind,2,s',r')) b (*,*)
==>^2 do ret(b) as b in (\t,r'. do s ++ [t] as s' in dbr(dec,base,ind,2,s',r')) b (*,*) 
==> (\t,r'. do s ++ [t] as s' in dbr(dec,base,ind,2,s',r')) b (*,*) 
==>^3 do s ++ [b] as s' in dbr(dec,base,ind,2,s',(*,*))
==>^2 do ret([s;b]) as s' in dbr(dec,base,ind,2,s',(*,*))
==> dbr(dec,base,ind,2,[s;b],(*,*))
\end{verbatim}

Number of steps is 41.
Again, \isVal{[s;b]}{\lift{\nseqc{\suc{n}}{a.P_{\suc{n}}}{a.b.\zp}}} and 
\isVal{(\triv,\triv)}{\squash{R(n,s)}} by construction. 

\paragraph{\textbf{Case} \eval{s(n-1)}{v}, $v \ne \zero$, $n > 1$:}

Let \eval{n-1}{n_1}, \eval{n-2}{n_2}, \eval{s(n_1)}{s_1} and \eval{s(n_2)}{s_2}.
\begin{verbatim}
dbr(dec,base,ind,n,s,r) 
==> do dec(n,s) as b in 
    case b of
    inl(p) => base n s r p
    inr(p) => ind n s r (\t,r'. do s ++ [t] as s' in dbr(dec,base,ind,n+1,s',r')) 
==>^(3+13) do ret(inr(\b.*)) as d in 
    case d of
    inl(p) => base n s r p 
    inr(p) => ind n s r (\t,r'. do s ++ [t] as s' in dbr(dec,base,ind,n+1,s',r')) 
==> case inr(\b.*) of
    inl(p) => base n s r p 
    inr(p) => ind n s r (\t,r'. do s ++ [t] as s' in dbr(dec,base,ind,n+1,s',r')) 
==> ind n s r (\t,r'. do s ++ [t] as s' in dbr(dec,base,ind,n+1,s',r')) 
==>^7 if(eq(n,0); ...; if(eq n 1; ...; ...))
==>^2 if(ret(ff); ...; if(eq n 1; ...; ...))
==> if(eq n 1; ...; ...) 
==>^2 if(ret(ff); ...; ...)
==> do s(n-1) as s1 in do s(n-2) as s2 in ... 
==>^2 do s(ret(n1)) as s1 in do s(n-2) as s2 in ... 
==> do s(n1) as s1 in do s(n-2) as s2 in ...
==>^(1+3) do ret(s1) as s1 in do s(n-2) as s2 in ...
==> do s(n-2) as s2 in ... 
==>^2 do s(ret(n2)) as s2 in ...
==> do s(n2) as s2 in ...
==>^(1+3) do ret(s2) as s2 in ...
==> do divmod(s2,s1) as q,r in ...
==>^2 do ret((q,r)) as q,r in ...
==>^5 do (\t,r'. do s ++ [t] as s' in dbr(dec,base,ind,n+1,s',r')) r * as g in M
==>^3 do do s ++ [r] as s' in dbr(dec,base,ind,n+1,s',*) as g in M
==>^2 do do ret([s;r]) as s' in dbr(dec,base,ind,n+1,s',*) as g in M 
==> do dbr(dec,base,ind,n+1,[s;r],*) as g in M 
\end{verbatim}

Where 
\begin{align*}
M \triangleq \bind{&\fst{g}}{d\\}{
                            \bind{&\fst{\snd{g}}}{k\\}{
                              \bind{&\fst{\snd{\snd{g}}}}{l\\}{
                    &\ret{(d,l,k-lq,\triv)}}}
          }
\end{align*}

By construction, \isVal{[s;r]}{\lift{\nseqc{\suc{n}}{a.P_{\suc{n}}}{a.b.\zp}}}
\isVal{\triv}{\squash{R(\suc{n},[s;r])}}, and 
\openComp{\isOf{g}{P(\suc{n},[s;r])}}{M}{P(n,s)}. We can show $M$ is a constant time computation:

Let \isVal{(d,k,l,\triv)}{P(\suc{n}, [s;r])}. Let \eval{l*q}{lq}, \eval{k-lq}{klq}.
\begin{verbatim}
do (d,k,l,*).1 as d in ...
==> do ret(d) as d in ...
==> do (d,k,l,*).2.1 as k in ...
==> do ret((k,l,*)).1 as k in ...
==> do (k,l,*).1 as k in ...
==> do ret(k) as k in ...
==> do (d,k,l,*).2.2.1 as l in ... 
==> do ret(k,l,*).2.1 as l in ...
==> do (k,l,*).2.1 as l in ...
==> do ret(l,*).1 as l in ...
==> do (l,*).1 as l in ...
==> do ret(l) as l in ...
==> ret((d,l,k-l*q,*)) 
==>^2 ret((d,l,k-ret(lq),*)) 
==> ret((d,l,k-lq,*)) 
==>^2 ret((d,l,ret(klq),*)) 
==> ret((d,l,klq,*)) 
\end{verbatim}

Number of steps is 18. Hence \openCComp{\isOf{g}{P(\suc{n},[s;r])}}{M}{P(n,s)}{\constp{18}}{\zp}

\end{proof}

So we know that there are a constant number of steps between each extension up the spread. 
Since the spread is linear (no branches), it suffices to find the length of the 
initial segment of non-zero entries.

\begin{lemma}
  Let \isVal{n}{\lift{\nat}}, \isVal{s}{\lift{\nseq{n}}}, and \isVal{r}{\squash{R(n,s)}}.
  Then the first $\zero$ in $s$ occurs at index $i$, such that:
  \begin{cases}
    x \ge F_i \text{ and } y \ge F_{i-1} \text{ if $x > y$ }\\
    y \ge F_i \text{ and } x \ge F_{i-1} \text{ o.w.}\\
  \end{cases}
\end{lemma}

Hence $max(x,y) \ge F_i$, and $i \le F^{-1}(max(x,y))$. We know the computation of gcd 
starts from the empty sequence, so:

\begin{verbatim}
dbr(dec,base,ind,0,[],r)
==>^38 dbr(dec,base,ind,1,[max(x,y)],r) 
==>^41 dbr(dec,base,ind,2,[max(x,y),min(x,y)],r) 
==>^61 do dbr(dec,base,ind,3,[max(x,y),min(x,y),s2],r) as g in M
==>^61 do 
        do 
          dbr(dec,base,ind,4,[max(x,y),min(x,y),s2,s3],r) 
        as g in M
       as g in M
...
==>^61 do 
        do 
        ...
          do
            dbr(dec,base,ind,i+1,s,r) 
          as g in M
        as g in M
       as g in M
==>^40 do 
        do 
        ...
          do
            ret(v0)
          as g in M
        as g in M
       as g in M
...
==>^18 do ret(v{i-2}) as g in M
==> M
==>^18 ret(v)
\end{verbatim}

Where we first build up a $(i-1)$-layer stack of \texttt{do} statements and then 
pop out $i-1$ times again. Number of steps is $38+41+61*(i-1) + 40 + (1+18)*(i-1) = 80i+39$. 
Thus 
\[\isCComp{gcd(x,y)}{\sigmatyc{\isOf{d}{\lift{\nat}}}{
    \sigmatyc{\isOf{l,k}{\lift{\z}}}{\ret{\pos{d}} =_{\z} kx + ly}}
}{\seq{F^{-1}(max(x,y))}{i}{80i+39}}{\zp}
\]


